El siguiente documento comprende el estudio y creación de un algoritmo híbrido entre Machine Learning y Loseless Compression Algorithm. Dado que  Internet crece cada día y los datos aumentan en volúmenes del orden de los Terabytes, por lo tanto nos interesa usar técnicas de compresión para realizar un procesamiento de mayor información con la menor cantidad de recursos.

Hoy existen variados tipos de web, redes sociales, microbloging, web informativas, etc. El contenido proporcionado a los usuarios finales  ya no es estático y esto permite que los mismos puedan generar, aportar, modificar contenido, en vista de esto la ingeniería ofrecida para construir web está en constante evolución lo que ha ayudado a generar más recursos para poder desarrollarla. 
Muchas de estas nuevas tecnologías han permitido entregar una mejor experiencia al momento de navergarla, aún cuando se haya avanzado, esto no ha permitido crear Webs que sean por sí mismas inteligentes y puedan ir anticipando el comportamiento del usuario,  por ejemplo; disminuir la latencia desde que se abre una web ya visitada o desde que se navega dentro de un sitio con alta demanda; también desde el punto de vista de la arquitectura como servicio que las hospedan no se ha visto incluída, dando un aspecto económico a los recursos utilizados. Si bien el crecimiento de los recursos de almacenamiento en la nube se encuentran en apogeo, las redes no crecen a la misma velocidad. 

Este proyecto busca predecir el comportamiento del usuario dentro de una web, usando técnicas de aprendizaje y algoritmo de compresión. Con este propósito se trabajará para crear un modelo híbrido.}
