% !TeX spellcheck=es
% @Author @jaimeguzman
% Base latex template @AdinRamirez
% Repo: git@giteit.udp.cl:udp/udp-latex.git

\documentclass{udparticle}

%% DEFINITIONS
% para usarse en la escritura técnica (investigativa)
% revisar uso en el internet
\usepackage{xspace}
\makeatletter
\DeclareRobustCommand\onedot{\futurelet\@let@token\@onedot}
\newcommand\@onedot{\ifx\@let@token.\else.\null\fi\xspace}

\newcommand\eg{\emph{e.g}\onedot} \newcommand\Eg{\emph{E.g}\onedot}
\newcommand\ie{\emph{i.e}\onedot} \newcommand\Ie{\emph{I.e}\onedot}
\newcommand\cf{\emph{cf}\onedot} \newcommand\Cf{\emph{Cf}\onedot}
\newcommand\etc{\emph{etc}\onedot} \newcommand\vs{\emph{vs}\onedot}
\newcommand\wrt{w.r.t\onedot} \newcommand\dof{d.o.f\onedot}
\newcommand\etal{\emph{et al}\onedot}
\newcommand\adhoc{\emph{ad hoc}\xspace}
\makeatother

\setlogo{EITFI}

\title{ Predicción de patrones de comportamiento de usuarios en la web basado en web access log para un Administrador de Contenidos Open Source}
\author{  
  Jaime Guzmán\\{\small\ttfamily mail@jguzman.cl}\protect\\[5pt]%
  {\small Adin Ramírez\thanks{Profesor guía} y Francisco Claude\thanks{Profesor comisión}}%
  }

\begin{document}

\maketitle

\section{Antecedentes y Motivación}

\subsection{Contexto}

  La Web crece constantemente y por ende su infraestructura, como también la concurrencia de los mismos sistemas. 
  Paralelamente se suma un costo exponencial de recursos que no son optimizados para poder dar una experiencia de usuario con calidad de servicio.
  Podemos entender, entonces, que llegará un punto en que el no tener servidores de gran rendimiento será lo óptimo para dar una calidad de servicio web, ya que el ancho de banda de Internet no crecerá a la misma proporción. 
  Adicionalmente, las tecnologías para la creación de web dinámicas han evolucionado a favor del cliente.
  Se tienen \emph{MEAN stacks} que disminuyen considerablemente la carga de un servidor, por lo cual, hoy en día, un buen servicio web es proveer una balanceada carga dentro del cliente y el servidor.

  Por lo mismo, es de gran interés predecir los movimientos siguientes que un usuario tendrá en una determinada web.
  Entendiendo que la forma en que navega una persona es su comportamiento web, y que se puede reflejar mediante Web Access Log. 
  El registro de los mismos, de manera procesada o pre-procesada, ayudaría a ingenieros de desarrollo web y diseñadores, como a los mismo usuarios finales a tener una experiencia de usuario mejor.
  
  Hoy en día, las web no pueden ser simplemente dinámicas, éstas deben poseer una adaptabilidad a la demanda del usuario o proveer información que permita adaptarse a los eventos.
  Por lo tanto, es de interés el profundizar en este tópico.


\subsection{Trabajos relacionados}

% @TODO: SEGUIR TRABAJANDO EN ESTA BREVE INTRODUCCION
En este tema convergen dos áreas, por un lado existe trabajo para crear estructuras eficientes para predicciones basadas en algoritmos de compresión~\cite{BWT}, y, por otro lado, el uso de algoritmos de aprendizaje para realizar clustering y predecir el comportamiento.

En la literatura, el tema de la predicción en la web se ha presentado como un tema concurrente, y ha sido abarcado por varios autores. 
Tenemos los siguientes trabajos en orden cronológico:

\begin{enumerate}
  \item Dongshan y Junyi~\cite{tmmd} destacan que un modelo de Markov puede ayudar a predecir el comportamiento de un usuario, pero con ciertas limitaciones.
  Para solucionarlo presentan un nuevo modelo de Markov basado en una representación de \emph{Tree Order Model}, el cual es un híbrido entre un modelo de markov tradicional y una representación de árbol, bautizada como HTMM (por sus siglas en inglés, \emph{Hybrid-Order Tree Markov Model}).
  Su modelo fue presentado en 2002, y da una importancia a conocer la predicción de los \emph{web access}, dada la importancia de creación de redes, la minería de datos, e-commerce, y otras áreas.

  \item Domenech \etal~\cite{domenech}, muestran un estudio de los rendimientos de técnicas de recuperación de datos.
  Las mismas se pueden utilizar para dar una entrada ideal a algoritmos de aprendizaje o algoritmos de predicción. 
  Los conceptos más importantes son las nuevas variables de caracterización, temporalidad, espacio y geografía, que se le suman a la predicción. 
  Además de comenzar un trabajo más elaborado de como tomar una predicción, se introducen conceptos como predicciones genéricas o específicas, variables de uso de recursos a nivel de red ó nivel procesamiento.
  Finalmente, se presenta un modelo predictivo que puede ayudar a disminuir la latencia entre la petición del cliente y la respuesta de la web, dando así un mejor rendimiento y \emph{QoS}.

  \item Casi cinco años después los sitios web son cada vez más dinámicos y responden a eventos cada vez más adaptables a los usuarios. 
  El trabajo de Eremic \etal~\cite{Dragica2010} propone una optimización de la ruta de navegación de los sitios y estructura de la navegación del sitio. 
  % @TODO detallar más explicarlo mas simple, darle mas enfoque al usuario segúnn del punto de vista que de los docuentos 
  % como los autores antteriores.

  \item Chen \etal~\cite{yuhua2011} dan una nueva perspectiva enfocada a entregar una clara recomendación a los usuarios basada en la misma propuesta de este proyecto, los access log.
  El primer análisis realizado por los autores cubre las reglas asociativas que requiere un sistema de recomendación, pero en las pruebas propiamente tales encuentran que el análisis de los patrones detectadados dan una representación clara de como optimizar la web, y finalmente mediante sus pruebas logran una recomendación de calidad.

  \item Rajimol y Raju~\cite{rajimol2012} minaron los patrones de los accesos web, donde el enfoque es usar los registros de acceso para crear subsecuencias y realizar comparaciones.
  La literatura presenta un interés para poder anticipar el patrón de comportamiento de la web.
  % @TODO reflexionar mas sobre este paper

  \item Kewen~\cite{kewen2012} realizó un análisis más profundo del \emph{web usage minning}.
  Parte de la importancia de este trabajo, es que después de minar los registros de accesos, logran reducir la ``\emph{bad data}''.
  %@TODO: Preguntar si este paper se escapa mucho del tema prinicipal, pero parece interesante  

  \item Poornalatha y Raghavendra~\cite{Poornalatha2012} establecen que se pueden utilizar máquinas de aprendizaje para predecir basándose en distintas entre clusters. Estos autores, al igual que Domenech \etal~\cite{Domenech2006} y Dongshan y Junyi~\cite{Dongshan2002}, comparan el objetivo de optimizar los recursos tanto en redes (disminución de latencia) y experiencia de usuario.

  \item Claude \etal~\cite{Claude2014} presentan una estructura de representación eficiente que permite dar una representación de \emph{web access log} y ofrecen las operaciones básicas de WUM.
\end{enumerate}


\subsection{Motivación}

La motivación de este proyecto de titulo es lograr una predicción mediante \emph{web access log}, determinando que metodología usada es la más eficiente, como también una predicción del comportamiento del usuario en la web, usando plataformas experimentales como administradores de contenidos, con rutas URL limpias predeterminadas.


\subsection{Descripción de la solución }

Se implementará un algoritmo Lempel Ziv~78 y un modelo tradicional de Markov, con un dataset predefinido.
Se medirá el rendimiento de ambas implementaciones, en varios experimentos y configuraciones, con el fin de encontrar un taxonomía de predicciones favorables en el set de datos.


\subsection{Objetivo General}
 
El objetivo general del proyecto de título es poder encontrar predicciones basadas en patrones encontrados en \emph{access web log}, que permitan a un \emph{CMS Open Source} adaptarse a los solicitudes del cliente final.
 
\subsection{Objetivo  Específicos }
 
A continuación se detallan los objetivos específicos:
 
\begin{enumerate}
  \item Estudiar y describir el estado del arte respecto a las variantes existentes de Modelos Predictivos, describiendo limitaciones pros y contras entre áreas de estudio, como futuras implementaciones y mejoras a la web.
  \item Implementar un algoritmo de compresión para usarlo como un algoritmo de predicción.
  \item Implementar un algoritmo basado en modelos de Markov que permita entregar una predicción.
  \item Preparar conjuntos de datos de prueba 
  \item Ejecutar pruebas para medir, y clasificar las predicciones de rendimiento usando nuestra implementación, y comparar con algoritmos expuestos en la literatura.
  \item Analizar los resultados obtenidos y mostrar el uso predictor.
\end{enumerate}


\section{Metodología de trabajo}

Se investigará en detalle el funcionamiento del algoritmo, para luego generar una implementación del algoritmo aproximado.
Durante el desarrollo del proyecto de título, se tomarán decisiones de implementación que permitan llevar la propuesta teórica a una implementación práctica.

Ya implementado el algoritmo, se realizarán pruebas en un dataset de la literatura y se creará un módulo para el CMS Drupal que permita generar predicciones.

Estas pruebas serán luego ejecutadas en una máquina de prueba y recopilados los resultados para ser analizados.
El análisis comprenderá como se relacionan las áreas de \emph{Information Retrieval}, \emph{Machine Learning} y Compresión de Algoritmos con el fin de lograr unificar la propuesta de solución. 

% Preguntar a Adin que recomendaciones colocar para las fechas.
\section{Cronograma de actividades, hitos y entregables}
  \begin{center}
  \begin{tabular}{ll}
  \hline\noalign{\smallskip}
  Fecha & Actividad \\
  \hline\noalign{\smallskip}
  31/07/2015 & Presentación anteproyecto (firmado por profesor guía y comisión).\\
  02/08/2015 & Entrega resultados anteproyectos.\\
  04/08/2015 & Entrega anteproyectos corregidos.\\
  09/08/2015 & Entrega resultados correcciones.\\
  25/09/2015 & Marco de trabajo.\\
  23/09/2015 & Primer prototipo de propuesta.\\
  13/10/2015 & Resultados parciales.\\
  27/10/2015 & Entrega Memoria Titulo I/II firmada por profesor guía.\\
  11/11/2015 & Fecha límite para que la comisión entregue correcciones.\\
  18/12/2015 & Fecha límite para que se realicen correcciones.\\
  \hline
  \end{tabular}
  \end{center}


\section{Resultados esperados}

Se espera crear un estudio completo de las áreas de interés y la implementación de ambas metodologías para ambos casos.
Además se espera entregar un listado de puntos de evaluación para implementar un algoritmo en el \emph{backend} de un CMS que permita crear las predicciones
del usuario basado en su comportamiento.

  
% ------------------------- %
% \bibliographystyle{IEEEtran}
\bibliographystyle{plain}
\bibliography{anteproyecto}
 

\end{document}
