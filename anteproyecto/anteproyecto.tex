% !TeX spellcheck=es
% @Author @jaimeguzman
% Base latex template @AdinRamirez
% Repo: git@giteit.udp.cl:udp/udp-latex.git

\documentclass{udparticle}

%% DEFINITIONS
% para usarse en la escritura técnica (investigativa)
% revisar uso en el internet
\usepackage{xspace}
\makeatletter
\DeclareRobustCommand\onedot{\futurelet\@let@token\@onedot}
\newcommand\@onedot{\ifx\@let@token.\else.\null\fi\xspace}

\newcommand\eg{\emph{e.g}\onedot} \newcommand\Eg{\emph{E.g}\onedot}
\newcommand\ie{\emph{i.e}\onedot} \newcommand\Ie{\emph{I.e}\onedot}
\newcommand\cf{\emph{cf}\onedot} \newcommand\Cf{\emph{Cf}\onedot}
\newcommand\etc{\emph{etc}\onedot} \newcommand\vs{\emph{vs}\onedot}
\newcommand\wrt{w.r.t\onedot} \newcommand\dof{d.o.f\onedot}
\newcommand\etal{\emph{et al}\onedot}
\newcommand\adhoc{\emph{ad hoc}\xspace}
\makeatother

\setlogo{EITFI}


\title{ Modelo Híbrido de Lempel-Ziv y Machine Learning para predicción de comportamiento de usuarios basado en Web Access Log }
\author{ Jaime Guzmán, \\ {\small mail@jguzman.cl}\protect\\[5pt]%
  {\small Adin Ramírez\thanks{Profesor guía}, y Francisco Claude \thanks{Profesor comisión}} \\
  {\small Escuela de Informática y Telecomunicaciones Universidad Diego Portales} \\
  {\small adin.ramirez@mail.udp.cl y fclaude@recoded.cl }\protect\\[5pt]% 
}


\begin{document}

\maketitle
\section{Antecedentes y Motivación}

\subsection{Contexto}

  La Web crece constantemente día a día, también los tipos de usuarios y la concurrencia de los mismos sistemas, por ende su infraestructura, la cual para usuarios finales se traduce en tener mayor latencia al acceder a recursos y un nivel de experiencia no deseado. 
  Paralelamente se suma un costo exponencial de recursos tanto en tecnologías de desarrollo como servicio que no son optimizados para poder dar una experiencia de usuario con calidad de servicio. Entonces podemos reflexionar que tener mayores recursos no mejorará el rendimiento ni tampoco será lo óptimo para dar una calidad de servicio web, ya que el ancho de banda de Internet no crecerá a la misma proporción.
   
  Adicionalmente, las tecnologías para la creación de web dinámicas y asíncronas ha evolucionado a favor del cliente. Hoy en día se poseen \emph{MEAN stacks} que disminuyen considerablemente la carga de un servidor, por lo cual, un buen servicio web es proveer una balanceada carga dentro del cliente y el servidor, pero cuando se poseen un volumen de datos considerables es requerido tomar decisiones que los recursos de infraestructura y lenguajes orientados a web no cubren, es ahí el interés de hacer web inteligente.

  Es de gran interés para los investigadores predecir los siguientes movimientos que un usuario tendrá en una determinada web. Entendiendo que la manera en que un usuario final navega es su comportamiento registrado en una web, y que se puede analizar, estudiar y registrar mediante \emph{Web Access Log} y a los cuales se puede hacer una minería de datos en estos registros, Web Usage Minning. El por qué de hacer minería de datos es debido a que cada día la web genera una innumerable cantidad de datos, por lo cual usar un algoritmo como LZ-78, que además de ser un algoritmo de compresión, se puede usar como un algoritmo de predicción y trabajar con una mayor cantidad de datos sin tener pérdida.
  
  Los registros de accesos de manera procesada o pre-procesada, ayudaría a ingenieros de desarrollo web, expertos de usabilidad, diseñadores y usuarios finales a tener una experiencia mucho mejor, disminuyendo por ejemplo la latencia en respuestas por parte de cada petición que realizan.
  
  Hoy en día, las web no pueden ser simplemente dinámicas en contenidos, debe poseer una adaptabilidad a la demanda del usuario o proveer información que permita adaptarse a los eventos, por lo tanto, es de interés el profundizar en este tópico.


\subsection{Trabajos relacionados}

% @TODO: SEGUIR TRABAJANDO EN ESTA BREVE INTRODUCCION
En este tema convergen tres áreas, por un lado existe trabajo para crear estructuras eficientes para predicciones basadas en algoritmos de compresión, como es en el caso de~\cite{Claude2014}, y, por otro lado, el uso de algoritmos de aprendizaje para realizar clustering y predecir el comportamiento basado en el mismo contenido o en la distancia del contenido que visita el usuario actual al contenido clusterizado, como es el caso de~\cite{Poornalatha2012}, inclusive se han utilizado modelos de Markov en~\cite{Dongshan2002}  para poder modelar el comportamiento de la web.La tercera área son los Sistemas de Recomendación, la cual en este proyecto no se tocará pero si se mencionará el enfoque práctico que presenta  esta área como un foco de múltiples implementaciones. 


En la literatura, el tema de la predicción en la web se ha presentado como un tema recurrente, y ha sido abarcado por varios autores. Tenemos los siguientes trabajos de interés:

\begin{enumerate}
  \item Dongshan y Junyi~\cite{Dongshan2002} destacan que un modelo de Markov puede ayudar a predecir el comportamiento de un usuario, pero con ciertas limitaciones .  Para solucionarlo presentan un nuevo modelo de Markov basado en una representación de \emph{Tree Order Model}, el cual es un híbrido entre un modelo de markov tradicional y una representación de árbol, bautizada como HTMM (por sus siglas en inglés, \emph{Hybrid-Order Tree Markov Model}).
  Su modelo fue presentado en 2002, y da una importancia a conocer la predicción de los \emph{web access}, dada la importancia de creación de redes, la minería de datos, e-commerce, y otras áreas.

  \item Domenech \etal~\cite{Domenech2006}, muestran un estudio de los rendimientos de técnicas de recuperación de datos.
  Las mismas se pueden utilizar para dar una entrada ideal a algoritmos de aprendizaje o algoritmos de predicción. 
  Los conceptos más importantes son las nuevas variables de caracterización, temporalidad, espacio y geografía, que se le suman a la predicción. 
  Además de comenzar un trabajo más elaborado de como tomar una predicción, se introducen conceptos como predicciones genéricas o específicas, variables de uso de recursos a nivel de red ó nivel procesamiento.
  Finalmente, se presenta un modelo predictivo que puede ayudar a disminuir la latencia entre la petición del cliente y la respuesta de la web, dando así un mejor rendimiento y \emph{QoS}.


  % @TODO detallar más explicarlo mas simple, darle mas enfoque al usuario segúnn del punto de vista que de los docuentos 
  % como los autores antteriores.

  \item Chen \etal~\cite{Chen2011} dan una nueva perspectiva enfocada a entregar una clara recomendación a los usuarios basada en la misma propuesta de este proyecto, los access log.
  El primer análisis realizado por los autores cubre las reglas asociativas que requiere un sistema de recomendación, pero en las pruebas propiamente tales encuentran que el análisis de los patrones detectadados dan una representación clara de como optimizar la web, y finalmente mediante sus pruebas logran una recomendación de calidad.

  \item Rajimol y Raju~\cite{Rajimol2012} minaron los patrones de los accesos web, donde el enfoque es usar los registros de acceso para crear subsecuencias y realizar comparaciones.
  La literatura presenta un interés para poder anticipar el patrón de comportamiento de la web.
  % @TODO reflexionar mas sobre este paper

  \item Kewen~\cite{kewen2012} realizó un análisis más profundo del \emph{web usage minning}.
  Parte de la importancia de este trabajo, es que después de minar los registros de accesos, logran reducir la ``\emph{bad data}''.
  %@TODO: Preguntar si este paper se escapa mucho del tema prinicipal, pero parece interesante  

  \item Poornalatha y Raghavendra~\cite{Poornalatha2012} establecen que se pueden utilizar máquinas de aprendizaje para predecir basándose en distintas entre clusters. Estos autores, al igual que Domenech \etal~\cite{Domenech2006} y Dongshan y Junyi~\cite{Dongshan2002}, comparan el objetivo de optimizar los recursos tanto en redes (disminución de latencia) y experiencia de usuario.

  \item Claude \etal~\cite{Claude2014} presentan una estructura de representación eficiente que permite dar una representación de \emph{web access log} y ofrecen las operaciones básicas de WUM.
\end{enumerate}


\subsection{Motivación}

La motivación de este proyecto de título es poder converger dos áreas de estudio para el mismo problema, predicción del comportamiento de usuario en una web.

Usar recursos que ofrece la industria y la academia para poder crear un algoritmo o modelo que permita dar predicciones del comportamiento de los usuarios en una web, basados en los registros de web access log.


\subsection{Descripción de la solución }
Básicamente se pretende crear un algoritmo que represente un punto en común entre las áreas de Machine Learning y Lossless Data Compression, tomando un compresor, ej. es el caso de LZ-78, el cual se se puede usar como un algoritmo de predicción.

Se busca alterar un algoritmo de compresión de la familia de Lempel Ziv y crear un híbrido con un Modelo de basado en Markov representado en \emph{trie}, lo anterior no es la solución definitiva pero es la ruta que se busca estudiar, acorde al transcurso del desarrollo puede sufrir variaciones, pero siguiendo la premisa de la integración de un algoritmo de compresión con técnicas de Machine Learning+, ofrecidas por \textit{Prediction.IO (PIO)}.

Se utilizará como base la implementación Open Source de Prediction.IO para poder hacer el entrenamiento de un data set provisto por el diario español \emph{Prisa}, el cual posee 1.000.000 de registros correspondientes al año 2009, septiembre. La data se adecuará al formato de la PIO, para la entradas y pruebas de la suite basado en \emph{DASE}, de las siglas de Data Source and Data Preparator, Algorithm, Serving, Evaluation Metrics.

Se realizará comparaciones con algoritmos de la familia de Lempel-Ziv, en el estado del arte propuesto se analizarán con mayor detalle, los cuales se seleccionaran acorde a literatura y recomendaciones de los profesores guía. Adicionalmente se usarán las implementaciones ofrecidas por \emph{PIO}.

En una etapa más experimental todas las mediciones y experimentos correrán sobre la infraestructura de \emph{PIO} con y sin el algoritmo propuesto con el fin de encontrar valores cuantitativos de predicciones favorables en el set de datos, para cada caso. La solución no pretende tener un mejor rendimiento o exactitud, pero sí poder combinar ambos campos en un algoritmo y ponerlo a disposición de herramientas de la industria.
%Con que algoritmos voy a comparar.


\subsection{Objetivo General}
 
	El objetivo general del proyecto de título es crear un algoritmo que use un algoritmo de compresión en conjunto con una infraestructura de Machine Learning, teniendo en consideración algoritmos de tipo \emph{Variable Markov Model} y encontrar predicciones basadas en patrones encontrados en \emph{access web log} previamente limpios, que permitan entregar mejor información sobre la minería en los accesos, con manipulación off-line .
 
\subsection{Objetivo  Específicos }
 
A continuación se detallan los objetivos específicos:
 
\begin{enumerate}
  \item Estudiar y describir el estado del arte respecto a las variantes existentes de Modelos Predictivos, describiendo limitaciones pros y contras entre áreas de estudio, como futuras implementaciones y mejoras para la web.
  \item Estudiar un algoritmo de compresión para usarlo como un algoritmo de predicción.
  \item Estudiar un algoritmo basado en modelos de Markov que permita entregar una predicción.
  \item Implementación de algoritmo híbrido propuesto.
  \item Usar un set de datos, con distintos algoritmos para establecer valores base y comparaciones.
  \item Definir métricas de efectividad de la predicción.
  \item Definir métricas de relación de comportamiento de usuario con la predicción. 
  \item Ejecutar pruebas para medir y clasificar las predicciones usando nuestra implementación y comparar con algoritmos seleccionados de la familia de Lempel-Ziv y algoritmos ofrecidos por prediction.io. % Aun no me caso con que algoritmo de ML voy a correr
  
  \item Analizar los resultados obtenidos y compararlos cuantitativamente en base a las métricas de predicción.
  \item Mostrar el uso del algoritmo y señalar casos de  implementaciones en la vida real.
\end{enumerate}


\section{Metodología de trabajo}

Se investigará en detalle el funcionamiento del algoritmo de compresión Lempel Ziv, determinando una combinación o viceversa de una representación arbórea de un modelo de Markov, para luego generar una implementación del algoritmo aproximado.
Durante el desarrollo del proyecto de título, se tomarán decisiones de implementación que permitan llevar la propuesta teórica a una implementación práctica. Ya implementado el algoritmo, se realizarán pruebas en un data set anteriormente señalado.

El trabajo será dirigido y monitoreado mediante el Gitlab de la escuela \emph{EIT} y su sistema de \emph{issues tracking}, para resolver hitos, componente y la escritura del documento de tesis. A nivel de código e implementación se ocuparán servicios recomendados por los profesores guías y recursos disponibles por la facultad.

Estas pruebas serán luego ejecutadas en una máquina de prueba y recopilados los resultados para ser analizados.
El análisis comprenderá como se relacionan las áreas de \emph{Machine Learning} y \emph{Compresión de Algoritmos} con el fin de lograr unificar una propuesta de algoritmo. 

\section{Resultados esperados}

Se espera crear un estudio que abarque las áreas de interés y la implementación de ambas metodologías que converjan en un algoritmo, 
también la implementación que la industria del Data Science, prediction.IO, ofrece y además ver sus fortalezas en aspectos de predicciones y comparar con nuestra propuesta.

Además se espera entregar un listado de puntos de evaluación y mejoras propuestas para el algoritmo propuesto.


% Preguntar a Adin que recomendaciones colocar para las fechas.
\section{Cronograma de actividades, hitos y entregables}
  \begin{center}
  	\begin{tabular}{ll}
  		\hline\noalign{\smallskip}
  		Fecha & Actividad  \\
  		\hline\noalign{\smallskip}
  		05/08/2015 & Presentación anteproyecto (firmado por profesor guía y comisión) \\
  		& Entregable: Documento de Anteproyecto \\
  		07/08/2015 & Entrega resultados anteproyectos. \\
  		& Entregable: Correcciones de Anteproyecto \\
  		10/08/2015 & Entrega anteproyectos corregidos. \\
  		& Entregable: Anteproyecto corregido \\
  		12/08/2015 & Entrega resultados correcciones. \\
  		17/08/2015 & Inicio Marco de trabajo. \\
  		20/08/2015 & Buscar todos los papers relacionados tanto en temas como trabajos posibles que tengan relación.\\
  		24/08/2015 & Escribir estado del arte. \\
  		25/08/2015 & Reutilizar textos de anteproyecto para re-utilización.\\
  		27/08/2015 & Revisión estado del arte por profesor Guía.\\
  		31/08/2015 & Hacer un análisis completo de Prediction.IO y comienzo de estudio de uso.\\
  		& Entregable: Avance documento escrito y estudio de container Docker-PIO \\
  		%Septiembre
  		03/09/2015 & Ver como incluir el algoritmo híbrido.\\
  		& Entregable: Correcciones de Anteproyecto \\
  		04/09/2015 & Estructurar capítulos de tesis.\\
  		05/09/2015 & Proponer un introducción.\\
  		& Entregable: Texto de introducción \\
  		06/09/2015 & Tabular datos provenientes del diario de Prisa.\\
  		07/09/2015 & Aprender a utilizar el algoritmo de Claude para poder dar representación. \\
  		10/09/2015 & Investigar representación de markov se puede implementar con el árbol del LZ-78.\\
  		10/09/2015 & Aprender noción básica de lenguaje de programación Pascal.\\
  	%	12/09/2015 & Levantar entorno Experimental basado en Dockers, en ambiente UDP o Propio.\\
  		14/09/2015 & Hacer pruebas de validación para el entorno de desarrollo y pruebas.\\
  		17/09/2015 & Documentación del proceso de entorno.\\
  		24/09/2015 & Iteración de validación con profesores Guías.\\
  		25/09/2015 & Redacción de capitulo ``Experimental''.\\
  		28/09/2015 & Recolección de datos para casos sin modelo híbrido.\\
  		29/09/2015 & Investigación teórica de implementación.\\
  		
  		%Octubre
  		01/10/2015 & Decisión del lenguaje de implementación.\\
  		05/10/2015 & Programación. \\
  		12/10/2015 & Iteración de Codificación y validación con profesor guía.\\
  		15/10/2015 & Arreglos de algoritmo propuesto.\\
  		19/10/2015 & Revisión e iteración de bugs. \\
  		22/10/2015 & Iteración de revisión. \\
  		30/10/2015 & Comienzo de etapa de pruebas.\\
  		& Entregable: Código/Algoritmo propuesto \\
  		
  		%Noviembre
  		02/11/2015 & Correcciones. \\
  		06/11/2015 & Comienzo de etapas de pruebas con dataset\\
  		09/11/2015 & Comparación con datos de literatura.\\
  		10/11/2015 & Escritura de capitulo.\\
  		12/11/2015 & Análisis de Resultados.\\
  		13/11/2015 & Comparativa.\\
  		& Entregable: Documento de tesis para revisión 1 \\
  		14/11/2015 & Trabajos futuro.\\
  		17/11/2015 & Conclusiones.\\
  		23/11/2015 & Revisión de capítulo.\\
  		25/11/2015 & Correcciones en \LaTeX. \\
  		27/11/2015 & Entrega Memoria Título I/II firmada por profesor guía.\\
  		& Entregable: Documento\\ 
  		
  		04/12/2015 & Fecha límite para que la comisión entregue correcciones.\\
  		11/12/2015 & Fecha límite para que se realicen correcciones.\\
  		
  		%Diciembre
  		04/12/2015 & Creación de presentación.\\
  		& Entregable: Documento de Presentación  \\
  		27/12/2015 & Presentación.\\
  		
  		
  		\hline
  	\end{tabular}
  \end{center}


\section{Cursos Relacionados}

Para el desarrollo de este proyecto se tomarán las bases teóricas y prácticas aprendidas en cursos como Estructuras de datos, 
el cual entrega el conocimiento esencial  para la compresión de estructuras como un \emph{TRIE}, sobre todo para el estudio de los arboles generados 
tanto como Markov y Lempel Ziv, cursos con fundamento matemática como Probabilidades y Estadísticas ayudarán a la compresión de los eventos
y los modelos de Markov, en conjunto con Modelos Estocásticos. 

Finalmente los cursos de Inteligencias Artificial, Minería de Datos e Implementación de Algoritmos Avanzados, dan el fundamento mas detallado
en las áreas que este proyecto se fundamenta, tanto en aspectos para las áreas de Machine Learning, Clustering, \emph{WUM} e implementación de algoritmos de compresión
vistos en cátedras electivas.



  
% ------------------------- %
% \bibliographystyle{IEEEtran}
\bibliographystyle{plain}
\bibliography{anteproyecto}
 

\end{document}
