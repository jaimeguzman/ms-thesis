\documentclass[thesis]{udpbook}



  
\begin{document}



\degreedoc
{ Degree Doc }{Ingeniero Civil Informático y Telecomunicaciones}{Jaime Guzmán}

\udplogo
%\UDP
 
\author{ Jaime Guzmán }








 
\abstract{ asdasdasdasd }
\dedicatory{ lorem lorem lorem }

%============================ANTECEDENTES Y MOTIVACION=======================%

\section{Título del Subcapítulo 1 }\label{chsub:Título del Subcapítulo 1}

\parindent=0pt Quo no falli viris intellegam, ut fugit veritus placerat per. Ius id vidit volumus mandamus, vide veritus democritum te nec, ei eos debet libris consulatu. No mei ferri graeco dicunt, ad cum veri accommodare. 


\subsection{Título de la sección 1 del subcapítulo 1}\label{chsub:Título de la sección 1 del subcapítulo 1}


\parindent=0pt Eos vocibus deserunt quaestio ei. Blandit incorrupte quaerendum in quo, nibh impedit id vis, vel no nullam semper audiam. 

\vspace{0.5cm}
\parindent=30pt Ei populo graeci consulatu mei, has ea stet modus phaedrum. Inani oblique ne has, duo et veritus detraxit. 


\subsubsection{Título de la subsección 1(no aparece en tabla de contenidos)}\label{chsub:Título de la subsección 1}

\parindent=0pt Lorem ipsum ad his scripta blandit partiendo, eum fastidii accumsan euripidis in, eum liber hendrerit an. 



\begin{description}
\item[a)] Punteo de primer nivel 1
\item[b)] Punteo de primer nivel 2

    \begin{description}
    \item[i)] Punteo de segundo nivel 1
    \item[ii)] Punteo de segundo nivel 2
  
      \begin{description}
      \item[-] Punteo de tercer nivel 1
      \item[-] Punteo de tercer nivel 2
    \end{description}  
    
    \item[iii)] Punteo de segundo nivel 3
  \end{description}

\item[c)] Punteo de primer nivel 3
\end{description}

\begin{enumerate}
\item Enumerar 1.
\item Enumerar 2.
\item Enumerar 3
\end{enumerate}

\begin{itemize}
\item Ítem de lista 1.
\item Ítem de lista 2.
\item Ítem de lista 3.
\end{itemize}


\vspace{0.5cm}
\parindent=30pt Quo mundi lobortis reformidans eu, legimus senserit definiebas an eos. Eu sit tincidunt incorrupte definitionem, vis mutat affert percipit cu, eirmod consectetuer signiferumque eu per. In usu latine equidem dolores.

\subsection{Título de la sección 2 del subcapítulo 1}\label{chsub:Título de la sección 2 del subcapítulo 1}

\parindent=0pt Quo no falli viris intellegam, ut fugit veritus placerat per. Ius id vidit volumus mandamus, vide veritus democritum te nec, ei eos debet libris consulatu. No mei ferri graeco dicunt, ad cum veri accommodare. 
 


\section{Título del Subcapítulo 2} \label{chsub:Título del Subcapítulo 2}

\parindent=0pt Lorem ipsum ad his scripta blandit partiendo, eum fastidii accumsan euripidis in, eum liber hendrerit an. 

\vspace{0.5cm}
\parindent=30pt Quo mundi lobortis reformidans eu, legimus senserit definiebas an eos. Eu sit tincidunt incorrupte definitionem, vis mutat affert percipit cu, eirmod consectetuer signiferumque eu per.

\subsection{Título de la sección 1 del subcapítulo 2}\label{chsub:Título de la sección 1 del subcapítulo 2}

\parindent=0pt Eos vocibus deserunt quaestio ei. Blandit incorrupte quaerendum in quo, nibh impedit id vis, vel no nullam semper audiam. 


\chapter[TÍTULO DEL SEGUNDO CAPÍTULO ]{TÍTULO DEL SEGUNDO CAPÍTULO }\label{ch:capitulo2}
 

%=================================================ANTECEDENTES Y MOTIVACION========================================%

\section{Ecuaciones} \label{chsub:ecuaciones}


Esta ecuación $3=\frac{6}{2}$ está en la frase, la siguiente no.


  \[3=\frac{6}{2}
\]


Para referenciar una ecuación~\ref{CODIGO ECUACION}.

\begin{equation}
\label{CODIGO ECUACION}
\lim_{n\to\infty}
\sum_{i=1}^n\frac{1}{i^2}=
\frac{\pi^2}{6}
\end{equation}


\section{Figuras} \label{chsub:figuras}

\parindent=0pt A continuación se pueden ver dos figuras, la primera figura~\ref{Imagen1png}, es un archivo .png 

 \begin{figure}
  \begin{center}
    \includegraphics[width=0.65\textwidth, trim = 0mm 0mm 0mm 0mm, clip]{Figuras/imagen1.png}
  \end{center}
  \caption{Imagen 1 en formato .png}
  \label{Imagen1png}
  \end{figure}

\vspace{0.5cm}
\parindent=30pt Y la segunda figura (Fig.~\ref{Imagen1pdf}), es un archivo .pdf

 \begin{figure}
  \begin{center}
    \includegraphics[width=0.65\textwidth, trim = 0mm 0mm 0mm 0mm, clip]{Figuras/imagen1.pdf}
  \end{center}
  \caption{Imagen 1 en formato .pdf}
  \label{Imagen1pdf}
  \end{figure}
  



%===============================Tablas===========================================

\section{Tablas} \label{chsub:Tablas}



\parindent=0pt  El tamaño ideal que se adapta perfecto a las páginas pares e impares es de \emph{15 cm}. Se puede crear una tabla de menor tamaño, pero no es recomendable crear una tabla mayor a \emph{15 cm}. 

\vspace{0.5cm}
\parindent=30pt Primero se puede apreciar una tabla de tamaño total \emph{15 cm}(tabla~\ref{Comparación de técnicas basadas en modelos}). Esta tabla tiene tres columnas y cinco filas, donde cada columna tiene un tamaño de \emph{5 cm}.


  \begin{table}
  \begin{center}
  \resizebox{15cm}{!} {

  \begin{tabular}{|p{5cm}|p{5cm}|p{5cm}|}

  \hline

  Concepto & Modelación Analítica & Modelación por Simulación \\

  \hline

  Tiempo de aprendizaje & 4 & 1\\



  \hline

  Costo de construcción & 1 & 5\\


  \hline

  Costo de ejecución & 1 & 20\\

  \hline

  Error & 20 & 1\\

  \hline

  \end{tabular}

  }

  \end{center}

  \caption{Comparación de técnicas basadas en modelos}

  \label{Comparación de técnicas basadas en modelos}

  \end{table} 



\vspace{0.5cm}
\parindent=30pt A continuación se presentan dos tablas, la primera de un tamaño total de \emph{10 cm} y la segunda de \emph{15 cm}. Estas dos tablas tienen dos columnas y cuatro filas cada uno, donde cada columna de la primera tabla tiene un tamaño de \emph{5 cm} y cada columna de la segunda tabla tiene un tamaño de \emph{7.5 cm}.

\begin{table}
\begin{center}
\resizebox{10cm}{!} {

\begin{tabular}{|p{5cm}|p{5cm}|}

\hline

Ventajas & Desventajas \\

\hline

El conjunto de aplicaciones Apache-PHP-MySQL es uno de los más utilizados en Internet en servicios web y de buscadores de aplicaciones. & Carece de soporte para transacciones, rollback's y subconsultas. \\



\hline

Tiene gran velocidad a la hora de realizar operaciones, lo que le hace uno de los gestores que ofrecen mayor rendimiento. & No es viable para su uso con grandes bases de datos, a las que se acceda continuamente, ya que no implementa una buena escalabilidad. \\


\hline

Las utilidades de administración de este gestor son envidiables para muchos de los gestores comerciales existentes, debido a su gran facilidad de configuración e instalación. &  El hecho de que no maneje la integridad referencial, hace de este gestor una solución pobre para muchos campos de aplicación.\\

\hline

\end{tabular}

}

\end{center}

\caption{Ventajas y desventajas de MySQL (tabla 10cm)}

\label{Ventajas y desventajas de MySQL}

\end{table} 



\begin{table}
\begin{center}
\resizebox{15cm}{!} {

\begin{tabular}{|p{7.5cm}|p{7.5cm}|}

\hline

Ventajas & Desventajas \\

\hline

El conjunto de aplicaciones Apache-PHP-MySQL es uno de los más utilizados en Internet en servicios web y de buscadores de aplicaciones. & Carece de soporte para transacciones, rollback's y subconsultas. \\



\hline

Tiene gran velocidad a la hora de realizar operaciones, lo que le hace uno de los gestores que ofrecen mayor rendimiento. & No es viable para su uso con grandes bases de datos, a las que se acceda continuamente, ya que no implementa una buena escalabilidad. \\


\hline

Las utilidades de administración de este gestor son envidiables para muchos de los gestores comerciales existentes, debido a su gran facilidad de configuración e instalación. &  El hecho de que no maneje la integridad referencial, hace de este gestor una solución pobre para muchos campos de aplicación.\\

\hline

\end{tabular}

}

\end{center}

\caption{Ventajas y desventajas de MySQL (tabla 15cm)}

\label{Ventajas y desventajas de MySQL}

\end{table} 






%===============================Citas y referencias===========================================

\section{Citas y referencias} \label{chsubsub:citas y referencias}

\parindent=0pt Referenciando la sección~\ref{chsub:figuras} de este mismo capítulo y referenciando la sección~\ref{chsub:Título de la sección 1 del subcapítulo 1} del capítulo anterior.

\vspace{0.5cm}
\parindent=30pt Se cita un libro~\cite{flanagan1998javascript}, se cita un artículo~\cite{stonebraker1991postgres} y se cita una página web~\cite{SCDPreguntas}.


 

%==================================================ANEXOS========================================%
\chapter{PRIMER ANEXO}
  \label{ch:capituloAA}

  Contenido primer anexo


  \chapter{SEGUNDO ANEXO}
  \label{ch:capituloAB}


  Contenido del segundo anexo.





\chapter[TÍTULO DEL PRIMER CAPÍTULO]{TÍTULO DEL PRIMER CAPÍTULO}\label{ch:capitulo1}

  \parindent=0pt Lorem ipsum ad his scripta blandit partiendo, eum fastidii accumsan euripidis in, eum liber hendrerit an. 

  \vspace{0.5cm}
  \parindent=30pt Quo mundi lobortis reformidans eu, legimus senserit definiebas an eos. Eu sit tincidunt incorrupte definitionem, vis mutat affert percipit cu, eirmod consectetuer signiferumque eu per. In usu latine equidem dolores.







\end{document}
