% !TeX spellcheck=es
% @Author @jaimeguzman
% Base latex template @AdinRamirez
% Repo: git@giteit.udp.cl:udp/udp-latex.git

\documentclass{udparticle}



\setlogo{EITFI}

\title{ Predicción de comportamiento de usuarios en la web basado en registro de accesos a servidor }
\author{  
  Jaime Guzmán\\{\small\ttfamily mail@jguzman.cl}\protect\\[5pt]%
  {\small Adin Ramirez\thanks{Profesor guía} y Francisco Claude\thanks{Profesor comisión}}%
  }

\begin{document}

\maketitle

\section{Antecedentes y Motivación}

\subsection{Contexto}


La Web crece constantemente y por ende su infraestructura como también la concurrencia de los mismos sistemas, paralelamente se suman un costo exponencial de recursos que no son optimizados para poder dar un experiencia de usuario con calidad de servicio.
Lo cual podemos entender que llegará un punto en que no tener servidores de gran rendimiento será lo óptimo para dar un calidad de servicio web, el ancho de banda de Internet no crecerá a la misma proporción.


Adicionalmente las tecnologías para la creación de  web dinámicas han evolucionado a favor del cliente, se tiene MEAN stacks que disminuyen considerablemente la carga de un servidor, por lo cual hoy en día un buen servicio web  es proveer una balanceada carga dentro del cliente y el servidor.

Por lo mismo es de gran interés predecir lo siguientes moviemientos que tendrá un usuario en un determinada web, entendiendo que la forma en que navega una persona es su comportamiento web, que se puede reflejar mediante Web Access Log.

El registro de los mismo, de manera procesada ó pre-procesada ayudaría a ingenieros de desarrollo web y diseñadores, como a los mismo usuarios finales a tener una experiencia de usuario mejor.

Hoy en día las web no pueden ser simplemente dinámicas estas deben poseer una adaptabilidad a la demanda del usuario ó proveer información que permita adaptarse a los eventos, por lo cual es sumamente de interés profundizar este tópico.








 

\subsection{Trabajos relacionados}






 




\subsection{Motivación}
 

\section{Descripción de la solución}
 

\section{Metodología de trabajo}
  \subsection{Etapa 1}
  \subsection{Etapa 2} 
  \subsection{Etapa 3}
  \subsection{Etapa 4}
 


\section{Cronograma de actividades, hitos y entregables}
  \begin{tabular}{ll}
  \hline\noalign{\smallskip}
  Fecha & Actividad \\
  \hline\noalign{\smallskip}
  21/03/2014 & Presentación anteproyecto (firmado por profesor guía y comisión).\\

  02/04/2014 & Entrega resultados anteproyectos.\\

  04/04/2014 & Entrega anteproyectos corregidos.\\

  09/04/2014 &  Entrega resultados correcciones.\\

  25/04/2014 & Marco de trabajo.\\

  23/05/2014 & Primer prototipo de propuesta.\\

  13/06/2014 & Resultados parciales.\\

  27/06/2014 & Entrega Memoria Titulo 1 firmada por profesor guía.\\

  11/07/2014 & Fecha límite para que la comisión entregue correcciones.\\

  18/07/2014 & Fecha límite para que se realicen correcciones.\\

  20/08/2014 & Avance.\\

  10/09/2014 & Avance.\\

  08/10/2014 & Entrega Descriptor.\\


  \hline

  \end{tabular}


\section{Resultados esperados}
  
% ------------------------- %
% \bibliographystyle{IEEEtran}
\bibliographystyle{plain}
\bibliography{anteproyecto}
\nocite{*}

\end{document}
