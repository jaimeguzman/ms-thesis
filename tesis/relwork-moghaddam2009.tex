    Realizan una comparación de \texttt{LZ78}  y el algoritmo \texttt{LZW}, el cual es una derivación del anterior. La mayoría de las aplicaciones actuales que predicen el siguiente acceso a un página web posee un  componente offline que hace la tarea de preparar data y luego disponer una sección en línea que permite personalizar cierto contenido para un usuario en particular basado en las actividades de navegación.
	
	En la mayoría de las técnicas de \emph{Web Usage Mining}, las secuencias se utilizan, ya sea para producir las reglas de asociación o para producir estructuras de datos de tipo árbol o cadenas de Markov para representar patrones de navegación. Los  Modelos de Markov, se basan en una teoría bien establecida y son fáciles de entender.  
	

	La propuesta es no crear un modelo predictivo por usuarios.
	Moghaddam y Kabir proponen modelar la navegación de usuarios mediante un \emph{trie} creado por un algoritmo de la familia \texttt{LZ} y usando muchas sesiones de usuarios, para tener un modelo predictivo de navegación.