\uncm
\section{Compresión para predicciones secuenciales}

%Introduccion de la sección
%%%%%%%  
%%%%%%%  INTRO COMPRESION
%%%%%%%  

 
La Compresión de Datos, en el contexto de las ciencias de la computación, es la ciencia  de representar información en una forma lo más compacta posible. Algunos ejemplos de áreas de aplicación que son relevantes y han sido motivación para la compresión de datos son:

\begin{itemize}
	\menorEspacioItemize

	\item Sistemas de comunicación, ejemplo fax, \emph{voice mail} y telefonía.
	\item Estructuras de memoria, discos y cintas de computadores.
	\item Mobile computing.
	\item Sistemas distribuidos.
	\item Redes informáticas e Internet.
	\item Evolución multimedia, imágenes, procesamiento de señales.
	\item Archivos de imagen y videoconferencia.
	\item TV digital y satelital.

\end{itemize}


Considerando que los usuarios de \inet constantemente  crean nuevos contenidos, imágenes y vídeos  entre otros tipos de datos, es de gran utilidad tener representaciones de estos datos mas compactas. Todos los recursos mencionados anteriormente al tener un menor tamaño o representación, se favorecen a su entorno y la red, en escenarios cuando no existen una buena calidad de conexión o infraestructura para mover de un lugar a otro. Esto implica un disminución del tiempo de transferencia en la \emph{web} en el que se mueven miles de \emph{gigabytes}. Dichos archivos crecen exponencialmente no solo en número, sino también en peso individual y he aquí uno de los mayores aportes que poseen los algoritmos de compresión con relación a transferencia de datos. Precisamente la compresión de datos en un escenario de transmisión en \inet, optimiza la transferencia de archivos desde servidores, como también la carga de archivos en el lado del cliente. A diferencia de la velocidad de conexión e infraestructura de redes, estas no crecen proporcionalmente si se comparasen, esto genera una cantidad de problemas para los usuarios e industria. Respecto a lo anterior podemos decir que el tiempo que demorar un usuario en pedir un archivo hasta recibirlo, se llama \emph{latencia}. Por lo cual se comprende que al transportar archivos de menor tamaño, la \emph{latencia} de una petición a un servicio, será menor.


En el capitulo~\ref{ch:preliminar} se ha planteado que ciertos algoritmos de compresión de datos, se usan para ayudar a tareas de \machinelearning. Los algoritmos de compresión de datos que se presentan e introducirá son algoritmos mayormente basados en métodos de diccionarios, existen otros como el algoritmo de \emph{Huffman} el cual usa métodos estadísticos que no abordaremos. Para lograr entender esta premisa usando los algoritmos señalados, se explicarán conceptos de compresión que darán los fundamentos para comprobar la factibilidad de nuestra propuesta. 



\uncm

% Preliminar 

% Llamaremos alfabeto a cualquier conjunto finito no vac ́ıo. Usualmente lo denotaremos como Σ. Los elementos de Σ se llamar ́an s ́ımbolos o caracteres.
% \cite{Navarro2014}

\subsection{Modelamiento matemático}

El modelamiento matemático es un proceso en el cual se configura un ambiente que permite a ciertas variables de interes ser observadas o explorar un cierto comportamiento  de un sistema. En si este proceso es una formalización y extensión de la descripción de un problema, todas las reglas y relaciones que presenta estas observaciones se pueden formular mediantes formulas matemáticas.

El modelamiento es una etapa muy importante en el diseño de algoritmos. Un modelo puede decir inmediatamente el acercamiento de n algoritmo, es decir, un  buen modelo puede llevar a soluciones eficientes de los algoritmos. Ciertas condiciones previas se deben asumir y el ambiente debe ser descrito  en términos matemáticos. En el área de Compresión de datos, los modelos son usados para describir ciertas fuentes de datos.



En los problemas que aborda la Compresión de Datos , el modelamiento puede ser visto como un proceso en el cual se identifica caracteres \emph{redundantes} de un fuente de datos y la busqueda eficiente para ser descritos. Algunos problemas, como en todo, no son de una resolución facil e inclusivo realizar el modelamiento resulta imposible. Es por eso que el modelamiento matemático resulta ser muy importante en el área de la ciencia de computacion, es por eso que la el área de Compresión de datos da una gran importancia a este modelamiento. Los modelos comunmente usados en Compresión de datos son los siguiente:

\begin{itemize}
	\menorEspacioItemize
	\item \textbf{Modelo físico:} Conocimiento de la físcia de los datos, como el origen de los datos de una cierte fuente, la cual puede ser un procesos generativo o mediante observación empírica.
	\item \textbf{Modelos de probabilidad:} Uso de teoría de probabilidades para describir la fuente de datos.
	\item \textbf{Modelo de Markov:} Uso de la teoría de cadena de Markov para modelar fuentes de datos.
	\item \textbf{Modelo compuesto:} Es la descripción de la fuente de datos como una combinación de varios tipos y el uso de un conmutador para activar un tipo a la vez.

\end{itemize}

El resultado final del modelamiento matemático en el contexto de la compresión, es un modelo factible en el cual la redundancia de datos y ciertas restricciones de salida se intercectan en una relación definidad, tanto para la entrada del algoritmo como su salida. En otras palabras los problemas de compresión  buscan un algoritmo eficiente para remover la mayor cantidad de redundacia (ver Sección~\ref{ch2:concept-redundacia}) desde una fuente de datos.

\uncm
 




\subsection{Teoría de la información elemental}


La Teoría de la Información fue propuesta por \emph{Claude E. Shannon} en los laboratorios \emph{Bell} en 1948 y se basa en el estudio de la información basada en teoría de probabilidades. Su objetivo es una la realización de forma matemática que pueda medir la cantidad de información.

La expectativa de los resultados de un evento se puede medir por la probabilidad del evento. La alta expectativa corresponde a una alta probabilidad de que el evento ocurra. Un evento que ocurra pocas veces significa que es un evento de baja probabilidad. Un evento que nunca sucede tiene una probabilidad cero. Por lo tanto la cantidad de información en un mensaje también se puede medir cuantitativamente de acuerdo con la incertidumbre o sorpresa de que un evento ocurra.





\subsubsection{Entropía}\label{ch2:concept-entropia}

La Entropía sirve para medir los eventos o símbolos con ocurrencia individual o a la vez. Sin embargo, a menudo estamos más interesados en la información de una fuente donde las ocurrencias de todos los símbolos posibles tienen que ser considerados. En otras palabras, la medida de la información de una fuente tiene que ser considerada en todo el alfabeto.





\subsubsection{Redundancia}\label{ch2:concept-redundacia}

La Redundancia de datos, se puede describir como la superposción de datos ó datos que poseen una base común, características comunes o equivalentes en alguna estructura. En el área de Compresión de Datos se vuelve una de las primeras tarea indentificar la redundacia en ciertos origenes de datos.


 





% \subsubsection{Compresión basada en Diccionarios}

\uncm
\subsection{Clasificación de algoritmos de compresión} 

Los algoritmos de compresión tienen como objetivo convertir datos de una fuente a una representación más comprimida, inversamente desde una representación comprimida a una fuente de datos incial. El primer proceso lo realiza un \emph{compresor de datos} y el segundo es un \emph{decodificador o decompresor de datos}. Estos algoritmos se clasifican en algoritmos sin pérdida de datos (\losslessdatacompression)y con pérdida(\emph{Lossy Data Compression}). Los algoritmos de compresión sin pérdida o  \losslessdatacompression(\LDC) son los que abordaremos. Estos usan técnicas de reconstrucción de información de un archivo comprimido sin perder información. Estos algoritmo se encuentran en el día a día, por ejemplo en programas de escritorio o en el \emph{Unix} o \emph{Linux}, comandos de compresión, y de extración, \emph{gzip}, \emph{gunzip} todos ocupan compresión basada en diccionarios~\cite{MengyiPu2006}. La motivación de profundizar en el área de compresión sin pérdida(\LDC), es la cantidad de posibilidades que entregan para mejorar ciertas operaciónes de manera más eficiente, por ejemplo la  transferencia de archivos en la web, que ayudan a transmistir información más rapidamente y abordar sus propiedades de transformación, como también la predicción.
% For example, in UNIX or Linux, commands compress, uncompress, gzip and gunzip have all used the dictionary compression methods at some stage.~\cite{MengyiPu2006}


Las propiedades de estos algoritmos no solo permiten juntar un colección de archivos y lograr un tasa de compresión óptima para ser transmitida por \inet, también pueden ayudar a realizar análisis predictivo en grandes volúmenes de información, por ejemplo; análisis de texto~\cite{}, clasificación de proteínas~\cite{}, moderación de contenidos en web y predicciones del comportamiento de usuarios que navegan en un sitio de \inet. Este último punto es nuestro mayor interés, predicciones de \webasccesslog de una \emph{web}. Para introducir el objetivo de usar \LDC para las predicciones se debe presentar formalmente los algoritmos de compresión que permitan realizar esta aproximación.
% JUSTO EN EL PARRAFO ANTERIOR FALTA UN NEXO A TEO DE INFORMACION

Existen varios \losslessdatacompression interesantes, algunos son basados en en diccionario. En esta sección veremos que se pueden usar para realizar modelos predictivos. Nos enfocaremos en los que estan basados en un modelo variable de Markov, los cuales nos ayudarán en nuestra etapa experimental a dar un modelamiento secuencial de la navegación de un usuario, como también crear funciones de predicciones en base a la probabilidad de ver cada nodo dado a su frecuencia. Aprender de predicciones sobre secuencia de datos discretas sigue siendo un ítem fundamental y un desafío en patrones de reconocimiento y \machinelearning, y con \LDC se pretende realizar.














\uncm
\subsection{Algoritmos de compresión}

%MEJORAR ESTA INTRO o BORRARLA
Existen varios algoritmos de compresión, nuestro enfoque es usar los algoritmos de compresión que tengan un espacio vectorial de características conjunto con \emph{Machine Learning} y además tengan propiedades para ser candidatos a un predictor. 


\subsubsection{Prediction by Partial Match (PPM)}
	


El algoritmo de Predicción por coincidencia parcial~\PPM~(por sus siglas en inglés \emph{Prediction by Partial Match})~\cite{Shkarin2002 }, es considerado uno de los mejores algoritmos del tipo \losslessdatacompression. El algoritmo requiere un tope superior $D$ en el orden máximo  de un modelo variable de Markov (\emph{\texttt{VMM}}) para construirse. \PPM maneja el problema de frecuencia cero~\cite{Begleiter2004} usando dos mecanismo llamados:
	\begin{itemize}
		\menorEspacioItemize
		\item Escape
		\item Exclusion
	\end{itemize}
	
Para un método que considera diferentes órdenes de modelos, retomamos una vez más a la compresión de datos y la familia de predictores.  Esto ha sido usado con gran efecto, para un marco predictivo basado en \lzSieteOcho. 

Algoritmos  \PPM consideran modelos de Markov de diferente orden,  con el fin de construir una distribución de probabilidad mediante la ponderación de modelos de diferente orden. En nuestro escenario predictivo, \emph{Active LeZi} construye un orden-k del modelo de Markov. Ahora empleamos la estrategia \PPM de exclusión para reunir información de los modelos de orden 1 a $k$ para asignar el siguiente símbolo de su valor de probabilidad. Este método se ilustra considerando la secuencia de ejemplo utilizado en los apartados anteriores: \texttt{aaababbbbbaabccddcbaaaa}.

% FATLA una CITA a ACTILEZI
La ventana mantenida por \emph{Active Lezi}~\cite{Gopalratnam2007} representa el conjunto de contextos utilizado para calcular la probabilidad  del siguiente símbolo. En nuestro ejemplo, el último se utiliza la frase \texttt{aaa}. Dentro de esta frase, los contextos que pueden ser utilizados son todos sufijos dentro de la frase, excepto la ventana en sí (es decir, \texttt{aa} , \texttt{a}, y el contexto nulo).

	
% @TODO: Trabajar mas en este tema. y mencionar mas adelante porque usar LZ y no este, rendimiento
 
\subsubsection{Probabilistic Suffix Tree (PST)}
 	 Los árboles de sufijos implementados como un algoritmo de predicción intentan construir el único y mejor \emph{VMM} con límite superior $D$, acorde a la secuencia de entrenamiento de entrada. Esto asume que un límite superior a la orden de Markov de un fuente certera es conocida como \emph{learner}.
 
 %@TODO: explayar mas aca

\subsubsection{Cadenas de Markov Dinámicas}
  	 %%%%%
 %%%%% ~\cite{PenaSordo2015}
 %%%%%

 Los \emph{Dynamic Markov Compression} o \DMC son modelos de información con máquinas de estados finitos. Las asociaciones están hechas entre todos los símbolos posibles en el alfabeto origen y la distribución de probabilidad sobre todos los símbolos en el alfabeto~\cite{PenaSordo2015}. 
 Esta distribución de probabilidad es usada para predecir el siguiente dígito binario. Los \DMC comienzan en un estado ya previamente definido, cambiando de estado cuando nuevos bits son leídos desde la entrada. La frecuencia de transmisión ya sea un $0$ o $1$ son sumados cuando un nuevo símbolo entra. %La estructura puede también ser actualizada usando \emph{state cloning method}.


% ~\cite{PenaSordo2015}
%  The compression algorithm dynamic Markov compression (DMC) [11] models information with a finite state machine. Associations are built between every possible symbol in the source alphabet and the probability distribution over those symbols. This probability distribution is used to predict the next binary digit. The DMC method starts in a already defined state, changing the state when new bits are read from the entry. The frequency of the transitions to either a 0 or a 1 are summed when a new symbol arrives. The structure can be also be updated using a state cloning method.
 

\subsubsection{Familia de algoritmos Lempel\&Ziv}\label{ch2:sec-lzfamily}
	%
% ~\cite{Moghaddam2009}
% ~\cite{Begleiter2004}
% ~\cite{Rissanen1983}
% ~\cite{Langdon1983}
% 
% shortcuts: \lzSieteOcho \lzSieteSiete
%

Los algoritmos originales de la familia fueron desarrollados por \emph{Jacob Ziv} y \emph{Abraham Lempel}, en sus trabajos publicados en 1977~\cite{ZivLempel1977} y 1978~\cite{ZivLempel1978}. Ambas publicaciones proporcionan dos enfoques para la creación de diccionarios adaptables, y cada enfoque ha dado lugar a una serie de variaciones. Se puede decir que existen dos grupos que generaron algoritmos derivados, es decir, \lzSieteSiete~\cite{ZivLempel1977} y \lzSieteOcho~\cite{ZivLempel1978}. Otra popular variante  de \lzSieteOcho es \texttt{LZW}, que fue una publicación de \emph{Terry Welch} en 1984. Existen muchas variantes de los algoritmos anteriormente señalados,  pero enfocaremos nuestro estudio y discusiones en el prominente \lzSieteOcho. 

%hacer este monito: https://dl.dropboxusercontent.com/spa/srrxi5k8b49mdt9/p-f_cjtr.png


Los enfoques de algoritmos de compresión basados en diccionario es no utilizar ningún modelo estadístico, pero se basan en la identificación de patrones repetidos. Por lo tanto, el efecto de compresión y descompresión no depende de la calidad del modelo estadístico, ni está limitado por la entropía de los datos. Por lo tanto, se puede lograr una mejor relación de compresión que los métodos basados en un modelo estadístico\footnote{\footnoteHuffman}, pero manteniendo el objetivo de eliminar la redundancia almacenada en series repetitivas de frases, dentro de una secuencia. Los algoritmos basados en diccionarios son normalmente más rápido que los basados en entropía. 

Una de los grandes problemas de \lzSieteSiete es que no puede reconocer ocurrencia de ciertos patrones del pasado, debido a que es posible que ya se haya desplazado hacia fuera de la memoria histórica (\emph{buffer}). En estas situaciones los patrones son ignorados o si son consultados el algoritmo no retorna nada. Para poder extender la memoria de los patrones encontrado, \lempelziv desarrollaron \lzSieteOcho con un diccionario que permite mantener los patrones permanentemente durante todo el proceso. El diccionario que se utilizado para almacenar los patrones antes visto  y los índices que son utilizados para comprimir los patrones repetidos. El contenido del diccionario varía en función de la secuencia de entrada de secuencias para ser comprimidas.


Esta familia de algoritmos tienen muchas aplicaciones y se han utilizado en una serie de programas de software comerciales. Por ejemplo, en UNIX o Linux, con comandos para comprimir y descomprimir, \emph{gzip}\footnote{\footnoteGZIP}. Sin embargo, hay otros aspectos a considerar, por ejemplo ¿Cómo se pueden identificar ciertos patrones de secuencias correctamente? ¿Cuáles son las buenas técnicas que se pueden utilizar para comprobar si un símbolo está en el diccionario?. Diferentes opciones de reconocimiento de secuencias, pueden dar diferentes resultados de la compresión realizada. Por otro lado, ¿Qué hacemos si el diccionario crece en un gran tamaño y también que hacer si lo hace rápidamente?. Tenemos que contemplar que ciertas estructuras de datos puede ser afectas a disminuir la eficacia de sus operaciones directamente. Por ejemplo, cuanto más grande es el diccionario, más tiempo se necesita para comprobar si una frase o secuencia está dentro del diccionario, esta idea la desarrollaremos en la próxima sección que detalla estas soluciones con \lzSieteOcho. 





 
% NO ME ACUERDO DE ESTA REFERENCIA 
% 
% El método de codificación también afecta a la eficacia de la compresión. 
% Para un código de longitud variable, las longitudes de las palabras de código tienen que satisfacer la desigualdad Kraft con el fin de ser únicamente descifrable. Esto, de alguna manera, ofrece orientación teórica de hasta qué punto un algoritmo de compresión puede ir; en otro, limita el rendimiento de estos algoritmos de compresión.

 
 % El diccionario aparece ya sea en un forma explícita o implícita.



% Dado que nuestro interés se centra en los enfoques de los algoritmos de diccionario, vamos a examinar tres algoritmos más populares de forma fundamental, a saber, LZ77, LZ78 y el LZW.

% todos han usado los métodos de compresión de diccionario en algún momento.



\uncm
\subsection{Algoritmo Lempel-Ziv 78}
	% Convenciones para esta sección:
% shortcuts: \lzSieteOcho \lzSieteSiete
% Se habla de secuencias
% ¿Como se construye el arbol?



% Seucodigo de algormito de compresión LZ78


\begin{algorithm}[t]
	\caption{Seudocódigo para Algoritmo \texttt{LZ78}.}
	\label{alg:pseudocode-lz78}
	\begin{algorithmic}[1]
		\State {initialize} \textbf{dictionary  }{:= null}
		\State {initialize} \textbf{phrase w  }{:= null}
		
		\While{wait for next symbol v }
			\If { ((w.v) in dictionary):}
				\State  \textbf{ w  }{:= w.v}	
			\Else 
				\State 	{add (w.v) to dictionary}
				\State {w := null}
				\State {increment frequency for every possible prefix of phrase}
				
			\EndIf	
		\EndWhile
	\end{algorithmic}
\end{algorithm}




% https://www.safaribooksonline.com/library/view/analytic-pattern-matching/9781316287392/Chapter_9.html


Lempel-Ziv 78~\cite{ZivLempel1978} es uno de los algoritmos \losslessdatacompression más populares~\cite{Begleiter2004}, en la sección anterior hemos vistos varios ejemplos en donde se usa este algoritmo, (ver sección~\ref{ch2:sec-lzfamily}).

Este divide una secuencia larga en frases o bloques de tamaño variable ,de tal manera que una nueva frase es la subcadena más corta que no se haya visto anteriormente como una frase.

Divide una secuencia en frases o bloques de tamaño variable de tal manera que una nueva frase es la subcadena más corta no vista en el pasado como una frase.

Cada frase es codificada por el índice de su prefijo anexado(appended) por un símbolo; por lo tanto el código LZ'78 contiene los pares (puntero, símbolos).

LZ78 compression algorithms use a trie

% Número de frases y redundanancia.
% La raíz contiene la frase vacía

Todas las otras frases del algoritmo de análisis de Lempel-Ziv se almacenan en los nodos internos .



El rendimiento de LZ78 depende del número de frases, pero el objetivo final es reducir al mínimo el código de compresión, y hablamos de este lado.


% Let n be a nonnegative integer. We denote by Mn the number of phrases M(w) and by Cn the code length C(w) when the original text w is of fixed length n. We shall assume throughout that the text is generated by a memoryless source over a finite alphabet A such that the entropy rate is h = − ∑a∈A pa log pa > 0, where pa is the probability of symbol a ∈ A. We respectively define the compression rate






\textbf{Observaciones sobre LZ78}


- LZ78 ha hecho algunas mejoras sobre LZ77. 
Por ejemplo, en teoría, el diccionario puede mantener a los 
patrones para siempre después de haber sido visto una vez. 
En la práctica, sin embargo, el tamaño del diccionario 
no puede crecer indefinidamente. 
Algunos patrones pueden necesitar ser reinstalado si el diccionario está lleno.

- Las palabras de código de salida contienen un componente 
menos que los de LZ77. Esto mejora la eficiencia de los datos.

- LZ78 tiene muchas variantes y LZW es la varianza más popular para LZ78, 
donde el diccionario comienza con todas las 256 símbolos iniciales 
y el par de salida se simplifica a la salida de un único elemento.



% LIMITACION DE CRECIMIENTO:
% While the LZ78 algorithm has the ability to capture patterns and hold them indefinitely,
%  it also has a rather serious drawback. 
%  As seen from the example, the dictionary keeps growing without bound. 
%  In a practical situation, we would have to stop the growth of the dictionary at some stage and then either prune it back or treat the encoding as a fixed dictionary scheme. 
%  We will discuss some possible approaches when we study applications of dictionary coding.


Sin embargo se presentan ciertos problemas de análisis con \lzSieteOcho. Cualquier aplicación práctica de \lzSieteOcho sufre  los siguientes inconvenientes: 

\begin{itemize}
	\menorEspacioItemize	
	\item En cualquier análisis \emph{Lempel} \& \emph{Ziv}, una cadena de entrada, toda la información cruzada de los bordes de las frases se pierden. En muchos casos, serian patrones y éstos afectarían al siguiente símbolo en la secuencia.
	
	\item La tasa de convergencia de \lzSieteOcho a la predictibilidad óptima como se definió anteriormente es lento. Los resultados experimentales que realizaremos  describirán que \lzSieteOcho se acerca de forma asintótica a un óptimo~(ver Ryabko~\etal \cite{Ryabko2002}) . Esta estrecha relación entre predicciones en secuencias discretas y algoritmo sin perdida, donde, en principio cualquier \texttt{LCA} es candidato a ser usado como un predictor y viceversa (ver Feder \etal~\cite{Feder1992}). 
	
\end{itemize}



The memory may be an explicit dictionary that can be extended infinitely, or an implicit limited dictionary as sliding windows. Each seen string is stored into a dictionary with an index. The indices of all the seen strings are used as codewords. The compression and decompression algorithm maintains individually its own dictionary but the two dictionaries are identical. Many variations are based on three representative families, namely LZ77, LZ78 and LZW. Implementation issues include the choice of the size of the buffers, the dictionary and indices.




% ESTO PUEDE SERVIR PAR EL CAPITULO DE EXPERIMENTACION
% Referencias
% ~\cite{Claude2014}
% ~\cite{Moghaddam2009}
% ~\cite{Begleiter2004}
% 

Sabemos que \lzSieteOcho es un algoritmo que no tendrá pérdida, si buscamos representar la navegación de un usuario no deseamos perder datos, ya que estos mismo nos servirán para realizar una predicción, y esta representación debe ser la base para ese objetivo, un modelo.

Para comenzar con el modelo, debemos tomar las ideas principales que tiene \lzSieteOcho. Primero en base a una sequencia de símbolos crear un diccionario, y después crear un \trie que represente la estructura y modelo. Uno de los factores de los sistemas que están en linea o \webs, es siempre reducir su complejidad y con la simple idea anterior es posible. Segundo debemos hacer que cada sesión que realice el usuario es representado por una secuencia, gracias el trabajo realizado por \emph{Claude}~\etal en~\cite{Claude2014}, podemos solamente preocuparnos de continuar la constucción, ya que los datos ya han sido minados de los \webasccesslog.



Supogamos que $S_{1}$ es la secuencia de un cierto usuario, con el  algoritmo \lzSieteOcho creará un árbol de navegación que representará la navegación de este usuario. Si ingresa otro usuario con una secuencia representada por $S_{2}$, con un comportamiento similar, se podría crear predicciones en base a como ya se ha comportado el usuario anterio. En caso contrario actualizará los nodos de cada símbolos~(representan una página web o una categoría de una web). Si el usuario se mantiene navegando el algoritmo irá actualizando su frecuencia en cada nodo del \trie. Podemos señalar además por el trabajo de~\cite{Begleiter2004}, en general, el número total de secuencias que inserta en el árbol es menos que el algoritmo \PPM.  

% TODO explicar esto con el modelo de navegacion que usa ~\cite{Moghaddam2009}


% Explicamos este algoritmo con un ejemplo. Supongamos que el usuario solicita las páginas \texttt{ABABCBC} secuencialmente. Si utilizamos el algoritmo \lzSieteOcho, entonces generaríamos un \emph{trie} con nodos  \texttt{A, B, AB, BC, C} que deberán insertar. 


% Cuando se inserta una secuencia en el árbol los contadores de las aristas que representan el paso desde la raíz hasta la última petición de secuencia se incrementa en cada inserción. 
% Supongamos ahora que el usuario B pide a la secuencia de páginas \texttt{ABCABCD}, estos generaría cambios en el \emph{trie} y de cumplir las condiciones se incrementarían los contadores. 


% TODO:
% Concluir y conectar con el que sigue

\textbf{CONECTOR pendiente a la siguiente subseccion}













