

% texto mencionado en mi intro, pero puede ser utilizado en related work

También se pueden utilizar para mejorar el rendimiento del caché en navegadores web como lo ha mencionado \emph{Josep Domènech} \etal~\cite{Domenech2006},  detección de intereses de usuarios y también recomendaciones de páginas o bienes relacionados para los sitios {web} de comercio electrónico, mejorar  resultados de los motores de búsqueda y personalización de contenido web con preferencias personalizadas por el usuario a medida que va navegando. Todos los ejemplos anteriormente mencionados usando un modelo predictivo con disponibilidad inmediata.



%  textos de introduccion por evaluar


Muchas técnicas de recuperación de datos y algunos sistemas de personalización usan algoritmos de predicción, podemos señalar que existen técnicas con componente online que entregan resultados inmediatos o \online, es bastante usual crear un entrenamiento estático y con un set de datos de volumen fijo, a este lo llamamos contrariamente una componente \offline de un modelo de predicción . La mayoría de las aplicaciones actuales que predicen la siguiente página web de un usuario tienen una componente  \offline que hacen la tarea de preparar los datos y su componente \online  proporciona contenido personalizado a los usuarios en función de sus actividades de navegación actuales.

Un modelo genérico no podría ser válido dada a la naturaleza básica de los conjuntos de datos que podemos recuperar de una \web, pero si un modelo predictivo liviano que vaya aprendiendo acorde a los patrones que se van recopilando en tiempo real, esto da inicio a tener un modelo predictivo con componentes tanto \online como \offline, el problema anterior es uno de nuestros intereses, consumir el resultado de un modelo de predicción confiable en que podamos ver los siguiente accesos de los usuarios y tomar acciones inmediatas. Muchos algoritmos de \machinelearning usan análisis de datos predictivos para generar modelos predictivos en variadas áreas y este último es nuestro segundo interés, crear un modelo predictivo que se pueda integrar con los componentes de un servicio de predicción discreta que normalmente es abarcado por el campo de \machinelearning.  