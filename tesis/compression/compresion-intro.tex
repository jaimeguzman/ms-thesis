%%%%%%%  
%%%%%%%  INTRO COMPRESION
%%%%%%%  

 
La Compresión de Datos, en el contexto de las ciencias de la computación, es la ciencia  de representar información en una forma lo más compacta posible. Algunos ejemplos de áreas de aplicación que son relevantes y han sido motivación para la compresión de datos son:

\begin{itemize}
	\menorEspacioItemize

	\item Sistemas de comunicación, ejemplo fax, \emph{voice mail} y telefonía.
	\item Estructuras de memoria, discos y cintas de computadores.
	\item Mobile computing.
	\item Sistemas distribuidos.
	\item Redes informáticas e Internet.
	\item Evolución multimedia, imágenes, procesamiento de señales.
	\item Archivos de imagen y videoconferencia.
	\item TV digital y satelital.

\end{itemize}


Considerando que los usuarios de \inet constantemente  crean nuevos contenidos, imágenes y vídeos  entre otros tipos de datos, es de gran utilidad tener representaciones de estos datos mas compactas. Todos los recursos mencionados anteriormente al tener un menor tamaño o representación, se favorecen a su entorno y la red, en escenarios cuando no existen una buena calidad de conexión o infraestructura para mover de un lugar a otro. Esto implica un disminución del tiempo de transferencia en la \emph{web} en el que se mueven miles de \emph{gigabytes}. Dichos archivos crecen exponencialmente no solo en número, sino también en peso individual y he aquí uno de los mayores aportes que poseen los algoritmos de compresión con relación a transferencia de datos. Precisamente la compresión de datos en un escenario de transmisión en \inet, optimiza la transferencia de archivos desde servidores, como también la carga de archivos en el lado del cliente. A diferencia de la velocidad de conexión e infraestructura de redes, estas no crecen proporcionalmente si se comparasen, esto genera una cantidad de problemas para los usuarios e industria. Respecto a lo anterior podemos decir que el tiempo que demorar un usuario en pedir un archivo hasta recibirlo, se llama \emph{latencia}. Por lo cual se comprende que al transportar archivos de menor tamaño, la \emph{latencia} de una petición a un servicio, será menor.


En el capitulo~\ref{ch:preliminar} se ha planteado que ciertos algoritmos de compresión de datos, se usan para ayudar a tareas de \machinelearning. Los algoritmos de compresión de datos que se presentan e introducirá son algoritmos mayormente basados en métodos de diccionarios, existen otros como el algoritmo de \emph{Huffman} el cual usa métodos estadísticos que no abordaremos. Para lograr entender esta premisa usando los algoritmos señalados, se explicarán conceptos de compresión que darán los fundamentos para comprobar la factibilidad de nuestra propuesta. 


