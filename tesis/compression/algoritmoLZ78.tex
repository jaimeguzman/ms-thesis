% Referencias
% 
% ~\cite{Moghaddam2009}
% ~\cite{Begleiter2004}
% ~\cite{Rissanen1983}
% ~\cite{Langdon1983}
% 
% shortcuts: \lzSieteOcho \lzSieteSiete
%
%
% TODO: colocar la jodida representaciones de math como lo explica ~\cite{Begleiter2004}
\lempelziv~78~\cite{ZivLempel1978} es uno de los algoritmos \LDC más populares~\cite{Begleiter2004}, en la sección anterior hemos vistos varios escenarios ejemplos en donde puede ser usado, (ver sección~\ref{ch2:sec-lzfamily}). Recordemos que este tipo de algoritmos es basado en diccionarios, que simultáneamente es representado por una estructura \trie. En la figura~\ref{alg:pseudocode-lz78} se tiene el algoritmo.
% Seucodigo de algormito de compresión LZ78


\begin{algorithm}[t]
	\caption{Seudocódigo para Algoritmo \texttt{LZ78}.}
	\label{alg:pseudocode-lz78}
	\begin{algorithmic}[1]
		\State {initialize} \textbf{dictionary  }{:= null}
		\State {initialize} \textbf{phrase w  }{:= null}
		
		\While{wait for next symbol v }
			\If { ((w.v) in dictionary):}
				\State  \textbf{ w  }{:= w.v}	
			\Else 
				\State 	{add (w.v) to dictionary}
				\State {w := null}
				\State {increment frequency for every possible prefix of phrase}
				
			\EndIf	
		\EndWhile
	\end{algorithmic}
\end{algorithm}
% https://www.safaribooksonline.com/library/view/analytic-pattern-matching/9781316287392/Chapter_9.html


Para poder explicar este algoritmo supongamos que tenemos una secuencia de un tamaño largo, y extraemos de esta una subsecuencia de menor tamaño, a este lo llamaremos \emph{frase}. Entonces se divide una secuencia larga en frases de tamaño variable, de tal manera que una nueva frase es la subsecuencia más corta que no se ha visto anteriormente. Cada frase es codificada por el índice de su prefijo anexado (\emph{append}) por un símbolo; por lo tanto el compresor contiene los pares (puntero, símbolos). Además, podemos mencionar que el  rendimiento de \lzSieteOcho depende del número de frases, pero el objetivo final es reducir, comprimir al mínimo la secuencia. El algoritmo comienza con una raíz que contiene un frase vacía y luego todas las otras frases que se le dan al algoritmo se almacenan en los nodos internos. 
% Número de frases y redundanancia.



\subsubsection{Observaciones sobre LZ78}

\lzSieteOcho con respecto a su predecesor~\cite{ZivLempel1977} ha tenido ciertas mejoras. Por ejemplo, el diccionario teóricamente puede mantener secuencias o frases que ha encontrado, para siempre. Por otro lado,  el tamaño del diccionario no tiene un limite y tampoco  debiese crecer infinitamente. Si es necesario algunos patrones deben ser re-ubicados, cuando el diccionario esta lleno. Otra opción, es directamente parar el crecimiento, evitando el ingreso de más frases ó permitir que el diccionario comience desde cero, cuando alcanza un cierto número de entradas. 

Al igual que todos los algoritmos \losslessdatacompression tienen una estrecha relación con las predicciones en secuencias discretas. En 1983 los investigadores \emph{Langdon}~\cite{Langdon1983} y \emph{Rissanen}~\cite{Rissanen1983}, comenzaron a discutir la componente de predicción de este algoritmo. Posteriormente, en el trabajo de \emph{Feder}~\etal señaló que cualquier \LDC es candidato a ser usado como un algoritmo predictor y viceversa (ver \emph{Feder}~\etal~\cite{Feder1992}). Por lo tanto, \lzSieteOcho es un algoritmo de predicción universal con respecto a la gran clase de fuentes de Markov estacionarias y ergódicos de orden finito~\cite{Begleiter2004}.

% Por ejemplo, cualquier análisis de \lempelziv con una determinada secuencia de entrada, puede perder toda la información de sus extremos de las frases se pierden. En muchos casos, serian patrones y éstos afectarían al siguiente símbolo en la secuencia.
	
% \item La tasa de convergencia de \lzSieteOcho a la predictibilidad óptima como se definió anteriormente es lento. Los resultados experimentales que realizaremos  describirán que \lzSieteOcho se acerca de forma asintótica a un óptimo~(ver Ryabko~\etal \cite{Ryabko2002}) . 




% CONECTOR CON LA SIGUINENTE SECCIóN
\uncm
Concluimos esta sección, en que ciertos algoritmos de compresión se pueden usar para predecir. Además, dentro de la clasificación de algoritmos sin pérdidas, existen algunos que son basados en diccionarios, como \lzSieteOcho. Estos se concentran en la memoria de las secuencias o frases que han visto, la cual memoria puede ser acotada explícitamente al diccionario o se puede extender infinitamente. Con lo visto en esta sección ya tenemos un acercamiento para predecir, pero en el próximo capítulo se estudiará y profundizará más sobre el tema de predicciones. En particular las que corresponden a \webasccesslog, que ocurren secuencialmente y se pueden ser representados por símbolos en una determinada secuencia.




 % La eficiencia de la compresión para una fuente de datos dada, depende del tamaño del alfabeto y lo cerca que su distribución de probabilidad se encuentra de las estadísticas es a las de la fuente. 