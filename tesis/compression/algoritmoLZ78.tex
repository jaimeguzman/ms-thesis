% Convenciones para esta sección:
% shortcuts: \lzSieteOcho \lzSieteSiete
% Se habla de secuencias
% ¿Como se construye el arbol?



% Seucodigo de algormito de compresión LZ78


\begin{algorithm}[t]
	\caption{Seudocódigo para Algoritmo \texttt{LZ78}.}
	\label{alg:pseudocode-lz78}
	\begin{algorithmic}[1]
		\State {initialize} \textbf{dictionary  }{:= null}
		\State {initialize} \textbf{phrase w  }{:= null}
		
		\While{wait for next symbol v }
			\If { ((w.v) in dictionary):}
				\State  \textbf{ w  }{:= w.v}	
			\Else 
				\State 	{add (w.v) to dictionary}
				\State {w := null}
				\State {increment frequency for every possible prefix of phrase}
				
			\EndIf	
		\EndWhile
	\end{algorithmic}
\end{algorithm}




% https://www.safaribooksonline.com/library/view/analytic-pattern-matching/9781316287392/Chapter_9.html


Lempel-Ziv 78~\cite{ZivLempel1978} es uno de los algoritmos \losslessdatacompression más populares~\cite{Begleiter2004}, en la sección anterior hemos vistos varios ejemplos en donde se usa este algoritmo, (ver sección~\ref{ch2:sec-lzfamily}).

Este divide una secuencia larga en frases o bloques de tamaño variable ,de tal manera que una nueva frase es la subcadena más corta que no se haya visto anteriormente como una frase.

Divide una secuencia en frases o bloques de tamaño variable de tal manera que una nueva frase es la subcadena más corta no vista en el pasado como una frase.

Cada frase es codificada por el índice de su prefijo anexado(appended) por un símbolo; por lo tanto el código LZ'78 contiene los pares (puntero, símbolos).

LZ78 compression algorithms use a trie

% Número de frases y redundanancia.
% La raíz contiene la frase vacía

Todas las otras frases del algoritmo de análisis de Lempel-Ziv se almacenan en los nodos internos .



El rendimiento de LZ78 depende del número de frases, pero el objetivo final es reducir al mínimo el código de compresión, y hablamos de este lado.


% Let n be a nonnegative integer. We denote by Mn the number of phrases M(w) and by Cn the code length C(w) when the original text w is of fixed length n. We shall assume throughout that the text is generated by a memoryless source over a finite alphabet A such that the entropy rate is h = − ∑a∈A pa log pa > 0, where pa is the probability of symbol a ∈ A. We respectively define the compression rate






\textbf{Observaciones sobre LZ78}


- LZ78 ha hecho algunas mejoras sobre LZ77. 
Por ejemplo, en teoría, el diccionario puede mantener a los 
patrones para siempre después de haber sido visto una vez. 
En la práctica, sin embargo, el tamaño del diccionario 
no puede crecer indefinidamente. 
Algunos patrones pueden necesitar ser reinstalado si el diccionario está lleno.

- Las palabras de código de salida contienen un componente 
menos que los de LZ77. Esto mejora la eficiencia de los datos.

- LZ78 tiene muchas variantes y LZW es la varianza más popular para LZ78, 
donde el diccionario comienza con todas las 256 símbolos iniciales 
y el par de salida se simplifica a la salida de un único elemento.



% LIMITACION DE CRECIMIENTO:
% While the LZ78 algorithm has the ability to capture patterns and hold them indefinitely,
%  it also has a rather serious drawback. 
%  As seen from the example, the dictionary keeps growing without bound. 
%  In a practical situation, we would have to stop the growth of the dictionary at some stage and then either prune it back or treat the encoding as a fixed dictionary scheme. 
%  We will discuss some possible approaches when we study applications of dictionary coding.


Sin embargo se presentan ciertos problemas de análisis con \lzSieteOcho. Cualquier aplicación práctica de \lzSieteOcho sufre  los siguientes inconvenientes: 

\begin{itemize}
	\menorEspacioItemize	
	\item En cualquier análisis \emph{Lempel} \& \emph{Ziv}, una cadena de entrada, toda la información cruzada de los bordes de las frases se pierden. En muchos casos, serian patrones y éstos afectarían al siguiente símbolo en la secuencia.
	
	\item La tasa de convergencia de \lzSieteOcho a la predictibilidad óptima como se definió anteriormente es lento. Los resultados experimentales que realizaremos  describirán que \lzSieteOcho se acerca de forma asintótica a un óptimo~(ver Ryabko~\etal \cite{Ryabko2002}) . Esta estrecha relación entre predicciones en secuencias discretas y algoritmo sin perdida, donde, en principio cualquier \texttt{LCA} es candidato a ser usado como un predictor y viceversa (ver Feder \etal~\cite{Feder1992}). 
	
\end{itemize}



The memory may be an explicit dictionary that can be extended infinitely, or an implicit limited dictionary as sliding windows. Each seen string is stored into a dictionary with an index. The indices of all the seen strings are used as codewords. The compression and decompression algorithm maintains individually its own dictionary but the two dictionaries are identical. Many variations are based on three representative families, namely LZ77, LZ78 and LZW. Implementation issues include the choice of the size of the buffers, the dictionary and indices.

