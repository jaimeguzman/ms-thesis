%
% ~\cite{Moghaddam2009}
% ~\cite{Begleiter2004}
% ~\cite{Rissanen1983}
% ~\cite{Langdon1983}
% 
% shortcuts: \lzSieteOcho \lzSieteSiete
%

Los algoritmos originales de la familia fueron desarrollados por \emph{Jacob Ziv} y \emph{Abraham Lempel}, en sus trabajos publicados en 1977~\cite{ZivLempel1977} y 1978~\cite{ZivLempel1978}. Ambas publicaciones proporcionan dos enfoques para la creación de diccionarios adaptables, y cada enfoque ha dado lugar a una serie de variaciones. Se puede decir que existen dos grupos que generaron algoritmos derivados, es decir, \lzSieteSiete~\cite{ZivLempel1977} y \lzSieteOcho~\cite{ZivLempel1978}. Otra popular variante  de \lzSieteOcho es \texttt{LZW}, que fue una publicación de \emph{Terry Welch} en 1984. Existen muchas variantes de los algoritmos anteriormente señalados,  pero enfocaremos nuestro estudio y discusiones en el prominente \lzSieteOcho. 

%hacer este monito: https://dl.dropboxusercontent.com/spa/srrxi5k8b49mdt9/p-f_cjtr.png


Los enfoques de algoritmos de compresión basados en diccionario es no utilizar ningún modelo estadístico, pero se basan en la identificación de patrones repetidos. Por lo tanto, el efecto de compresión y descompresión no depende de la calidad del modelo estadístico, ni está limitado por la entropía de los datos. Por lo tanto, se puede lograr una mejor relación de compresión que los métodos basados en un modelo estadístico\footnote{\footnoteHuffman}, pero manteniendo el objetivo de eliminar la redundancia almacenada en series repetitivas de frases, dentro de una secuencia. Los algoritmos basados en diccionarios son normalmente más rápido que los basados en entropía. 

Una de los grandes problemas de \lzSieteSiete es que no puede reconocer ocurrencia de ciertos patrones del pasado, debido a que es posible que ya se haya desplazado hacia fuera de la memoria histórica (\emph{buffer}). En estas situaciones los patrones son ignorados o si son consultados el algoritmo no retorna nada. Para poder extender la memoria de los patrones encontrado, \lempelziv desarrollaron \lzSieteOcho con un diccionario que permite mantener los patrones permanentemente durante todo el proceso. El diccionario que se utilizado para almacenar los patrones antes visto  y los índices que son utilizados para comprimir los patrones repetidos. El contenido del diccionario varía en función de la secuencia de entrada de secuencias para ser comprimidas.


Esta familia de algoritmos tienen muchas aplicaciones y se han utilizado en una serie de programas de software comerciales. Por ejemplo, en UNIX o Linux, con comandos para comprimir y descomprimir, \emph{gzip}\footnote{gzip es un software libre GNU que reemplaza al programa \texttt{compress} de UNIX, ver~\url{http://www.gzip.org/}}. Sin embargo, hay otros aspectos a considerar, por ejemplo ¿Cómo se pueden identificar ciertos patrones de secuencias correctamente? ¿Cuáles son las buenas técnicas que se pueden utilizar para comprobar si un símbolo está en el diccionario?. Diferentes opciones de reconocimiento de secuencias, pueden dar diferentes resultados de la compresión realizada. Por otro lado, ¿Qué hacemos si el diccionario crece en un gran tamaño y también que hacer si lo hace rápidamente?. Tenemos que contemplar que ciertas estructuras de datos puede ser afectas a disminuir la eficacia de sus operaciones directamente. Por ejemplo, cuanto más grande es el diccionario, más tiempo se necesita para comprobar si una frase o secuencia está dentro del diccionario, esta idea la desarrollaremos en la próxima sección, que detalla estas soluciones con \lzSieteOcho. 





 
% NO ME ACUERDO DE ESTA REFERENCIA 
% 
% El método de codificación también afecta a la eficacia de la compresión. 
% Para un código de longitud variable, las longitudes de las palabras de código tienen que satisfacer la desigualdad Kraft con el fin de ser únicamente descifrable. Esto, de alguna manera, ofrece orientación teórica de hasta qué punto un algoritmo de compresión puede ir; en otro, limita el rendimiento de estos algoritmos de compresión.

 
 % El diccionario aparece ya sea en un forma explícita o implícita.



% Dado que nuestro interés se centra en los enfoques de los algoritmos de diccionario, vamos a examinar tres algoritmos más populares de forma fundamental, a saber, LZ77, LZ78 y el LZW.

% todos han usado los métodos de compresión de diccionario en algún momento.