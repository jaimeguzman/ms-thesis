\chapter[Conceptos Básicos]{Conceptos básicos y trabajos relacionados} \label{ch:Conceptos-Basicos}





En este capítulo se explicarán brevemente los conceptos principales que se usarán en los siguientes capítulos:


\section{Conceptos}

\subsection{Access Log}\label{concept-accesslog}

Los \emph{Access Log} son registros que se almacenan en un servidor, los cuales dependiendo del servidor, sistema operativo y las configuraciones del ambiente pueden tener mayor o menor información. Cuando los usuarios acceden a diversos sitios web, estos registros dejan una gran cantidad de información, un ejemplo se puede ver en la Figura~\ref{fig:accesslog-apache-teleton}. Si se extraen de forma correcta se puede obtener patrones de navegación de usuarios. 


\subsection{Árboles Trie} \label{concept-trie}

Los \emph{trie} son un tipo de árbol de prefijos, estructuras de datos en forma de árbol que almacenan datos en sus nodos y es muy fácil la recuperación de información de estos. Se caracterizan por ser un conjunto de llaves que se representan en el árbol y sus nodos internos representan la información, en nuestro caso una secuencia de acceso o un símbolo (secuencia de largo 1). 

Un árbol es una estructura general de nodos recursivos. Hay muchos tipos de árboles. Los populares son árbol binario y el árbol balanceado. Un \emph{trie} es conocido por muchos nombres incluyendo árbol prefijo, árbol de búsqueda digital, árbol de la recuperación (de ahí el nombre de \textquotedbl \emph{trie}\textquotedbl, por la palabra del inglés recuperación, \emph{retrieval}).

Cada tipo de árbol tiene distinta finalidad, estructura y comportamiento. Por ejemplo, un árbol binario almacena una colección de elementos comparables (por ejemplo, números). Por lo tanto, se puede utilizar para almacenar un conjunto de números, o al índice de otros datos que pueden ser representados por los números (por ejemplo, objetos que pueden ser \emph{hash}). Su estructura está ordenada por lo que se puede buscar rápidamente para encontrar nodo. Otras estructuras de árbol, como un árbol balanceado son similares en principio al \emph{trie} que se implementará en la etapa experimental.

Un \emph{trie} representa una forma  estructurada de nodos, la cual puede almacenar secuencias. % (ver el artículo de wiki). 
Es muy diferente cuando almacena secuencias de valores en lugar de valores individuales. Cada nivel representa un incremento en la altura del árbol.
%'¿cuál es el valor del punto I de la lista de entrada'. Esto es diferente a un árbol binario que compara el valor buscado único a cada nodo.


Durante este trabajo mostraremos que nuestro modelo de predicción usa un \emph{trie}, para representar un diccionario, en los cuales podemos señalar las siguientes operaciones disponibles:

	\begin{itemize}	
		\item \textbf{findByPrefix(x):}  Retornar una una lista de todos los nodos hasta llegar a un nodo que posea un \emph{String} de parámetro equivalente a $x$.
		
		\item \textbf{contains(x):} Retornar una lista de nodos intermedios hasta que el contenido del nodo final sea hoja o el intermedio corresponda a valor de la secuencia \emph{String} $x$.
		
		\item \textbf{remove(x):} Retorna \emph{true} o \emph{false} cuando es posible remover el nodo con el contenido equivalente a $x$.
		
		\item \textbf{pathTo(x):} Retorna una lista de nodos hasta llegar a un nodo que posea el contenido equivalente del valor $x$.
	\end{itemize}







\subsection{Alfabeto} \label{concept-alphabet}

Un alfabeto es un conjunto ordenado de elementos. Una de sus características es que la cantidad de elementos puede ser de valores  discreto, representaremos por $\Sigma$ y una de las características básica de este alfabeto es consiste en símbolos $\sigma$ que pertenecen a $\Sigma$, durante el resto de nuestra investigación usaremos solo un $\Sigma = \sigma_{i},\sigma_{i+1}\ =\ 17\  \mbox{símbolos}$.



Dado un volumen de datos experimental, nuestro alfabeto es mostrado simbólicamente como la representación de varios nodos contenido en un \emph{Trie} que modela la navegación de usuarios para un sitio web.

%Donde $\sigma_{1}$, puede ser definido como la página inicial de una cierta secuencia  o sesión de usuario. Este alfabeto es finito y acotado por una minería de datos ya realizada por \cite{Claude2014} en \emph{Efficient Indexing and Representation of Web Access Logs}. % Referencia al trabajo de claud

%Let A be a discrete alphabet consisting of M  2 symbols; x = x ; : : : ; x ; x 2 A
% be rst n symbols of message; jj be the length of sequence  or the cardinality of
% set . The probability of symbol x = a 2 A for a source with memory depends on
% current context s = x ; : : : ; x 2 A ; d  D. The set of all possible contexts can
% be presented as nodes in M -ary tree with depth D. Context (sequence) s describes a
% path from tree root to the current node also denoted as s.
% Usually the true conditional probabilities are unknown and the coding conditional
% pr ob abilitiesq(ajs) depend on characteristics of one or sev eralsubsequences x (s),
% x (s) is the subsequence of all symbols x such that x ; : : : ; x
% = s . These
% characteristics are: frequency f (ajs) = f (ajx (s)) of symbol a in x (s), alphabet
% A(s) = A(s; x (s)) = fa : f (ajs) > 0g, its cardinality m(s) = jA(s)j etc. Even at
% small D, the n umber of model states is big, subsequences x (s) are short ina verage,
% and their statistics is insuÆcient for eective compression.
% PPM algorithm [2] is based on an (implicit) assumption: the longer is the common initial part of contexts, the more (in average)similarity there is between their
% conditional probability distributions. High PPM eÆciency means this assumption is
% fair forthe majority of real sources.



\subsection{Secuencias discretas}\label{concept-discret-seq}

Definimos una secuencia de accesos discreta y finita, dado los accesos que tiene un usuario frente a una web, lo anterior es acotado por el concepto de sesión, el cual es desde que se inicia la navegación por parte de un usuario, es decir secuencia de tamaño $Seq\ \leq 1$ y de tamaño no superior a un alfabeto $\Sigma$.



% Compresor tonto o Machine Learning lento para predecir
\subsection{Lossless Data Compression (LDC)} \label{concept-LDC}

La compresión sin pérdida o \emph{LDC}, es el arte de poder comprimir \emph{bits} y poder hacer el proceso inverso, es decir poder codificar y decodificar. En capítulos posteriores se detallará más sobre el tema compresión y como esta área ayuda a crear un modelo de predicción.





\subsection{Motor de Predicción}\label{concept-enginepredict}

Es la parte fundamental de un sistema dirigida a adivinar el futuro acceso de un usuario. La salida del motor de predicción es la siguiente página, que se compone de un símbolo representando la dirección \emph{url} o una sección en particular de una cierta web. 
% que son propensos a ser solicitado por el usuario en las solicitudes posteriores.

 


\subsection{Resilient Distributed Datasets }\label{concept-RDD}

	Los \texttt{RDD}, permiten que en un servidor de \emph{Machine Learning } pueda mantener un modelo o motor de aprendizaje persistente sin importar en el flujo que se encuentre.

	Esta estructura es fundamental dentro de la librería que se introducirá mas adelante, \emph{Apache Spark}. Esta estructura es una colección distribuida de objetos inmutables, cada 
	set de datos en un \texttt{RDD} es divido en particiones lógicas, las cuales puedes ser computadas en distintos cluster. Los \texttt{RDD} pueden contener cualquier tipo de objeto de los siguientes lenguajes: \emph{Python}, \emph{Scala} y \emph{Java}, incluyendo clases definidas por el usuario. 

	Formalmente los \texttt{RDD} son solo de lectura, una colección de objetos divididos. Estos pueden ser creados a través de  operaciones deterministas en una cierta tabla o un almacenamiento externo u otra \texttt{RDD}.
	Otra de las características de los \texttt{RDD}, es que son colecciones de elementos tolerantes a fallas que pueden ser operadas en si mismas o en paralelo.
	\emph{Apache Spark} hace el uso del concepto de \texttt{RDD} para lograr rapidez y eficiencia en las operaciones de \emph{MapReduce}, de ser requeridas. Destacamos la escabilidad de esta librería para un gran nivel de cómputo, pero en este trabajo no se explicará el uso de \emph{Spark}, pero si se utilizarán algunos conceptos como \texttt{RDD} y otros.




\subsection{Data Source y Dataset }

	Los \emph{Data Source} son distintas fuentes de datos que podemos ir sacando set de datos (\emph{Dataset}). Ambos conceptos están enfocados a proveer información tanto para el servidor de \emph{Machine Learning}, como para el procesamiento y análisis.
	En este trabajo el dataset con que realizaremos nuestro estudio son los registros de accesos de la web española \emph{MSNBC}\cite{Claude2014}, los cuales representan aproximadamente 1.000.000 de registros de \emph{webaccess log}.

	Nuestro set de  datos esta basado en los registros (webaccess log), ya previamente depurados con una representación numérica desde 0 hasta 16, el cual como antes ya se ha señalado estará relacionado al concepto de alfabeto. También crearemos \emph{dataset} sintéticos para poder realizar depuraciones de nuestra implementación y casos de bordes.
	
	 



 



% \subsection{DataFrame}


	% A DataFrame is a distributed collection of data, which is organized into named columns. Conceptually, it is equivalent to relational tables with good optimization techniques.
	% Here is a set of few characteristic features of DataFrame −
	% Ability to process the data in the size of Kilobytes to Petabytes on a single node cluster to large cluster.
	% Supports different data formats (Avro, csv, elastic search, and Cassandra) and storage systems (HDFS, HIVE tables, mysql, etc).
	% State of art optimization and code generation through the Spark SQL Catalyst optimizer (tree transformation framework).
	% Can be easily integrated with all Big Data tools and frameworks via Spark-Core.
	% Provides API for Python, Java, Scala, and R Programming.



% easy text
% https://es.wikipedia.org/wiki/Trie
%Definición interpretada de esot

%\subsection{Cadenas de Markov}





\subsection{Transferencia de Estado Representacional}~\label{concept-rest}

Es un estilo de desarrollo de software para sistemas distribuidos mediante Internet. El uso de REST, de sus siglas en ingles, ha sido adoptado mucho más adoptado que  \emph{Simple Object Access Protocol}, ya que los servicios \emph{REST} aprovechan mayor cantidad el ancho de banda, lo que hace que sea una mejor opción para su uso a través de Internet, los mensajes tanto originados por el servidor o cliente puede ser enviados en formato \emph{xml} o \emph{json}. 



%@TODO:
% Este tema debería detallarse en las siguientes secciones
\section{Trabajos relacionados}

En la literatura, el tema de la predicción en la web se ha presentado como un tema recurrente provocando bastante atención durante los último años y ha sido tratado por varios autores de áreas de \emph{Machine Learning} y \emph{Lossless Data Compression}. Tenemos los siguientes trabajos de interés:


\begin{enumerate}


  %%%%%%%%%%%%%%%%%%%%%%%%%%%%%%%%%%%
  \item \textbf{Dynamic and memory efficient web page prediction model using LZ78 and LZW algorithms }

	Moghaddam y Kabir~\cite{Moghaddam2009}  realizan una comparación de LZ78  y el algoritmo LZW, el cual es una derivación del anterior. La mayoría de las aplicaciones actuales que predicen el siguiente acceso a un página web posee una  componente offline que hace la tarea de preparar data y luego disponer una sección en linea que permite personalizar cierto contenido para un usuario en particular basado en las actividades de navegación
	
	En la mayoría de las técnicas de \emph{Web Usage Mining}, las secuencias se utilizan ya sea para producir las reglas de asociación o para producir estructuras de datos de tipo árbol o cadenas de Markov para representar patrones de navegación. Los  Modelos de Markov se basan en una teoría bien establecida y son fáciles de entender.  
	

	La propuesta es no crear un modelo predictivo por usuarios.
	Moghaddam y Kabir proponen modelar la navegación de usuarios mediante un Trie creado por un algoritmo de la familia LZ y usando muchas sesiones de usuarios, tener un modelo predic






  %%%%%%%%%%%%%%%%%%%%%%%%%%%%%%%%%%%
  \item \textbf{Prediction Algorithms for User Actions} ( Hartmann \& Schreiber,2007  \etal~\cite{hartmann2007}) , 

	{
		% Proactive User Interface
	Se requiere la predicción de la siguiente acción del usuario basada en la historia de interacción que ha tenido con una interfaz. 
	En su trabajo dan una revisión a los algoritmos de predicción Discreta (SPA) y desarrollan dos propuestas de algoritmos basadas en Modelos de Markov que convinan distintos orden de Markov. Y desarrollan una librería en PERL para su propuesta y evaluación.
	
	% NOTAS MIAS
	
	%Acorde a lo que Hartman propone nuestro trabajo esta totalmente alineado, debido a que optimizar o predecir el siguiente acción/acceso puede ayudar enormemente a el problema de PUI Proactive User Interfaces que el plantea
	
	% Ellos comparan comparan el mejor rendimiento del algoritmo acorde a los requerimientos de memoria y tiempo de procesamiento
	
	%Hartman en su paper dice que las Interfaces de Usuario Intelignente(Intelligent User Interfaces) tb dan uso y son un campo de aplicación para las predicciones secuenciales 
	
	% Algo significativo que menciona es que muchos de los algoritmos toman k- elementos de una secuencia para hacer una predicción, en mi caso yo tomo la secunecia completa en base a un entrenamiento
	
	% Plantea que hay dos maneras de hacer predicciones secuenciales On-Demand or Live, en mi caso es Live y OnDemand la implementación que ofrezco cumple ambos campos.
	% Online Learning Algorithm -> Muy similar a mi modelo hibirido ML y LDC
	% Mi Propuesta podría ayudar mucho en PUI para caso como el desarrollo movil,esto es obiviamente por las limitaciones de pantalla que presenta 
	
	% Modelos/Algortimos propuestos   ONISI y IPAM Jacobs Blockeel, ActiveLeZi
	}
  %%%%%%%%%%%%%%%%%%%%%%%%%%%%%%%%%%%
  \item \textbf{ActiveLezi} ( Gopalratnam \& Cook,2007 \etal\cite{Gopalratnam2007}) 
  {  
  Proponen un Algoritmo \emph{On-Demand} que considera varios modelos de Markov.  
  El funcionamiento es basado en almacenar la frecuencia del patrón de \emph{input} en un \emph{Trie} acorde al algoritmo de compresión de \emph{LZ78} para superar algunos de los problemas que surgen con \emph{LZ78}, se usa una ventana de largo variable de los simbolos previamente usados en la construcción del \emph{Trie}. El tamaño de la ventana crece con el numero de las diferentes subsecuencias que se van viendo en la entrada de cada secuencia nueva que ingresa.  Sea $suff_{l}$  el sufijo de largo $l+1$ el sufijo de longitud l 1 de las inmediatamente historial de interacción a, que es un hacha .... la probabilidad se define de la siguiente forma recursiva.:
 }



  %%%%%%%%%%%%%%%%%%%%%%%%%%%%%%%%%%%
  \item Dongshan y Junyi~\cite{Dongshan2002} 
  {
	  Destacan que un modelo de Markov puede ayudar a predecir el comportamiento de un usuario, pero con ciertas limitaciones .  Para solucionarlo presentan un nuevo modelo de Markov basado en una representación de \emph{Tree Order Model}, el cual es un híbrido entre un modelo de markov tradicional y una representación de árbol, bautizada como HTMM (por sus siglas en inglés, \emph{Hybrid-Order Tree Markov Model}).
	  Su modelo fue presentado en 2002, y da una importancia a conocer la predicción de los \emph{web access}, dada la importancia de creación de redes, la minería de datos, e-commerce, y otras áreas.
	}
  %%%%%%%%%%%%%%%%%%%%%%%%%%%%%%%%%%%
  \item Domenech \etal~\cite{Domenech2006}
  
	{  
  Muestran un estudio de los rendimientos de técnicas de recuperación de datos.
  Las mismas se pueden utilizar para dar una entrada ideal a algoritmos de aprendizaje o algoritmos de predicción. 
  Los conceptos más importantes son las nuevas variables de caracterización, temporalidad, espacio y geografía, que se le suman a la predicción. 
  Además de comenzar un trabajo más elaborado de como tomar una predicción, se introducen conceptos como predicciones genéricas o específicas, variables de uso de recursos a nivel de red ó nivel procesamiento.
  Finalmente, se presenta un modelo predictivo que puede ayudar a disminuir la latencia entre la petición del cliente y la respuesta de la web, dando así un mejor rendimiento y \emph{QoS}.
  
    % @TODO detallar más explicarlo mas simple, darle mas enfoque al usuario segúnn del punto de vista que de los docuentos 
    % como los autores antteriores.
 }


  %%%%%%%%%%%%%%%%%%%%%%%%%%%%%%%%%%%
  \item Chen \etal~\cite{Chen2011} 
  
	{  
	Dan una nueva perspectiva enfocada a entregar una clara recomendación a los usuarios basada en la misma propuesta de este proyecto, los access log.
	El primer análisis realizado por los autores cubre las reglas asociativas que requiere un sistema de recomendación, pero en las pruebas propiamente tales encuentran que el análisis de los patrones detectadados dan una representación clara de como optimizar la web, y finalmente mediante sus pruebas logran una recomendación de calidad.}
  %%%%%%%%%%%%%%%%%%%%%%%%%%%%%%%%%%%
  \item Rajimol y Raju~\cite{Rajimol2012} 
	{  
	Minaron los patrones de los accesos web, donde el enfoque es usar los registros de acceso para crear subsecuencias y realizar comparaciones.
	La literatura presenta un interés para poder anticipar el patrón de comportamiento de la web.
  % @TODO reflexionar mas sobre este paper


	}
  %%%%%%%%%%%%%%%%%%%%%%%%%%%%%%%%%%%
  \item Kewen~\cite{kewen2012} 
  
	{  
	Realizó un análisis más profundo del \emph{web usage minning}.
	Parte de la importancia de este trabajo, es que después de minar los registros de accesos, logran reducir la ``\emph{bad data}''.
	  %@TODO: Preguntar si este paper se escapa mucho del tema prinicipal, pero parece interesante  
	}

  %%%%%%%%%%%%%%%%%%%%%%%%%%%%%%%%%%%
  \item Poornalatha y Raghavendra~\cite{Poornalatha2012}
  
	{ 
		
	  Establecen que se pueden utilizar máquinas de aprendizaje para predecir basándose en distintas entre clusters. Estos autores, al igual que Domenech \etal~\cite{Domenech2006} y Dongshan y Junyi~\cite{Dongshan2002}, comparan el objetivo de optimizar los recursos tanto en redes (disminución de latencia) y experiencia de usuario.
	  }
  
  %%%%%%%%%%%%%%%%%%%%%%%%%%%%%%%%%%%
  \item Claude \etal~\cite{Claude2014} 
  
	{  
	presentan una estructura de representación eficiente que permite dar una representación de \emph{web access log} y ofrecen las operaciones básicas de WUM.
  }
\end{enumerate}

%%%%%%%Moghaddam_Kabir
%Various models have been proposed for modeling the user navigation behavior and predicting the next requests of users. According to [12], association rules, sequential pattern discovery, clustering, and classification are most popular methods for web usage mining. Collaborative filtering is another method for modeling users' behaviors. Association rules [6] were proposed to capture the co- occurrences of buying different items in a supermarket shopping. Association rules indicate groups that are related together. Methods that use association rules can be found in [6, 7]. Collaborative filtering techniques are often based on matching the current user's profile against similar data obtained by the system over time from other users. It tries to make useful recommendations users based on discovered cluster of similar categories. But for web sites where the number of web pages is quite large it can be quite inefficient.



