

\section{Definición del problema}


El problema de realizar modelos predictivo que minimicen su conjunto de entrenamientos, ha surgido hace años y diversos investigadores han trabajado con distintos enfoques. Rissman \cite{Rissanen1983} y Langdom \cite{Langdon1983} en los laboratorios \emph{Bell} al realizar pruebas y experimentar con un robot que tiraba una moneda compitiendo con humano, el robot realizaba todos los cálculos {markovianos} y las probabilidades condicionales para que cierto evento ocurra, a diferencia del sujeto que sólo estaba esperando un resultado aleatorio.

Predecir no es trivial y requiere de gran cantidad información para poder realizar un modelo que abarque varios escenarios, pero sí podemos llegar a acercarnos y minimizar el error  estaremos más cerca a predicciones exactas. Sin embargo, dos áreas han tratado de resolver el problema;  \emph{Lossless Data Compression}(\texttt{LDC}) y \emph{Machine Learning} de manera separada. Por parte de \texttt{LDC} los mayores problemas son que los algoritmos predictivos  funcionan totalmente desconectados de la fuente de datos, lo que implica que la validez del modelo solo es factible cuando esta realizando predicciones sin usuarios concurrentes o como se ha señalado anteriormente con un modelo con una componente \emph{offline}, lo que inhabilita rápidamente al modelo y en general no dan un resultado inmediato, en cambios en el esquema de \emph{Machine Learning} debemos crear un modelo para entrenar y luego poder generar una función predictiva a lo cual se le suma un gran cantidad de datos para lograr un buen entrenamiento que produce un modelo bastante pesado para poder funcionar como modelo predictivo \emph{online}. 

Acotaremos nuestro problema a resolver predicciones discretas secuenciales con un modelo predictivo propuesto en un set de datos discretos con data sintética y real. Teniendo un modelo híbrido de funcionamiento justamos los algoritmos y arquitectura de funcionamiento de cada área para disponerlo como un servicio inmediato, es decir crear la componente \emph{online} del modelo predictivo; dando una predictibilidad inmediata que hoy en la industria es necesaria para poder hacer útiles estos algoritmo y dar un valor a los avances. Usando un algoritmo de tipo \texttt{LDC} corriendo sobre una arquitectura de  servicios de \emph{Machine Learning} podemos alcanzar a una solución óptima o a lo menos estar sobre el promedio aleatorio de ocurrencia de la predicción.

