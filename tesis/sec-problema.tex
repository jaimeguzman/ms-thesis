

\section{Definición del Problema}


El problema de la Predicción, ha surgido hace años y diversos investigadores han trabajado con distintos enfoques. Rissman y Langdom en los laboratorios Bell al realizar pruebas con un robot y hacer un experimento con un robot que tiraba una moneda compitiendo con humano, realizaba todos los calculos markovianos y calculos de las probabilidades condicionales para que cierto evento ocurra, a diferencia del sujeto que solo estaba esperando un resultado.

Predecir no es trivial, pero si podemos llegar a cercanos y minimizar el error de equivocarnos. Sin embargo, dos áreas han tratado de resolver el problema, LDC y Machine Learning de manera separada. Por parte de LDC los mayores problemas son que los predictores funcionan totalmente desconectados y no dan una de disponibilidad inmediata de los resultados, en cambios en el área de Machine Learning debemos crear un modelo para entrenar y luego poder generar una función predictiva.

Planteamos el problema de poder resolver tener un modelos híbrido juntando los patrones de cada área y disponerlo como un servicio inmediato dando una predictibilidad inmediata que hoy en la industria es necesaria para poder hacer útiles estos algoritmo y dar un valor a los avances.

 