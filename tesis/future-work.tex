\newpage
\section{Trabajos Futuro}

Esta memoria forma parte del plan de continuidad en el postgrado de la Escuela de Ingeniería Informática y telecomunicaciones, por lo cual se desea profundizar este trabajo en las discusiones realizadas. Los temas deseados por abarcar:

\begin{itemize}

\item Crear un estudio comparativo con Modelos de \emph{Machine Learning} como (\emph{RNN}) Redes Neuronales, Reglas de Asociación, \emph{Deep Learning} y algoritmo de tipo \emph{Frequent Pattern Growth}.

\item Técnicas para mejorar el modelo de predicción 
\item Mejorar la implementación de \texttt{LZ78}, realizado con lenguaje funcional y objetos(\emph{Scala}) y hacer una comparación de rendimientos contra implementaciones clásicas en lenguajes de más bajo nivel (\texttt{C++}).


\item Crear técnicas como la usada por Claude \etal~\cite{Claude2014}, para crear representaciones eficientes en función de este modelo predictivo.


\item Investigar los factores teóricos y técnicos para poder mejorar la exactitud de la predicción.

\item Ambiciosamente a realizar un estudio comparativo para encontrar puntos en común de estas áreas de la ciencia de la computación, se desea crear un nuevo algoritmo basado en la familia de \emph{Lempel Ziv}, el cual pueda tener un complemento para la selección de sesiones antes ser ingresadas en el modelo de navegación predictivo. 


	
\end{itemize}	

