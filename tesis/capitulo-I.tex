% Capitulo I


\chapter[Introducción]{Introducción}\label{capitulo-I}



%%%%%%%%%%
% Uno de los interes en poder crear un un aplicación de ambas areas es la convergenia de las areas en la cual un proceso de aprendizaje ó predicción secuencial se puede usar con ML y Compress data

% @TODO: SEGUIR TRABAJANDO EN ESTA BREVE INTRODUCCION
%En este tema convergen tres áreas, por un lado existe trabajo para crear estructuras eficientes para predicciones basadas en algoritmos de compresión, como es en el caso de~\cite{Claude2014}, y, por otro lado, el uso de algoritmos de aprendizaje para realizar clustering y predecir el comportamiento basado en el mismo contenido o en la distancia del contenido que visita el usuario actual al contenido clusterizado, como es el caso de ~\cite{Poornalatha2012}, inclusive se han utilizado modelos de Markov en ~\cite{Dongshan2002}  para poder modelar el comportamiento de la web.
%La tercera área son los Sistemas de Recomendación, la cual en este proyecto no se tocará pero si se mencionará el enfoque práctico que presenta área como un foco de múltiples implementaciones. 



\chapter[Introducción]{Introducción}
\label{ch:intro}

%chacharara
{

Los nuevos avances tecnológicos  y la inclusión de la ciencia de la computación en distintas campos han permitido hacer colecciones de datos muy grandes, las cuales se deben procesar, clasificar y analizar. Encontrar información útil dentro de estos grandes volúmenes de datos significa poder mejorar la decisiones a las que se pueden tomar dada a una base de conocimiento.

La minería de datos basado en web access se ha convertido en un área importante de investigación en los últimos años. Se puede utilizar para mejorar el rendimiento de la caché web, la detección de intereses de los usuarios y así recomendar páginas o bienes relacionados para los sitios web de comercio electrónico, mejorar los resultados de los motores de búsqueda  y por ejemplo personalizar el contenido de la web con las preferencias deseadas para el usuarios . En la predicción de navegación web logrando entender el patrón de navegación del usuario y luego la predicción de las páginas siguientes es el principal problema que buscamos solucionar. Con un sistema de predicción fiable podemos ver la siguiente acción de los usuarios de Internet y tomar acciones.

Normalmente se ocupa muchos algoritmos de Machine Learning para poder hacer predicciones en variadas áreas.  Hoy en día las aplicaciones en que los usuarios se enfrentan no pueden ser estáticas y sin tener un comportamiento que ayuden a predecir la forma en que actúan los usuarios. Esta área toma mucha relevancia al encontrarnos en una auge de la información generada por usuarios, redes sociales y variadas plataformas. Podemos mencionar que esta es una de las razones para poder requerir disponer de herramientas de análisis y predicción que nos permitan saber como se comporta usuarios sobre una web, conocer la frecuencia en que se accede a un recurso en Internet, etc.

El momento en que un usuario entre a un web se establece una conexión cliente-servidor. Una gran cantidad de servicios proporcionan datos de accesos de los usuarios que acceden, sobre estos mismos se puede realizar variadas investigaciones sobre como predecir cual es la siguiente página que podrá visitar.

%%%%%%Moghaddam_Kabir
Las predicciones en los registros de web access han atraído una atención significativa en los últimos años. Muchas técnicas de recuperación de datos  y algunos sistemas de personalización usan algoritmos de predicción. La mayoría de las aplicaciones actuales que predicen la siguiente página Web de un usuario tienen una componente en línea que hace la tarea de preparación de datos y una sección en línea que proporciona contenido personalizado a los usuarios en función de sus actividades de navegación actuales. En este trabajo se presenta un modelo de predicción en línea que se puede consumir como un servicio de API el cual da una integración ha variadas plataformas y sistemas que no tenga un componente en línea y con una buena precisión de la predicción. 
Nuestro algoritmo se basa en algoritmos LZ78 que están adaptados para modelar  y representar la navegación secuencia del usuario en una web. Nuestro modelo disminuye la complejidad computacional, que es un problema grave en el desarrollo de sistemas de predicción en línea. 


  }

%%%%%%Moghaddam_Kabir
%Web access prediction has attracted significant attention in recent years. Web prefetching and some personalization systems use prediction algorithms. Most current applications that predict the next user web page have an offline component that does the data preparation task and an online section that provides personalized content to the users based on their current navigational activities. In this paper we present an online prediction model that does not have an offline component and fit in the memory with good prediction accuracy. Our algorithm is based on LZ78 and LZW algorithms that are adapted for modeling the user navigation in web. Our model decreases computational complexities which is a serious problem in developing online prediction systems. A performance evaluation is presented using real web logs. This evaluation shows that our model needs much less memory than PPM family of algorithms with good prediction accuracy.

%Web mining has become an important research area in recent years. It can be used to improve the web cache performance[1, 2], Detecting the user interests and so recommending related pages or goods for e-commerce web sites [3], improving search engines results [4] , and personalizing the web content as the users like [5]. In web page prediction understanding the user navigation pattern and then predicting the next pages is the main problem. With a reliable prediction system we can see the next action of web users.






%When user requests inserted and deleted incrementally the online models are desirable. In this paper we present efficient techniques for modeling user navigation behavior. Our model is online so changes in user request patterns will update our prediction model incrementally.
%%%%%%%%%%%%%%%%

% 

% Nuestro intere es hacer que las rpecicciones sean modeladas por un compresaor de datos basado en Lempel ziv



%%%%%%%%%%
% Uno de los interes en poder crear un un aplicación de ambas areas es la convergenia de las areas en la cual un proceso de aprendizaje ó predicción secuencial se puede usar con ML y Compress data
% Cuando se hace el pio train se hace un entrenamiento de modelo lo que resulta ser es que genera el arbol que es un trie de LZ que será el predicto
% Una función de predición implementada con MLIB y PIO sería predict () en scala




%%%%%%% CONTEXTO PRELIMINAR 
\section{Contexto preliminar} 
\label{sec:preliminar}

% Idealmente aca explicar el problema

  La Web crece constantemente y por ende su infraestructura, también la información que podemos obtener de los  usuarios y  por consecuencia mayor concurrencia de los sistemas, la cual para los usuarios finales se traduce en un incremento de latencia y una mejor o peor experiencia de usuario. Paralelamente se suma un costo exponencial de recursos tanto en tecnologías de desarrollo como servicio que no son optimizados para poder dar una experiencia de usuario con calidad de servicio. Podemos reflexionar, entonces, que tener mayores recursos no mejorará el rendimiento, ni tampoco será lo óptimo para dar una calidad de servicio web ya que el ancho de banda de Internet no crecerá en la misma proporción.
   
  Adicionalmente, las tecnologías para la creación de web dinámicas y asíncronas han evolucionado a favor de traspasar la carga cliente.
  Hoy en día ya se poseen lenguajes y {framework} que disminuyen considerablemente la carga de un servidor, por lo cual, un buen servicio web es proveer una balanceada carga dentro del cliente y el servidor, pero cuando se poseen un gran volumen de datos es fundamental tomar decisiones que los recursos y lenguajes no cubren, es ahí el interés de dar inteligencia a servicios de la web.

  Interpretaremos que la manera en que un usuario navega es su comportamiento o patrón registrado en \emph{webacces log}, y que se pueden analizar, estudiar y modelar con algoritmos que tengan enfoques predictivos. 

  Sobre estos registros se pueden hacer representaciones eficientes como las realizadas por Claude \etal~\cite{Claude2014},  minería de datos para buscar \emph{Web Usage Mining} (WUM), el porque de hacer minería de datos radica en que cada día la web genera una innumerable cantidad de información, por lo cual usar algoritmos que se puedan opera de manera comprimida o con una representación lo más liviana posible es de interés ya que además de disminuir el espacio físico o recursos utilizados, se puede usar como un algoritmo de predicción y trabajar con una mayor cantidad de datos.
  
  Teniendo el conocimiento que los \emph{webaccess logs} se pueden estudiar de manera procesada o pre-procesada, ayudaría a ingenieros de desarrollo web y diseñadores de experiencia de usuarios, como también en general a mejorar la experiencia de usuario final, también disminuyendo por ejemplo la latencia en respuestas por parte de cada petición realizada, con técnicas de \emph{pre-fetching predictivo} en el lado del cliente.
  

  Actualmente, los sitios web han evolucionado de ser contenido estático a dinámico, también moderado por administradores a contenido orgánico creado por usuarios finales. Se debe poseer una adaptabilidad a la demanda o proveer información que permita adaptarse a los eventos, por lo tanto, hacer un estudio sobre esto y poder hacer integraciones en áreas como uso de algoritmos de compresión y \emph{Machine Learning} presenta un gran desafío. Independientemente del área, el problema común  se puede resolver pero se busca encontrar las fortalezas de cada área y usarlas en conjunto. 

  Durante este trabajo se usarán técnicas de compresión de datos, se utilizará una infraestructura y patrón de implementación para modelos de \emph{Machine Learning} como servicio. Adicionalmente toda la experimentación se llevará acabo ofreciendo los algoritmos y modelos como servicio basado en una Transferencia de Estado Representacional (\emph{REST}~\ref{concept-rest}), y así implementarlo en áreas productivas las cuales pueden presentar interés dando nuevos escenarios de estudio.


  La sesión de un usuario comienza cuando se conecta a un servicio web, éstas pueden ser páginas informativas, redes sociales, web dinámicas, contenido colaborativo, etc. Estableciendo dicha conexión a una página, se crea en ese momento una sesión de navegación automáticamente. Dado esto es posible almacenar datos muy relevantes los cuales son ''webaccess log'' ó registros de accesos web, durante el texto se mantendrán las referencias en inglés. Un ejemplo de \emph{webaccess log} es lo que se observa en la figura ~\ref{fig-ejemplo-webaccesslogbruto}:

\begin{figure}[tb]\label{fig-ejemplo-webaccesslogbruto} 
	\centering
	\begin{lstlisting}[frame=single,basicstyle=\ttfamily\tiny,]
	172.31.33.116 - - [26/Nov/2015:00:12:12 +0000] "HTTP/1.1" 200 1784 "http://localhost/home" 
	"Mozilla/5.0 (Linux; Android 5.1.1; SAMSUNG SM-G920I Build/LMY47X) 
	SamsungBrowser/3.2 Chrome/38.0.2125.102 Mobile Safari/537.36"
	172.31.33.116 - - [26/Nov/2015:00:12:12 +0000] "HTTP/1.1" 200 179333 "http://localhost/news" 
	"Mozilla/5.0 (Linux; Android 5.1.1; SAMSUNG SM-G920I Build/LMY47X) 
	SamsungBrowser/3.2 Chrome/38.0.2125.102 Mobile Safari/537.36"
	172.31.33.116 - - [26/Nov/2015:00:12:12 +0000] "HTTP/1.1" 200 24660 "http://localhost/health" 
	"Mozilla/5.0 (Linux; Android 5.1.1; SAMSUNG SM-G920I Build/LMY47X) 
	SamsungBrowser/3.2 Chrome/38.0.2125.102 Mobile Safari/537.36"
	172.31.33.116 - - [26/Nov/2015:00:15:12 +0000] "HTTP/1.1" 200 24604 "http://localhost/sports" 
	"Mozilla/5.0 (Linux; Android 5.1.1; SAMSUNG SM-G920I Build/LMY47X) 
	SamsungBrowser/3.2 Chrome/38.0.2125.102 Mobile Safari/537.36"
	172.31.33.116 - - [26/Nov/2015:00:20:12 +0000] "HTTP/1.1" 200 4860 "http://localhost/home" 
	"Mozilla/5.0 (Linux; Android 5.1.1; SAMSUNG SM-G920I Build/LMY47X) 
	SamsungBrowser/3.2 Chrome/38.0.2125.102 Mobile Safari/537.36"
	172.31.33.116 - - [26/Nov/2015:00:22:19 +0000] "HTTP/1.1" 200 4841 "http://localhost/finances" 
	\end{lstlisting}
	
	
	
	\caption{Ejemplo de un \emph{webaccess Log} de un servidor Apache.}
	\label{fig:accesslog-apache-teleton}
\end{figure}


  En la Figura \ref{fig-ejemplo-webaccesslogbruto} nos entrega mucha información interesante como la \texttt{IP} desde donde se conecta, el tipo de navegador, el dispositivo si es un teléfono inteligente o un navegador de escritorio, la fecha en que se realizó el acceso y también lo más relevante el destino del usuario.
  
  
  
  




\section{Definición del Problema}


El problema de la Predicción, ha surgido hace años y diversos investigadores han trabajado con distintos enfoques. Rissman y Langdom en los laboratorios Bell al realizar pruebas con un robot y hacer un experimento con un robot que tiraba una moneda compitiendo con humano, realizaba todos los calculos markovianos y calculos de las probabilidades condicionales para que cierto evento ocurra, a diferencia del sujeto que solo estaba esperando un resultado.

Predecir no es trivial, pero si podemos llegar a cercanos y minimizar el error de equivocarnos. Sin embargo, dos áreas han tratado de resolver el problema, LDC y Machine Learning de manera separada. Por parte de LDC los mayores problemas son que los predictores funcionan totalmente desconectados y no dan una de disponibilidad inmediata de los resultados, en cambios en el área de Machine Learning debemos crear un modelo para entrenar y luego poder generar una función predictiva.

Planteamos el problema de poder resolver tener un modelos híbrido juntando los patrones de cada área y disponerlo como un servicio inmediato dando una predictibilidad inmediata que hoy en la industria es necesaria para poder hacer útiles estos algoritmo y dar un valor a los avances.

 




% @TODO: SEGUIR TRABAJANDO EN ESTA BREVE INTRODUCCION
%En este tema convergen tres áreas, por un lado existe trabajo para crear estructuras eficientes para predicciones basadas en algoritmos de compresión, como es en el caso de~\cite{Claude2014}, y, por otro lado, el uso de algoritmos de aprendizaje para realizar clustering y predecir el comportamiento basado en el mismo contenido o en la distancia del contenido que visita el usuario actual al contenido clusterizado, como es el caso de ~\cite{Poornalatha2012}, inclusive se han utilizado modelos de Markov en ~\cite{Dongshan2002}  para poder modelar el comportamiento de la web.
%La tercera área son los Sistemas de Recomendación, la cual en este proyecto no se tocará pero si se mencionará el enfoque práctico que presenta área como un foco de múltiples implementaciones. 







%Algoritmos como serivcio
\section{Algoritmos como servicio web }

	Los avances en el desarrollo de nuevas tecnologías que brinden mejores experiencias en su uso día a día, deriva en cómo podamos llevar varios escenarios idealizados a implementaciones empresariales reales. Es bastante común encontrar librerías que son bastante útiles para hacer Minería de Datos, agrupación y muchas operaciones que pueden recurrir en cálculos muy complejos, pero no se pueden ofrecer como servicio. Ya en pleno auge de las infraestructuras en la nube, la capacidad de cómputo que se puede alcanzar no es un problema como antes lo era para un Científico de Datos.


	Una \emph{API} es un interfaz de programación de Aplicaciones que nos permiten intermediar el \emph{Servicio $A$} con el \emph{Servicio $B$}. Respectivamente $A$ puede ser el proveedor y $B$ el demandante del servicio. Si quisiéramos analizar datos que se encuentran dentro de un servidor específico, estos se podrían consumir por esta interfaz. Existen variados clientes que nos permiten ayudar en esta comunicación, incluso se pueden utilizar por una terminal de {Unix} que es posible dialogar mediante el programa \emph{curl}.
	
	Ya se dispone de infraestructura como servicio (\emph{IaaS}) , software como servicio (\emph{SaaS}), plataformas como servicios (\emph{PaaS}). Dado lo anterior ofrecer estos algoritmos para hacer que las soluciones de desarrollo den valor agregado a la experiencia requerida por el usuario final. Por esto hemos decidido utilizar una librería y {framework} que nos de esta posibilidad. Ofrecer algoritmos a la industria como un servicio que ayuda de manera eficiente e Inteligente desarrollado como una API REST, que permitirá una fácil integración. 
	
	Todas las ventajas de este patrón son heredados de las características que ofrece una API, interoperabilidad, evitar problemas de Infraestructura, Resiliencia de Datos, Persistencia de Datos, Análisis y Procesamiento sin afectar un curso operacional de una aplicación. Un ejemplo claro de esto es el análisis de datos en sistemas legados los cuales en plan de mejoras, no poseen la compatibilidad para poder realizarlo. Por otro lado, los algoritmos de compresión o algoritmos de \emph{Machine Learning} tienden a ser muy complejos de implementar o ocupan muchso recursos y este hecho pueden ser la razón para no implementarlos. 
	
	
	
	




%introducción a las Prediccion


\section{Predecir con \emph{PredictionIO}}

  	
En esta sección se presenta formalmente el ambiente de desarrollo que se utilizará durante este trabajo. \emph{PredictionIO} es un servidor de \emph{Machine Learning} de código abierto para Científico de Datos y Desarrolladores que permite crear motores de predicción para aplicaciones en producción, con un bajo tiempo de entrenamiento y despliegue. Principalmente está construido en \emph{Apache Spark, HBase} y \emph{Spray}. 

Este ambiente de trabajo se encuentra en un maduración estable y constante que permite tanto disponer servidores con motores predictivos, como también toda una infraestructura distribuida para hacer que complejos algoritmos que sean utilizados para solucionar problemas reales de mayor escala.



% PIO, tiene practicamente todo armadao, ello no hicieron nada nuevo ... solo juntaron  todo...



% He estado investigando y revisando documentación, Yelp, Skype, Hubot de github y otras implementaciones tienen usando prediction.io




% La otra opción es meterle a este "DASE" un algoritmo  que mezcle una representación de cadenas de markov mezclado con LZ78. no se en que punto mezclarlo en el diccionario, o la verdad es como hacer el compresor sea "mas inteligente", encontré un papaer que te adjunto en el cual usan lz78 y lzw, esta interesante ya que le hacen un acercamiento mas al tema de de ser un predictor online.


% Ahora entiendo que la cadenas de markov son y se han ocupado para las predicciones, pero no veo la necesidad de ocuparlas mayormente. Adin me inisiste en que le de una vuelta.... pero mi sensación  es que tengo separada las ideas en dos extremos. 

% Ya revise Suffix Tree para predicciones LZ77 y LZW, también PPM y HMM.

 
 



\subsection{Arquitectura DASE}


Un motor de predicción es un tipo de proceso en \emph{Machine Learning}. Siguiendo una arquitectura de tipo \emph{DASE}, contendríamos los siguientes componentes.



\begin{itemize}

  \item\label{dase-datasource} \textbf{ $[D]$ Data Source y Data Preparator}. Los Data Source leen la data desde la entrada original y la transforman en un formato deseado para hacer análisis de estos. En cambio \emph{Data Preparator} pre-procesa la información y la reenvía a los algoritmos para   hacer el modelo de entrenamiento.


  \item\label{dase-algoritmo} \textbf{ $[A]$ Algoritmo}. Los componentes de \emph{PredictionIO}, dada sus librerías incluyen algoritmos de \emph{Machine Learning}, estos, pueden ser provistos por \emph{Apache Spark} o se pueden incluir algoritmos propios como también de terceros.
    Adicionalmente a los algoritmos podemos asignarle parámetros, para determinar como debiese ser construido el motor ó si es requerido para un cierto algoritmo.



  \item\label{dase-servicio} \textbf{ $[S]$ Servicio}. El componente servicio toma las consultas ó \emph{queries} de predicción y retorna los resultados, en nuestro modelo propuesto en la etapa experimental veremos el siguiente símbolo de una secuencia. 
  Si el motor de predicción tiene múltiples algoritmos, combinará los resultados en uno. Adicionalmente, la lógica específica de negocios puede ser añadida para especificar aún más el resultado final. 
 
  \item\label{dase-eval} \textbf{ $[E]$ Evaluación de Métricas}.
Las métricas de evaluación cuantifican la precisión de la predicción con una puntuación numérica. Puede ser utilizado para la comparación de algoritmos o ajustes de los parámetros del algoritmo.
\end{itemize}




  \tikzstyle{decision} = [diamond, draw,text width=4.5em, text badly centered, node distance=2.5cm, inner sep=0pt]
  \tikzstyle{block} = [rectangle, draw,text width=5em, text centered, rounded corners, minimum height=4em]
  \tikzstyle{line} = [draw, very thick, color=black!50, -latex']
  \tikzstyle{cloud} = [draw, ellipse, node distance=2.5cm,
  minimum height=2em]


\begin{figure}[t]
	\centering	
	\resizebox{0.8\textwidth}{!}{% <------ Don't forget this %
	
		\begin{tikzpicture}[scale=1, node distance = 2.5cm, auto]
		% Place nodes
		
		\node [block] (init) {Data de la Aplicación};
		\draw [color=gray,thick](1.3,1) rectangle (11.2,-3.5);    
		
		\node [block,right of=init] 		  (datasource) {Data Source};
		\node [block,right of=datasource]     (datapreparator) {Data Preparator};
		\node [block,right of=datapreparator] (alg1) {Algoritmo };
		\node [block,right of=alg1] 		  (serving) {Servicio};
		\node [block,below of=serving] 		  (evalmetric) {Evaluación Métrica};
		\node [block,right of=serving] 		  (resultpredict) {Resultado Predicción};
		\node [block,below of=resultpredict]  (resulteval) {Resultado Evaluación};
		
		\path [line] (init) -- (datasource);
		\path [line] (datasource) -- (datapreparator);
		\path [line] (datapreparator) -- (alg1);
		\path [line] (alg1) -- (serving);
		\path [line] (serving) -- (evalmetric);
		\path [line] (serving) -- (resultpredict);
		\path [line] (evalmetric) -- (resulteval);    
		
		\end{tikzpicture}
	}
	\caption{Diagrama de componentes Arquitectura DASE.}
	\label{fig:arquitectura-dase}
\end{figure}



\vspace{1cm}

\emph{PredictionIO} ayuda a tener componentes muy modulares, las que ya hemos descrito como  arquitectura \emph{DASE} (\ref{fig:arquitectura-dase})  que puede construir modelos de predicción de manera sencilla y ya contando con toda la arquitectura y algoritmos de \emph{MLIB}\footnote{Machine Learning Library Apache Spark, \url{http://spark.apache.org/mllib/}} de \emph{Apache Spark}. También poder integrarlos con gran facilidad a cualquier sistema o plataforma, por ejemplo, es posible elegir cual de todos los componentes se podrá desplegar al momento de crear un \emph{Engine} (Motor de Predicción.)



\vspace{0.5cm}
\subsection{Despliegue de motor de predicción}

  Un \emph{Motor de predicción} pone todos los componentes del diseño de arquitectura \emph{DASE} en un estado especifico de despliegue 
  \begin{enumerate}
  		\setlength{\itemsep}{1pt}
  		\setlength{\parskip}{0pt}
  		\setlength{\parsep}{0pt}
    \item Data Source
    \item Data Preparator
    \item Uno o más Algoritmos generadores de Modelos
    \item Un Servicio 
  \end{enumerate}

  Si se especifica más de un algoritmo, cada uno de los resultados de los modelos de predicción se entregará para ser consumido por cualquier cliente.
  Cada \emph{motor de predicción} procesa los datos y construye un modelos de forma independiente. Por lo tanto, todos los motores de predicción que usaremos sirven a su propio conjunto de resultados. Por ejemplo, se puede desplegar dos \emph{motores predictivos} para una aplicación móvil: uno para recomendar noticias a los usuarios y otro para sugerir nuevos amigos a los usuarios.


\vspace{1cm}
\subsection{Evaluación del motor de predicción }

  Para evaluar el \emph{Accuracy} de un motor de predicción, se debe especificar la métrica seleccionad cuando se corre el motor de evaluación, en los capítulos experimentales se verá como se generan métricas y como se desempeña esta métrica para ser evaluada.











\subsection{Modelamiento de eventos}

%https://docs.prediction.io/datacollection/eventmodel/




  El modelamiento de eventos es fundamental para un predictor \emph{online}, el hecho de poder llevar un vector con ciertas propiedades característica\footnote{Característica de un cierto dataset para entrenar.} de una representación vectorial del mundo del \emph{Machine Learning} a un modelamiento secuencial, es en realidad el modelamiento que se debe realizar de como  tener las muestras de datos para generar \emph{Resilient Distributed Dataset} (\texttt{RDD}) que son un parte principal de nuestro ambiente de trabajo con \emph{PredictionIO} para poder acceder posterior o inmediatamente. 

  Un evento lo definiremos como entidad que nos permite dar una representación temporalizada de información que será procesada por un motor de predicción. Analizaremos los eventos que un usuarios realiza para poder acceder a una web. Adicionalmente cuando cada usuario ingresa a una web automáticamente este genera una sesión, desde que que llega hasta que abandona la web.

  Usaremos un \emph{dataset} con información que esta totalmente depurada y recuperada de los \emph{access log} (provista por Claude \etal~\cite{Claude2014}), los cuales a efectos de temporalidad solo nos interesa conocer la secuencialidad de estos accesos.
  


%Poner algo mas matematico.
% EVENT API 
% https://docs.prediction.io/datacollection/eventapi/

  El modelamiento que realizaremos contempla los siguientes campos :

    \begin{itemize}
    		\setlength{\itemsep}{1pt}
    		\setlength{\parskip}{0pt}
    		\setlength{\parsep}{0pt}
      \item Tipo de Evento: Visitar
      \item Entidad que ejecuta el evento: Usuario
      \item Propiedades:
          \begin{enumerate}
          		\setlength{\itemsep}{1pt}
          		\setlength{\parskip}{0pt}
          		\setlength{\parsep}{0pt}
            \item Página actual
            \item Página siguiente
            \item Cierre de Sesión
          \end{enumerate}
    \end{itemize}



    El interés de tener un modelo totalmente atómico es poder contemplar la información que nos entrega, destacando sus variables y propiedades como restricciones.



\subsection{Ventajas de PredictionIO }


  Es posible mezclar y aplicar distintas característica, si el modelo no puede ser persistido por \emph{PredictionIO} automáticamente. Se requiere un objeto  heredado de una clase que permita lograr la persistencia en memoria, esto permite cargar el modelo persistentemente e instanciar automáticamente durante la ejecución del motor de predicciones.

  % Tenga en cuenta que los modelos generados por PAlgorithm no pueden ser persistieron automáticamente por naturaleza y deben implementar estas características, si se desea modelo de persistencia.


  Comprendiendo el concepto de \emph{Resilient Distributed Dataset} (\texttt{RDD} \ref{concept-RDD}), ésta es la abstracción básica de \emph{Apache Spark}, aún más, esto es una de las grandes cualidades de PredictionIO, ya que no solamente podemos disponer de una máquina para hacer estudios o implementar algoritmos, este servidor de \emph{Machine Learning} permite, gracias a sus componentes poder hacer un cluster para entregar mayor eficiencia acorde a los datos o algoritmo a implementar.

  Ya hemos mencionado que los \texttt{RDD} son una representación inmutable, una colección divida de elementos que pueden ser operadas en paralelo. Internamente cada \texttt{RDD} tiene cinco principales propiedades:


  \begin{enumerate}
    \item Una lista de particiones.
    \item Una función para procesar cada porción de datos.
    \item Una lista de dependencias en otros \texttt{RDD}. 
    \item Opcionalmente una partición de un \texttt{RDD} puede ser representada como una combinación de llave y valor. 
    \item Una lista optativa de los lugares preferidos para calcular cada una dividida en (por ejemplo, lugares de bloque para un archivo \emph{HDFS}); para procesamiento en \emph{Clustering}.

  \end{enumerate}
  




% https://docs.prediction.io/templates/recommendation/customize-serving/


















%Se consideran y ex- plican detalladamen- te todos los concep- tos, teor ́ıas, y aspec- tos pertinentes al te- ma tratado.
%El marco teo ́rico con- cuerda con los ob jeti- vos y el tipo de traba- jo.











\section{Descripción del Contenido}

%Ejemplo :

%This paper is organized as follows. We begin in Section 2 with some preliminary defi- nitions. In Section 3, we present six VMM prediction algorithms. In Section 4, we present the experimental setup. In Sections 5-6, we discuss the experimental results. Part of the experimental related details are discussed in Appendices A-C. In Section 7, we discuss the related work. In Section 8, we conclude by introducing a number of open problems raised by this research. Note that the source code of all the algorithms we consider is available at http://www.cs.technion.ac.il/~ronbeg/vmm.




\section{Contexto preliminar} 
\label{sec:preliminar}

% Idealmente aca explicar el problema

  La Web crece constantemente y por ende su infraestructura, también la información que podemos obtener de los  usuarios y  por consecuencia mayor concurrencia de los sistemas, la cual para los usuarios finales se traduce en un incremento de latencia y una mejor o peor experiencia de usuario. Paralelamente se suma un costo exponencial de recursos tanto en tecnologías de desarrollo como servicio que no son optimizados para poder dar una experiencia de usuario con calidad de servicio. Podemos reflexionar, entonces, que tener mayores recursos no mejorará el rendimiento, ni tampoco será lo óptimo para dar una calidad de servicio web ya que el ancho de banda de Internet no crecerá en la misma proporción.
   
  Adicionalmente, las tecnologías para la creación de web dinámicas y asíncronas han evolucionado a favor de traspasar la carga cliente.
  Hoy en día ya se poseen lenguajes y {framework} que disminuyen considerablemente la carga de un servidor, por lo cual, un buen servicio web es proveer una balanceada carga dentro del cliente y el servidor, pero cuando se poseen un gran volumen de datos es fundamental tomar decisiones que los recursos y lenguajes no cubren, es ahí el interés de dar inteligencia a servicios de la web.

  Interpretaremos que la manera en que un usuario navega es su comportamiento o patrón registrado en \emph{webacces log}, y que se pueden analizar, estudiar y modelar con algoritmos que tengan enfoques predictivos. 

  Sobre estos registros se pueden hacer representaciones eficientes como las realizadas por Claude \etal~\cite{Claude2014},  minería de datos para buscar \emph{Web Usage Mining} (WUM), el porque de hacer minería de datos radica en que cada día la web genera una innumerable cantidad de información, por lo cual usar algoritmos que se puedan opera de manera comprimida o con una representación lo más liviana posible es de interés ya que además de disminuir el espacio físico o recursos utilizados, se puede usar como un algoritmo de predicción y trabajar con una mayor cantidad de datos.
  
  Teniendo el conocimiento que los \emph{webaccess logs} se pueden estudiar de manera procesada o pre-procesada, ayudaría a ingenieros de desarrollo web y diseñadores de experiencia de usuarios, como también en general a mejorar la experiencia de usuario final, también disminuyendo por ejemplo la latencia en respuestas por parte de cada petición realizada, con técnicas de \emph{pre-fetching predictivo} en el lado del cliente.
  

  Actualmente, los sitios web han evolucionado de ser contenido estático a dinámico, también moderado por administradores a contenido orgánico creado por usuarios finales. Se debe poseer una adaptabilidad a la demanda o proveer información que permita adaptarse a los eventos, por lo tanto, hacer un estudio sobre esto y poder hacer integraciones en áreas como uso de algoritmos de compresión y \emph{Machine Learning} presenta un gran desafío. Independientemente del área, el problema común  se puede resolver pero se busca encontrar las fortalezas de cada área y usarlas en conjunto. 

  Durante este trabajo se usarán técnicas de compresión de datos, se utilizará una infraestructura y patrón de implementación para modelos de \emph{Machine Learning} como servicio. Adicionalmente toda la experimentación se llevará acabo ofreciendo los algoritmos y modelos como servicio basado en una Transferencia de Estado Representacional (\emph{REST}~\ref{concept-rest}), y así implementarlo en áreas productivas las cuales pueden presentar interés dando nuevos escenarios de estudio.


  La sesión de un usuario comienza cuando se conecta a un servicio web, éstas pueden ser páginas informativas, redes sociales, web dinámicas, contenido colaborativo, etc. Estableciendo dicha conexión a una página, se crea en ese momento una sesión de navegación automáticamente. Dado esto es posible almacenar datos muy relevantes los cuales son ''webaccess log'' ó registros de accesos web, durante el texto se mantendrán las referencias en inglés. Un ejemplo de \emph{webaccess log} es lo que se observa en la figura ~\ref{fig-ejemplo-webaccesslogbruto}:

\begin{figure}[tb]\label{fig-ejemplo-webaccesslogbruto} 
	\centering
	\begin{lstlisting}[frame=single,basicstyle=\ttfamily\tiny,]
	172.31.33.116 - - [26/Nov/2015:00:12:12 +0000] "HTTP/1.1" 200 1784 "http://localhost/home" 
	"Mozilla/5.0 (Linux; Android 5.1.1; SAMSUNG SM-G920I Build/LMY47X) 
	SamsungBrowser/3.2 Chrome/38.0.2125.102 Mobile Safari/537.36"
	172.31.33.116 - - [26/Nov/2015:00:12:12 +0000] "HTTP/1.1" 200 179333 "http://localhost/news" 
	"Mozilla/5.0 (Linux; Android 5.1.1; SAMSUNG SM-G920I Build/LMY47X) 
	SamsungBrowser/3.2 Chrome/38.0.2125.102 Mobile Safari/537.36"
	172.31.33.116 - - [26/Nov/2015:00:12:12 +0000] "HTTP/1.1" 200 24660 "http://localhost/health" 
	"Mozilla/5.0 (Linux; Android 5.1.1; SAMSUNG SM-G920I Build/LMY47X) 
	SamsungBrowser/3.2 Chrome/38.0.2125.102 Mobile Safari/537.36"
	172.31.33.116 - - [26/Nov/2015:00:15:12 +0000] "HTTP/1.1" 200 24604 "http://localhost/sports" 
	"Mozilla/5.0 (Linux; Android 5.1.1; SAMSUNG SM-G920I Build/LMY47X) 
	SamsungBrowser/3.2 Chrome/38.0.2125.102 Mobile Safari/537.36"
	172.31.33.116 - - [26/Nov/2015:00:20:12 +0000] "HTTP/1.1" 200 4860 "http://localhost/home" 
	"Mozilla/5.0 (Linux; Android 5.1.1; SAMSUNG SM-G920I Build/LMY47X) 
	SamsungBrowser/3.2 Chrome/38.0.2125.102 Mobile Safari/537.36"
	172.31.33.116 - - [26/Nov/2015:00:22:19 +0000] "HTTP/1.1" 200 4841 "http://localhost/finances" 
	\end{lstlisting}
	
	
	
	\caption{Ejemplo de un \emph{webaccess Log} de un servidor Apache.}
	\label{fig:accesslog-apache-teleton}
\end{figure}


  En la Figura \ref{fig-ejemplo-webaccesslogbruto} nos entrega mucha información interesante como la \texttt{IP} desde donde se conecta, el tipo de navegador, el dispositivo si es un teléfono inteligente o un navegador de escritorio, la fecha en que se realizó el acceso y también lo más relevante el destino del usuario.
  
  
  
  




\section{Definición del Problema}


El problema de la Predicción, ha surgido hace años y diversos investigadores han trabajado con distintos enfoques. Rissman y Langdom en los laboratorios Bell al realizar pruebas con un robot y hacer un experimento con un robot que tiraba una moneda compitiendo con humano, realizaba todos los calculos markovianos y calculos de las probabilidades condicionales para que cierto evento ocurra, a diferencia del sujeto que solo estaba esperando un resultado.

Predecir no es trivial, pero si podemos llegar a cercanos y minimizar el error de equivocarnos. Sin embargo, dos áreas han tratado de resolver el problema, LDC y Machine Learning de manera separada. Por parte de LDC los mayores problemas son que los predictores funcionan totalmente desconectados y no dan una de disponibilidad inmediata de los resultados, en cambios en el área de Machine Learning debemos crear un modelo para entrenar y luego poder generar una función predictiva.

Planteamos el problema de poder resolver tener un modelos híbrido juntando los patrones de cada área y disponerlo como un servicio inmediato dando una predictibilidad inmediata que hoy en la industria es necesaria para poder hacer útiles estos algoritmo y dar un valor a los avances.

 



\chapter[Conceptos Básicos]{Conceptos básicos y trabajos relacionados} \label{ch:Conceptos-Basicos}





En este capítulo se explicarán brevemente los conceptos principales que se usarán en los siguientes capítulos:


\section{Conceptos}

\subsection{Access Log}\label{concept-accesslog}

Los \emph{Access Log} son registros que se almacenan en un servidor, los cuales dependiendo del servidor, sistema operativo y las configuraciones del ambiente pueden tener mayor o menor información. Cuando los usuarios acceden a diversos sitios web, estos registros dejan una gran cantidad de información, un ejemplo se puede ver en la Figura~\ref{fig:accesslog-apache-teleton}. Si se extraen de forma correcta se puede obtener patrones de navegación de usuarios. 


\subsection{Árboles Trie} \label{concept-trie}

Los \emph{trie} son un tipo de árbol de prefijos, estructuras de datos en forma de árbol que almacenan datos en sus nodos y es muy fácil la recuperación de información de estos. Se caracterizan por ser un conjunto de llaves que se representan en el árbol y sus nodos internos representan la información, en nuestro caso una secuencia de acceso o un símbolo (secuencia de largo 1). 

Un árbol es una estructura general de nodos recursivos. Hay muchos tipos de árboles. Los populares son árbol binario y el árbol balanceado. Un \emph{trie} es conocido por muchos nombres incluyendo árbol prefijo, árbol de búsqueda digital, árbol de la recuperación (de ahí el nombre de \textquotedbl \emph{trie}\textquotedbl, por la palabra del inglés recuperación, \emph{retrieval}).

Cada tipo de árbol tiene distinta finalidad, estructura y comportamiento. Por ejemplo, un árbol binario almacena una colección de elementos comparables (por ejemplo, números). Por lo tanto, se puede utilizar para almacenar un conjunto de números, o al índice de otros datos que pueden ser representados por los números (por ejemplo, objetos que pueden ser \emph{hash}). Su estructura está ordenada por lo que se puede buscar rápidamente para encontrar nodo. Otras estructuras de árbol, como un árbol balanceado son similares en principio al \emph{trie} que se implementará en la etapa experimental.

Un \emph{trie} representa una forma  estructurada de nodos, la cual puede almacenar secuencias. % (ver el artículo de wiki). 
Es muy diferente cuando almacena secuencias de valores en lugar de valores individuales. Cada nivel representa un incremento en la altura del árbol.
%'¿cuál es el valor del punto I de la lista de entrada'. Esto es diferente a un árbol binario que compara el valor buscado único a cada nodo.


Durante este trabajo mostraremos que nuestro modelo de predicción usa un \emph{trie}, para representar un diccionario, en los cuales podemos señalar las siguientes operaciones disponibles:

	\begin{itemize}	
		\item \textbf{findByPrefix(x):}  Retornar una una lista de todos los nodos hasta llegar a un nodo que posea un \emph{String} de parámetro equivalente a $x$.
		
		\item \textbf{contains(x):} Retornar una lista de nodos intermedios hasta que el contenido del nodo final sea hoja o el intermedio corresponda a valor de la secuencia \emph{String} $x$.
		
		\item \textbf{remove(x):} Retorna \emph{true} o \emph{false} cuando es posible remover el nodo con el contenido equivalente a $x$.
		
		\item \textbf{pathTo(x):} Retorna una lista de nodos hasta llegar a un nodo que posea el contenido equivalente del valor $x$.
	\end{itemize}







\subsection{Alfabeto} \label{concept-alphabet}

Un alfabeto es un conjunto ordenado de elementos. Una de sus características es que la cantidad de elementos puede ser de valores finitos y discreto lo representaremos por $\Sigma$ y una de las características básica de este alfabeto es consiste en símbolos $\sigma$ que pertenecen a $\Sigma$, durante el resto de nuestra investigación usaremos solo un $\Sigma = \sigma_{i},\sigma_{i+1}\ =\ 17\  \mbox{símbolos}$.



Dado un volumen de datos experimental, nuestro alfabeto es mostrado simbólicamente como la representación de varios nodos contenido en un \emph{Trie} que modela la navegación de usuarios para un sitio web.

%Donde $\sigma_{1}$, puede ser definido como la página inicial de una cierta secuencia  o sesión de usuario. Este alfabeto es finito y acotado por una minería de datos ya realizada por \cite{Claude2014} en \emph{Efficient Indexing and Representation of Web Access Logs}. % Referencia al trabajo de claud

%Let A be a discrete alphabet consisting of M  2 symbols; x = x ; : : : ; x ; x 2 A
% be rst n symbols of message; jj be the length of sequence  or the cardinality of
% set . The probability of symbol x = a 2 A for a source with memory depends on
% current context s = x ; : : : ; x 2 A ; d  D. The set of all possible contexts can
% be presented as nodes in M -ary tree with depth D. Context (sequence) s describes a
% path from tree root to the current node also denoted as s.
% Usually the true conditional probabilities are unknown and the coding conditional
% pr ob abilitiesq(ajs) depend on characteristics of one or sev eralsubsequences x (s),
% x (s) is the subsequence of all symbols x such that x ; : : : ; x
% = s . These
% characteristics are: frequency f (ajs) = f (ajx (s)) of symbol a in x (s), alphabet
% A(s) = A(s; x (s)) = fa : f (ajs) > 0g, its cardinality m(s) = jA(s)j etc. Even at
% small D, the n umber of model states is big, subsequences x (s) are short ina verage,
% and their statistics is insuÆcient for eective compression.
% PPM algorithm [2] is based on an (implicit) assumption: the longer is the common initial part of contexts, the more (in average)similarity there is between their
% conditional probability distributions. High PPM eÆciency means this assumption is
% fair forthe majority of real sources.



\subsection{Secuencias discretas}\label{concept-discret-seq}

Definimos una secuencia de accesos discreta y finita, dado los accesos que tiene un usuario frente a una web, lo anterior es acotado por el concepto de sesión, el cual es desde que se inicia la navegación por parte de un usuario, es decir secuencia de tamaño $Seq\ \leq 1$ y de tamaño no superior a un alfabeto $\Sigma$.



% Compresor tonto o Machine Learning lento para predecir
\subsection{Lossless Data Compression (LDC)} \label{concept-LDC}

La compresión sin pérdida o \emph{LDC}, es el arte de poder comprimir \emph{bits} y poder hacer el proceso inverso, es decir poder codificar y decodificar. En capítulos posteriores se detallará más sobre el tema compresión y como esta área ayuda a crear un modelo de predicción.





\subsection{Motor de Predicción}\label{concept-enginepredict}

Es la parte fundamental de un sistema dirigida a adivinar el futuro acceso de un usuario. La salida del motor de predicción es la siguiente página, que se compone de un símbolo representando la dirección \emph{url} o una sección en particular de una cierta web. 
% que son propensos a ser solicitado por el usuario en las solicitudes posteriores.

 


\subsection{Resilient Distributed Datasets }\label{concept-RDD}

	Los \texttt{RDD}, permiten que en un servidor de \emph{Machine Learning } pueda mantener un modelo o motor de aprendizaje persistente sin importar en el flujo que se encuentre.

	Esta estructura es fundamental dentro de la librería que se introducirá mas adelante, \emph{Apache Spark}. Esta estructura es una colección distribuida de objetos inmutables, cada 
	set de datos en un \texttt{RDD} es divido en particiones lógicas, las cuales puedes ser computadas en distintos cluster. Los \texttt{RDD} pueden contener cualquier tipo de objeto de los siguientes lenguajes: \emph{Python}, \emph{Scala} y \emph{Java}, incluyendo clases definidas por el usuario. 

	Formalmente los \texttt{RDD} son solo de lectura, una colección de objetos divididos. Estos pueden ser creados a través de  operaciones deterministas en una cierta tabla o un almacenamiento externo u otra \texttt{RDD}.
	Otra de las características de los \texttt{RDD}, es que son colecciones de elementos tolerantes a fallas que pueden ser operadas en si mismas o en paralelo.
	\emph{Apache Spark} hace el uso del concepto de \texttt{RDD} para lograr rapidez y eficiencia en las operaciones de \emph{MapReduce}, de ser requeridas. Destacamos la escabilidad de esta librería para un gran nivel de cómputo, pero en este trabajo no se explicará el uso de \emph{Spark}, pero si se utilizarán algunos conceptos como \texttt{RDD} y otros.




\subsection{Data Source y Dataset }

	Los \emph{Data Source} son distintas fuentes de datos que podemos ir sacando set de datos (\emph{Dataset}). Ambos conceptos están enfocados a proveer información tanto para el servidor de \emph{Machine Learning}, como para el procesamiento y análisis.
	En este trabajo el dataset con que realizaremos nuestro estudio son los registros de accesos de la web española \emph{MSNBC}\cite{Claude2014}, los cuales representan aproximadamente 1.000.000 de registros de \emph{webaccess log}.

	Nuestro set de  datos esta basado en los registros (webaccess log), ya previamente depurados con una representación numérica desde 0 hasta 16, el cual como antes ya se ha señalado estará relacionado al concepto de alfabeto. También crearemos \emph{dataset} sintéticos para poder realizar depuraciones de nuestra implementación y casos de bordes.
	
	 



 



% \subsection{DataFrame}


	% A DataFrame is a distributed collection of data, which is organized into named columns. Conceptually, it is equivalent to relational tables with good optimization techniques.
	% Here is a set of few characteristic features of DataFrame −
	% Ability to process the data in the size of Kilobytes to Petabytes on a single node cluster to large cluster.
	% Supports different data formats (Avro, csv, elastic search, and Cassandra) and storage systems (HDFS, HIVE tables, mysql, etc).
	% State of art optimization and code generation through the Spark SQL Catalyst optimizer (tree transformation framework).
	% Can be easily integrated with all Big Data tools and frameworks via Spark-Core.
	% Provides API for Python, Java, Scala, and R Programming.



% easy text
% https://es.wikipedia.org/wiki/Trie
%Definición interpretada de esot

%\subsection{Cadenas de Markov}





\subsection{Transferencia de Estado Representacional}~\label{concept-rest}

Es un estilo de desarrollo de software para sistemas distribuidos mediante Internet. El uso de REST, de sus siglas en ingles, ha sido adoptado mucho más adoptado que  \emph{Simple Object Access Protocol}, ya que los servicios \emph{REST} aprovechan mayor cantidad el ancho de banda, lo que hace que sea una mejor opción para su uso a través de Internet, los mensajes tanto originados por el servidor o cliente puede ser enviados en formato \emph{xml} o \emph{json}. 



%@TODO:
% Este tema debería detallarse en las siguientes secciones
\section{Trabajos relacionados}

En la literatura, el tema de la predicción en la web se ha presentado como un tema recurrente provocando bastante atención durante los último años y ha sido tratado por varios autores de áreas de \emph{Machine Learning} y \emph{Lossless Data Compression}. Tenemos los siguientes trabajos de interés:


\begin{enumerate}


  %%%%%%%%%%%%%%%%%%%%%%%%%%%%%%%%%%%
  \item \textbf{Dynamic and memory efficient web page prediction model using LZ78 and LZW algorithms }

	Moghaddam y Kabir~\cite{Moghaddam2009}  realizan una comparación de LZ78  y el algoritmo LZW, el cual es una derivación del anterior. La mayoría de las aplicaciones actuales que predicen el siguiente acceso a un página web posee una  componente offline que hace la tarea de preparar data y luego disponer una sección en linea que permite personalizar cierto contenido para un usuario en particular basado en las actividades de navegación
	
	En la mayoría de las técnicas de \emph{Web Usage Mining}, las secuencias se utilizan ya sea para producir las reglas de asociación o para producir estructuras de datos de tipo árbol o cadenas de Markov para representar patrones de navegación. Los  Modelos de Markov se basan en una teoría bien establecida y son fáciles de entender.  
	

	La propuesta es no crear un modelo predictivo por usuarios.
	Moghaddam y Kabir proponen modelar la navegación de usuarios mediante un Trie creado por un algoritmo de la familia LZ y usando muchas sesiones de usuarios, tener un modelo predic






  %%%%%%%%%%%%%%%%%%%%%%%%%%%%%%%%%%%
  \item \textbf{Prediction Algorithms for User Actions} ( Hartmann \& Schreiber,2007  \etal~\cite{hartmann2007}) , 

	{
		% Proactive User Interface
	Se requiere la predicción de la siguiente acción del usuario basada en la historia de interacción que ha tenido con una interfaz. 
	En su trabajo dan una revisión a los algoritmos de predicción Discreta (SPA) y desarrollan dos propuestas de algoritmos basadas en Modelos de Markov que convinan distintos orden de Markov. Y desarrollan una librería en PERL para su propuesta y evaluación.
	
	% NOTAS MIAS
	
	%Acorde a lo que Hartman propone nuestro trabajo esta totalmente alineado, debido a que optimizar o predecir el siguiente acción/acceso puede ayudar enormemente a el problema de PUI Proactive User Interfaces que el plantea
	
	% Ellos comparan comparan el mejor rendimiento del algoritmo acorde a los requerimientos de memoria y tiempo de procesamiento
	
	%Hartman en su paper dice que las Interfaces de Usuario Intelignente(Intelligent User Interfaces) tb dan uso y son un campo de aplicación para las predicciones secuenciales 
	
	% Algo significativo que menciona es que muchos de los algoritmos toman k- elementos de una secuencia para hacer una predicción, en mi caso yo tomo la secunecia completa en base a un entrenamiento
	
	% Plantea que hay dos maneras de hacer predicciones secuenciales On-Demand or Live, en mi caso es Live y OnDemand la implementación que ofrezco cumple ambos campos.
	% Online Learning Algorithm -> Muy similar a mi modelo hibirido ML y LDC
	% Mi Propuesta podría ayudar mucho en PUI para caso como el desarrollo movil,esto es obiviamente por las limitaciones de pantalla que presenta 
	
	% Modelos/Algortimos propuestos   ONISI y IPAM Jacobs Blockeel, ActiveLeZi
	}
  %%%%%%%%%%%%%%%%%%%%%%%%%%%%%%%%%%%
  \item \textbf{ActiveLezi} ( Gopalratnam \& Cook,2007 \etal\cite{Gopalratnam2007}) 
  {  
  Proponen un Algoritmo \emph{On-Demand} que considera varios modelos de Markov.  
  El funcionamiento es basado en almacenar la frecuencia del patrón de \emph{input} en un \emph{Trie} acorde al algoritmo de compresión de \emph{LZ78} para superar algunos de los problemas que surgen con \emph{LZ78}, se usa una ventana de largo variable de los simbolos previamente usados en la construcción del \emph{Trie}. El tamaño de la ventana crece con el numero de las diferentes subsecuencias que se van viendo en la entrada de cada secuencia nueva que ingresa.  Sea $suff_{l}$  el sufijo de largo $l+1$ el sufijo de longitud l 1 de las inmediatamente historial de interacción a, que es un hacha .... la probabilidad se define de la siguiente forma recursiva.:
 }



  %%%%%%%%%%%%%%%%%%%%%%%%%%%%%%%%%%%
  \item Dongshan y Junyi~\cite{Dongshan2002} 
  {
	  Destacan que un modelo de Markov puede ayudar a predecir el comportamiento de un usuario, pero con ciertas limitaciones .  Para solucionarlo presentan un nuevo modelo de Markov basado en una representación de \emph{Tree Order Model}, el cual es un híbrido entre un modelo de markov tradicional y una representación de árbol, bautizada como HTMM (por sus siglas en inglés, \emph{Hybrid-Order Tree Markov Model}).
	  Su modelo fue presentado en 2002, y da una importancia a conocer la predicción de los \emph{web access}, dada la importancia de creación de redes, la minería de datos, e-commerce, y otras áreas.
	}
  %%%%%%%%%%%%%%%%%%%%%%%%%%%%%%%%%%%
  \item Domenech \etal~\cite{Domenech2006}
  
	{  
  Muestran un estudio de los rendimientos de técnicas de recuperación de datos.
  Las mismas se pueden utilizar para dar una entrada ideal a algoritmos de aprendizaje o algoritmos de predicción. 
  Los conceptos más importantes son las nuevas variables de caracterización, temporalidad, espacio y geografía, que se le suman a la predicción. 
  Además de comenzar un trabajo más elaborado de como tomar una predicción, se introducen conceptos como predicciones genéricas o específicas, variables de uso de recursos a nivel de red ó nivel procesamiento.
  Finalmente, se presenta un modelo predictivo que puede ayudar a disminuir la latencia entre la petición del cliente y la respuesta de la web, dando así un mejor rendimiento y \emph{QoS}.
  
    % @TODO detallar más explicarlo mas simple, darle mas enfoque al usuario segúnn del punto de vista que de los docuentos 
    % como los autores antteriores.
 }


  %%%%%%%%%%%%%%%%%%%%%%%%%%%%%%%%%%%
  \item Chen \etal~\cite{Chen2011} 
  
	{  
	Dan una nueva perspectiva enfocada a entregar una clara recomendación a los usuarios basada en la misma propuesta de este proyecto, los access log.
	El primer análisis realizado por los autores cubre las reglas asociativas que requiere un sistema de recomendación, pero en las pruebas propiamente tales encuentran que el análisis de los patrones detectadados dan una representación clara de como optimizar la web, y finalmente mediante sus pruebas logran una recomendación de calidad.}
  %%%%%%%%%%%%%%%%%%%%%%%%%%%%%%%%%%%
  \item Rajimol y Raju~\cite{Rajimol2012} 
	{  
	Minaron los patrones de los accesos web, donde el enfoque es usar los registros de acceso para crear subsecuencias y realizar comparaciones.
	La literatura presenta un interés para poder anticipar el patrón de comportamiento de la web.
  % @TODO reflexionar mas sobre este paper


	}
  %%%%%%%%%%%%%%%%%%%%%%%%%%%%%%%%%%%
  \item Kewen~\cite{kewen2012} 
  
	{  
	Realizó un análisis más profundo del \emph{web usage minning}.
	Parte de la importancia de este trabajo, es que después de minar los registros de accesos, logran reducir la ``\emph{bad data}''.
	  %@TODO: Preguntar si este paper se escapa mucho del tema prinicipal, pero parece interesante  
	}

  %%%%%%%%%%%%%%%%%%%%%%%%%%%%%%%%%%%
  \item Poornalatha y Raghavendra~\cite{Poornalatha2012}
  
	{ 
		
	  Establecen que se pueden utilizar máquinas de aprendizaje para predecir basándose en distintas entre clusters. Estos autores, al igual que Domenech \etal~\cite{Domenech2006} y Dongshan y Junyi~\cite{Dongshan2002}, comparan el objetivo de optimizar los recursos tanto en redes (disminución de latencia) y experiencia de usuario.
	  }
  
  %%%%%%%%%%%%%%%%%%%%%%%%%%%%%%%%%%%
  \item Claude \etal~\cite{Claude2014} 
  
	{  
	presentan una estructura de representación eficiente que permite dar una representación de \emph{web access log} y ofrecen las operaciones básicas de WUM.
  }
\end{enumerate}

%%%%%%%Moghaddam_Kabir
%Various models have been proposed for modeling the user navigation behavior and predicting the next requests of users. According to [12], association rules, sequential pattern discovery, clustering, and classification are most popular methods for web usage mining. Collaborative filtering is another method for modeling users' behaviors. Association rules [6] were proposed to capture the co- occurrences of buying different items in a supermarket shopping. Association rules indicate groups that are related together. Methods that use association rules can be found in [6, 7]. Collaborative filtering techniques are often based on matching the current user's profile against similar data obtained by the system over time from other users. It tries to make useful recommendations users based on discovered cluster of similar categories. But for web sites where the number of web pages is quite large it can be quite inefficient.



 





\begin{enumerate}


  %%%%%%%%%%%%%%%%%%%%%%%%%%%%%%%%%%%
  \item \textbf{Dynamic and memory efficient web page prediction model using LZ78 and LZW algorithms }

	Moghaddam y Kabir~\cite{Moghaddam2009}  realizan una comparación de LZ78  y el algoritmo LZW, el cual es una derivación del anterior. La mayoría de las aplicaciones actuales que predicen el siguiente acceso a un página web posee una  componente offline que hace la tarea de preparar data y luego disponer una sección en linea que permite personalizar cierto contenido para un usuario en particular basado en las actividades de navegación
	
	En la mayoría de las técnicas de \emph{Web Usage Mining}, las secuencias se utilizan ya sea para producir las reglas de asociación o para producir estructuras de datos de tipo árbol o cadenas de Markov para representar patrones de navegación. Los  Modelos de Markov se basan en una teoría bien establecida y son fáciles de entender.  
	

	La propuesta es no crear un modelo predictivo por usuarios.
	Moghaddam y Kabir proponen modelar la navegación de usuarios mediante un Trie creado por un algoritmo de la familia LZ y usando muchas sesiones de usuarios, tener un modelo predic






  %%%%%%%%%%%%%%%%%%%%%%%%%%%%%%%%%%%
  \item \textbf{Prediction Algorithms for User Actions} ( Hartmann \& Schreiber,2007  \etal~\cite{hartmann2007}) , 

	{
		% Proactive User Interface
	Se requiere la predicción de la siguiente acción del usuario basada en la historia de interacción que ha tenido con una interfaz. 
	En su trabajo dan una revisión a los algoritmos de predicción Discreta (SPA) y desarrollan dos propuestas de algoritmos basadas en Modelos de Markov que convinan distintos orden de Markov. Y desarrollan una librería en PERL para su propuesta y evaluación.
	
	% NOTAS MIAS
	
	%Acorde a lo que Hartman propone nuestro trabajo esta totalmente alineado, debido a que optimizar o predecir el siguiente acción/acceso puede ayudar enormemente a el problema de PUI Proactive User Interfaces que el plantea
	
	% Ellos comparan comparan el mejor rendimiento del algoritmo acorde a los requerimientos de memoria y tiempo de procesamiento
	
	%Hartman en su paper dice que las Interfaces de Usuario Intelignente(Intelligent User Interfaces) tb dan uso y son un campo de aplicación para las predicciones secuenciales 
	
	% Algo significativo que menciona es que muchos de los algoritmos toman k- elementos de una secuencia para hacer una predicción, en mi caso yo tomo la secunecia completa en base a un entrenamiento
	
	% Plantea que hay dos maneras de hacer predicciones secuenciales On-Demand or Live, en mi caso es Live y OnDemand la implementación que ofrezco cumple ambos campos.
	% Online Learning Algorithm -> Muy similar a mi modelo hibirido ML y LDC
	% Mi Propuesta podría ayudar mucho en PUI para caso como el desarrollo movil,esto es obiviamente por las limitaciones de pantalla que presenta 
	
	% Modelos/Algortimos propuestos   ONISI y IPAM Jacobs Blockeel, ActiveLeZi
	}
  %%%%%%%%%%%%%%%%%%%%%%%%%%%%%%%%%%%
  \item \textbf{ActiveLezi} ( Gopalratnam \& Cook,2007 \etal\cite{Gopalratnam2007}) 
  {  
  Proponen un Algoritmo \emph{On-Demand} que considera varios modelos de Markov.  
  El funcionamiento es basado en almacenar la frecuencia del patrón de \emph{input} en un \emph{Trie} acorde al algoritmo de compresión de \emph{LZ78} para superar algunos de los problemas que surgen con \emph{LZ78}, se usa una ventana de largo variable de los simbolos previamente usados en la construcción del \emph{Trie}. El tamaño de la ventana crece con el numero de las diferentes subsecuencias que se van viendo en la entrada de cada secuencia nueva que ingresa.  Sea $suff_{l}$  el sufijo de largo $l+1$ el sufijo de longitud l 1 de las inmediatamente historial de interacción a, que es un hacha .... la probabilidad se define de la siguiente forma recursiva.:
 }



  %%%%%%%%%%%%%%%%%%%%%%%%%%%%%%%%%%%
  \item Dongshan y Junyi~\cite{Dongshan2002} 
  {
	  Destacan que un modelo de Markov puede ayudar a predecir el comportamiento de un usuario, pero con ciertas limitaciones .  Para solucionarlo presentan un nuevo modelo de Markov basado en una representación de \emph{Tree Order Model}, el cual es un híbrido entre un modelo de markov tradicional y una representación de árbol, bautizada como HTMM (por sus siglas en inglés, \emph{Hybrid-Order Tree Markov Model}).
	  Su modelo fue presentado en 2002, y da una importancia a conocer la predicción de los \emph{web access}, dada la importancia de creación de redes, la minería de datos, e-commerce, y otras áreas.
	}
  %%%%%%%%%%%%%%%%%%%%%%%%%%%%%%%%%%%
  \item Domenech \etal~\cite{Domenech2006}
  
	{  
  Muestran un estudio de los rendimientos de técnicas de recuperación de datos.
  Las mismas se pueden utilizar para dar una entrada ideal a algoritmos de aprendizaje o algoritmos de predicción. 
  Los conceptos más importantes son las nuevas variables de caracterización, temporalidad, espacio y geografía, que se le suman a la predicción. 
  Además de comenzar un trabajo más elaborado de como tomar una predicción, se introducen conceptos como predicciones genéricas o específicas, variables de uso de recursos a nivel de red ó nivel procesamiento.
  Finalmente, se presenta un modelo predictivo que puede ayudar a disminuir la latencia entre la petición del cliente y la respuesta de la web, dando así un mejor rendimiento y \emph{QoS}.
  
    % @TODO detallar más explicarlo mas simple, darle mas enfoque al usuario segúnn del punto de vista que de los docuentos 
    % como los autores antteriores.
 }


  %%%%%%%%%%%%%%%%%%%%%%%%%%%%%%%%%%%
  \item Chen \etal~\cite{Chen2011} 
  
	{  
	Dan una nueva perspectiva enfocada a entregar una clara recomendación a los usuarios basada en la misma propuesta de este proyecto, los access log.
	El primer análisis realizado por los autores cubre las reglas asociativas que requiere un sistema de recomendación, pero en las pruebas propiamente tales encuentran que el análisis de los patrones detectadados dan una representación clara de como optimizar la web, y finalmente mediante sus pruebas logran una recomendación de calidad.}
  %%%%%%%%%%%%%%%%%%%%%%%%%%%%%%%%%%%
  \item Rajimol y Raju~\cite{Rajimol2012} 
	{  
	Minaron los patrones de los accesos web, donde el enfoque es usar los registros de acceso para crear subsecuencias y realizar comparaciones.
	La literatura presenta un interés para poder anticipar el patrón de comportamiento de la web.
  % @TODO reflexionar mas sobre este paper


	}
  %%%%%%%%%%%%%%%%%%%%%%%%%%%%%%%%%%%
  \item Kewen~\cite{kewen2012} 
  
	{  
	Realizó un análisis más profundo del \emph{web usage minning}.
	Parte de la importancia de este trabajo, es que después de minar los registros de accesos, logran reducir la ``\emph{bad data}''.
	  %@TODO: Preguntar si este paper se escapa mucho del tema prinicipal, pero parece interesante  
	}

  %%%%%%%%%%%%%%%%%%%%%%%%%%%%%%%%%%%
  \item Poornalatha y Raghavendra~\cite{Poornalatha2012}
  
	{ 
		
	  Establecen que se pueden utilizar máquinas de aprendizaje para predecir basándose en distintas entre clusters. Estos autores, al igual que Domenech \etal~\cite{Domenech2006} y Dongshan y Junyi~\cite{Dongshan2002}, comparan el objetivo de optimizar los recursos tanto en redes (disminución de latencia) y experiencia de usuario.
	  }
  
  %%%%%%%%%%%%%%%%%%%%%%%%%%%%%%%%%%%
  \item Claude \etal~\cite{Claude2014} 
  
	{  
	presentan una estructura de representación eficiente que permite dar una representación de \emph{web access log} y ofrecen las operaciones básicas de WUM.
  }
\end{enumerate}

%%%%%%%Moghaddam_Kabir
%Various models have been proposed for modeling the user navigation behavior and predicting the next requests of users. According to [12], association rules, sequential pattern discovery, clustering, and classification are most popular methods for web usage mining. Collaborative filtering is another method for modeling users' behaviors. Association rules [6] were proposed to capture the co- occurrences of buying different items in a supermarket shopping. Association rules indicate groups that are related together. Methods that use association rules can be found in [6, 7]. Collaborative filtering techniques are often based on matching the current user's profile against similar data obtained by the system over time from other users. It tries to make useful recommendations users based on discovered cluster of similar categories. But for web sites where the number of web pages is quite large it can be quite inefficient.










% Se consideran y explican detalladamente 
% todos los conceptos, teorıas, y aspectos pertinentes al tema tratado.
% El marco teorico concuerda con los objetivos y el tipo de trabajo.

\section{Descripción del contenido} 

Este trabajo esta organizado de la siguiente manera. En el Capítulo \ref{cap:ch1-intro} describiremos el contexto preliminar que rodea este trabajo y definiremos el problema de la predicción de secuencias discretas para \emph{webaccess log}. Veremos como poder dar una inducción a un entorno de trabajo llamado \emph{PredictionIO} el cual nos ayudará a entregar los algoritmos como servicios consumibles, por cualquier aplicación cliente que pueda comunicarse con un servidor web.
En el Capítulo \ref{ch:Conceptos-Basicos} explicaremos todos los conceptos básicos para el entendimiento sobre esta investigación, como también conceptos para el uso de \emph{PredictionIO} y cerraremos con todos los trabajos relacionados más recientes que involucran nuestra investigación. En el Capítulo \ref{ch:predicciones-webaccess} veremos las predicciones sobre \emph{webaccess log}, los modelos propuestos por varios investigadores y sus limitaciones, daremos una revisión del trabajo realizado por \emph{Rissanen}\cite{Rissanen1984} que da el inicio a esta área de Investigación.
En el Capítulo \ref{ch:Compresion-Machine-Learning} veremos los temas de \emph{Machine Learning} y \emph{Lossless Compression Data} y como pretendemos crear un modelo predictivo con recursos de ambas áreas. Para cerrar este capitulo explicaremos el uso de del algoritmo Lempel \& Ziv, el uso para secuencias discretas su convergencia a un modelo de predicción eficiente y exacto. 

Finalmente el Capítulo \ref{ch:experimetal-all}, se presentará nuestros experimentos realizados sobre la implementación de un algoritmo de compresión en un servidor de \emph{Machine Learning}: \emph{PredictionIO}, analizaremos el comportamiento del algoritmo y como se desempeña este en distintos ambientes, propondremos discusiones de como mejorar nuestra implementación y los trabajos a futuro que puede presentar esta investigación.

Se deja como anexo una guía básica de uso para \emph{PredictionIO}, todos los datos y nuestra implementación se puede encontrar en nuestro repositorio \textbf{git público}\footnote{\url{https://github.com/jaimeguzman/PredictionIO-LZmodel}}, en se encontrará  los datos para replicar las experimentos que hemos realizado. 







% TODO: REALMENTE NO SE SI VALE LA PENA ESTO
\section{Algoritmos como servicio web }

	Los avances en el desarrollo de nuevas tecnologías que brinden mejores experiencias en su uso día a día, deriva en cómo podamos llevar varios escenarios idealizados a implementaciones empresariales reales. Es bastante común encontrar librerías que son bastante útiles para hacer Minería de Datos, agrupación y muchas operaciones que pueden recurrir en cálculos muy complejos, pero no se pueden ofrecer como servicio. Ya en pleno auge de las infraestructuras en la nube, la capacidad de cómputo que se puede alcanzar no es un problema como antes lo era para un Científico de Datos.


	Una \emph{API} es un interfaz de programación de Aplicaciones que nos permiten intermediar el \emph{Servicio $A$} con el \emph{Servicio $B$}. Respectivamente $A$ puede ser el proveedor y $B$ el demandante del servicio. Si quisiéramos analizar datos que se encuentran dentro de un servidor específico, estos se podrían consumir por esta interfaz. Existen variados clientes que nos permiten ayudar en esta comunicación, incluso se pueden utilizar por una terminal de {Unix} que es posible dialogar mediante el programa \emph{curl}.
	
	Ya se dispone de infraestructura como servicio (\emph{IaaS}) , software como servicio (\emph{SaaS}), plataformas como servicios (\emph{PaaS}). Dado lo anterior ofrecer estos algoritmos para hacer que las soluciones de desarrollo den valor agregado a la experiencia requerida por el usuario final. Por esto hemos decidido utilizar una librería y {framework} que nos de esta posibilidad. Ofrecer algoritmos a la industria como un servicio que ayuda de manera eficiente e Inteligente desarrollado como una API REST, que permitirá una fácil integración. 
	
	Todas las ventajas de este patrón son heredados de las características que ofrece una API, interoperabilidad, evitar problemas de Infraestructura, Resiliencia de Datos, Persistencia de Datos, Análisis y Procesamiento sin afectar un curso operacional de una aplicación. Un ejemplo claro de esto es el análisis de datos en sistemas legados los cuales en plan de mejoras, no poseen la compatibilidad para poder realizarlo. Por otro lado, los algoritmos de compresión o algoritmos de \emph{Machine Learning} tienden a ser muy complejos de implementar o ocupan muchso recursos y este hecho pueden ser la razón para no implementarlos. 
	
	
	
	  

