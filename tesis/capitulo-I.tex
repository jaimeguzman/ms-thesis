% 
% 
% Capitulo I
% 
% 
\chapter[Introducción]{Introducción}\label{ch1:intro}

%%%%%%%%%%
% Uno de los interes en poder crear un un aplicación de ambas areas es la convergenia de las areas en la cual un proceso de aprendizaje ó predicción secuencial se puede usar con ML y Compress data

% @TODO: SEGUIR TRABAJANDO EN ESTA BREVE INTRODUCCION
%En este tema convergen tres áreas, por un lado existe trabajo para crear estructuras eficientes para predicciones basadas en algoritmos de compresión, como es en el caso de~\cite{Claude2014}, y, por otro lado, el uso de algoritmos de aprendizaje para realizar clustering y predecir el comportamiento basado en el mismo contenido o en la distancia del contenido que visita el usuario actual al contenido clusterizado, como es el caso de ~\cite{Poornalatha2012}, inclusive se han utilizado modelos de Markov en ~\cite{Dongshan2002}  para poder modelar el comportamiento de la web.
%La tercera área son los Sistemas de Recomendación, la cual en este proyecto no se tocará pero si se mencionará el enfoque práctico que presenta área como un foco de múltiples implementaciones. 

%
% Introduccion 
%
%
Internet es una fuente de información que crece constantemente y es visitada con una gran frecuencia, posee billones de \webs públicas. Dado este rápido desarrollo y accesos a grandes colecciones de datos se vuelve fundamental y desafiante lograr anticipar las acciones que un usuario puede realizar, tanto para investigadores e ingenieros. Los beneficios de lograr predicciones con la suficiente precisión pueden  mejorar la experiencia de usuarios en base a su comportamiento ó mejorar la experiencia en comercios electrónicos, por ejemplo como un escenario de estudio.

Cuando se navega en un sitio \web, se puede recolecta una gran cantidad de registros guardados desde el lado del servidor, los cuales por ejemplo son datos de accesos de la sesión realizada. También estos registros representan usuarios de una \web, que se encuentran activos (\online), llamaremos en adelante a estos registros de acceso, \webasccesslog. En estos se pueden realizar variados análisis, un escenario ejemplo es: dado un usuario, el cual intenta comprar un cierto producto en una \web de comercio electrónico. Este sigue un cierto comportamiento hasta el momento de generar una comprar, supongamos que el usuario se encuentra en un estado de búsqueda o visita todas las posibles elecciones, puede volver inclusive a las que ya ha visitado, también abandonar la actual página de cierto productos u otro, hasta encontrar lo que desea y finalmente comprar. En este caso se vuelve interesante como poder predecir la transición de una página a otra. 

Existen varias área que buscan un estudio de este problema, como también al ejemplo que hemos mencionado, el de las predicciones de páginas \web, también conocido como \texttt{WPP}. Algunos de estos enfoques para  abordarlo, por ejemplo son:  Análisis Predictivos de Datos, \emph{Web Usage Minning}, \emph{Web Access Pattern}, \machinelearning y predicciones secuenciales con algoritmos de compresión. Esta tesis usará estas dos ultimas áreas para la implementación de un modelo predictivo simple, en un servidor de \machinelearning. La  propuesta se basa en la implementación del algoritmo \lzSieteOcho que está adaptado para modelar y representar la navegación secuencial del usuario. Nuestra propuesta disminuye el tiempo de actualización del modelo predictivo, con actualizaciones a medida que se va usando. Además la puesta en marcha del servicio, que es un desventaja en el desarrollo de sistemas predictivos \online.


Predecir no es trivial, existen modelos que se han implementado que no logran dar con un patrón para la navegación de usuario, de manera genérica. Dado la diversidad de perfiles de usuarios que se pueden encontrar en ciertas \webs  y distintos flujos de navegación de contenido de las mismas.  

En este trabajo se presenta una implementación de un modelo de predicción \online, que poseen un entrenamiento inicial \offline si es deseado y que se puede consumir como una \API de servicios \REST, la cual permite una integración ha variadas plataformas y sistemas que no tenga un componente \online y con una buena precisión de predicción. 

% conector:
En las próximas secciones pretendemos que el lector se contextualice más en el escenario que deseamos realizar predicciones, trabajos relacionados que presentan otros enfoques y los fundamentos que soportan esta tesis al trabajar con \webasccesslog.





% DEFINIR EL PROBLEMA

\section{Definición del problema}
% IDEA: me gustaria que esta sección  no se llamara DEFINICION DE PROBLEMA, intrinsicamente he hablado que el problema es como mejorar las predicciones secienciales usando lo mejor de dos áreas.



% esto creo que se repite en contexto o intro.
En esta tesis el problema que se busca solucionar es la predicción a páginas web, que se representan mediante símbolos que pertenecen a una determinada secuencia de accesos, la cual representa una sesión. Este problemática esta presente hace años y diversos autores han trabajado con distintos enfoques. 

% TODO: Este parrafo es desastroso
%\emph{Rissanen}\cite{Rissanen1983} y Langdom\cite{Langdon1983} en los laboratorios \emph{Bell}, al realizar pruebas y experimentar con un \emph{robot} que tiraba una moneda compitiendo con una persona, aquel robot realizaba todos los cálculos {markovianos} y las probabilidades condicionales para que cierto evento ocurra, a diferencia del sujeto que sólo estaba esperando un resultado aleatorio, la diferencia se marco en los costos de tiempo y de computo que se realizaron, por un parte la demora del \emph{robot} haciendo sus cálculos no mejoraba a la probabilidad aleatoria con que la persona a que enfrentaba. 


La solución a este problema que pretendemos abordar, es mediante modelos predictivos. Como el lector puede imaginar predecir no es trivial y requiere de una basta cantidad de  información para poder analizar y desarrollar un modelo, que logre idealmente abarcar varios escenarios reales en que un usuario se enfrenta a una \web. Sí podemos llegar a acercarnos y minimizar el error estaremos cerca a predicciones aceptables. Sin embargo, dos áreas han tratado de resolver el problema; \machinelearning y \datacompression. Este último presenta condiciones desfavorables para los algoritmos predictivos que se pueden implementar, es decir, funcionan totalmente desconectados de la fuente de datos de entrada, esto implica que la validez del modelo solo es factible cuando esta realizando predicciones sin usuarios concurrentes o como un modelo desconectado de la fuente, lo que inhabilita rápidamente al modelo y en general no logrará un resultado en demanda. Por otro lado, en el acercamiento que ofrece \machinelearning debemos crear un modelo, tal que  deba entrenar y luego generar una función predictiva a lo cual se le suma un gran cantidad de datos. El proceso anterior puede producir un modelo bastante pesado para poder funcionar como un modelo predictivo \online. 

Para acotar nuestro problema, buscaremos resolver predicciones secuenciales discretas con un modelo generado por un algoritmo de compresión, aplicado a un conjunto de datos generados sintética-mente y otros datos real provistos por \emph{MSNBC}\cite{Claude2014}, un sitio \web de noticias. 

% ESTE TEXTO ESTA HORRIBLE:
% Este modelo propuesto de componentes híbridas, es decir, algoritmos de compresión  y un servidor de \machinelearning, buscaremos la solución a nuestro problema planteado, disponiendo como servicio, es decir crear, manteniendo la  componente \online del modelo predictivo vigente con ayuda de \losslessdatacompression y el famoso algoritmo \texttt{LZ78}; dando una predictibilidad inmediata que hoy en la industria es necesaria para usar los datos recolectados y entregar nuevos enfoques a las decisiones basadas en esta perspectiva predictiva, como también dando un avance en el análisis de datos predictivos. Usando un algoritmo de tipo \LDC reemplazando en el proceso de aprendizaje de una arquitectura de  servicios  \machinelearning podemos alcanzar a una buena solución o a lo menos estar sobre el promedio aleatorio de ocurrencia de eventos.





\section{Contribución de esta Tesis}


En este trabajo,  nuestra principal contribución es el desarrollo del primer servidor de predicción de secuencias discretas con funcionalidad online y \emph{offline} para secuencias de  sitios web. Adicionalmente nuestro modelo en conjunto \emph{PredictionIO}, es el primer servidor de Machine Learning usando un modelo de predicción con un algoritmo de \emph{Lossless Data Compression}. Hay varios temas propuestos en el que este trabajo pueda tener iteraciones  futuras de estas sistema. 

% Se consideran y explican detalladamente 
% todos los conceptos, teorıas, y aspectos pertinentes al tema tratado.
% El marco teorico concuerda con los objetivos y el tipo de trabajo.
%
% ch2 ~\ref{ch:Compresion-Machine-Learning}
% ch3 ~\ref{ch:predicciones-webaccess}
% ch4 ~\ref{ch:experimetal-all}
% anexo ~\ref{ch:anexos}
%
\uncm
\section{Estructura de la tesis} 




Para una mejor lectura y compresión de esta tesis, este trabajo se ha compuestos de cinco capítulos, incluyendo este introductorio, más un anexo técnico de las herramientas usadas. Los capítulos siguiente abordarán los siguientes temas:

 % En el Capítulo \ref{cap:ch1-intro} hemos realizado una introducción y  explicado un contexto preliminar que rodea este trabajo y hemos definido el problema que buscamos abordar, como las posibles aproximaciones de soluciones, también hemos dado un interés en el uso de algoritmo que puedan ser consumidos como un servicio \emph{web}.
 
En el \textbf{Capítulo~\ref{ch:preliminar}}, se le recordará al lector todas los conceptos que se usarán en esta tesis, adicionalmente los trabajos relacionados y distintos puntos de vistas que se han tornado sobre al problema de las predicciones de páginas \web.


En el \textbf{Capítulo~\ref{ch:Compresion-Machine-Learning}}, presentaremos los temas de \machinelearning y \losslessdatacompression, para llevar acabo un estudio de predicciones sobre \webasccesslog, para poder predecir la siguiente página web que un usuario accede.


% explicaremos todos los conceptos básicos para el entendimiento sobre esta investigación, como también conceptos para el uso de \emph{PredictionIO} y cerraremos con todos los trabajos relacionados más recientes que involucran nuestra investigación. Adicionalmente veremos como poder dar una inducción a un entorno de trabajo llamado \emph{PredictionIO},el cual nos ayudará a entregar los algoritmos como servicios consumibles, por cualquier aplicación cliente que pueda comunicarse con un servidor \emph{web}. 
% veremos las predicciones sobre \emph{webaccess log}, los modelos propuestos por varios investigadores y sus limitaciones, daremos una revisión del trabajo realizado por \emph{Rissanen}\cite{Rissanen1984} que da el inicio a esta área de Investigación.
% En el Capítulo \ref{ch:Compresion-Machine-Learning} veremos los temas de \emph{Machine Learning} y \emph{Lossless Compression Data} y como pretendemos crear un modelo predictivo con recursos de ambas áreas. Para cerrar este capitulo explicaremos el uso de del algoritmo Lempel \& Ziv, el uso para secuencias discretas su convergencia a un modelo de predicción eficiente y exacto. 

En el \textbf{Capítulo~\ref{ch:predicciones-webaccess}}, se estudia ciertos fundamentos para comprender el tópico de los análisis predictivos y se describe como crear un modelo predictivo de navegación de usuarios, usando un algoritmo de compresión.





En el \textbf{Capítulo~\ref{ch:experimetal-all}}, finalmente se presentará nuestros experimentos realizados con nuestra implementación. Posteriormente, veremos nuestras conclusiones y discusiones, como también nuestro trabajos futuros.





Además se adjunta un anexo, que corresponde a una guía básica para él uso de \predictionio, todos los datos y nuestra implementación se puede encontrar en nuestro repositorio público\footnote{\footnotePublicRepo}, en este se encontrarán los datos para replicar los experimentos que hemos realizado. 


% sobre la implementación de un algoritmo de compresión en un servidor de \emph{Machine Learning}: \emph{PredictionIO}, analizaremos el comportamiento del algoritmo y como se desempeña este en distintos ambientes. propondremos discusiones de como mejorar nuestra implementación y los trabajos a futuro que puede presentar esta investigación.

% Explicaremos nuestras conclusiones obtenidas en base al desarrollo e implementación de la propuesta.
% Ademas dejaremos planteado nuestros trabajos futuros.


% TODO: REALMENTE NO SE SI VALE LA PENA ESTO
\section{Algoritmos como servicio web }

	Los avances en el desarrollo de nuevas tecnologías que brinden mejores experiencias en su uso, deriva en cómo podamos llevar varios escenarios idealizados a implementaciones empresariales reales. Es bastante común encontrar librerías que son útiles para hacer Minería de Datos, agrupación y muchas operaciones que pueden recurrir en cálculos muy complejos, pero no se pueden ofrecer como servicio y en la industria es difícil implementarlos. Ya en pleno auge de las infraestructuras en la nube, la capacidad de cómputo que se puede alcanzar no es un problema como antes lo era para un Científico de Datos.


	Una \texttt{API} es un interfaz de programación de Aplicaciones que nos permiten intermediar el \emph{Servicio $A$} con el \emph{Servicio $B$}. Respectivamente $A$ puede ser el proveedor y $B$ el demandante del servicio. Si quisiéramos analizar datos que se encuentran dentro de un servidor específico, estos se podrían consumir por esta interfaz. Existen variados clientes que nos permiten ayudar en esta comunicación, incluso se pueden utilizar por una terminal de {Unix} que es posible dialogar mediante el programa \emph{curl}.
	
	Ya se dispone de infraestructura como servicio (\emph{IaaS}) , software como servicio (\emph{SaaS}), plataformas como servicios (\emph{PaaS}). Esta lógica de llevar todo a un nivel más abstracto de computación nos permite ofrecer  algoritmos que sean consumibles para hacer soluciones de desarrollo que den valor agregado a la experiencia requerida por el usuario final. Por esto hemos decidido utilizar una librería y ambiente de desarrollo que nos entregue esta posibilidad. Algoritmos como servicios que ayuden de manera eficiente e Inteligente ha mejorar la \emph{web} y con una fácil integración. 
	
	Todas las ventajas de este patrón son heredados de las características que ofrece una \texttt{API}, evita problemas de Infraestructura, Resiliencia de Datos, Persistencia de Datos, Análisis y Procesamiento sin afectar un curso operacional de una aplicación. Un ejemplo claro de esto es el análisis de datos en sistemas legados, los cuales en plan de mejoras no poseen la compatibilidad para poder implementarse. Por otro lado, los algoritmos de compresión o  \emph{Machine Learning} tienden a ser muy complejos de implementar, ocupan muchos recursos o la infraestructura disponible no da abasto, el personal que puede implementarlo es costoso. Lo anterior son algunas, pero no todas las razones para no  implementarlos.   

