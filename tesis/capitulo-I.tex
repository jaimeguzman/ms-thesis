% 
% 
% Capitulo I
% 
% 
\chapter[Introducción]{Introducción}\label{ch1:intro}

%%%%%%%%%%
% Uno de los interes en poder crear un un aplicación de ambas areas es la convergenia de las areas en la cual un proceso de aprendizaje ó predicción secuencial se puede usar con ML y Compress data

% @TODO: SEGUIR TRABAJANDO EN ESTA BREVE INTRODUCCION
%En este tema convergen tres áreas, por un lado existe trabajo para crear estructuras eficientes para predicciones basadas en algoritmos de compresión, como es en el caso de~\cite{Claude2014}, y, por otro lado, el uso de algoritmos de aprendizaje para realizar clustering y predecir el comportamiento basado en el mismo contenido o en la distancia del contenido que visita el usuario actual al contenido clusterizado, como es el caso de ~\cite{Poornalatha2012}, inclusive se han utilizado modelos de Markov en ~\cite{Dongshan2002}  para poder modelar el comportamiento de la web.
%La tercera área son los Sistemas de Recomendación, la cual en este proyecto no se tocará pero si se mencionará el enfoque práctico que presenta área como un foco de múltiples implementaciones. 

%
% Introduccion 
%
%
Internet es una fuente de información que crece constantemente y es visitada con una gran frecuencia, posee billones de \webs públicas. Dado este rápido desarrollo y accesos a grandes colecciones de datos se vuelve fundamental y desafiante lograr anticipar las acciones que un usuario puede realizar, tanto para investigadores e ingenieros. Los beneficios de lograr predicciones con la suficiente precisión pueden  mejorar la experiencia de usuarios en base a su comportamiento ó mejorar la experiencia en comercios electrónicos, por ejemplo como un escenario de estudio.

Cuando se navega en un sitio \web, se puede recolecta una gran cantidad de registros guardados desde el lado del servidor, los cuales por ejemplo son datos de accesos de la sesión realizada. También estos registros representan usuarios de una \web, que se encuentran activos (\online), llamaremos en adelante a estos registros de acceso, \webasccesslog. En estos se pueden realizar variados análisis, un escenario ejemplo es: dado un usuario, el cual intenta comprar un cierto producto en una \web de comercio electrónico. Este sigue un cierto comportamiento hasta el momento de generar una comprar, supongamos que el usuario se encuentra en un estado de búsqueda o visita todas las posibles elecciones, puede volver inclusive a las que ya ha visitado, también abandonar la actual página de cierto productos u otro, hasta encontrar lo que desea y finalmente comprar. En este caso se vuelve interesante como poder predecir la transición de una página a otra. 

Existen varias área que buscan un estudio de este problema, como también al ejemplo que hemos mencionado, el de las predicciones de páginas \web, también conocido como \texttt{WPP}. Algunos de estos enfoques para  abordarlo, por ejemplo son:  Análisis Predictivos de Datos, \emph{Web Usage Minning}, \emph{Web Access Pattern}, \machinelearning y predicciones secuenciales con algoritmos de compresión. Esta tesis usará estas dos ultimas áreas para la implementación de un modelo predictivo simple, en un servidor de \machinelearning. La  propuesta se basa en la implementación del algoritmo \lzSieteOcho que está adaptado para modelar y representar la navegación secuencial del usuario. Nuestra propuesta disminuye el tiempo de actualización del modelo predictivo, con actualizaciones a medida que se va usando. Además la puesta en marcha del servicio, que es un desventaja en el desarrollo de sistemas predictivos \online.


Predecir no es trivial, existen modelos que se han implementado que no logran dar con un patrón para la navegación de usuario, de manera genérica. Dado la diversidad de perfiles de usuarios que se pueden encontrar en ciertas \webs  y distintos flujos de navegación de contenido de las mismas.  

En este trabajo se presenta una implementación de un modelo de predicción \online, que poseen un entrenamiento inicial \offline si es deseado y que se puede consumir como una \API de servicios \REST, la cual permite una integración ha variadas plataformas y sistemas que no tenga un componente \online y con una buena precisión de predicción. 

% conector:
En las próximas secciones pretendemos que el lector se contextualice más en el escenario que deseamos realizar predicciones, trabajos relacionados que presentan otros enfoques y los fundamentos que soportan esta tesis al trabajar con \webasccesslog.





\section{Contexto preliminar}\label{sec:preliminar}

Los modelos de predicción secuenciales son uno de los temas de interés tanto para el área de \machinelearning  y \datacompression. La búsqueda de un algoritmo que pueda entregar un análisis  y resultado predictivo con la mejor probabilidad de acierto o con el menor posible.
En los escenarios que involucran a \emph{webs}, la cantidad de variables y problemas a estudiar, es proporcional a la complejidad de la misma. 
Una característica común, es como los usuarios navegan su contenido en una sesión o acceso de un usuario. Podemos formular un modelo que refleje el comportamiento de accesos y realizar predicciones secuenciales con este. Tomando como punto inicial la llegada del usuario a la web, tendremos una cantidad finita de vistas de una sección a otra, por ejemplo. Si esta web que se describe fuera una comercio electrónico análogamente podríamos visualizar que se podría obtener predicciones de acceso a ciertos cierto catalogo de producto y podríamos tomar una decisión de negocio que logre concretar una venta. 

%  Como explique en ese parrafo esta muy ligado a lo que es recomendacion 
% ¿Cual es la necesidad?
Tener información antes que suceda o tener una aproximación cercana, es una de los principales desafío que busca solucionar en el área  de \machinelearning dando en el caso de las \emph{webs} una inteligencia para entregar una mejor experiencia cuando el usuario interactúa. Un ejemplo frecuente son los motores de búsqueda como \emph{Google}, \emph{Yahoo} y \emph{Bing}, los cuales en cada interacción de búsqueda muestran el tiempo demorado. Hoy en día se intentan anticipar el contexto de la búsqueda con la menor cantidad ed palabras claves, con el menor esfuerzo posible del usuario ó inclusive recomendar exactamente mientras se esta escribiendo. Esto nos da un estrecha relación con los sistemas de recomendación, en el cual no es de nuestro interés. 

Es común que en \machinelearning se realicen varios entrenamientos a un cierto algoritmo, que de como resultado un buen aprendizaje del dominio, también es requerido la mayor cantidad de datos posibles. Lo anterior da una complejidad de recursos en los cuales el algoritmo debe ser muy eficiente o exacto. En muchos casos es la segunda propiedad la predominante. 


% Los motores de busqueda es un claro ejemplo del gran manejo de volumenes de datos en el cual estos algoritmos de prediccion, si el objetivo fuera búsqueda, no se pueden acotar.

Podemos deducir de lo anterior que no es suficiente aumentar los recursos, para obtener  mejor rendimiento para un cierto algoritmo, tampoco será óptimo o económicamente viable. Como ejemplo tenemos las infraestructuras de telecomunicaciones, el ancho de banda de \emph{Internet} en una relación con el volumen de datos generado es inversamente proporcional al crecimiento en un mismo periodo de tiempo. La industria con ofertas de computación en la nube (\emph{cloud computing}), ejemplo \emph{Amazon AWS, Google Cloud, Azure Microsoft, Oracle Cloud}, han buscado disminuir esta brecha técnica y de recursos físicos. Cada vez mas muchas empresas cuentan con sistemas de información en estas nuevas arquitecturas en la nube. Adicionalmente, las tecnologías para la creación de \www dinámica y asíncrona, entre el cliente y el servidor evoluciona a favor de dar la carga al cliente, también los lenguajes de programación y \emph{framework} que disminuyen considerablemente la carga de peticiones al servidor que se realizan.

Sobre escenarios que se posee un gran volumen de datos, es fundamental usarlos para tener análisis predictivos más exactos y tomar decisiones o acciones. Estas decisiones  pueden ser determinadas bajo un algoritmo probabilista o de frecuencia, es posible ser consultado  mientas el usuario esta navegando en una web. 

El contenido de \emph{Internet} tiene un crecimiento exponencial y muchos usuarios son atraídos a visitarlos.  En gran medida para satisfacer ciertas necesidades comunicación, información, ocio y entretenimiento, redes sociales,  publicidad o  comprar de bienes y servicios. Los escenarios anteriores deben estar sujetos a ciertas restricciones que para el usuario debe estar en el menor tiempo posible, dado el poco tiempo de atención que se dispone o la gran cantidad de tareas que se realizan. Este constante y complicado desafío genera una gran consumo de recursos de infraestructura informáticas y sistemas de telecomunicaciones. Además  requiere de \emph{webs} con ciertos algoritmos que aprendan silenciosamente desde información técnica que es almacenada en cada visita, con el fin de entregar una gran experiencia al usuario.
 
% %%% ejemplos de mas ideas dnd ser usado
Existen varios escenarios de estudios en que las predicciones sobre  volúmenes de datos, también pueden ser usadas para la {detección de fraudes financieros, predicción de valores bursátiles y también para hacer diagnostico en áreas de la salud, basado un conjuntos de datos genéticos o problemas de salud hereditarios. Los ejemplos anteriormente mencionados tienen una variable  común y es que toda predicción ayuda a tomar una decisión.}\label{ejemplos-casos-contextopreliminar}

%
%
%
% hacer link al concepto de trie
%
%  REFERENCIAS
\losslessdatacompression, es un área que permite tener representaciones de la información mas compactas sin perder información. Dado el caso que se detallara en la sección de compresión, existe un punto acorde a la literatura, en que un compresor y debido a la entropía de los datos se puede convertir en un predictor, que a su vez es una representación secuencial de datos comprimidas en un \emph{trie}, maneja datos discretas al igual que algoritmos de \machinelearning. En la literatura encontramos casos en que existen aproximaciones a usar estas áreas, para converger en una propuesta llamada \emph{modelos de markov variables}, los cuales veremos como se relacionan con el algoritmo de compresión \emph{LZ78}.

%
% JUNTAR MAS IDEAS  ML con LDC
%

% IDEA de que porque es comportamiento
Variados escenarios de estudio como  detección de  patrones para cierto dominio de problemas y aprenderlos rápidamente con un cierto algoritmo,, posteriormente se puede proveer resultados para ser analizados o procesados pueden ser desarrollados gracias a \machinelearning, mediante un modelo predictivo.  La navegación de un usuario  en una web, es nuestro contexto de estudio y en adelante diremos que la  navegación de usuarios usuarios es su patrón de navegación o comportamiento, registrado en  \webasccesslog. Estos se pueden analizar, estudiar y modelar con algoritmos que tengan enfoques predictivos. Se pueden usar de manera procesada o pre-procesada. Acorde a los ejemplos que hemos mencionado en esta sección~(\ref{ejemplos-casos-contextopreliminar})  y nuestro interés para predecir el comportamiento de navegación de usuarios en determinadas \emph{webs}.

%% Estos 2 parrafos aun no logros relacionarlos


 Es fundamental estudiar los datos que se buscan modelar para predecir y encontrar patrones frecuentes que  se encuentran para poder crear un modelo predictivo, intentando de evitar un sobre-entrenamiento del modelo. Este trabajo usará los avances en \emph{Web Usage Minning} y \emph{Web Access Pattern} para buscar una implementación de un modelos predictivos con una característica \emph{online} e híbrido. Ciertamente \machinelearning usa \emph{features}\footnote{\emph{feature:} Las característica de un \machinelearning son propiedades individual y cuantitativas de un fenómeno que se observa dentro de un conjunto de datos.} y entre más información de entrenamiento se obtendrá  mejores predicciones. Por ejemplo la sesión de un usuario podría modelarse con gran exactitud en el mundo \machinelearning, ya que nos entrega una gran cantidad de datos para hacer análisis predictivos. Estas sesiones comienza cuando se establece la conexión a una \www. Estableciendo dicha conexión, se crea instantáneamente una sesión de navegación automáticamente, que se almacena en el lado del cliente o que hospeda la \www. Un ejemplo de  \webasccesslog es lo que se observa en la Figura \ref{fig:accesslog-apache-teleton}.

\begin{figure}[tb]
	\centering
	\begin{lstlisting}[frame=single,basicstyle=\ttfamily\tiny,]
	172.31.33.116 - - [26/Nov/2015:00:12:12 +0000] "HTTP/1.1" 200 1784 "http://localhost/home" 
	"Mozilla/5.0 (Linux; Android 5.1.1; SAMSUNG SM-G920I Build/LMY47X) 
	SamsungBrowser/3.2 Chrome/38.0.2125.102 Mobile Safari/537.36"
	172.31.33.116 - - [26/Nov/2015:00:12:12 +0000] "HTTP/1.1" 200 179333 "http://localhost/news" 
	"Mozilla/5.0 (Linux; Android 5.1.1; SAMSUNG SM-G920I Build/LMY47X) 
	SamsungBrowser/3.2 Chrome/38.0.2125.102 Mobile Safari/537.36"
	172.31.33.116 - - [26/Nov/2015:00:12:12 +0000] "HTTP/1.1" 200 24660 "http://localhost/health" 
	"Mozilla/5.0 (Linux; Android 5.1.1; SAMSUNG SM-G920I Build/LMY47X) 
	SamsungBrowser/3.2 Chrome/38.0.2125.102 Mobile Safari/537.36"
	172.31.33.116 - - [26/Nov/2015:00:15:12 +0000] "HTTP/1.1" 200 24604 "http://localhost/sports" 
	"Mozilla/5.0 (Linux; Android 5.1.1; SAMSUNG SM-G920I Build/LMY47X) 
	SamsungBrowser/3.2 Chrome/38.0.2125.102 Mobile Safari/537.36"
	172.31.33.116 - - [26/Nov/2015:00:20:12 +0000] "HTTP/1.1" 200 4860 "http://localhost/home" 
	"Mozilla/5.0 (Linux; Android 5.1.1; SAMSUNG SM-G920I Build/LMY47X) 
	SamsungBrowser/3.2 Chrome/38.0.2125.102 Mobile Safari/537.36"
	172.31.33.116 - - [26/Nov/2015:00:22:19 +0000] "HTTP/1.1" 200 4841 "http://localhost/finances" 
	\end{lstlisting}
		\caption{Ejemplo de un \emph{webaccess Log} de un servidor Apache.}
		\label{fig:accesslog-apache-teleton}
\end{figure}

%TODO:
%En el párrafo anterior cerré hablando de una figura no puedo seguir hablando de la misma

En la Figura \ref{fig:accesslog-apache-teleton}, existe mucha información interesante como la \texttt{IP} desde donde se conecta, el tipo de navegador, el dispositivo desde donde se conecta, si es un teléfono inteligente o un navegador de escritorio, la fecha en que se realizó el acceso y también lo más relevante el destino del usuario. Anteriormente mencionamos la importancia para nuestro trabajo tener un versión simplificada de la sesión. Al tener un simplificada la representación podemos usar mas datos y generar un modelo de datos para predecir, en este instante se comprenderá la importancia de el uso de un algoritmo de compresión de tipo  \losslessdatacompression, el cual nos permitirá crear modelos que sean creados con mayor volumen de datos que las técnicas tradicionales de \machinelearning y además usando propiedades de la compresión y la \emph{Teoría de la Información}, que nos entreguen resultados interesantes dado nuestro escenario de modelo discreto de predicción para navegación de  usuarios en una.
   
Esta nueva perspectiva posee una adaptabilidad a la demanda o proveer información que permita adaptarse a los eventos, por lo tanto, presenta una gran ayuda para conocer futuros eventos y ayudar a tomar decisiones, por ejemplo el siguiente acceso, basado en un mayor cantidad de datos históricos. Sobre esta idea se pueden hacer integraciones en áreas  \losslessdatacompression y \machinelearning siendo un gran desafío. Independientemente del área, el problema común  se puede resolver de manera acotada en cada área, pero se busca encontrar las propiedades en común y usarlas para lograr un resultado predictivo en demanda.  






%; moderado por administradores a ser  generado orgánicamente por los usuarios. Estas nuevas propiedades es una nueva evolución de \inet. 
% Enfocando nuestros interés en encontrar comportamientos de usuarios, se buscar predecir accesos web para dar la mejora continua de la \www. y evolución. Este trabajo no profundizará en tópicos completos de recuperación de la información (\emph{Information Retrieval}), ni en el procesamiento analítico de estos datos, pero si el estudio del aprendizaje de patrones e implementación tendrán un protagonismo en nuestra propuesta y se tendrá un completa sección de los conceptos básico~(\ref{ch:Conceptos-Basicos}) sobre \emph{\losslessdatacompression} y \machinelearning.
%%%%%%%%%%% evaluar si se migra a intro --- esta idea es como quiero usarlo on ... algo asi..


% bla bla
% Nos enfocamos en la arquitectura de un servidor de \machinelearning para realizar experimentos y alterar su funcionamiento. En las etapas de un estudio usando \machinelearning existe el entrenamiento de un conjunto de datos, un aprendizaje, una etapa de servir resultados y finalmente una evaluación.  Para el aprendizaje, es requerido tener un modelo, que funcione como un algoritmo de predicción y posteriormente pueda dar una interpretación de los datos que son entregados por este modelo.  

 
% Evaluar para intro
 % entrega una posible ayuda a ingenieros de desarrollo \www y diseñadores de experiencia de usuarios, como también en general a mejorar la experiencia del usuario, podemos mencionar que podría disminuir latencia en respuestas por parte de cada petición realizada a los servidores, con técnicas de \emph{pre-fetching predictivo} en el lado del cliente, pero para lograr esto debemos tener una registros de accesos válidos. 

 % sar representaciones eficientes como las realizadas por Claude \etal~\cite{Claude2014} facilitaría el estudio al tener un foco en las predicciones secuenciales y no en la recuperación de estos registros. El uso de esta  minería de datos realizada en un conjunto de datos real, radica en que día a día \www genera  innumerables cantidades de información, lo que conlleva a usar algoritmos que puedan operar de manera comprimida grandes volúmenes de información o representación más liviana de datos trabajar. 
% DEFINIR EL PROBLEMA

\section{Definición del problema}
% IDEA: me gustaria que esta sección  no se llamara DEFINICION DE PROBLEMA, intrinsicamente he hablado que el problema es como mejorar las predicciones secienciales usando lo mejor de dos áreas.



% esto creo que se repite en contexto o intro.
En esta tesis el problema que se busca solucionar es la predicción a páginas web, que se representan mediante símbolos que pertenecen a una determinada secuencia de accesos, la cual representa una sesión. Este problemática esta presente hace años y diversos autores han trabajado con distintos enfoques. 

% TODO: Este parrafo es desastroso
%\emph{Rissanen}\cite{Rissanen1983} y Langdom\cite{Langdon1983} en los laboratorios \emph{Bell}, al realizar pruebas y experimentar con un \emph{robot} que tiraba una moneda compitiendo con una persona, aquel robot realizaba todos los cálculos {markovianos} y las probabilidades condicionales para que cierto evento ocurra, a diferencia del sujeto que sólo estaba esperando un resultado aleatorio, la diferencia se marco en los costos de tiempo y de computo que se realizaron, por un parte la demora del \emph{robot} haciendo sus cálculos no mejoraba a la probabilidad aleatoria con que la persona a que enfrentaba. 


La solución a este problema que pretendemos abordar, es mediante modelos predictivos. Como el lector puede imaginar predecir no es trivial y requiere de una basta cantidad de  información para poder analizar y desarrollar un modelo, que logre idealmente abarcar varios escenarios reales en que un usuario se enfrenta a una \web. Sí podemos llegar a acercarnos y minimizar el error estaremos cerca a predicciones aceptables. Sin embargo, dos áreas han tratado de resolver el problema; \machinelearning y \datacompression. Este último presenta condiciones desfavorables para los algoritmos predictivos que se pueden implementar, es decir, funcionan totalmente desconectados de la fuente de datos de entrada, esto implica que la validez del modelo solo es factible cuando esta realizando predicciones sin usuarios concurrentes o como un modelo desconectado de la fuente, lo que inhabilita rápidamente al modelo y en general no logrará un resultado en demanda. Por otro lado, en el acercamiento que ofrece \machinelearning debemos crear un modelo, tal que  deba entrenar y luego generar una función predictiva a lo cual se le suma un gran cantidad de datos. El proceso anterior puede producir un modelo bastante pesado para poder funcionar como un modelo predictivo \online. 

Para acotar nuestro problema, buscaremos resolver predicciones secuenciales discretas con un modelo generado por un algoritmo de compresión, aplicado a un conjunto de datos generados sintética-mente y otros datos real provistos por \emph{MSNBC}\cite{Claude2014}, un sitio \web de noticias. 

% ESTE TEXTO ESTA HORRIBLE:
% Este modelo propuesto de componentes híbridas, es decir, algoritmos de compresión  y un servidor de \machinelearning, buscaremos la solución a nuestro problema planteado, disponiendo como servicio, es decir crear, manteniendo la  componente \online del modelo predictivo vigente con ayuda de \losslessdatacompression y el famoso algoritmo \texttt{LZ78}; dando una predictibilidad inmediata que hoy en la industria es necesaria para usar los datos recolectados y entregar nuevos enfoques a las decisiones basadas en esta perspectiva predictiva, como también dando un avance en el análisis de datos predictivos. Usando un algoritmo de tipo \LDC reemplazando en el proceso de aprendizaje de una arquitectura de  servicios  \machinelearning podemos alcanzar a una buena solución o a lo menos estar sobre el promedio aleatorio de ocurrencia de eventos.




\section{Contribución de esta Tesis}


En este trabajo,  nuestra principal contribución es el desarrollo del primer servidor de predicción de secuencias discretas con funcionalidad online y \emph{offline} para secuencias de  sitios web. Adicionalmente nuestro modelo en conjunto \emph{PredictionIO}, es el primer servidor de Machine Learning usando un modelo de predicción con un algoritmo de \emph{Lossless Data Compression}. Hay varios temas propuestos en el que este trabajo pueda tener iteraciones  futuras de estas sistema. 

%
% ch2 ~\ref{ch:Compresion-Machine-Learning}
% ch3 ~\ref{ch:predicciones-webaccess}
% ch4 ~\ref{ch:experimetal-all}
% anexo ~\ref{ch:anexos}
%
\uncm
\section{Estructura de la tesis} 




Para una mejor lectura y compresión de esta tesis, este trabajo se ha compuestos de cinco capítulos, incluyendo este introductorio, más un anexo técnico de las herramientas usadas. Los capítulos siguiente abordarán los siguientes temas:

 % En el Capítulo \ref{cap:ch1-intro} hemos realizado una introducción y  explicado un contexto preliminar que rodea este trabajo y hemos definido el problema que buscamos abordar, como las posibles aproximaciones de soluciones, también hemos dado un interés en el uso de algoritmo que puedan ser consumidos como un servicio \emph{web}.
 
En el \textbf{Capítulo~\ref{ch:preliminar}}, se le recordará al lector todas los conceptos que se usarán en esta tesis, adicionalmente los trabajos relacionados y distintos puntos de vistas que se han tornado sobre al problema de las predicciones de páginas \web.


En el \textbf{Capítulo~\ref{ch:Compresion-Machine-Learning}}, presentaremos los temas de \machinelearning y \losslessdatacompression, para llevar acabo un estudio de predicciones sobre \webasccesslog, para poder predecir la siguiente página web que un usuario accede.


% explicaremos todos los conceptos básicos para el entendimiento sobre esta investigación, como también conceptos para el uso de \emph{PredictionIO} y cerraremos con todos los trabajos relacionados más recientes que involucran nuestra investigación. Adicionalmente veremos como poder dar una inducción a un entorno de trabajo llamado \emph{PredictionIO},el cual nos ayudará a entregar los algoritmos como servicios consumibles, por cualquier aplicación cliente que pueda comunicarse con un servidor \emph{web}. 
% veremos las predicciones sobre \emph{webaccess log}, los modelos propuestos por varios investigadores y sus limitaciones, daremos una revisión del trabajo realizado por \emph{Rissanen}\cite{Rissanen1984} que da el inicio a esta área de Investigación.
% En el Capítulo \ref{ch:Compresion-Machine-Learning} veremos los temas de \emph{Machine Learning} y \emph{Lossless Compression Data} y como pretendemos crear un modelo predictivo con recursos de ambas áreas. Para cerrar este capitulo explicaremos el uso de del algoritmo Lempel \& Ziv, el uso para secuencias discretas su convergencia a un modelo de predicción eficiente y exacto. 

En el \textbf{Capítulo~\ref{ch:predicciones-webaccess}}, se estudia ciertos fundamentos para comprender el tópico de los análisis predictivos y se describe como crear un modelo predictivo de navegación de usuarios, usando un algoritmo de compresión.





En el \textbf{Capítulo~\ref{ch:experimetal-all}}, finalmente se presentará nuestros experimentos realizados con nuestra implementación. Posteriormente, veremos nuestras conclusiones y discusiones, como también nuestro trabajos futuros.





Además se adjunta un anexo, que corresponde a una guía básica para él uso de \predictionio, todos los datos y nuestra implementación se puede encontrar en nuestro repositorio público\footnote{\footnotePublicRepo}, en este se encontrarán los datos para replicar los experimentos que hemos realizado. 


% sobre la implementación de un algoritmo de compresión en un servidor de \emph{Machine Learning}: \emph{PredictionIO}, analizaremos el comportamiento del algoritmo y como se desempeña este en distintos ambientes. propondremos discusiones de como mejorar nuestra implementación y los trabajos a futuro que puede presentar esta investigación.

% Explicaremos nuestras conclusiones obtenidas en base al desarrollo e implementación de la propuesta.
% Ademas dejaremos planteado nuestros trabajos futuros.


% TODO: REALMENTE NO SE SI VALE LA PENA ESTO
%\section{Algoritmos como servicio web }

	Los avances en el desarrollo de nuevas tecnologías que brinden mejores experiencias en su uso, deriva en cómo podamos llevar varios escenarios idealizados a implementaciones empresariales reales. Es bastante común encontrar librerías que son útiles para hacer Minería de Datos, agrupación y muchas operaciones que pueden recurrir en cálculos muy complejos, pero no se pueden ofrecer como servicio y en la industria es difícil implementarlos. Ya en pleno auge de las infraestructuras en la nube, la capacidad de cómputo que se puede alcanzar no es un problema como antes lo era para un Científico de Datos.


	Una \texttt{API} es un interfaz de programación de Aplicaciones que nos permiten intermediar el \emph{Servicio $A$} con el \emph{Servicio $B$}. Respectivamente $A$ puede ser el proveedor y $B$ el demandante del servicio. Si quisiéramos analizar datos que se encuentran dentro de un servidor específico, estos se podrían consumir por esta interfaz. Existen variados clientes que nos permiten ayudar en esta comunicación, incluso se pueden utilizar por una terminal de {Unix} que es posible dialogar mediante el programa \emph{curl}.
	
	Ya se dispone de infraestructura como servicio (\emph{IaaS}) , software como servicio (\emph{SaaS}), plataformas como servicios (\emph{PaaS}). Esta lógica de llevar todo a un nivel más abstracto de computación nos permite ofrecer  algoritmos que sean consumibles para hacer soluciones de desarrollo que den valor agregado a la experiencia requerida por el usuario final. Por esto hemos decidido utilizar una librería y ambiente de desarrollo que nos entregue esta posibilidad. Algoritmos como servicios que ayuden de manera eficiente e Inteligente ha mejorar la \emph{web} y con una fácil integración. 
	
	Todas las ventajas de este patrón son heredados de las características que ofrece una \texttt{API}, evita problemas de Infraestructura, Resiliencia de Datos, Persistencia de Datos, Análisis y Procesamiento sin afectar un curso operacional de una aplicación. Un ejemplo claro de esto es el análisis de datos en sistemas legados, los cuales en plan de mejoras no poseen la compatibilidad para poder implementarse. Por otro lado, los algoritmos de compresión o  \emph{Machine Learning} tienden a ser muy complejos de implementar, ocupan muchos recursos o la infraestructura disponible no da abasto, el personal que puede implementarlo es costoso. Lo anterior son algunas, pero no todas las razones para no  implementarlos.   

