
\section{Conceptos Básicos}


En esta sección se introducira los conceptos principales que se trabajarán en esta memoria.

\subsection{Web Usage Minning}


\subsection{Secuencias discretas}

Definimos una secuencia de accesos discreta y finita, dado los acccesos que tiene un usuario frente a una web, lo anterio es acotado por el concepto de sesión, el cual es desde que se inicia la navegación, es decir secuencia de tamaño $Seq\ \leq 1$ y de tamalo no superior a un alfabeto $A$.


\subsection{Alfabeto}

Dado un volumen de datos experimental, nuestro alfabeto es representado simbólicamente como la representación de un nodo de contenido de un sitio web.
Donde $A $, puede ser definido como la página inical. Este alfabeto es finito y acotoda por la mineria de datos de uso web.



\subsection{Arboles Trie}


Son estructuras de datos de tipo de árbol que almacenan datos en nodos y es de muy fácil la recuperación de información de estos mismo. Sus características generales es ser un conjunto de llaves las cuales se representan en el arbol y sus nodos internos representan la información, en nuestro caso una caracter o string de tamaño 1.

% easy text
% https://es.wikipedia.org/wiki/Trie
%Definición interpretada de esot

\subsection{Cadenas de Markov}



\subsection{Transferencia de Estado Representacional}

 REST es un estilo de arquitectura software para sistemas hipermedia distribuidos como la World Wide Web. El término se originó en el año 2000, en una tesis doctoral sobre la web escrita por Roy Fielding, uno de los principales autores de la especificación del protocolo HTTP y ha pasado a ser ampliamente utilizado por la comunidad de 
 %@TODO: poner una sucia referencia a wiki o algun paper mas sensato