
Este trabajo esta organizado de la siguiente manera. En el Capítulo \ref{cap:ch1-intro} describiremos el contexto preliminar que rodea este trabajo y definiremos el problema de la predicción de secuencias discretas para \emph{webaccess log}. Veremos como poder dar una inducción a un entorno de trabajo llamado \emph{PredictionIO} el cual nos ayudará a entregar los algoritmos como servicios consumibles, por cualquier aplicación cliente que pueda comunicarse con un servidor web.
En el Capítulo \ref{ch:Conceptos-Basicos} explicaremos todos los conceptos básicos para el entendimiento sobre esta investigación, como también conceptos para el uso de \emph{PredictionIO} y cerraremos con todos los trabajos relacionados más recientes que involucran nuestra investigación. En el Capítulo \ref{ch:predicciones-webaccess} veremos las predicciones sobre \emph{webaccess log}, los modelos propuestos por varios investigadores y sus limitaciones, daremos una revisión del trabajo realizado por \emph{Rissanen}\cite{Rissanen1984} que da el inicio a esta área de Investigación.
En el Capítulo \ref{ch:Compresion-Machine-Learning} veremos los temas de \emph{Machine Learning} y \emph{Lossless Compression Data} y como pretendemos crear un modelo predictivo con recursos de ambas áreas. Para cerrar este capitulo explicaremos el uso de del algoritmo Lempel \& Ziv, el uso para secuencias discretas su convergencia a un modelo de predicción eficiente y exacto. 

Finalmente el Capítulo \ref{ch:experimetal-all}, se presentará nuestros experimentos realizados sobre la implementación de un algoritmo de compresión en un servidor de \emph{Machine Learning}: \emph{PredictionIO}, analizaremos el comportamiento del algoritmo y como se desempeña este en distintos ambientes, propondremos discusiones de como mejorar nuestra implementación y los trabajos a futuro que puede presentar esta investigación.

Se deja como anexo una guía básica de uso para \emph{PredictionIO}, todos los datos y nuestra implementación se puede encontrar en nuestro repositorio \textbf{git público}\footnote{\url{https://github.com/jaimeguzman/PredictionIO-LZmodel}}, en se encontrará  los datos para replicar las experimentos que hemos realizado. 
