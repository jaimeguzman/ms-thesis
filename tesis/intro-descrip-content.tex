
Este trabajo esta organizado de la siguiente manera: En el capitulo 1 describiremos el contexto preliminar que rodea este trabajo y definiremos el problema de la predicción de secuencias discretas para \emph{webaccess log}. Veremos como poder dar una inducción a un framework llamada \emph{PredictionIO} el cual nos ayudará a entregar los algoritmos como servicios consumibles por cualquier aplicación cliente que pueda comunicarse con un servidor.
En capitulo 2 explicaremos todos los conceptos básicos para el entendimiento sobre esta investigación, como también conceptos para el uso de \emph{PredictionIO} y cerraremos con todos los trabajos  reaccionados mas recientes que involucran nuestra investigación de interés. En capitulo 3 veremos las predicciones sobre webaccess log, los modelos propuestos por varios investigadores y sus limitaciones, daremos una revisión del trabajo realizado por \emph{Rissanen}\cite{Rissanen1984} que da el inicio a esta área de Investigación.
En el capitulo 4 veremos los temas de \emph{Machine Learning} y \emph{Lossless Compression Data} y como pretendemos crear un modelo de predicción con recursos de ambas áreas. Para cerrar este capitulo explicaremos el uso de del algoritmo Lempel \& Ziv, el uso para secuencias discretas su convergencia a un modelo de predicción eficiente y exacto. 

Finalmente el capitulo 5, presentará nuestros experimentos realizados sobre la implementación de un algoritmo de compresión en un servidor de \emph{Machine Learning} como es \emph{PredictionIO}, analizaremos el comportamiento del algoritmo y como se desempeña en este ambiente, propondremos discusiones de como mejorar nuestra implementación y los trabajos a futuro que puede presentar esta investigación.

Se deja como anexo una guía básica de uso para \emph{PredictionIO}, todos los datos y nuestra implementación se puede encontrar en nuestro repositorio git público \footnote{\url{https://github.com/jaimeguzman/PredictionIO-LZmodel}}, en donde usted encontrará todos los datos para replicar las experiencias experimentales que hemos realizado. 
