\chapter[Introducción]{Introducción}\label{cap:ch1-intro}


{
Los nuevos avances tecnológicos y la inclusión de la ciencia de la computación en distintos campos han permitido crear grandes colecciones de datos que se pueden analizar, clasificar y procesar. Encontrar información útil y relevante dentro de estos grandes volúmenes de datos ayuda a mejorar las decisiones basándose en un conocimiento histórico de estos mismo y a la vez determinando patrones cuando se van generando a medida que transcurre el tiempo. 


Hoy en día las aplicaciones en que los usuarios se enfrentan no pueden ser estáticas y no tener características que ayuden a predecir la forma en que interactúan los usuarios. Estas nuevas características deseadas toman mucha relevancia al encontrarnos en un auge de información generada por varios usuarios, redes sociales y variadas plataformas. Podemos mencionar que ésta es una de las razones para poder  crear o usar  herramientas para el análisis y predicción de datos que nos permitan saber cómo se comportan un patrón encontrado dentro de una \emph{web}, conocer su frecuencia de accesos a un recurso en Internet, etc.

Al navegar en una \emph{web} establecemos una conexión {cliente--servidor}, en ese preciso instante una gran cantidad de registros son almacenados en el lado del servidor, los cuales pueden ser por ejemplo, datos de accesos de la sesión realizada. Estos registros  son normalmente usuarios de la \emph{web}  que se encuentran activos, llamaremos en adelante a estos registros de acceso: \emph{webaccess logs},  de los cuales se pueden realizar variadas investigaciones, por ejemplo:  cómo predecir cuál es la siguiente acceso de un usuario activo en una colección secuencial de estos registros. 

Ya hemos mencionado  que  en general las webs, redes sociales y en sí \emph{Internet} genera un gran volumen de datos constantemente, y se deben usar técnicas específicas que nos den resultados cuantitativos para darle una interpretación a nuestro análisis de datos predictivo. Una de las técnicas que toma sentido y nos da un gran aporte en este ámbito es el uso de Minería de datos  en los \emph{webaccess logs}.  Estás técnicas están a la disposición  de estudios y análisis de datos tanto para las industria, como para la academia y  se han convertido en un tema importante en esta área  de investigación, en los últimos años, en la cual podemos extender el campo usando estás técnicas en la búsqueda de nuevas aproximaciones para modelos predictivos en demanda. También se pueden utilizar para mejorar el rendimiento del caché en navegadores web como lo ha mencionado \emph{Moghaddam} \etal~\cite{Moghaddam2009},  detección de intereses de usuarios y también recomendaciones de páginas o bienes relacionados para los sitios {web} de comercio electrónico, mejorar  resultados de los motores de búsqueda y personalización de contenido web con preferencias personalizadas por el usuario a medida que va navegando. Todos los ejemplos anteriormente mencionados usando un modelo predictivo con disponibilidad inmediata.



Predecir estos  accesos no es trivial, aún los modelos predictivos que se han implementado no logran dar con un patrón para la navegación de usuario, de manera genérica. Dado la diversidad de perfiles de usuarios que se pueden encontrar en ciertas \emph{webs}  y distintos flujos de navegación de contenido de las mismas.  

Muchas técnicas de recuperación de datos y algunos sistemas de personalización usan algoritmos de predicción, podemos señalar que existen técnicas con componente online que entregan resultados inmediatos o \emph{online}, es bastante usual crear un entrenamiento estático y con un set de datos de volumen fijo, a este lo llamamos contrariamente una componente \emph{offline} de un modelo de predicción . La mayoría de las aplicaciones actuales que predicen la siguiente página web de un usuario tienen una componente  \emph{offline} que hacen la tarea de preparar los datos y su componente \emph{online}  proporciona contenido personalizado a los usuarios en función de sus actividades de navegación actuales.

Un modelo genérico no podría ser válido dada a la naturaleza básica de los conjuntos de datos que podemos recuperar de una \emph{web}, pero si un modelo predictivo liviano que vaya aprendiendo acorde a los patrones que se van recopilando en tiempo real, esto da inicio a tener un modelo predictivo con componentes tanto \emph{online} como \emph{offline}, el problema anterior es uno de nuestros intereses, consumir el resultado de un modelo de predicción confiable en que podamos ver los siguiente accesos de los usuarios y tomar acciones inmediatas. Muchos algoritmos de \emph{Machine Learning} usan análisis de datos predictivos para generar modelos predictivos en variadas áreas y este último es nuestro segundo interés, crear un modelo predictivo que se pueda integrar con los componentes de un servicio de predicción discreta que normalmente es abarcado por el campo de \emph{Machine Learning}.  

En este trabajo se presenta un modelo de predicción \emph{online}, que poseen un entrenamiento inicial \emph{offline} si es deseado y que se puede consumir como una \texttt{API} de servicios \texttt{REST}, la cual permite una integración ha variadas plataformas clientes y sistemas que no tenga un componente \emph{online} y con una buena exactitud de predicción. 

Nuestro modelo propuesto se basa en la implementación del algoritmo \texttt{LZ78} que está adaptado para modelar y representar la navegación secuencial del usuario. Nuestra propuesta disminuye la complejidad computacional  y la puesta en marcha del servicio, que es un desventaja en el desarrollo de sistemas predictivos \emph{online}.
}
%%%%%%Moghaddam_Kabir
%Web access prediction has attracted significant attention in recent years. Web prefetching and some personalization systems use prediction algorithms. Most current applications that predict the next user web page have an offline component that does the data preparation task and an online section that provides personalized content to the users based on their current navigational activities. In this paper we present an online prediction model that does not have an offline component and fit in the memory with good prediction accuracy. Our algorithm is based on LZ78 and LZW algorithms that are adapted for modeling the user navigation in web. 
%  A performance evaluation is presented using real web logs. 
%This evaluation shows that our model needs much less memory than PPM family of algorithms with good prediction accuracy.

  
%When user requests inserted and deleted incrementally the online models are desirable. 
%In this paper we present efficient techniques for modeling user navigation behavior. 
%Our model is online so changes in user request patterns will update our prediction model incrementally.
%%%%%%%%%%%%%%%%



%%%%%%%%%%
% Uno de los interes en poder crear un un aplicación de ambas areas es la convergenia de las areas en la cual un proceso de aprendizaje ó predicción secuencial se puede usar con ML y Compress data






\section{Contexto preliminar} 
\label{sec:preliminar}

% Idealmente aca explicar el problema

  La Web crece constantemente y por ende su infraestructura, también la información que podemos obtener de los  usuarios y  por consecuencia mayor concurrencia de los sistemas, la cual para los usuarios finales se traduce en un incremento de latencia y una mejor o peor experiencia de usuario. Paralelamente se suma un costo exponencial de recursos tanto en tecnologías de desarrollo como servicio que no son optimizados para poder dar una experiencia de usuario con calidad de servicio. Podemos reflexionar, entonces, que tener mayores recursos no mejorará el rendimiento, ni tampoco será lo óptimo para dar una calidad de servicio web ya que el ancho de banda de Internet no crecerá en la misma proporción.
   
  Adicionalmente, las tecnologías para la creación de web dinámicas y asíncronas han evolucionado a favor de traspasar la carga cliente.
  Hoy en día ya se poseen lenguajes y {framework} que disminuyen considerablemente la carga de un servidor, por lo cual, un buen servicio web es proveer una balanceada carga dentro del cliente y el servidor, pero cuando se poseen un gran volumen de datos es fundamental tomar decisiones que los recursos y lenguajes no cubren, es ahí el interés de dar inteligencia a servicios de la web.

  Interpretaremos que la manera en que un usuario navega es su comportamiento o patrón registrado en \emph{webacces log}, y que se pueden analizar, estudiar y modelar con algoritmos que tengan enfoques predictivos. 

  Sobre estos registros se pueden hacer representaciones eficientes como las realizadas por Claude \etal~\cite{Claude2014},  minería de datos para buscar \emph{Web Usage Mining} (WUM), el porque de hacer minería de datos radica en que cada día la web genera una innumerable cantidad de información, por lo cual usar algoritmos que se puedan opera de manera comprimida o con una representación lo más liviana posible es de interés ya que además de disminuir el espacio físico o recursos utilizados, se puede usar como un algoritmo de predicción y trabajar con una mayor cantidad de datos.
  
  Teniendo el conocimiento que los \emph{webaccess logs} se pueden estudiar de manera procesada o pre-procesada, ayudaría a ingenieros de desarrollo web y diseñadores de experiencia de usuarios, como también en general a mejorar la experiencia de usuario final, también disminuyendo por ejemplo la latencia en respuestas por parte de cada petición realizada, con técnicas de \emph{pre-fetching predictivo} en el lado del cliente.
  

  Actualmente, los sitios web han evolucionado de ser contenido estático a dinámico, también moderado por administradores a contenido orgánico creado por usuarios finales. Se debe poseer una adaptabilidad a la demanda o proveer información que permita adaptarse a los eventos, por lo tanto, hacer un estudio sobre esto y poder hacer integraciones en áreas como uso de algoritmos de compresión y \emph{Machine Learning} presenta un gran desafío. Independientemente del área, el problema común  se puede resolver pero se busca encontrar las fortalezas de cada área y usarlas en conjunto. 

  Durante este trabajo se usarán técnicas de compresión de datos, se utilizará una infraestructura y patrón de implementación para modelos de \emph{Machine Learning} como servicio. Adicionalmente toda la experimentación se llevará acabo ofreciendo los algoritmos y modelos como servicio basado en una Transferencia de Estado Representacional (\emph{REST}~\ref{concept-rest}), y así implementarlo en áreas productivas las cuales pueden presentar interés dando nuevos escenarios de estudio.


  La sesión de un usuario comienza cuando se conecta a un servicio web, éstas pueden ser páginas informativas, redes sociales, web dinámicas, contenido colaborativo, etc. Estableciendo dicha conexión a una página, se crea en ese momento una sesión de navegación automáticamente. Dado esto es posible almacenar datos muy relevantes los cuales son ''webaccess log'' ó registros de accesos web, durante el texto se mantendrán las referencias en inglés. Un ejemplo de \emph{webaccess log} es lo que se observa en la figura ~\ref{fig-ejemplo-webaccesslogbruto}:

\begin{figure}[tb]\label{fig-ejemplo-webaccesslogbruto} 
	\centering
	\begin{lstlisting}[frame=single,basicstyle=\ttfamily\tiny,]
	172.31.33.116 - - [26/Nov/2015:00:12:12 +0000] "HTTP/1.1" 200 1784 "http://localhost/home" 
	"Mozilla/5.0 (Linux; Android 5.1.1; SAMSUNG SM-G920I Build/LMY47X) 
	SamsungBrowser/3.2 Chrome/38.0.2125.102 Mobile Safari/537.36"
	172.31.33.116 - - [26/Nov/2015:00:12:12 +0000] "HTTP/1.1" 200 179333 "http://localhost/news" 
	"Mozilla/5.0 (Linux; Android 5.1.1; SAMSUNG SM-G920I Build/LMY47X) 
	SamsungBrowser/3.2 Chrome/38.0.2125.102 Mobile Safari/537.36"
	172.31.33.116 - - [26/Nov/2015:00:12:12 +0000] "HTTP/1.1" 200 24660 "http://localhost/health" 
	"Mozilla/5.0 (Linux; Android 5.1.1; SAMSUNG SM-G920I Build/LMY47X) 
	SamsungBrowser/3.2 Chrome/38.0.2125.102 Mobile Safari/537.36"
	172.31.33.116 - - [26/Nov/2015:00:15:12 +0000] "HTTP/1.1" 200 24604 "http://localhost/sports" 
	"Mozilla/5.0 (Linux; Android 5.1.1; SAMSUNG SM-G920I Build/LMY47X) 
	SamsungBrowser/3.2 Chrome/38.0.2125.102 Mobile Safari/537.36"
	172.31.33.116 - - [26/Nov/2015:00:20:12 +0000] "HTTP/1.1" 200 4860 "http://localhost/home" 
	"Mozilla/5.0 (Linux; Android 5.1.1; SAMSUNG SM-G920I Build/LMY47X) 
	SamsungBrowser/3.2 Chrome/38.0.2125.102 Mobile Safari/537.36"
	172.31.33.116 - - [26/Nov/2015:00:22:19 +0000] "HTTP/1.1" 200 4841 "http://localhost/finances" 
	\end{lstlisting}
	
	
	
	\caption{Ejemplo de un \emph{webaccess Log} de un servidor Apache.}
	\label{fig:accesslog-apache-teleton}
\end{figure}


  En la Figura \ref{fig-ejemplo-webaccesslogbruto} nos entrega mucha información interesante como la \texttt{IP} desde donde se conecta, el tipo de navegador, el dispositivo si es un teléfono inteligente o un navegador de escritorio, la fecha en que se realizó el acceso y también lo más relevante el destino del usuario.
  
  
  
  


\section{Definición del Problema}


El problema de la Predicción, ha surgido hace años y diversos investigadores han trabajado con distintos enfoques. Rissman y Langdom en los laboratorios Bell al realizar pruebas con un robot y hacer un experimento con un robot que tiraba una moneda compitiendo con humano, realizaba todos los calculos markovianos y calculos de las probabilidades condicionales para que cierto evento ocurra, a diferencia del sujeto que solo estaba esperando un resultado.

Predecir no es trivial, pero si podemos llegar a cercanos y minimizar el error de equivocarnos. Sin embargo, dos áreas han tratado de resolver el problema, LDC y Machine Learning de manera separada. Por parte de LDC los mayores problemas son que los predictores funcionan totalmente desconectados y no dan una de disponibilidad inmediata de los resultados, en cambios en el área de Machine Learning debemos crear un modelo para entrenar y luego poder generar una función predictiva.

Planteamos el problema de poder resolver tener un modelos híbrido juntando los patrones de cada área y disponerlo como un servicio inmediato dando una predictibilidad inmediata que hoy en la industria es necesaria para poder hacer útiles estos algoritmo y dar un valor a los avances.

 




% @TODO: SEGUIR TRABAJANDO EN ESTA BREVE INTRODUCCION
%En este tema convergen tres áreas, por un lado existe trabajo para crear estructuras eficientes para predicciones basadas en algoritmos de compresión, como es en el caso de~\cite{Claude2014}, y, por otro lado, el uso de algoritmos de aprendizaje para realizar clustering y predecir el comportamiento basado en el mismo contenido o en la distancia del contenido que visita el usuario actual al contenido clusterizado, como es el caso de ~\cite{Poornalatha2012}, inclusive se han utilizado modelos de Markov en ~\cite{Dongshan2002}  para poder modelar el comportamiento de la web.
%La tercera área son los Sistemas de Recomendación, la cual en este proyecto no se tocará pero si se mencionará el enfoque práctico que presenta área como un foco de múltiples implementaciones. 








\section{Algoritmos como servicio web }

	Los avances en el desarrollo de nuevas tecnologías que brinden mejores experiencias en su uso día a día, deriva en cómo podamos llevar varios escenarios idealizados a implementaciones empresariales reales. Es bastante común encontrar librerías que son bastante útiles para hacer Minería de Datos, agrupación y muchas operaciones que pueden recurrir en cálculos muy complejos, pero no se pueden ofrecer como servicio. Ya en pleno auge de las infraestructuras en la nube, la capacidad de cómputo que se puede alcanzar no es un problema como antes lo era para un Científico de Datos.


	Una \emph{API} es un interfaz de programación de Aplicaciones que nos permiten intermediar el \emph{Servicio $A$} con el \emph{Servicio $B$}. Respectivamente $A$ puede ser el proveedor y $B$ el demandante del servicio. Si quisiéramos analizar datos que se encuentran dentro de un servidor específico, estos se podrían consumir por esta interfaz. Existen variados clientes que nos permiten ayudar en esta comunicación, incluso se pueden utilizar por una terminal de {Unix} que es posible dialogar mediante el programa \emph{curl}.
	
	Ya se dispone de infraestructura como servicio (\emph{IaaS}) , software como servicio (\emph{SaaS}), plataformas como servicios (\emph{PaaS}). Dado lo anterior ofrecer estos algoritmos para hacer que las soluciones de desarrollo den valor agregado a la experiencia requerida por el usuario final. Por esto hemos decidido utilizar una librería y {framework} que nos de esta posibilidad. Ofrecer algoritmos a la industria como un servicio que ayuda de manera eficiente e Inteligente desarrollado como una API REST, que permitirá una fácil integración. 
	
	Todas las ventajas de este patrón son heredados de las características que ofrece una API, interoperabilidad, evitar problemas de Infraestructura, Resiliencia de Datos, Persistencia de Datos, Análisis y Procesamiento sin afectar un curso operacional de una aplicación. Un ejemplo claro de esto es el análisis de datos en sistemas legados los cuales en plan de mejoras, no poseen la compatibilidad para poder realizarlo. Por otro lado, los algoritmos de compresión o algoritmos de \emph{Machine Learning} tienden a ser muy complejos de implementar o ocupan muchso recursos y este hecho pueden ser la razón para no implementarlos. 
	
	
	
	 %Algoritmos como serivcio




% Se consideran y explican detalladamente todos los conceptos, teorıas, y aspectos pertinentes al tema tratado.
% El marco teorico concuerda con los objetivos y el tipo de trabajo.



\newpage
\section{Descripción del contenido} 
Este trabajo esta organizado de la siguiente manera. En el Capítulo \ref{cap:ch1-intro} describiremos el contexto preliminar que rodea este trabajo y definiremos el problema de la predicción de secuencias discretas para \emph{webaccess log}. Veremos como poder dar una inducción a un entorno de trabajo llamado \emph{PredictionIO} el cual nos ayudará a entregar los algoritmos como servicios consumibles, por cualquier aplicación cliente que pueda comunicarse con un servidor web.
En el Capítulo \ref{ch:Conceptos-Basicos} explicaremos todos los conceptos básicos para el entendimiento sobre esta investigación, como también conceptos para el uso de \emph{PredictionIO} y cerraremos con todos los trabajos relacionados más recientes que involucran nuestra investigación. En el Capítulo \ref{ch:predicciones-webaccess} veremos las predicciones sobre \emph{webaccess log}, los modelos propuestos por varios investigadores y sus limitaciones, daremos una revisión del trabajo realizado por \emph{Rissanen}\cite{Rissanen1984} que da el inicio a esta área de Investigación.
En el Capítulo \ref{ch:Compresion-Machine-Learning} veremos los temas de \emph{Machine Learning} y \emph{Lossless Compression Data} y como pretendemos crear un modelo predictivo con recursos de ambas áreas. Para cerrar este capitulo explicaremos el uso de del algoritmo Lempel \& Ziv, el uso para secuencias discretas su convergencia a un modelo de predicción eficiente y exacto. 

Finalmente el Capítulo \ref{ch:experimetal-all}, se presentará nuestros experimentos realizados sobre la implementación de un algoritmo de compresión en un servidor de \emph{Machine Learning}: \emph{PredictionIO}, analizaremos el comportamiento del algoritmo y como se desempeña este en distintos ambientes, propondremos discusiones de como mejorar nuestra implementación y los trabajos a futuro que puede presentar esta investigación.

Se deja como anexo una guía básica de uso para \emph{PredictionIO}, todos los datos y nuestra implementación se puede encontrar en nuestro repositorio \textbf{git público}\footnote{\url{https://github.com/jaimeguzman/PredictionIO-LZmodel}}, en se encontrará  los datos para replicar las experimentos que hemos realizado. 
