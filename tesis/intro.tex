\chapter[Introducción]{Introducción}
\label{ch:intro}

%chacharara
{
  El área de las ciencias de la computación que se encarga de estudiar todos los datos de maneras en que entregue información relevante, para poder ser estudiada es la Minería de Datos. Hoy en día esta área toma mucha relevancia al encontrarnos en una auge de la información generada por usuarios, redes sociales y variadas plataformas, podemos mencionar que esta es una de las razones para poder requerir disponer de herramientas de analisis que nos permitan saber como se comporta usuarios sobre una web, conocer la frecuencia en que se accede a un recurso en internet, etc.
  
  Sea el caso que entre una conexión cliente-servidor, una gran cantidad de servicios proporcionan datos de acccesos de los usuarios que acceden, sobre estos mismos existe se puede realizar un estudio sobre como predecir cual es la siguiente página que van a visitar.
  }

% Normalmente se ocupa muchos algorimot de maquinas de aprendizaje para poder hacer predicciones en variadas areas,

% Nuestro intere es hacer que las rpecicciones sean modeladas por un compresaor de datos basado en Lempel ziv



%%%%%%%%%%
% Uno de los interes en poder crear un un aplicación de ambas areas es la convergenia de las areas en la cual un proceso de aprendizaje ó predicción secuencial se puede usar con ML y Compress data

% Cuando se hace el pio train se hace un entrenamiento de modelo lo que resulta ser es que genera el arbol que es un trie de LZ que será el predicto

% Una función de predición implementada con MLIB y PIO sería predict () en scala




%%%%%%% CONTEXTO PRELIMINAR 
\section{Contexto preliminar} 
\label{sec:preliminar}

% Idealmente aca explicar el problema

  La Web crece constantemente y por ende su infraestructura, también la información que podemos obtener de los  usuarios y  por consecuencia mayor concurrencia de los sistemas, la cual para los usuarios finales se traduce en un incremento de latencia y una mejor o peor experiencia de usuario. Paralelamente se suma un costo exponencial de recursos tanto en tecnologías de desarrollo como servicio que no son optimizados para poder dar una experiencia de usuario con calidad de servicio. Podemos reflexionar, entonces, que tener mayores recursos no mejorará el rendimiento, ni tampoco será lo óptimo para dar una calidad de servicio web ya que el ancho de banda de Internet no crecerá en la misma proporción.
   
  Adicionalmente, las tecnologías para la creación de web dinámicas y asíncronas han evolucionado a favor de traspasar la carga cliente.
  Hoy en día ya se poseen lenguajes y {framework} que disminuyen considerablemente la carga de un servidor, por lo cual, un buen servicio web es proveer una balanceada carga dentro del cliente y el servidor, pero cuando se poseen un gran volumen de datos es fundamental tomar decisiones que los recursos y lenguajes no cubren, es ahí el interés de dar inteligencia a servicios de la web.

  Interpretaremos que la manera en que un usuario navega es su comportamiento o patrón registrado en \emph{webacces log}, y que se pueden analizar, estudiar y modelar con algoritmos que tengan enfoques predictivos. 

  Sobre estos registros se pueden hacer representaciones eficientes como las realizadas por Claude \etal~\cite{Claude2014},  minería de datos para buscar \emph{Web Usage Mining} (WUM), el porque de hacer minería de datos radica en que cada día la web genera una innumerable cantidad de información, por lo cual usar algoritmos que se puedan opera de manera comprimida o con una representación lo más liviana posible es de interés ya que además de disminuir el espacio físico o recursos utilizados, se puede usar como un algoritmo de predicción y trabajar con una mayor cantidad de datos.
  
  Teniendo el conocimiento que los \emph{webaccess logs} se pueden estudiar de manera procesada o pre-procesada, ayudaría a ingenieros de desarrollo web y diseñadores de experiencia de usuarios, como también en general a mejorar la experiencia de usuario final, también disminuyendo por ejemplo la latencia en respuestas por parte de cada petición realizada, con técnicas de \emph{pre-fetching predictivo} en el lado del cliente.
  

  Actualmente, los sitios web han evolucionado de ser contenido estático a dinámico, también moderado por administradores a contenido orgánico creado por usuarios finales. Se debe poseer una adaptabilidad a la demanda o proveer información que permita adaptarse a los eventos, por lo tanto, hacer un estudio sobre esto y poder hacer integraciones en áreas como uso de algoritmos de compresión y \emph{Machine Learning} presenta un gran desafío. Independientemente del área, el problema común  se puede resolver pero se busca encontrar las fortalezas de cada área y usarlas en conjunto. 

  Durante este trabajo se usarán técnicas de compresión de datos, se utilizará una infraestructura y patrón de implementación para modelos de \emph{Machine Learning} como servicio. Adicionalmente toda la experimentación se llevará acabo ofreciendo los algoritmos y modelos como servicio basado en una Transferencia de Estado Representacional (\emph{REST}~\ref{concept-rest}), y así implementarlo en áreas productivas las cuales pueden presentar interés dando nuevos escenarios de estudio.


  La sesión de un usuario comienza cuando se conecta a un servicio web, éstas pueden ser páginas informativas, redes sociales, web dinámicas, contenido colaborativo, etc. Estableciendo dicha conexión a una página, se crea en ese momento una sesión de navegación automáticamente. Dado esto es posible almacenar datos muy relevantes los cuales son ''webaccess log'' ó registros de accesos web, durante el texto se mantendrán las referencias en inglés. Un ejemplo de \emph{webaccess log} es lo que se observa en la figura ~\ref{fig-ejemplo-webaccesslogbruto}:

\begin{figure}[tb]\label{fig-ejemplo-webaccesslogbruto} 
	\centering
	\begin{lstlisting}[frame=single,basicstyle=\ttfamily\tiny,]
	172.31.33.116 - - [26/Nov/2015:00:12:12 +0000] "HTTP/1.1" 200 1784 "http://localhost/home" 
	"Mozilla/5.0 (Linux; Android 5.1.1; SAMSUNG SM-G920I Build/LMY47X) 
	SamsungBrowser/3.2 Chrome/38.0.2125.102 Mobile Safari/537.36"
	172.31.33.116 - - [26/Nov/2015:00:12:12 +0000] "HTTP/1.1" 200 179333 "http://localhost/news" 
	"Mozilla/5.0 (Linux; Android 5.1.1; SAMSUNG SM-G920I Build/LMY47X) 
	SamsungBrowser/3.2 Chrome/38.0.2125.102 Mobile Safari/537.36"
	172.31.33.116 - - [26/Nov/2015:00:12:12 +0000] "HTTP/1.1" 200 24660 "http://localhost/health" 
	"Mozilla/5.0 (Linux; Android 5.1.1; SAMSUNG SM-G920I Build/LMY47X) 
	SamsungBrowser/3.2 Chrome/38.0.2125.102 Mobile Safari/537.36"
	172.31.33.116 - - [26/Nov/2015:00:15:12 +0000] "HTTP/1.1" 200 24604 "http://localhost/sports" 
	"Mozilla/5.0 (Linux; Android 5.1.1; SAMSUNG SM-G920I Build/LMY47X) 
	SamsungBrowser/3.2 Chrome/38.0.2125.102 Mobile Safari/537.36"
	172.31.33.116 - - [26/Nov/2015:00:20:12 +0000] "HTTP/1.1" 200 4860 "http://localhost/home" 
	"Mozilla/5.0 (Linux; Android 5.1.1; SAMSUNG SM-G920I Build/LMY47X) 
	SamsungBrowser/3.2 Chrome/38.0.2125.102 Mobile Safari/537.36"
	172.31.33.116 - - [26/Nov/2015:00:22:19 +0000] "HTTP/1.1" 200 4841 "http://localhost/finances" 
	\end{lstlisting}
	
	
	
	\caption{Ejemplo de un \emph{webaccess Log} de un servidor Apache.}
	\label{fig:accesslog-apache-teleton}
\end{figure}


  En la Figura \ref{fig-ejemplo-webaccesslogbruto} nos entrega mucha información interesante como la \texttt{IP} desde donde se conecta, el tipo de navegador, el dispositivo si es un teléfono inteligente o un navegador de escritorio, la fecha en que se realizó el acceso y también lo más relevante el destino del usuario.
  
  
  
  



%%%%%%% CONCEPTOS BASICOS

\section{Conceptos Básicos}


En esta sección se introducira los conceptos principales que se trabajarán en esta memoria.

\subsection{Web Usage Minning}


\subsection{Secuencias discretas}

Definimos una secuencia de accesos discreta y finita, dado los acccesos que tiene un usuario frente a una web, lo anterio es acotado por el concepto de sesión, el cual es desde que se inicia la navegación, es decir secuencia de tamaño $Seq\ \leq 1$ y de tamalo no superior a un alfabeto $A$.


\subsection{Alfabeto}

Dado un volumen de datos experimental, nuestro alfabeto es representado simbólicamente como la representación de un nodo de contenido de un sitio web.
Donde $A $, puede ser definido como la página inical. Este alfabeto es finito y acotoda por la mineria de datos de uso web.



\subsection{Arboles Trie}


Son estructuras de datos de tipo de árbol que almacenan datos en nodos y es de muy fácil la recuperación de información de estos mismo. Sus características generales es ser un conjunto de llaves las cuales se representan en el arbol y sus nodos internos representan la información, en nuestro caso una caracter o string de tamaño 1.

% easy text
% https://es.wikipedia.org/wiki/Trie
%Definición interpretada de esot

\subsection{Cadenas de Markov}



\subsection{Transferencia de Estado Representacional}

 REST es un estilo de arquitectura software para sistemas hipermedia distribuidos como la World Wide Web. El término se originó en el año 2000, en una tesis doctoral sobre la web escrita por Roy Fielding, uno de los principales autores de la especificación del protocolo HTTP y ha pasado a ser ampliamente utilizado por la comunidad de 
 %@TODO: poner una sucia referencia a wiki o algun paper mas sensato




% @TODO: SEGUIR TRABAJANDO EN ESTA BREVE INTRODUCCION
%En este tema convergen tres áreas, por un lado existe trabajo para crear estructuras eficientes para predicciones basadas en algoritmos de compresión, como es en el caso de~\cite{Claude2014}, y, por otro lado, el uso de algoritmos de aprendizaje para realizar clustering y predecir el comportamiento basado en el mismo contenido o en la distancia del contenido que visita el usuario actual al contenido clusterizado, como es el caso de ~\cite{Poornalatha2012}, inclusive se han utilizado modelos de Markov en ~\cite{Dongshan2002}  para poder modelar el comportamiento de la web.
%La tercera área son los Sistemas de Recomendación, la cual en este proyecto no se tocará pero si se mencionará el enfoque práctico que presenta área como un foco de múltiples implementaciones. 







%Algoritmos como serivcio
\section{Algoritmos como servicio web }

	Los avances en el desarrollo de nuevas tecnologías que brinden mejores experiencias en su uso día a día, deriva en cómo podamos llevar varios escenarios idealizados a implementaciones empresariales reales. Es bastante común encontrar librerías que son bastante útiles para hacer Minería de Datos, agrupación y muchas operaciones que pueden recurrir en cálculos muy complejos, pero no se pueden ofrecer como servicio. Ya en pleno auge de las infraestructuras en la nube, la capacidad de cómputo que se puede alcanzar no es un problema como antes lo era para un Científico de Datos.


	Una \emph{API} es un interfaz de programación de Aplicaciones que nos permiten intermediar el \emph{Servicio $A$} con el \emph{Servicio $B$}. Respectivamente $A$ puede ser el proveedor y $B$ el demandante del servicio. Si quisiéramos analizar datos que se encuentran dentro de un servidor específico, estos se podrían consumir por esta interfaz. Existen variados clientes que nos permiten ayudar en esta comunicación, incluso se pueden utilizar por una terminal de {Unix} que es posible dialogar mediante el programa \emph{curl}.
	
	Ya se dispone de infraestructura como servicio (\emph{IaaS}) , software como servicio (\emph{SaaS}), plataformas como servicios (\emph{PaaS}). Dado lo anterior ofrecer estos algoritmos para hacer que las soluciones de desarrollo den valor agregado a la experiencia requerida por el usuario final. Por esto hemos decidido utilizar una librería y {framework} que nos de esta posibilidad. Ofrecer algoritmos a la industria como un servicio que ayuda de manera eficiente e Inteligente desarrollado como una API REST, que permitirá una fácil integración. 
	
	Todas las ventajas de este patrón son heredados de las características que ofrece una API, interoperabilidad, evitar problemas de Infraestructura, Resiliencia de Datos, Persistencia de Datos, Análisis y Procesamiento sin afectar un curso operacional de una aplicación. Un ejemplo claro de esto es el análisis de datos en sistemas legados los cuales en plan de mejoras, no poseen la compatibilidad para poder realizarlo. Por otro lado, los algoritmos de compresión o algoritmos de \emph{Machine Learning} tienden a ser muy complejos de implementar o ocupan muchso recursos y este hecho pueden ser la razón para no implementarlos. 
	
	
	
	




%introducción a las Prediccion


\section{Predecir con \emph{PredictionIO}}

  	
En esta sección se presenta formalmente el ambiente de desarrollo que se utilizará durante este trabajo. \emph{PredictionIO} es un servidor de \emph{Machine Learning} de código abierto para Científico de Datos y Desarrolladores que permite crear motores de predicción para aplicaciones en producción, con un bajo tiempo de entrenamiento y despliegue. Principalmente está construido en \emph{Apache Spark, HBase} y \emph{Spray}. 

Este ambiente de trabajo se encuentra en un maduración estable y constante que permite tanto disponer servidores con motores predictivos, como también toda una infraestructura distribuida para hacer que complejos algoritmos que sean utilizados para solucionar problemas reales de mayor escala.



% PIO, tiene practicamente todo armadao, ello no hicieron nada nuevo ... solo juntaron  todo...



% He estado investigando y revisando documentación, Yelp, Skype, Hubot de github y otras implementaciones tienen usando prediction.io




% La otra opción es meterle a este "DASE" un algoritmo  que mezcle una representación de cadenas de markov mezclado con LZ78. no se en que punto mezclarlo en el diccionario, o la verdad es como hacer el compresor sea "mas inteligente", encontré un papaer que te adjunto en el cual usan lz78 y lzw, esta interesante ya que le hacen un acercamiento mas al tema de de ser un predictor online.


% Ahora entiendo que la cadenas de markov son y se han ocupado para las predicciones, pero no veo la necesidad de ocuparlas mayormente. Adin me inisiste en que le de una vuelta.... pero mi sensación  es que tengo separada las ideas en dos extremos. 

% Ya revise Suffix Tree para predicciones LZ77 y LZW, también PPM y HMM.

 
 



\subsection{Arquitectura DASE}


Un motor de predicción es un tipo de proceso en \emph{Machine Learning}. Siguiendo una arquitectura de tipo \emph{DASE}, contendríamos los siguientes componentes.



\begin{itemize}

  \item\label{dase-datasource} \textbf{ $[D]$ Data Source y Data Preparator}. Los Data Source leen la data desde la entrada original y la transforman en un formato deseado para hacer análisis de estos. En cambio \emph{Data Preparator} pre-procesa la información y la reenvía a los algoritmos para   hacer el modelo de entrenamiento.


  \item\label{dase-algoritmo} \textbf{ $[A]$ Algoritmo}. Los componentes de \emph{PredictionIO}, dada sus librerías incluyen algoritmos de \emph{Machine Learning}, estos, pueden ser provistos por \emph{Apache Spark} o se pueden incluir algoritmos propios como también de terceros.
    Adicionalmente a los algoritmos podemos asignarle parámetros, para determinar como debiese ser construido el motor ó si es requerido para un cierto algoritmo.



  \item\label{dase-servicio} \textbf{ $[S]$ Servicio}. El componente servicio toma las consultas ó \emph{queries} de predicción y retorna los resultados, en nuestro modelo propuesto en la etapa experimental veremos el siguiente símbolo de una secuencia. 
  Si el motor de predicción tiene múltiples algoritmos, combinará los resultados en uno. Adicionalmente, la lógica específica de negocios puede ser añadida para especificar aún más el resultado final. 
 
  \item\label{dase-eval} \textbf{ $[E]$ Evaluación de Métricas}.
Las métricas de evaluación cuantifican la precisión de la predicción con una puntuación numérica. Puede ser utilizado para la comparación de algoritmos o ajustes de los parámetros del algoritmo.
\end{itemize}




  \tikzstyle{decision} = [diamond, draw,text width=4.5em, text badly centered, node distance=2.5cm, inner sep=0pt]
  \tikzstyle{block} = [rectangle, draw,text width=5em, text centered, rounded corners, minimum height=4em]
  \tikzstyle{line} = [draw, very thick, color=black!50, -latex']
  \tikzstyle{cloud} = [draw, ellipse, node distance=2.5cm,
  minimum height=2em]


\begin{figure}[t]
	\centering	
	\resizebox{0.8\textwidth}{!}{% <------ Don't forget this %
	
		\begin{tikzpicture}[scale=1, node distance = 2.5cm, auto]
		% Place nodes
		
		\node [block] (init) {Data de la Aplicación};
		\draw [color=gray,thick](1.3,1) rectangle (11.2,-3.5);    
		
		\node [block,right of=init] 		  (datasource) {Data Source};
		\node [block,right of=datasource]     (datapreparator) {Data Preparator};
		\node [block,right of=datapreparator] (alg1) {Algoritmo };
		\node [block,right of=alg1] 		  (serving) {Servicio};
		\node [block,below of=serving] 		  (evalmetric) {Evaluación Métrica};
		\node [block,right of=serving] 		  (resultpredict) {Resultado Predicción};
		\node [block,below of=resultpredict]  (resulteval) {Resultado Evaluación};
		
		\path [line] (init) -- (datasource);
		\path [line] (datasource) -- (datapreparator);
		\path [line] (datapreparator) -- (alg1);
		\path [line] (alg1) -- (serving);
		\path [line] (serving) -- (evalmetric);
		\path [line] (serving) -- (resultpredict);
		\path [line] (evalmetric) -- (resulteval);    
		
		\end{tikzpicture}
	}
	\caption{Diagrama de componentes Arquitectura DASE.}
	\label{fig:arquitectura-dase}
\end{figure}



\vspace{1cm}

\emph{PredictionIO} ayuda a tener componentes muy modulares, las que ya hemos descrito como  arquitectura \emph{DASE} (\ref{fig:arquitectura-dase})  que puede construir modelos de predicción de manera sencilla y ya contando con toda la arquitectura y algoritmos de \emph{MLIB}\footnote{Machine Learning Library Apache Spark, \url{http://spark.apache.org/mllib/}} de \emph{Apache Spark}. También poder integrarlos con gran facilidad a cualquier sistema o plataforma, por ejemplo, es posible elegir cual de todos los componentes se podrá desplegar al momento de crear un \emph{Engine} (Motor de Predicción.)



\vspace{0.5cm}
\subsection{Despliegue de motor de predicción}

  Un \emph{Motor de predicción} pone todos los componentes del diseño de arquitectura \emph{DASE} en un estado especifico de despliegue 
  \begin{enumerate}
  		\setlength{\itemsep}{1pt}
  		\setlength{\parskip}{0pt}
  		\setlength{\parsep}{0pt}
    \item Data Source
    \item Data Preparator
    \item Uno o más Algoritmos generadores de Modelos
    \item Un Servicio 
  \end{enumerate}

  Si se especifica más de un algoritmo, cada uno de los resultados de los modelos de predicción se entregará para ser consumido por cualquier cliente.
  Cada \emph{motor de predicción} procesa los datos y construye un modelos de forma independiente. Por lo tanto, todos los motores de predicción que usaremos sirven a su propio conjunto de resultados. Por ejemplo, se puede desplegar dos \emph{motores predictivos} para una aplicación móvil: uno para recomendar noticias a los usuarios y otro para sugerir nuevos amigos a los usuarios.


\vspace{1cm}
\subsection{Evaluación del motor de predicción }

  Para evaluar el \emph{Accuracy} de un motor de predicción, se debe especificar la métrica seleccionad cuando se corre el motor de evaluación, en los capítulos experimentales se verá como se generan métricas y como se desempeña esta métrica para ser evaluada.











\subsection{Modelamiento de eventos}

%https://docs.prediction.io/datacollection/eventmodel/




  El modelamiento de eventos es fundamental para un predictor \emph{online}, el hecho de poder llevar un vector con ciertas propiedades característica\footnote{Característica de un cierto dataset para entrenar.} de una representación vectorial del mundo del \emph{Machine Learning} a un modelamiento secuencial, es en realidad el modelamiento que se debe realizar de como  tener las muestras de datos para generar \emph{Resilient Distributed Dataset} (\texttt{RDD}) que son un parte principal de nuestro ambiente de trabajo con \emph{PredictionIO} para poder acceder posterior o inmediatamente. 

  Un evento lo definiremos como entidad que nos permite dar una representación temporalizada de información que será procesada por un motor de predicción. Analizaremos los eventos que un usuarios realiza para poder acceder a una web. Adicionalmente cuando cada usuario ingresa a una web automáticamente este genera una sesión, desde que que llega hasta que abandona la web.

  Usaremos un \emph{dataset} con información que esta totalmente depurada y recuperada de los \emph{access log} (provista por Claude \etal~\cite{Claude2014}), los cuales a efectos de temporalidad solo nos interesa conocer la secuencialidad de estos accesos.
  


%Poner algo mas matematico.
% EVENT API 
% https://docs.prediction.io/datacollection/eventapi/

  El modelamiento que realizaremos contempla los siguientes campos :

    \begin{itemize}
    		\setlength{\itemsep}{1pt}
    		\setlength{\parskip}{0pt}
    		\setlength{\parsep}{0pt}
      \item Tipo de Evento: Visitar
      \item Entidad que ejecuta el evento: Usuario
      \item Propiedades:
          \begin{enumerate}
          		\setlength{\itemsep}{1pt}
          		\setlength{\parskip}{0pt}
          		\setlength{\parsep}{0pt}
            \item Página actual
            \item Página siguiente
            \item Cierre de Sesión
          \end{enumerate}
    \end{itemize}



    El interés de tener un modelo totalmente atómico es poder contemplar la información que nos entrega, destacando sus variables y propiedades como restricciones.



\subsection{Ventajas de PredictionIO }


  Es posible mezclar y aplicar distintas característica, si el modelo no puede ser persistido por \emph{PredictionIO} automáticamente. Se requiere un objeto  heredado de una clase que permita lograr la persistencia en memoria, esto permite cargar el modelo persistentemente e instanciar automáticamente durante la ejecución del motor de predicciones.

  % Tenga en cuenta que los modelos generados por PAlgorithm no pueden ser persistieron automáticamente por naturaleza y deben implementar estas características, si se desea modelo de persistencia.


  Comprendiendo el concepto de \emph{Resilient Distributed Dataset} (\texttt{RDD} \ref{concept-RDD}), ésta es la abstracción básica de \emph{Apache Spark}, aún más, esto es una de las grandes cualidades de PredictionIO, ya que no solamente podemos disponer de una máquina para hacer estudios o implementar algoritmos, este servidor de \emph{Machine Learning} permite, gracias a sus componentes poder hacer un cluster para entregar mayor eficiencia acorde a los datos o algoritmo a implementar.

  Ya hemos mencionado que los \texttt{RDD} son una representación inmutable, una colección divida de elementos que pueden ser operadas en paralelo. Internamente cada \texttt{RDD} tiene cinco principales propiedades:


  \begin{enumerate}
    \item Una lista de particiones.
    \item Una función para procesar cada porción de datos.
    \item Una lista de dependencias en otros \texttt{RDD}. 
    \item Opcionalmente una partición de un \texttt{RDD} puede ser representada como una combinación de llave y valor. 
    \item Una lista optativa de los lugares preferidos para calcular cada una dividida en (por ejemplo, lugares de bloque para un archivo \emph{HDFS}); para procesamiento en \emph{Clustering}.

  \end{enumerate}
  




% https://docs.prediction.io/templates/recommendation/customize-serving/




















%@TODO:
% Este tema debería detallarse en las siguientes secciones
\section{Literatura}
En la literatura, el tema de la predicción en la web se ha presentado como un tema concurrente, y ha sido abarcado por varios autores. Tenemos los siguientes trabajos de interés:

\begin{enumerate}
  \item Dongshan y Junyi~\cite{Dongshan2002} destacan que un modelo de Markov puede ayudar a predecir el comportamiento de un usuario, pero con ciertas limitaciones .  Para solucionarlo presentan un nuevo modelo de Markov basado en una representación de \emph{Tree Order Model}, el cual es un híbrido entre un modelo de markov tradicional y una representación de árbol, bautizada como HTMM (por sus siglas en inglés, \emph{Hybrid-Order Tree Markov Model}).
  Su modelo fue presentado en 2002, y da una importancia a conocer la predicción de los \emph{web access}, dada la importancia de creación de redes, la minería de datos, e-commerce, y otras áreas.

  \item Domenech \etal~\cite{Domenech2006}, muestran un estudio de los rendimientos de técnicas de recuperación de datos.
  Las mismas se pueden utilizar para dar una entrada ideal a algoritmos de aprendizaje o algoritmos de predicción. 
  Los conceptos más importantes son las nuevas variables de caracterización, temporalidad, espacio y geografía, que se le suman a la predicción. 
  Además de comenzar un trabajo más elaborado de como tomar una predicción, se introducen conceptos como predicciones genéricas o específicas, variables de uso de recursos a nivel de red ó nivel procesamiento.
  Finalmente, se presenta un modelo predictivo que puede ayudar a disminuir la latencia entre la petición del cliente y la respuesta de la web, dando así un mejor rendimiento y \emph{QoS}.


  % @TODO detallar más explicarlo mas simple, darle mas enfoque al usuario segúnn del punto de vista que de los docuentos 
  % como los autores antteriores.

  \item Chen \etal~\cite{Chen2011} dan una nueva perspectiva enfocada a entregar una clara recomendación a los usuarios basada en la misma propuesta de este proyecto, los access log.
  El primer análisis realizado por los autores cubre las reglas asociativas que requiere un sistema de recomendación, pero en las pruebas propiamente tales encuentran que el análisis de los patrones detectadados dan una representación clara de como optimizar la web, y finalmente mediante sus pruebas logran una recomendación de calidad.

  \item Rajimol y Raju~\cite{Rajimol2012} minaron los patrones de los accesos web, donde el enfoque es usar los registros de acceso para crear subsecuencias y realizar comparaciones.
  La literatura presenta un interés para poder anticipar el patrón de comportamiento de la web.
  % @TODO reflexionar mas sobre este paper

  \item Kewen~\cite{kewen2012} realizó un análisis más profundo del \emph{web usage minning}.
  Parte de la importancia de este trabajo, es que después de minar los registros de accesos, logran reducir la ``\emph{bad data}''.
  %@TODO: Preguntar si este paper se escapa mucho del tema prinicipal, pero parece interesante  

  \item Poornalatha y Raghavendra~\cite{Poornalatha2012} establecen que se pueden utilizar máquinas de aprendizaje para predecir basándose en distintas entre clusters. Estos autores, al igual que Domenech \etal~\cite{Domenech2006} y Dongshan y Junyi~\cite{Dongshan2002}, comparan el objetivo de optimizar los recursos tanto en redes (disminución de latencia) y experiencia de usuario.

  \item Claude \etal~\cite{Claude2014} presentan una estructura de representación eficiente que permite dar una representación de \emph{web access log} y ofrecen las operaciones básicas de WUM.
  
\end{enumerate}



\section{Descripción del Contenido y Contribuciones}