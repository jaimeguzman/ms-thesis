\section{Algoritmos como servicio web }

	Los avances en el desarrollo de nuevas tecnologías que brinden mejores experiencias en su uso día a día, deriva en cómo podamos llevar varios escenarios idealizados a implementaciones empresariales reales. Es bastante común encontrar librerías que son bastante útiles para hacer Minería de Datos, agrupación y muchas operaciones que pueden recurrir en cálculos muy complejos, pero no se pueden ofrecer como servicio. Ya en pleno auge de las infraestructuras en la nube, la capacidad de cómputo que se puede alcanzar no es un problema como antes lo era para un Científico de Datos.


	Una \emph{API} es un interfaz de programación de Aplicaciones que nos permiten intermediar el \emph{Servicio $A$} con el \emph{Servicio $B$}. Respectivamente $A$ puede ser el proveedor y $B$ el demandante del servicio. Si quisiéramos analizar datos que se encuentran dentro de un servidor específico, estos se podrían consumir por esta interfaz. Existen variados clientes que nos permiten ayudar en esta comunicación, incluso se pueden utilizar por una terminal de {Unix} que es posible dialogar mediante el programa \emph{curl}.
	
	Ya se dispone de infraestructura como servicio (\emph{IaaS}) , software como servicio (\emph{SaaS}), plataformas como servicios (\emph{PaaS}). Dado lo anterior ofrecer estos algoritmos para hacer que las soluciones de desarrollo den valor agregado a la experiencia requerida por el usuario final. Por esto hemos decidido utilizar una librería y {framework} que nos de esta posibilidad. Ofrecer algoritmos a la industria como un servicio que ayuda de manera eficiente e Inteligente desarrollado como una API REST, que permitirá una fácil integración. 
	
	Todas las ventajas de este patrón son heredados de las características que ofrece una API, interoperabilidad, evitar problemas de Infraestructura, Resiliencia de Datos, Persistencia de Datos, Análisis y Procesamiento sin afectar un curso operacional de una aplicación. Un ejemplo claro de esto es el análisis de datos en sistemas legados los cuales en plan de mejoras, no poseen la compatibilidad para poder realizarlo. Por otro lado, los algoritmos de compresión o algoritmos de \emph{Machine Learning} tienden a ser muy complejos de implementar o ocupan muchso recursos y este hecho pueden ser la razón para no implementarlos. 
	
	
	
	