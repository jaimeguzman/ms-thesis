\section{Algoritmos como servicio web }

	Los avances en el desarrollo de nuevas tecnología que brinden mejores experiencia en su uso del día a día, lleva a dar como podemos llevar varios escenarios idealizados a implementaciones empresariales reales. Es bastante común encontrar librerías que son bastante útiles para hacer Minería de Datos, Clustering y muchas operaciones que pueden recurrir en cálculos muy complejos pero no se pueden ofrecer como servicio. Ya en auge de las Infraestructuras Cloud, la capacidad de computo que se puede alcanzar no es un problema a lo que antes se enfrenta un Cientista de Datos.


	Una API es un interfaz de programación de Aplicaciones que nos permiten intermediar el Servicio A con el Servicio B. Respectivamente A puede ser el proveedor y B el Demandante de servicios. Si quisiéramos analizar datos que se encuentran dentro de un servidor especifico, estos se podrían consumir por esta interfaz. Existen variados clientes que nos permiten ayudar a esta comunicación, incluso se pueden utilizar por una terminal de Unix la cual es posible que mediante el programa $curl$, el cual no permita dialogar.
	
	Ya se dispone de Infraestructura como servicio, Software como servicio, Plataformas como Servicios. Dado lo anterior ofrecer estos algoritmos para hacer que cada vez soluciones de desarrollo den valor agregado a la experiencia requerida por usuario. 
	Es por lo cual hemos decidido utilizar una librería y framework que nos de esta posibilidad. Ofrecer algoritmos a la Industria como un servicio que ayuda de manera inteligente y multiplataforma  que es el caso de una API. 
	
	Las ventajas de este patrón son heredados  todo lo que ofrece un API, interoperabilidad, evitar problemas de Infraestructura, Resiliencia de Datos, Persistencia de Datos, Análisis y Procesamiento sin afectar un curso operacional de una aplicación.
	
	Un ejemplo claro de esto son el análisis de datos en sistemas legados los cuales en plan de mejoras, no poseen la compatibilidad para poder realizarlo. Por otro lado no En sí los algoritmos de Compresión ó algoritmos de Machine Learning tienden ocupar recursos los cuales pueden ser razones para no implementarlos.
	
	
	
	