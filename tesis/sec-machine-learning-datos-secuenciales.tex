%Thomas G. Dietterich Oregon State University, Corvallis, Oregon, USA, tgd@cs.orst.edu,
%Como introducción 


Podemos señalar que existen muchos problemas en que el factor de la secuencialidad de los datos se convierten en un principal actor. Hemos estado atacando un escenario en que la ocurrencia de los datos, sin ser afectos al tiempo, el orden que van ocurriendo generan puntos a desarrollar.

Básicamente aquí tengo que explicar en que sirve o apartan las HMM, esta claro que estos modelos son de Machine Learning, o por lo menos se le debe dejar claro al lector que es así.

Si dieramos una introducción al modelamiento secuencial, es necesario introducir modelos o los efectos probabilisticos.



Aqui también se debe dar una intro pequeña a que se utilizará matlab, una de las validaciones que se espera es hacer correr el modelo de LZ para compararlo con los resultados típicos que tienen el LZ.



