


Este trabajo se ha centrado en la generación de un modelo predictivo de secuencias discretas sobre un alfabeto finito, con las definiciones anteriores podemos profundizando como usar técnicas de Machine Learning, que nos permitan avanzar y estar en la búsqueda de nuevas comparaciones.

El aprendizaje sobre datos secuenciales, como también el reconocimiento de patrones sigue siendo una de los desafíos del área de \machinelearning.
La literatura en estos temas es extensa y se ofrecen muchos acercamientos para análisis y predicción sobre secuencias en un alfabeto finito.

Una de las técnicas mas usadas son basadas sobre los \HMM, siglas en ingles de \hiddenmarkovmodels (Cadenas ocultas de Markov) CITA A RABINEER 1989 . Los \HMM nos ofrecen un estructura flexible que puede modelar distinto orígenes de datos secuenciales. Sin embargo, trabajar con los \HMM requieren una basta compresión en el dominio de problema, para poder modelar todas sus posibles restricciones.


Existen muchos problemas en que el factor de la secuencialidad de los datos se convierten en un principal actor. Hemos estado atacando un escenario en que la ocurrencia de los datos, sin ser afectos al tiempo, el orden que van ocurriendo generan puntos a desarrollar. También dado a la flexibilidad proporcionada, un entrenamiento exitoso requiere un gran conjunto de datos para ser entrenado.

 




Si diéramos una introducción al modelamiento secuencial, es necesario introducir modelos o los efectos probabilistas


De alguna manera hay que mencionar los VMM o Modelos de markov variables, son la base de los algoritmos de predicción usando LZ78



Aquí también se debe dar una intro pequeña a que se utilizará matlab, una de las validaciones que se espera es hacer correr el modelo de LZ para compararlo con los resultados típicos que tienen el LZ.

