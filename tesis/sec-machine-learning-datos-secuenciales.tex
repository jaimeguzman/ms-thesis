%%%%% SECCION DE MACHINE LEARNIGN PARA ANALISIS DE DATOS SECUENCIALES


%Thomas G. Dietterich Oregon State University, Corvallis, Oregon, USA, tgd@cs.orst.edu,
%Como introducción 

Este trabajo se ha centrado en la generación de un modelo predictivo de secuencias discretas sobre un alfabeto finito.

El aprendizaje sobre datos secuenciales, como tambíen el reconocimiento de patrones sigue siendo una de los desafios del área de \emph{Machine Learning}.
La literura en estos temas es extensa y se ofrecen muchos acercamientos para análisis y predicción sobre secuencias en un alfabeto finito.

Una de las técnicas mas usadas son basadas sobre los \emph{HMM}, siglas en ingles de \emph{Hidden Markov Models} (Cadenas ocultas de Markov) CITA A RABINEER 1989 . Los \emph{HMM} mos ofrecen un estructura flexible que puede modelar distinto origenes de datos secuenciales. Sin embargo, trabajar con los \emph{HMM} requieren una basta compresión en el dominio de problema, para poder modelar todas sus posibles restricciones.


Existen muchos problemas en que el factor de la secuencialidad de los datos se convierten en un principal actor. Hemos estado atacando un escenario en que la ocurrencia de los datos, sin ser afectos al tiempo, el orden que van ocurriendo generan puntos a desarrollar. También dado a la flexibilidad proporcionada, un entrenamiento exitoso requiere un gran conjunto de datos para ser entrenado.

 




Si dieramos una introducción al modelamiento secuencial, es necesario introducir modelos o los efectos probabilisticos.



Aqui también se debe dar una intro pequeña a que se utilizará matlab, una de las validaciones que se espera es hacer correr el modelo de LZ para compararlo con los resultados típicos que tienen el LZ.



