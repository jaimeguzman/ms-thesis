\section{Contexto Preliminar} 
\label{sec:preliminar}

% Idealmente aca explicar el problema

  La Web crece constantemente y por ende su infraestructura, también la información que podemos obtener de los  usuarios y  concurrencia de los sistemas, la cual para los usuario finales se traduce en latencia y una mejor o peor experiencia de usuario. Paralelamente se suma un costo exponencial de recursos tanto en tecnologías de desarrollo como servicio que no son optimizados para poder dar una experiencia de usuario con calidad de servicio (QoS). Podemos reflexionar, entonces, que tener mayores recursos no mejorará el rendimiento, ni tampoco será lo óptimo para dar una calidad de servicio web ya que el ancho de banda de Internet no crecerá en la misma proporción.
   
  Adicionalmente, las tecnologías para la creación de web dinámicas e asíncronas han evolucionado a favor de traspasar la carga cliente.
  Hoy en día ya se poseen lenguajes y framework que disminuyen considerablemente la carga de un servidor, por lo cual, un buen servicio web es proveer una balanceada carga dentro del cliente y el servidor, pero cuando se poseen un volumen de datos grandes es fundamental tomar decisiones que los recursos y lenguajes no cubren, es ahí el interés de dar inteligencia a los servicios web.

  Predecir los futuros accesos que un usuario tendrá en una determinada web. Interpretaremos que la manera en que un usuario navega es su comportamiento registrado en una web, y que se puede analizar, estudiar y registrar mediante \emph{Web Access Log},sobre estos mismo se puede hacer representaciones eficientes\cite{Claude2014} las de los registros y una minería de datos, Web Usage Mining (WUM), El por qué de hacer minería de datos es que cada día la web genera una innumerable cantidad de datos, por lo cual usar un algoritmos que se puedan operan comprimidos presenta un interés ya que además de disminuir el espacio físico o recursos utilizados, este se puede usar como un algoritmo de predicción y trabajar con una mayor cantidad de datos.
  
  Tener conocimiento de la predicción de los \emph{webaccess logs} de manera procesada o pre-procesada, ayudaría a ingenieros de desarrollo web y diseñadores de experiencia de usuario, como a  usuarios finales a mejorar la experiencia de usuario final, disminuyendo por ejemplo la latencia en respuestas por parte de cada petición que realizan, con técnicas de \emph{pre-fetching predictivo}.
  
  Hoy en día, las web no pueden ser simplemente dinámicas en contenidos, debe poseer una adaptabilidad a la demanda del usuario o proveer información que permita adaptarse a los eventos, por lo tanto, es de interés el profundizar en este tópico.

  Actualmente, las web no pueden ser simplemente dinámicas en contenidos, deben poseer una adaptabilidad a la demanda del usuario o proveer información que permita adaptarse a los eventos, por lo tanto, es de interés el hacer un estudio sobre ésto y poder hacer integraciones en áreas como compresión y maquinas de aprendizaje. Han sido abordadas independientemente para un problema en común que se puede resolver de manera eficiente. 

  Durante este trabajo se usarán técnicas de compresión de datos, se utilizará una infraestructura y patrón de implementación para modelos de Machine Learning. Adicionalmente toda la experimentación se llevará acabo ofreciendo los algoritmos y modelos como servicio REST (el cual se explicará más adelante) y así es posible implementarlo en áreas productivas las cuales pueden presentar interés.

  Definiremos el concepto de sesión cuando usuario se conecta a un servicio web, estas pueden ser paginas informativas, redes sociales, generalmente web dinámicas ó contenido colaborativo, etc. Este usuario establece una conexión directa con una página al momento de realizar esta operación. Dato esto es posible almacenar datos muy relevantes los cuales vamos a llamar ''we access log'' ó registros de accesos web, durante el texto se mantendrán las referencias en inglés.

  Un ejemplo de \emph{web access log} es el siguiente:


\begin{lstlisting}[frame=single,basicstyle=\ttfamily\tiny,]

172.31.33.116 - - [26/Nov/2015:00:12:12 +0000] "HTTP/1.1" 200 1784 "http://localhost/home" 
"Mozilla/5.0 (Linux; Android 5.1.1; SAMSUNG SM-G920I Build/LMY47X) 
SamsungBrowser/3.2 Chrome/38.0.2125.102 Mobile Safari/537.36"
172.31.33.116 - - [26/Nov/2015:00:12:12 +0000] "HTTP/1.1" 200 179333 "http://localhost/news" 
"Mozilla/5.0 (Linux; Android 5.1.1; SAMSUNG SM-G920I Build/LMY47X) 
SamsungBrowser/3.2 Chrome/38.0.2125.102 Mobile Safari/537.36"
172.31.33.116 - - [26/Nov/2015:00:12:12 +0000] "HTTP/1.1" 200 24660 "http://localhost/health" 
"Mozilla/5.0 (Linux; Android 5.1.1; SAMSUNG SM-G920I Build/LMY47X) 
SamsungBrowser/3.2 Chrome/38.0.2125.102 Mobile Safari/537.36"
172.31.33.116 - - [26/Nov/2015:00:15:12 +0000] "HTTP/1.1" 200 24604 "http://localhost/sports" 
"Mozilla/5.0 (Linux; Android 5.1.1; SAMSUNG SM-G920I Build/LMY47X) 
SamsungBrowser/3.2 Chrome/38.0.2125.102 Mobile Safari/537.36"
172.31.33.116 - - [26/Nov/2015:00:20:12 +0000] "HTTP/1.1" 200 4860 "http://localhost/home" 
"Mozilla/5.0 (Linux; Android 5.1.1; SAMSUNG SM-G920I Build/LMY47X) 
SamsungBrowser/3.2 Chrome/38.0.2125.102 Mobile Safari/537.36"
172.31.33.116 - - [26/Nov/2015:00:22:19 +0000] "HTTP/1.1" 200 4841 "http://localhost/finances" 
"Mozilla/5.0 (Linux; Android 5.1.1; SAMSUNG SM-G920I Build/LMY47X) 
SamsungBrowser/3.2 Chrome/38.0.2125.102 Mobile Safari/537.36"

  	
  \end{lstlisting}

  El ejemplo anterior nos da mucha información interesante como la IP desde donde se conecta, el tipo de navegador, el dispositivo si es un teléfono inteligente o un navegador de escritorio, la fecha en que se realizo el acceso y también lo mas relevante el destino del usuario.