\chapter[Introducción]{Introducción}
\label{ch:intro}

%charara
{
	El área de las ciencias de la computación que se encarga de estudiar todos los datos de maneras en que entregue información relevante, para poder ser estudiada es la Minería de Datos. Hoy en día esta área toma mucha relevancia al encontrarnos en una auge de la información generada por usuarios, redes sociales y variadas plataformas, podemos mencionar que esta es una de las razones para poder requerir disponer de herramientas de analisis que nos permitan saber como se comporta usuarios sobre una web, conocer la frecuencia en que se accede a un recurso en internet, etc.
	
	Sea el caso que entre una conexión cliente-servidor, una gran cantidad de servicios proporcionan datos de acccesos de los usuarios que acceden, sobre estos mismos existe se puede realizar un estudio sobre como predecir cual es la siguiente página que van a visitar.
	}


\section{Preliminar} \label{sec:preliminar}

  El interés en hacer un estudio sobre esto es poder hacer integraciones en áreas como compresión y la áreas que se dedican a completamente a hacer estudios sobre maquinas de aprendizaje. De por sí solas cada una se abordado independientemente lo cual es un interés converge en un problema en común que se puede resolver de manera eficiente.

  Durante este trabajo se usarán técnicas de compresión de datos, se utilizará una infraestructura y patrón de implementación para modelos de Machine Learning. Adicionalmente toda la experimentación se llevará acabo disponibilizando los algoritmos y modelos como servicio REST, el cual se explicará mas adelante, ya mencionado lo anterior este trabajo es implementable en áreas productivas las cuales pueden presentar interés.

  Definiremos que un usuarios es un la que se conecta a un servicio web, estos pueden ser paginas web informativas, redes sociales, etc. Este usuario establece una conexión directa con una pagina al momento de realizar esta operación, es posible almacenar datos los cuales vamos a llamar "access log" ó registros de accesos, durante el texto se mantendrán las referencias en ingles.

  Un ejemplo de access log podría ser el siguiente:


\begin{verbatim}
  172.31.33.116 - - [26/Nov/2015:00:12:11 +0000] "GET /wp-content/themes/corridacuprum/images/cuprum-footer.png HTTP/1.1" 200 4447 "http://virtual.corridacuprumteleton.cl/?utm_source=masterbase&utm_medium=mail2&utm_campaign=corrida-cuprum-teleton&utm_campaign=8474:%20%23!nombre!%23%2c+PARTICIPA+EN+LA+CORRIDA+VIRTUAL+Y+APOYA++A+LA+TELETON&utm_source=MasterBase%20CUPRUM&utm_medium=email&utm_content=2&utm_term=none" "Mozilla/5.0 (Linux; Android 5.1.1; SAMSUNG SM-G920I Build/LMY47X) AppleWebKit/537.36 (KHTML, like Gecko) SamsungBrowser/3.2 Chrome/38.0.2125.102 Mobile Safari/537.36"
  172.31.33.116 - - [26/Nov/2015:00:12:12 +0000] "GET /wp-content/themes/corridacuprum/images/principal-footer.png HTTP/1.1" 200 1784 "http://virtual.corridacuprumteleton.cl/?utm_source=masterbase&utm_medium=mail2&utm_campaign=corrida-cuprum-teleton&utm_campaign=8474:%20%23!nombre!%23%2c+PARTICIPA+EN+LA+CORRIDA+VIRTUAL+Y+APOYA++A+LA+TELETON&utm_source=MasterBase%20CUPRUM&utm_medium=email&utm_content=2&utm_term=none" "Mozilla/5.0 (Linux; Android 5.1.1; SAMSUNG SM-G920I Build/LMY47X) AppleWebKit/537.36 (KHTML, like Gecko) SamsungBrowser/3.2 Chrome/38.0.2125.102 Mobile Safari/537.36"
  172.31.33.116 - - [26/Nov/2015:00:12:12 +0000] "GET /wp-content/themes/corridacuprum/images/main.png HTTP/1.1" 200 179333 "http://virtual.corridacuprumteleton.cl/wp-content/themes/corridacuprum/style.css?ver=3.8.1" "Mozilla/5.0 (Linux; Android 5.1.1; SAMSUNG SM-G920I Build/LMY47X) AppleWebKit/537.36 (KHTML, like Gecko) SamsungBrowser/3.2 Chrome/38.0.2125.102 Mobile Safari/537.36"
  172.31.33.116 - - [26/Nov/2015:00:12:12 +0000] "GET /wp-content/themes/corridacuprum/fonts/akzidenzgrotesk-eb.woff2 HTTP/1.1" 200 24660 "http://virtual.corridacuprumteleton.cl/wp-content/themes/corridacuprum/style.css?ver=3.8.1" "Mozilla/5.0 (Linux; Android 5.1.1; SAMSUNG SM-G920I Build/LMY47X) AppleWebKit/537.36 (KHTML, like Gecko) SamsungBrowser/3.2 Chrome/38.0.2125.102 Mobile Safari/537.36"
  172.31.33.116 - - [26/Nov/2015:00:12:12 +0000] "GET /wp-content/themes/corridacuprum/fonts/akzidenzgrotesk-mc.woff2 HTTP/1.1" 200 24604 "http://virtual.corridacuprumteleton.cl/wp-content/themes/corridacuprum/style.css?ver=3.8.1" "Mozilla/5.0 (Linux; Android 5.1.1; SAMSUNG SM-G920I Build/LMY47X) AppleWebKit/537.36 (KHTML, like Gecko) SamsungBrowser/3.2 Chrome/38.0.2125.102 Mobile Safari/537.36"
  172.31.33.116 - - [26/Nov/2015:00:12:12 +0000] "GET /wp-content/themes/corridacuprum/images/lines-left.png HTTP/1.1" 200 4860 "http://virtual.corridacuprumteleton.cl/wp-content/themes/corridacuprum/style.css?ver=3.8.1" "Mozilla/5.0 (Linux; Android 5.1.1; SAMSUNG SM-G920I Build/LMY47X) AppleWebKit/537.36 (KHTML, like Gecko) SamsungBrowser/3.2 Chrome/38.0.2125.102 Mobile Safari/537.36"
  172.31.33.116 - - [26/Nov/2015:00:12:12 +0000] "GET /wp-content/themes/corridacuprum/images/lines-right.png HTTP/1.1" 200 4841 "http://virtual.corridacuprumteleton.cl/wp-content/themes/corridacuprum/style.css?ver=3.8.1" "Mozilla/5.0 (Linux; Android 5.1.1; SAMSUNG SM-G920I Build/LMY47X) AppleWebKit/537.36 (KHTML, like Gecko) SamsungBrowser/3.2 Chrome/38.0.2125.102 Mobile Safari/537.36"
  	
  \end{verbatim}

  El ejemplo anterior nos da mucha información interesante como la IP desde donde se conecta, el tipo de navegador, el dispositivo si es un telefono inteligente o un navegador de escritorio, la fecha en que se realizo el acceso y también lo mas relevante el destino del usuario.


\subsection{Contexto}
  \label{sec:contexto}

  La Web crece constantemente y por ende su infraestructura, también los tipos de usuarios y  también la concurrencia de los mismos sistemas, la cual para usuarios finales se traduce en latencia y una mejor o peor experiencia de usuario. 
  Paralelamente se suma un costo exponencial de recursos tanto en tecnologías de desarrollo como servicio que no son optimizados para poder dar una experiencia de usuario con calidad de servicio. Podemos reflexionar, entonces, que el no tener mayores recursos mejorará el rendimiento ni tampoco será lo óptimo para dar una calidad de servicio web, ya que el ancho de banda de Internet no crecerá a la misma proporción.
   
  Adicionalmente, las tecnologías para la creación de web dinámicas e asíncronas han evolucionado a favor del cliente.
  Hoy en día ya se poseen \emph{MEAN stacks} que disminuyen considerablemente la carga de un servidor, por lo cual, un buen servicio web es proveer una balanceada carga dentro del cliente y el servidor, pero cuando se poseen un volumen de datos considerables es requerido tomar decisiones que los recursos y lenguajes no cubren, es ahí el interés de hacer  web inteligente.

  Es de gran interés predecir los movimientos siguientes que un usuario tendrá en una determinada web.
  Entendiendo que la manera en que un usuario final navega es su comportamiento registrado en una web, y que se puede analizar, estudiar y registrar mediante \emph{Web Access Log} y a los cuales se puede hacer una minería de datos, Web Usage Minning. El por qué de hacer minería de datos es que cada día la web genera un innumerable cantidad de datos, por lo cual usar un algoritmo como LZ 78 presenta un interés ya que además de ser un algoritmo de compresión, este se puede usar como un algoritmo de predicción y trabajar con una mayor cantidad de datos.
  
  Los registros de accesos de manera procesada o pre-procesada, ayudaría a ingenieros de desarrollo web y diseñadores, como a  usuarios finales a tener una experiencia de usuario mejor, disminuyendo por ejemplo la latencia en respuestas por parte de cada petición que realizan.
  
  Hoy en día, las web no pueden ser simplemente dinámicas en contenidos, debe poseer una adaptabilidad a la demanda del usuario o proveer información que permita adaptarse a los eventos, por lo tanto, es de interés el profundizar en este tópico.






% @TODO: SEGUIR TRABAJANDO EN ESTA BREVE INTRODUCCION
%En este tema convergen tres áreas, por un lado existe trabajo para crear estructuras eficientes para predicciones basadas en algoritmos de compresión, como es en el caso de~\cite{Claude2014}, y, por otro lado, el uso de algoritmos de aprendizaje para realizar clustering y predecir el comportamiento basado en el mismo contenido o en la distancia del contenido que visita el usuario actual al contenido clusterizado, como es el caso de ~\cite{Poornalatha2012}, inclusive se han utilizado modelos de Markov en ~\cite{Dongshan2002}  para poder modelar el comportamiento de la web.
%La tercera área son los Sistemas de Recomendación, la cual en este proyecto no se tocará pero si se mencionará el enfoque práctico que presenta área como un foco de múltiples implementaciones. 



\section{Minería Web}


\section{Compresión}



\section{Algoritmos como servicio web }


  En esta sección hablare de como disponer los algoritmos como un servicio el cual se puede consumir, como si fuera una API.

Una API es un interfaz de programación de Aplicaciones que nos permiten intermediar el Servicio A con el Servicio B.
Por ejemplo si quisieramos disponer de datos que se encuentran dentro de un servidor especifico, estos se podrían consumir por esta interfaz. Existen variados clientes que nos permiten ayudar a esta comunicación, incluso se puede hacer que por una terminal de unix la cual es posible que mediante el programa curl entregue la información.

Disponer de algorimtos como servicio ayuda en el caso de que podemos disminiuir la carga de un servidor en particula y ayuda a la integración. Un ejemplo claro de esto son el analisis de datos en sistemas legados, es normal que estos sistemas legados no posean la Infraestructura para poder ser procesados, por lo cual desligar la carga de procesamiento, memoria ayuda a una integración.

En sí los algormimos de Compresión ó algoritmos de Machine Learning pueden ocupar muchos recursos los cuales pueden ser razones para no implementarlos.






\section{Predicción }

  \subsection{Ventajas }


  \subsection{Desventajas }


\section{Literatura}
En la literatura, el tema de la predicción en la web se ha presentado como un tema concurrente, y ha sido abarcado por varios autores. Tenemos los siguientes trabajos de interés:

\begin{enumerate}
  \item Dongshan y Junyi~\cite{Dongshan2002} destacan que un modelo de Markov puede ayudar a predecir el comportamiento de un usuario, pero con ciertas limitaciones .  Para solucionarlo presentan un nuevo modelo de Markov basado en una representación de \emph{Tree Order Model}, el cual es un híbrido entre un modelo de markov tradicional y una representación de árbol, bautizada como HTMM (por sus siglas en inglés, \emph{Hybrid-Order Tree Markov Model}).
  Su modelo fue presentado en 2002, y da una importancia a conocer la predicción de los \emph{web access}, dada la importancia de creación de redes, la minería de datos, e-commerce, y otras áreas.

  \item Domenech \etal~\cite{Domenech2006}, muestran un estudio de los rendimientos de técnicas de recuperación de datos.
  Las mismas se pueden utilizar para dar una entrada ideal a algoritmos de aprendizaje o algoritmos de predicción. 
  Los conceptos más importantes son las nuevas variables de caracterización, temporalidad, espacio y geografía, que se le suman a la predicción. 
  Además de comenzar un trabajo más elaborado de como tomar una predicción, se introducen conceptos como predicciones genéricas o específicas, variables de uso de recursos a nivel de red ó nivel procesamiento.
  Finalmente, se presenta un modelo predictivo que puede ayudar a disminuir la latencia entre la petición del cliente y la respuesta de la web, dando así un mejor rendimiento y \emph{QoS}.


  % @TODO detallar más explicarlo mas simple, darle mas enfoque al usuario segúnn del punto de vista que de los docuentos 
  % como los autores antteriores.

  \item Chen \etal~\cite{Chen2011} dan una nueva perspectiva enfocada a entregar una clara recomendación a los usuarios basada en la misma propuesta de este proyecto, los access log.
  El primer análisis realizado por los autores cubre las reglas asociativas que requiere un sistema de recomendación, pero en las pruebas propiamente tales encuentran que el análisis de los patrones detectadados dan una representación clara de como optimizar la web, y finalmente mediante sus pruebas logran una recomendación de calidad.

  \item Rajimol y Raju~\cite{Rajimol2012} minaron los patrones de los accesos web, donde el enfoque es usar los registros de acceso para crear subsecuencias y realizar comparaciones.
  La literatura presenta un interés para poder anticipar el patrón de comportamiento de la web.
  % @TODO reflexionar mas sobre este paper

  \item Kewen~\cite{kewen2012} realizó un análisis más profundo del \emph{web usage minning}.
  Parte de la importancia de este trabajo, es que después de minar los registros de accesos, logran reducir la ``\emph{bad data}''.
  %@TODO: Preguntar si este paper se escapa mucho del tema prinicipal, pero parece interesante  

  \item Poornalatha y Raghavendra~\cite{Poornalatha2012} establecen que se pueden utilizar máquinas de aprendizaje para predecir basándose en distintas entre clusters. Estos autores, al igual que Domenech \etal~\cite{Domenech2006} y Dongshan y Junyi~\cite{Dongshan2002}, comparan el objetivo de optimizar los recursos tanto en redes (disminución de latencia) y experiencia de usuario.

  \item Claude \etal~\cite{Claude2014} presentan una estructura de representación eficiente que permite dar una representación de \emph{web access log} y ofrecen las operaciones básicas de WUM.
  
\end{enumerate}