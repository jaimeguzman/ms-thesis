%%%%%%%  
%%%%%%%  INTRO COMPRESION
%%%%%%%  

 
La Compresión de Datos, en el contexto de las ciencias de la computación, es la ciencia (y arte) de representar información en una forma lo más compacta posible. Algunos ejemplos de áreas de aplicación que son relevantes y han sido motivación para la compresión de datos, son:

\begin{itemize}
	\menorEspacioItemize

	\item Sistemas de comunicación, ejemplo fax, \emph{voice mail} y telefonía.
	\item Estructuras de memoria, discos y cintas de computadores.
	\item Mobile computing.
	\item Sistemas distribuidos.
	\item Redes informáticas e Internet.
	\item Evolución multimedia, imágenes, procesamiento de señales.
	\item Archivos de imagen y videoconferencia.
	\item TV digital y satelital.

\end{itemize}


Considerando que los usuarios de \inet constantemente  crean nuevos contenidos, imágenes y videos  entre otros tipos de datos. Todos los recursos mencionados anteriorment al tener un menor tamaño o representación, se favorece al usuario y la red cuando no existen una buena calidad de conexión o infraestructura para mover un lugar a otro. Esto implica un disminución del tiempo de transferencia en la \emph{web} en el que se mueven miles de \emph{gigabytes}. Dichos archivos crecen exponencialmente no solo en número, sino también en peso individual y he aquí uno de los mayores aportes que poseen los algoritmos de compresión con relación a transferencia de datos. Precisamente la compresión de datos en un escenario de transmisión de datos por \inet optimiza la transferencia de archivos desde servidores, como también la carga de archivos en el lado del cliente. A diferencia de la velocidad de conexión y infraestructura de redes, estas no crecen proporcionalmente a los mismo volúmenes de transferencias de datos, esto genera una cantidad de problemas para los usuarios e industria web. Respecto a lo anterior podemos decir que el tiempo que demorar un usuario en pedir un archivo hasta recibirlo, se llama latencia. Por lo cual se comprende que al transportar archivos de menor tamaño la latencia de una petición a un servicio, será menor.


En el capitulo~\ref{capitulo-I} se ha planteado que ciertos algoritmos de compresión de datos, se usan para ayudar a tareas de \machinelearning. Para lograr entender esta premisa, se explicarán conceptos de compresión que darán los fundamentos para comprobar esta idea.


