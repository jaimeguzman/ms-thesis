% Puede generar borradores si omite la opción "final" de la clase.
% \documentclass{udpthesis}
\documentclass[final]{udpthesis}

% Establecemos el sistema para uso del español
% Babel ya esta cargado dentro de updthesis
\usepackage[T1]{fontenc}			%output
\usepackage[utf8]{inputenc}	%input
\usepackage{calc}
\usepackage{ifthen}
\usepackage{lmodern}
\usepackage{listings}
\usepackage{tikz}
\usetikzlibrary{shapes,arrows}
\usepackage{verbatim}
\usepackage{nomencl}
\usepackage{pgfplots} % fuer plots
\usepackage[external]{forest}
%jguzman
\usepackage{xspace} 




% Leyendas
\usepackage[font=footnotesize,labelfont=bf,labelsep=period]{caption}

% Agregue acá otros paquetes que le sean de utilidad
\usepackage{amsmath}	% Matemáticas\bigtriangleup
\usepackage{graphicx}	  % Gráficos
\usepackage[font=footnotesize,labelformat=simple]{subfig}
\usepackage[acronym]{glossaries}



\makeglossaries


\newglossaryentry{latex}
{
	name=latex,
	description={Is a mark up language specially suited for 
		scientific documents}
}

\newglossaryentry{maths}
{
	name=mathematics,
	description={Mathematics is what mathematicians do}
}

\newglossaryentry{formula}
{
	name=formula,
	description={A mathematical expression}
}

\newacronym{gcd}{GCD}{Greatest Common Divisor}
\newacronym{lcm}{LCM}{Least Common Multiple}




% Cambiamos el formato de las leyendas: finalizan en punto, y en negrita.
\captionsetup{labelsep=period,labelfont=bf}

% Habilitamos el uso de paréntesis al citar las figuras con subfiguras dentro, e.g., Fig. 1(a)
\renewcommand\thesubfigure{(\alph{subfigure})}
\renewcommand\thesubtable{(\alph{subtable})}
\newcommand{\subfigureautorefname}{\figureautorefname}


\lstset{
 % language=[LaTeX]TeX,
 breaklines=true,
 basicstyle=\tt\scriptsize,
 keywordstyle=\color{blue},
 identifierstyle=\color{black},
 commentstyle=\color{green!40!black},
 % frame 
 frame=tb,
 captionpos=t,
 xleftmargin=1em,
 numbersep=0.3em,
 % numbers=left,
 framexleftmargin=1.1em,
 framexrightmargin=0pt,
 % additional letters for accents in spanish
 literate=%
   {á}{{\'{a}}}1
   {é}{{\'{e}}}1
   {í}{{\'{i}}}1
   {ó}{{\'{o}}}1
   {ú}{{\'{u}}}1
   {ñ}{{\~{n}}}1
   {Ñ}{{\~{N}}}1
}


% Código
% Para generar código fuente usar listings.sty
%\usepackage{listings}
%\usepackage{tikz}
%\lstset{
%  language=[LaTeX]TeX,
%  breaklines=true,
%  basicstyle=\tt\scriptsize,
%  keywordstyle=\color{blue},
%  identifierstyle=\color{magenta},
%  commentstyle=\color{green!40!black},
%  % frame 
%  frame=tb,
%  captionpos=t,
%  xleftmargin=1em,
%  numbersep=0.3em,
%  numbers=left,
%  framexleftmargin=1.1em,
%  framexrightmargin=0pt,
%  % additional letters for accents in spanish
%  literate=%
%    {á}{{\'{a}}}1
%    {é}{{\'{e}}}1
%    {í}{{\'{i}}}1
%    {ó}{{\'{o}}}1
%    {ú}{{\'{u}}}1
%    {ñ}{{\~{n}}}1
%    {Ñ}{{\~{N}}}1
%}
%
%\renewcommand{\lstlistingname}{Código}% Listing -> Código
%\DeclareCaptionFormat{listing}{\rule{\dimexpr\linewidth\relax}{0.4pt}\par\vskip1pt#1#2#3}
%\captionsetup[lstlisting]{format=listing,singlelinecheck=false, margin=0pt,position=bottom}

% O para generar algoritmos en pseudocódigo usar algpseudocode.sty
\usepackage{algorithm}% http://ctan.org/pkg/algorithms
\usepackage{algpseudocode}

%\makeatletter
%\renewcommand{\ALG@name}{Algoritmo}% Algorithm -> Algoritmo
%\makeatother
%\captionsetup[algorithm]{font=footnotesize,labelsep=period}

\usepackage{cite}% Referencias (este paquete ordena y comprime las referencias)
\udptheme{EIT}

\newcommand{\uncm}{\vspace{1cm} }
\newcommand{\mediocm}{\vspace{0.5cm} }

%\newcommand{}{}
%\xspace This command decides whether to insert a space or not, usually this works well.


\newcommand{\www}		{\emph{www}\xspace}
\newcommand{\webs}		{\emph{webs}\xspace}
\newcommand{\inet}		{\emph{Internet}\xspace}
\newcommand{\online}	{\emph{online}\xspace}
\newcommand{\offline}	{\emph{offline}\xspace}
\newcommand{\cloudcomputing}{\emph{cloud computing}\xspace}


\newcommand{\webasccesslog}{\emph{webaccess log}\xspace}


\newcommand{\losslessdatacompression}{\emph{Lossless Data Compression}\xspace}


% SIGLAS todas con ttt
\newcommand{\LDC}{\texttt{LDC}\xspace}
\newcommand{\HMM}{\texttt{HMM}\xspace}



\newcommand{\machinelearning}{\emph{Machine Learning}\xspace}
\newcommand{\hiddenmarkovmodels}{\emph{Hidden Markov Models}\xspace}


% citas
% LDC un papaer con relacion a compresion
% PDA un paper con relacion a predicion
% ML un paper con relacion a machine learming
\newcommand{\LDCMoghaddam}{\cite{Moghaddam2009}\xspace}


\newcommand{\MLPDASunila}{\cite{gollapudi2016practical}\xspace}


\newcommand{\MLGuller}{\cite{guller2015big}\xspace}










 %myHacks





\begin{document}




\frontmatter %% Inicio de la portada
%%%%%%%%%%%%%%%%%%%%%%%%%%%
\title{Modelo de Predicción Secuencial de Webaccess Log basado en el Algoritmo de Compresión LZ78 y Machine Learning}
%%%%%%%%%%%%%%%%%%%%%%%%%%%
% para precisar aún más su tema, use un subtítulo
%\subtitle{Subtítulo explicativo del tema}
\author{Jaime Guzmán} \email{mail@jguzman.cl} \date{2016}
%%%%%%%%%%%%%%%%%%%%%%%%%%%
%%%%%%%%%%%%%%%%%%%%%%%%%%%
% Profesor guía
\professor{Adín Ramirez}
% Comité
\committee{Francisco Claude}{Martin Gutierrez}
%%%%%%%%%%%%%%%%%%%%%%%%%%%
% Dedicatoria
\dedicatory{Dedicado a mi hijo, mis padres y las personas que no dejaron de apoyarme a pesar de las adversidades.}
%%%%%%%%%%%%%%%%%%%%%%%%%%%
% Agradecimientos
\acknowledgment{ Nota redactada sobriamente en la cual se agradece a quienes han colaborado en la elaboración del trabajo. No puede exceder más de una página. }
% Abstract en inglés
\abstract{ Internet is growing every day and data volumes grow in the order of 
terabytes, for this reason it is of interest to use compression 
techniques for larger-scale processing of information with the fewest 
possible resources. Today there are various types of web: social networks, microbloging, web information, etc. The content provided to end 
users is not static and this allows them to create, contribute, and 
modify content, so Web Development Industry is constantly evolving to generate 
better solutions and friendly interaction with the users. 
Many of these new technologies have to deliver a better experience when browsing, the breakthrough has made it possible to create Web that are themselves intelligent and anticipating their behavior can go to, for example, decrease latency from a website and accessed or since navigate within a site with high concurrency; also from the point of view of the current web service patterns that are consumed immediately give a wide application area and provides a strategic aspect to the use of the information. While the growth of storage resources in the cloud is in full swing, networks don't grow at the same speed, which gives an area of interest to deepen.


So in this thesis we  propose the creation of a hybrid model between Machine Learning and Lossless Data Compression  for predict the next page that some user can access within a web; using training techniques and compression algorithms like well know \texttt{LZ78}, on a server Machine Learning. To this end we will work to create and study a predictive model that uses both areas and provide the results of the predictive model with integrated online component to any type of platform. }
% Resumen - Abstract en español
\resumen{ 	%El resumen no debe contener menos de 100 palabras ni mas de 300 palabras.

Internet crece cada día y los datos crecen en grandes volúmenes del orden de los Terabytes, por lo cual es de interés usar técnicas de compresión para realizar procesamiento a mayor escala de información con la menor cantidad de recursos posibles. Hoy en día existen variados tipos de web, redes sociales, microbloging, web informativas, etc. El contenido proporcionado a los usuarios finales ya no es estático y esto permite que los mismos puedan generar, aportar, modificar contenido, así es que la industria del desarrollo web está en constante evolución para generar más recursos para poder desarrollarla. Muchas de estas nuevas tecnologías han permitido entregar una mejor experiencia al momento de navegar, el gran avance realizado no ha permitido crear Web que sean por sí mismas inteligentes y puedan ir anticipando su comportamiento ó recomendando, por ejemplo; disminuir la latencia desde que se abre una web o el reconocimiento discreto de patrones cuando se navega dentro de un sitio con alta demanda de usuarios; también desde el punto de vista de los patrones web actuales los servicio que se consumen inmediatamente dan una área de aplicación extensa y aporta un aspecto estratégico al uso de de la información. Si bien el crecimiento de los recursos de almacenamiento en la nube se encuentra en apogeo, las redes de telecomunicaciones  no crecen a la misma velocidad, lo cual da un área de interés para profundizar. 

En este trabajo proponemos  la creación de un modelo híbrido entre \emph{Machine Learning} y Algoritmos de tipo \emph{Lossless Data Compression} para predecir la siguiente secuencias de acceso que usuario puede realizará dentro en una web; usando modelamiento de navegación basado en un Algoritmo de Compresión como \texttt{LZ78}, sobre un servidor de Machine Learning. Con este propósito se trabajará para crear y estudiar un modelo predictivo que use ambas áreas y  ofrecer los resultados del modelo predictivo con una componente \emph{online} integrable a cualquier tipo de plataforma. }
%%%%%%%%%%%%%%%%%%%%%%%%%%%
\makecover				 % Generamos la portada
\tableofcontents % tabla de contenido
\listoftables					% Indice de tablas
\listoffigures			 % Indice de figuras
\mainmatter	 			 % Inicio del contenido
% LA GRACI ES SER REPETITIVO SIN QUE SE NOTE.
%%%%%%%%%%%%%%%%%%%%%%%%%%%
% 
% 
% Capitulo I
% 
% 
\chapter[Introducción]{Introducción}\label{ch1:intro}

%%%%%%%%%%
% Uno de los interes en poder crear un un aplicación de ambas areas es la convergenia de las areas en la cual un proceso de aprendizaje ó predicción secuencial se puede usar con ML y Compress data

% @TODO: SEGUIR TRABAJANDO EN ESTA BREVE INTRODUCCION
%En este tema convergen tres áreas, por un lado existe trabajo para crear estructuras eficientes para predicciones basadas en algoritmos de compresión, como es en el caso de~\cite{Claude2014}, y, por otro lado, el uso de algoritmos de aprendizaje para realizar clustering y predecir el comportamiento basado en el mismo contenido o en la distancia del contenido que visita el usuario actual al contenido clusterizado, como es el caso de ~\cite{Poornalatha2012}, inclusive se han utilizado modelos de Markov en ~\cite{Dongshan2002}  para poder modelar el comportamiento de la web.
%La tercera área son los Sistemas de Recomendación, la cual en este proyecto no se tocará pero si se mencionará el enfoque práctico que presenta área como un foco de múltiples implementaciones. 

%
% Introduccion 
%
%
Internet es una fuente de información que crece constantemente y es visitada con una gran frecuencia, posee billones de web públicas. Dado este rápido desarrollo y accesos a grandes colecciones de datos, se vuelve fundamental y desafiante lograr anticipar las acciones de un usuario realice para investigadores e ingenierios. Los beneficios de lograr predicciones con la suficiente precisión pueden por ejemplo, mejorar la experiencia de usuarios en base a su comportamiento, registrados por \webasccesslog, ó mejorar la experiencia de compras en comercios electrónicos.

Existen varias área que buscan un estudio de este problema, el de las predicciones web de usuarios, también conocido como WPP. Algunos de estos enfoques para  abordarlo, por ejemplo son:  Análisis Predictivos de Patrones, Web Usage Minning, Web Access Pattern, Machine Learning y predicciones secuenciales con algoritmos de compresión. Esta tesis usará estas dos ultimas áreas para la implementación de un modelo predictivo simple y en un servidor de Machine Learning.

La implementación  propuesto se basa en la implementación del algoritmo \lzSieteOcho que está adaptado para modelar y representar la navegación secuencial del usuario. Nuestra propuesta disminuye el tiempo de actualización del modelo predictivo, con actualizaciones a medida que se va usando. Además la puesta en marcha del servicio, que es un desventaja en el desarrollo de sistemas predictivos \online.


Cuando se navega en un sitio web se puede recolecta una gran cantidad de registros almacenados en el lado del servidor, los cuales pueden ser por ejemplo, datos de accesos de la sesión realizada, también estos  registros representan usuarios de la \web  que se encuentran activos, llamaremos en adelante a estos registros de acceso: \webasccesslog,   los cuales se pueden realizar variadas análisis, por ejemplo:  Existe un usuario el cual intenta comprar un cierto producto en una web de comercio electrónico. Este sigue un cierto comportamiento ya que hasta el momento de comprar, se genera un un estado de búsqueda o visita en todas las elecciones posibles, puede volver como abandonar variados productos, hasta encontrar y comprar, o no. 

Predecir estos  accesos no es trivial, aún los modelos predictivos que se han implementado no logran dar con un patrón para la navegación de usuario, de manera genérica. Dado la diversidad de perfiles de usuarios que se pueden encontrar en ciertas \webs  y distintos flujos de navegación de contenido de las mismas.  

En este trabajo se presenta un modelo de predicción \online, que poseen un entrenamiento inicial \offline si es deseado y que se puede consumir como una \API de servicios \REST, la cual permite una integración ha variadas plataformas clientes y sistemas que no tenga un componente \online y con una buena exactitud de predicción. 

% conector:
En las próximas secciones pretendemos que el lector se contextualice más en el escenario que deseamos realizar predicciones, trabajos relacionados que presentan otros enfoques y los fundamentos que soportan esta tesis al trabajar con \webasccesslog.





\section{Contexto preliminar}\label{sec:preliminar}

Los modelos de predicción secuenciales son uno de los temas de interés tanto para el área de \machinelearning  y \datacompression. La búsqueda de un algoritmo que pueda entregar un análisis  y resultado predictivo con la mejor probabilidad de acierto o con el menor posible.
En los escenarios que involucran a \emph{webs}, la cantidad de variables y problemas a estudiar, es proporcional a la complejidad de la misma. 
Una característica común, es como los usuarios navegan su contenido en una sesión o acceso de un usuario. Podemos formular un modelo que refleje el comportamiento de accesos y realizar predicciones secuenciales con este. Tomando como punto inicial la llegada del usuario a la web, tendremos una cantidad finita de vistas de una sección a otra, por ejemplo. Si esta web que se describe fuera una comercio electrónico análogamente podríamos visualizar que se podría obtener predicciones de acceso a ciertos cierto catalogo de producto y podríamos tomar una decisión de negocio que logre concretar una venta. 

%  Como explique en ese parrafo esta muy ligado a lo que es recomendacion 
% ¿Cual es la necesidad?
Tener información antes que suceda o tener una aproximación cercana, es una de los principales desafío que busca solucionar en el área  de \machinelearning dando en el caso de las \emph{webs} una inteligencia para entregar una mejor experiencia cuando el usuario interactúa. Un ejemplo frecuente son los motores de búsqueda como \emph{Google}, \emph{Yahoo} y \emph{Bing}, los cuales en cada interacción de búsqueda muestran el tiempo demorado. Hoy en día se intentan anticipar el contexto de la búsqueda con la menor cantidad ed palabras claves, con el menor esfuerzo posible del usuario ó inclusive recomendar exactamente mientras se esta escribiendo. Esto nos da un estrecha relación con los sistemas de recomendación, en el cual no es de nuestro interés. 

Es común que en \machinelearning se realicen varios entrenamientos a un cierto algoritmo, que de como resultado un buen aprendizaje del dominio, también es requerido la mayor cantidad de datos posibles. Lo anterior da una complejidad de recursos en los cuales el algoritmo debe ser muy eficiente o exacto. En muchos casos es la segunda propiedad la predominante. 


% Los motores de busqueda es un claro ejemplo del gran manejo de volumenes de datos en el cual estos algoritmos de prediccion, si el objetivo fuera búsqueda, no se pueden acotar.

Podemos deducir de lo anterior que no es suficiente aumentar los recursos, para obtener  mejor rendimiento para un cierto algoritmo, tampoco será óptimo o económicamente viable. Como ejemplo tenemos las infraestructuras de telecomunicaciones, el ancho de banda de \emph{Internet} en una relación con el volumen de datos generado es inversamente proporcional al crecimiento en un mismo periodo de tiempo. La industria con ofertas de computación en la nube (\emph{cloud computing}), ejemplo \emph{Amazon AWS, Google Cloud, Azure Microsoft, Oracle Cloud}, han buscado disminuir esta brecha técnica y de recursos físicos. Cada vez mas muchas empresas cuentan con sistemas de información en estas nuevas arquitecturas en la nube. Adicionalmente, las tecnologías para la creación de \www dinámica y asíncrona, entre el cliente y el servidor evoluciona a favor de dar la carga al cliente, también los lenguajes de programación y \emph{framework} que disminuyen considerablemente la carga de peticiones al servidor que se realizan.

Sobre escenarios que se posee un gran volumen de datos, es fundamental usarlos para tener análisis predictivos más exactos y tomar decisiones o acciones. Estas decisiones  pueden ser determinadas bajo un algoritmo probabilista o de frecuencia, es posible ser consultado  mientas el usuario esta navegando en una web. 

El contenido de \emph{Internet} tiene un crecimiento exponencial y muchos usuarios son atraídos a visitarlos.  En gran medida para satisfacer ciertas necesidades comunicación, información, ocio y entretenimiento, redes sociales,  publicidad o  comprar de bienes y servicios. Los escenarios anteriores deben estar sujetos a ciertas restricciones que para el usuario debe estar en el menor tiempo posible, dado el poco tiempo de atención que se dispone o la gran cantidad de tareas que se realizan. Este constante y complicado desafío genera una gran consumo de recursos de infraestructura informáticas y sistemas de telecomunicaciones. Además  requiere de \emph{webs} con ciertos algoritmos que aprendan silenciosamente desde información técnica que es almacenada en cada visita, con el fin de entregar una gran experiencia al usuario.
 
% %%% ejemplos de mas ideas dnd ser usado
Existen varios escenarios de estudios en que las predicciones sobre  volúmenes de datos, también pueden ser usadas para la {detección de fraudes financieros, predicción de valores bursátiles y también para hacer diagnostico en áreas de la salud, basado un conjuntos de datos genéticos o problemas de salud hereditarios. Los ejemplos anteriormente mencionados tienen una variable  común y es que toda predicción ayuda a tomar una decisión.}\label{ejemplos-casos-contextopreliminar}

%
%
%
% hacer link al concepto de trie
%
%  REFERENCIAS
\losslessdatacompression, es un área que permite tener representaciones de la información mas compactas sin perder información. Dado el caso que se detallara en la sección de compresión, existe un punto acorde a la literatura, en que un compresor y debido a la entropía de los datos se puede convertir en un predictor, que a su vez es una representación secuencial de datos comprimidas en un \emph{trie}, maneja datos discretas al igual que algoritmos de \machinelearning. En la literatura encontramos casos en que existen aproximaciones a usar estas áreas, para converger en una propuesta llamada \emph{modelos de markov variables}, los cuales veremos como se relacionan con el algoritmo de compresión \emph{LZ78}.

%
% JUNTAR MAS IDEAS  ML con LDC
%

% IDEA de que porque es comportamiento
Variados escenarios de estudio como  detección de  patrones para cierto dominio de problemas y aprenderlos rápidamente con un cierto algoritmo,, posteriormente se puede proveer resultados para ser analizados o procesados pueden ser desarrollados gracias a \machinelearning, mediante un modelo predictivo.  La navegación de un usuario  en una web, es nuestro contexto de estudio y en adelante diremos que la  navegación de usuarios usuarios es su patrón de navegación o comportamiento, registrado en  \webasccesslog. Estos se pueden analizar, estudiar y modelar con algoritmos que tengan enfoques predictivos. Se pueden usar de manera procesada o pre-procesada. Acorde a los ejemplos que hemos mencionado en esta sección~(\ref{ejemplos-casos-contextopreliminar})  y nuestro interés para predecir el comportamiento de navegación de usuarios en determinadas \emph{webs}.

%% Estos 2 parrafos aun no logros relacionarlos


 Es fundamental estudiar los datos que se buscan modelar para predecir y encontrar patrones frecuentes que  se encuentran para poder crear un modelo predictivo, intentando de evitar un sobre-entrenamiento del modelo. Este trabajo usará los avances en \emph{Web Usage Minning} y \emph{Web Access Pattern} para buscar una implementación de un modelos predictivos con una característica \emph{online} e híbrido. Ciertamente \machinelearning usa \emph{features}\footnote{\emph{feature:} Las característica de un \machinelearning son propiedades individual y cuantitativas de un fenómeno que se observa dentro de un conjunto de datos.} y entre más información de entrenamiento se obtendrá  mejores predicciones. Por ejemplo la sesión de un usuario podría modelarse con gran exactitud en el mundo \machinelearning, ya que nos entrega una gran cantidad de datos para hacer análisis predictivos. Estas sesiones comienza cuando se establece la conexión a una \www. Estableciendo dicha conexión, se crea instantáneamente una sesión de navegación automáticamente, que se almacena en el lado del cliente o que hospeda la \www. Un ejemplo de  \webasccesslog es lo que se observa en la Figura \ref{fig:accesslog-apache-teleton}.

\input{experiments/logs-teleton}

%TODO:
%En el párrafo anterior cerré hablando de una figura no puedo seguir hablando de la misma

En la Figura \ref{fig:accesslog-apache-teleton}, existe mucha información interesante como la \texttt{IP} desde donde se conecta, el tipo de navegador, el dispositivo desde donde se conecta, si es un teléfono inteligente o un navegador de escritorio, la fecha en que se realizó el acceso y también lo más relevante el destino del usuario. Anteriormente mencionamos la importancia para nuestro trabajo tener un versión simplificada de la sesión. Al tener un simplificada la representación podemos usar mas datos y generar un modelo de datos para predecir, en este instante se comprenderá la importancia de el uso de un algoritmo de compresión de tipo  \losslessdatacompression, el cual nos permitirá crear modelos que sean creados con mayor volumen de datos que las técnicas tradicionales de \machinelearning y además usando propiedades de la compresión y la \emph{Teoría de la Información}, que nos entreguen resultados interesantes dado nuestro escenario de modelo discreto de predicción para navegación de  usuarios en una.
   
Esta nueva perspectiva posee una adaptabilidad a la demanda o proveer información que permita adaptarse a los eventos, por lo tanto, presenta una gran ayuda para conocer futuros eventos y ayudar a tomar decisiones, por ejemplo el siguiente acceso, basado en un mayor cantidad de datos históricos. Sobre esta idea se pueden hacer integraciones en áreas  \losslessdatacompression y \machinelearning siendo un gran desafío. Independientemente del área, el problema común  se puede resolver de manera acotada en cada área, pero se busca encontrar las propiedades en común y usarlas para lograr un resultado predictivo en demanda.  






%; moderado por administradores a ser  generado orgánicamente por los usuarios. Estas nuevas propiedades es una nueva evolución de \inet. 
% Enfocando nuestros interés en encontrar comportamientos de usuarios, se buscar predecir accesos web para dar la mejora continua de la \www. y evolución. Este trabajo no profundizará en tópicos completos de recuperación de la información (\emph{Information Retrieval}), ni en el procesamiento analítico de estos datos, pero si el estudio del aprendizaje de patrones e implementación tendrán un protagonismo en nuestra propuesta y se tendrá un completa sección de los conceptos básico~(\ref{ch:Conceptos-Basicos}) sobre \emph{\losslessdatacompression} y \machinelearning.
%%%%%%%%%%% evaluar si se migra a intro --- esta idea es como quiero usarlo on ... algo asi..


% bla bla
% Nos enfocamos en la arquitectura de un servidor de \machinelearning para realizar experimentos y alterar su funcionamiento. En las etapas de un estudio usando \machinelearning existe el entrenamiento de un conjunto de datos, un aprendizaje, una etapa de servir resultados y finalmente una evaluación.  Para el aprendizaje, es requerido tener un modelo, que funcione como un algoritmo de predicción y posteriormente pueda dar una interpretación de los datos que son entregados por este modelo.  

 
% Evaluar para intro
 % entrega una posible ayuda a ingenieros de desarrollo \www y diseñadores de experiencia de usuarios, como también en general a mejorar la experiencia del usuario, podemos mencionar que podría disminuir latencia en respuestas por parte de cada petición realizada a los servidores, con técnicas de \emph{pre-fetching predictivo} en el lado del cliente, pero para lograr esto debemos tener una registros de accesos válidos. 

 % sar representaciones eficientes como las realizadas por Claude \etal~\cite{Claude2014} facilitaría el estudio al tener un foco en las predicciones secuenciales y no en la recuperación de estos registros. El uso de esta  minería de datos realizada en un conjunto de datos real, radica en que día a día \www genera  innumerables cantidades de información, lo que conlleva a usar algoritmos que puedan operar de manera comprimida grandes volúmenes de información o representación más liviana de datos trabajar. 
% DEFINIR EL PROBLEMA

\section{Definición del problema}
% IDEA: me gustaria que esta sección  no se llamara DEFINICION DE PROBLEMA, intrinsicamente he hablado que el problema es como mejorar las predicciones secienciales usando lo mejor de dos áreas.



% esto creo que se repite en contexto o intro.
El problema de realizar modelos predictivos que minimicen su conjunto de entrenamientos, ha surgido hace años y diversos investigadores han trabajado con distintos enfoques. Los principales Rissanen\cite{Rissanen1983} y Langdom\cite{Langdon1983} en los laboratorios \emph{Bell}, al realizar pruebas y experimentar con un \emph{robot} que tiraba una moneda compitiendo con una persona, aquel robot realizaba todos los cálculos {markovianos} y las probabilidades condicionales para que cierto evento ocurra, a diferencia del sujeto que sólo estaba esperando un resultado aleatorio, la diferencia se marco en los costos de tiempo y de computo que se realizaron, por un parte la demora del \emph{robot} haciendo sus cálculos no mejoraba a la probabilidad aleatoria con que la persona a que enfrentaba 

Predecir no es trivial y requiere de gran cantidad información para poder realizar un modelo que logre abarcar varios escenarios reales, pero sí podemos llegar a acercarnos y minimizar el error  estaremos más cerca a predicciones exactas. Sin embargo, dos áreas han tratado de resolver el problema;  \losslessdatacompression~(\LDC) y \machinelearning de manera separada. Por parte de \LDC los mayores problemas son que los algoritmos predictivos  funcionan totalmente desconectados de la fuente de datos de entrada, esto implica que la validez del modelo solo es factible cuando esta realizando predicciones sin usuarios concurrentes o como se ha señalado anteriormente con un modelo con una componente \offline, lo que inhabilita rápidamente al modelo y en general no dan un resultado inmediato, en cambios en el esquema de \machinelearning debemos crear un modelo para entrenar y luego poder generar una función predictiva a lo cual se le suma un gran cantidad de datos para lograr un buen entrenamiento que produce un modelo bastante pesado para poder funcionar como modelo predictivo \online. 

Nuestro problema se acota a resolver predicciones secuenciales discretas con un modelo generado por un algoritmo de compresión, aplicado a un conjunto de datos discretos generados sintéticamente y otros real provistos en \emph{MSNBC}\cite{Claude2014}. 

Teniendo en funcionamiento un modelo de componentes híbrido juntando algoritmos y arquitectura de cada área para disponerlo como un servicio inmediato, es decir crear la componente \online del modelo predictivo con ayuda de \losslessdatacompression y el famoso algoritmo \texttt{LZ78}; dando una predictibilidad inmediata que hoy en la industria es necesaria para usar los datos recolectados y entregar nuevos enfoques a las decisiones basadas en esta perspectiva predictiva, como también dando un avance en el análisis de datos predictivos. 
Usando un algoritmo de tipo \LDC reemplazando en el proceso de aprendizaje de una arquitectura de  servicios  \machinelearning podemos alcanzar a una buena solución o a lo menos estar sobre el promedio aleatorio de ocurrencia de eventos.




\section{Contribución de esta Tesis}


En este trabajo,  nuestra principal contribución es el desarrollo del primer servidor de \machinelearning con un modelo predictivo de secuencias discretas, con funcionalidad \online y \offline para secuencias de  sitios \web. Adicionalmente nuestra implementación en conjunto \emph{PredictionIO}, es el primer servidor de \machinelearning usando un algoritmo \losslessdatacompression para realizar predicciones.



% de predicción con un algoritmo de \emph{Lossless Data Compression}. Hay varios temas propuestos en el que este trabajo pueda tener iteraciones  futuras de estas sistema. 

%
% ch2 ~\ref{ch:Compresion-Machine-Learning}
% ch3 ~\ref{ch:predicciones-webaccess}
% ch4 ~\ref{ch:experimetal-all}
% anexo ~\ref{ch:anexos}
%
\section{Estructura de la tesis} 




Para una mejor lectura y compresión del proyecto el documento ha sido organizado de la siguiente forma: 

 % En el Capítulo \ref{cap:ch1-intro} hemos realizado una introducción y  explicado un contexto preliminar que rodea este trabajo y hemos definido el problema que buscamos abordar, como las posibles aproximaciones de soluciones, también hemos dado un interés en el uso de algoritmo que puedan ser consumidos como un servicio \emph{web}.

En el Capítulo~\ref{ch:Compresion-Machine-Learning}, presentaremos los temas de \machinelearning y \losslessdatacompression, para llevar acabo un estudio de predicciones sobre \webasccesslog, para poder predecir la siguiente página web que un usuario accede.


% explicaremos todos los conceptos básicos para el entendimiento sobre esta investigación, como también conceptos para el uso de \emph{PredictionIO} y cerraremos con todos los trabajos relacionados más recientes que involucran nuestra investigación. Adicionalmente veremos como poder dar una inducción a un entorno de trabajo llamado \emph{PredictionIO},el cual nos ayudará a entregar los algoritmos como servicios consumibles, por cualquier aplicación cliente que pueda comunicarse con un servidor \emph{web}. 
% veremos las predicciones sobre \emph{webaccess log}, los modelos propuestos por varios investigadores y sus limitaciones, daremos una revisión del trabajo realizado por \emph{Rissanen}\cite{Rissanen1984} que da el inicio a esta área de Investigación.
% En el Capítulo \ref{ch:Compresion-Machine-Learning} veremos los temas de \emph{Machine Learning} y \emph{Lossless Compression Data} y como pretendemos crear un modelo predictivo con recursos de ambas áreas. Para cerrar este capitulo explicaremos el uso de del algoritmo Lempel \& Ziv, el uso para secuencias discretas su convergencia a un modelo de predicción eficiente y exacto. 

En el Capítulo~\ref{ch:predicciones-webaccess} se describe como crear un modelo predictivo usando un algoritmo de compresión y se presentan los fundamentos teóricos.





Finalmente el Capítulo \ref{ch:experimetal-all}, se presentará nuestros experimentos realizados con nuestra implementación. Posteriormente, veremos nuestras conclusiones y discusiones, como también nuestro trabajos futuros.


% sobre la implementación de un algoritmo de compresión en un servidor de \emph{Machine Learning}: \emph{PredictionIO}, analizaremos el comportamiento del algoritmo y como se desempeña este en distintos ambientes. propondremos discusiones de como mejorar nuestra implementación y los trabajos a futuro que puede presentar esta investigación.

% Explicaremos nuestras conclusiones obtenidas en base al desarrollo e implementación de la propuesta.
% Ademas dejaremos planteado nuestros trabajos futuros.


Además se adjunta un anexo, que corresponde a una guía básica para él uso de \emph{PredictionIO}, todos los datos y nuestra implementación se puede encontrar en nuestro repositorio \textbf{git público}\footnote{Repositorio público, \url{https://github.com/jaimeguzman/PredictionIO-LZmodel}}, en este se encontrarán los datos para replicar los experimentos que hemos realizado. 





% TODO: REALMENTE NO SE SI VALE LA PENA ESTO
%\input{context/capitulo-I-algoritimos-saas}  


% 
% 
% Capitulo Preliminar
% 
% 
% Se consideran y explican detalladamente 
% todos los conceptos, teorıas, y aspectos pertinentes al tema tratado.
% El marco teorico concuerda con los objetivos y el tipo de trabajo.
\chapter[Preliminar]{Preliminar}\label{ch:preliminar}

En esta sección se explicarán brevemente los conceptos principales que se usarán en los siguientes capítulos.
En este capítulo repasaremos brevemente los conceptos elementales que el lector ya debiera conocer, para evitar cualquier ambigüedad se han tomado ejemplos y referencias de autores que pueden consultar en la bibliografía usada para esta tesis. Luego introduciremos los trabajos relacionados a esta tesis y discutiremos las distintas aproximaciones al problema enunciado en el capitulo~\ref{ch1:intro}.

\section[Conceptos Básicos]{Conceptos básicos y trabajos relacionados} \label{sec:Conceptos-Basicos}


En esta sección se explicarán brevemente los conceptos principales que se usarán en los siguientes capítulos.




\subsection{Access Log}\label{concept-accesslog}
	Los \emph{Web Access Log} son registros que se almacenan en un servidor, los cuales dependiendo del servidor, sistema operativo y las configuraciones del ambiente pueden tener mayor o menor información. Cuando los usuarios acceden a diversos sitios \emph{web}, estos registros dejan una gran cantidad de información, un ejemplo se puede ver en la figura~\ref{fig:accesslog-apache-teleton}. Si se extraen de forma correcta se puede obtener patrones de navegación de usuarios. 

\subsection{Árboles Trie} \label{concept-trie}

	
Los \emph{trie} son un tipo de árboles conocido por muchos nombres incluyendo árboles de prefijos, árbol de búsqueda digital, árbol de recuperación (de ahí su nombre ``\trie'', por la palabra en inglés recuperación, \emph{retrieval}). Básicamente son estructuras de datos en forma de árbol que almacenan valores en sus nodos y es muy fácil recuperar la información. Se caracterizan por ser un conjunto de llaves que se representan en el árbol y en sus nodos internos se encuentra la información asociada a las llaves, en nuestro caso una secuencia de acceso de usuarios o un símbolo. 

Hay muchos tipos de árboles, uno de los populares son los árboles binario y los árboles balanceado (\emph{B-tree}). Cada árbol tiene una finalidad, estructura y comportamiento distintos, por ejemplo un árbol binario almacena una colección de elementos comparables ( números o cadenas de caracteres). Por lo tanto, se puede utilizar para almacenar un conjunto de números, o también  índices de otros datos que pueden ser representados por números (por ejemplo, objetos que pueden tener un cierto \emph{hash} de identificación). Su estructura está ordenada por lo que se puede buscar rápidamente un nodo. Otras estructuras de árbol, como un \emph{B-tree} son similares en principio al \emph{trie} que se implementará en la etapa experimental, ya que también la velocidad en la búsqueda del \emph{trie} es directamente proporcional a la altura y el balance de sus nodos.

Un \emph{trie} en este trabajo representa una estructurada de nodos, los cuales almacenan \emph{webaccess log}.  Es muy diferente cuando se almacena secuencias de valores, en lugar de valores individuales, la deducción de esto puede ser trivial debido a que entre menor es la secuencia de \emph{webaccess log} el nodo estará más cercano a la raíz, esto se profundizará en el Capitulo~\ref{ch:experimetal-all}. %Cada nivel representa un incremento en la altura del árbol, al igual .
%'¿cuál es el valor del punto I de la lista de entrada'. Esto es diferente a un árbol binario que compara el valor buscado único a cada nodo.


Durante este trabajo mostraremos que nuestro modelo de predicción usa un \emph{trie}, para representar un diccionario generado por un algoritmo de compresión, en los cuales podemos señalar las siguientes operaciones disponibles que serán útiles conocer. Consideremos $x$ una cadena de caracteres

\begin{itemize}	
	\menorEspacioItemize
	\item \textbf{findByPrefix(x):}  Retorna una lista de todos los nodos 	que se recorren hasta llegar al nodo que posea un \emph{webaccess log} o valor del nodo  equivalente a $x$.
	
	\item \textbf{contains(x):} Retorna una lista de nodos intermedios entre la raíz del \emph{trie} hasta el contenido del nodo sea hoja o intermedio que corresponda al valor de $x$.
	
	\item \textbf{remove(x):} Retorna \emph{true} o \emph{false} cuando es posible remover el nodo con valor equivalente a $x$.
	
	\item \textbf{pathTo(x):} Retorna una lista de nodos, representativa a la ruta recursiva para llegar desde la raíz a un nodo que posea un valor equivalente a $x$, representativo a un \emph{webaccess log}.
\end{itemize}


\subsection{Alfabeto} \label{concept-alphabet}
	
Un alfabeto es un conjunto ordenado de elementos. Una de sus características es que la cantidad de elementos puede ser un valor discreto. Representamos por $\Sigma$ al alfabeto y los elementos de este  alfabeto por $\sigma^{n}$ símbolos, tal que $\sigma^{n}= \sigma_{1}, \dots, \sigma_{n} \mbox{ tal que } \sigma_{i} \in \Sigma$, donde {$\sigma_{1}$}\label{concept-alphabet-homepage} es definido como la página inicial de una cierta secuencia  o sesión de usuario. Sea $|\Sigma|$ la longitud de secuencia o la cardinalidad del alfabeto daremos como restricción que $|\Sigma| \geq 2$ símbolos~\cite{Dmitry2002} necesarios para realizar experimentos acotados. Posteriormente en el capítulo experimental~(\ref{ch:experimetal-all}) usaremos un conjunto de datos de tamaño $|\Sigma| = 17\  \mbox{símbolos}$, para datos reales recolectados de \emph{MSNBC}\cite{Claude2014} y un $|\Sigma|$ variable para experimentos con datos sintéticos.

El alfabeto sera utilizado simbólicamente como la representación de varios nodos contenido en un \emph{trie} que {modela la navegación de usuarios}~(\ref{sec:nuestromodelopredict-mlldc}) para un sitio \emph{web}.

\subsection{Secuencias discretas}\label{concept-discret-seq}
	
Definimos una secuencia de accesos discreta y finita  $S$, dado ciertos accesos de usuarios que tiene una \emph{web}, lo anterior es acotado por el concepto de \emph{sesión de navegación}, que es desde que se inicia la el primer acceso, $\sigma_{1}$~(\ref{concept-alphabet-homepage}) por parte de un usuario hasta su que finaliza su navegación. Las secuencias de navegación serán de tamaño $S\ \leq 1$ y  no superior a la cardinalidad del alfabeto $|\Sigma|$.


% Compresor tonto o Machine Learning lento para predecir
\subsection{Lossless Data Compression (LDC)} \label{concept-LDC}
	La compresión sin pérdida o \emph{LDC}, es el arte de poder comprimir \emph{bits} y poder hacer el proceso inverso, en otras palabras,  poder codificar y decodificar. En la  sección~\ref{ch2:compressdata-predict-seq} del próximo capítulo 
se detallará más sobre el tema compresión y como  crear un modelo de predicción.



\subsection{Motor de Predicción}\label{concept-enginepredict}

	Los motores de Predicción son la parte fundamental de un sistema predictivo, dirigido a adivinar el futuro eventos dado un entrenamiento para un escenario o conjunto de datos en particular. La salida del motor de predicción es un evento siguiente. Para nuestro trabajo la salida se compone de un símbolo representando la dirección \emph{url} o una sección en particular de una cierta \emph{web}. 
	% que son propensos a ser solicitado por el usuario en las solicitudes posteriores.

 


\subsection{Resilient Distributed Datasets}\label{concept-RDD}
	Los \emph{Resilient Distributed Datasets} (\texttt{RDD}) permiten que un servidor o arquitectura de servidores de \emph{Machine Learning} puedan mantener los datos de un modelo de predicción, fuente de datos o valores de evaluación persistente sin importar en el flujo que sea invocado, esto es muy útil cuando se realizan procesamineto en \emph{clusters} y los datos se encuentran transversalmente sin tener pérdida procesamiento ni ejecución.

Esta colección es fundamental para \emph{PredictionIO} y \emph{Apache Spark}, que se verán en el Capitulo~\ref{ch:experimetal-all}. Esta estructura distribuida de almacenamiento de objetos inmutables separa los conjuntos de datos en un \texttt{RDD}\cite{zaharia2010}, que  es divido en particiones lógicas, las cuales puedes ser computadas en distintos cluster. Los \texttt{RDD} pueden contener cualquier tipo de objeto de los siguientes lenguajes: \emph{Python}, \emph{Scala} y \emph{Java}, incluyendo clases definidas por el usuario. 

Formalmente los \texttt{RDD} son solo de lectura como colecciones de objetos distribuidas. Pueden ser creados a través de  operaciones deterministas en una cierta tabla o un almacenamiento externo u otra \texttt{RDD}.
Otra de las características es que son colecciones tolerantes a fallas que pueden operar individualmente o en paralelo.
\emph{Apache Spark}\footnote{Apache Spark, \url{http://spark.apache.org}} hace el uso del concepto de \texttt{RDD} para lograr rapidez y eficiencia en las operaciones de \emph{MapReduce}, si es requerido. Destacamos la escabilidad de esta librería para un gran nivel de cómputo, pero en este trabajo no se explicará el fundamento de \emph{Apache Spark}, pero si se utilizarán algunos conceptos como \texttt{RDD} mediante el entorno de trabajo de \emph{PredictionIO}.







%TODO ver mas aca
\subsection{Data Source y Dataset }
	
Los \emph{Data Source} son distintas fuentes de datos que podemos ir sacando set de datos (\emph{Dataset}). Ambos conceptos están enfocados a proveer información de entrada tanto para el servidor de \emph{Machine Learning}, como para el procesamiento y análisis.
En este trabajo el conjunto de datos experimental con que realizaremos nuestro estudio son los registros de accesos de la \emph{web} española \emph{MSNBC}\cite{Claude2014}, los cuales representan aproximadamente de registros de \emph{webaccess log}.

Nuestro set de  datos esta basado en los registros (webaccess log), ya previamente depurados con una representación numérica desde 0 hasta 16, el cual como antes ya se ha señalado estará relacionado al concepto de alfabeto. También crearemos conjuntos de datos sintéticos para poder realizar depuraciones de nuestra implementación y casos de bordes.
	
	 




% Predictive Analysis

% Predictive analysis helps you predict what will happen in the future. It is used to predict the probability of an uncertain outcome. For example, it can be used to predict if a credit card transaction is fraudulent, or if a given customer is likely to upgrade to a premium phone plan. Statistics and machine learning offer great techniques for prediction. This includes techniques such as neural networks, decision trees, random forests, boosted decision trees, Monte Carlo simulation, and regression.


 



% \subsection{DataFrame}


	% A DataFrame is a distributed collection of data, which is organized into named columns. Conceptually, it is equivalent to relational tables with good optimization techniques.
	% Here is a set of few characteristic features of DataFrame −
	% Ability to process the data in the size of Kilobytes to Petabytes on a single node cluster to large cluster.
	% Supports different data formats (Avro, csv, elastic search, and Cassandra) and storage systems (HDFS, HIVE tables, mysql, etc).
	% State of art optimization and code generation through the Spark SQL Catalyst optimizer (tree transformation framework).
	% Can be easily integrated with all Big Data tools and frameworks via Spark-Core.
	% Provides API for Python, Java, Scala, and R Programming.



% easy text
% https://es.wikipedia.org/wiki/Trie
%Definición interpretada de esot
%\subsection{Cadenas de Markov}





 

\input{context/capitulo-I-related-work}



%intro general de ambos temas

\chapter[Machine Learning y Lossless Data Compression]{Machine Learning y Lossless Data Compression}\label{ch:Compresion-Machine-Learning}
%%%%%%%%%%%%%%%%%%%%%%%%%%%%%%%%%%%%%%%%
% Que busco inducir de machine learnig
% Que buscon inducir de LDC
% Como los cruzo
% Donde el lector puede ir buscando lo que necesita para comprender
% Una idea donde se junta la prediccion con la prediccion son los VMM

%%%%% SECCION DE MACHINE LEARNIGN PARA ANALISIS DE DATOS SECUENCIALES

Una de las áreas que comprende la Inteligencia Artificial es \machinelearning, en la cual existen algoritmos, técnicas y metodología que permiten realizar un entrenamiento a un sistema, con el fin de aprender desde un conjunto de datos. En los últimos años esta área ha tenido mayor atención, debido a la gran cantidad de datos que están disponibles y el mayor interés por encontrar información relevante dentro de estos datos. Anteriormente existían limitación de recursos, que en la actualidad el \cloudcomputing logra solucionar, de tal manera que se abren instancias para nuevas investigaciones y desarrollos. Adicionalmente al tener mejor infraestructura para funcionar, existe una gran demanda de soluciones a problemas y necesidades que la industria presenta. Esto produce  a que exista una constante competencia en la extracción de conocimiento, desde conjuntos de datos no procesados.

El conocimiento en este contexto, es a menudo definido como un modelo que puede ser actualizado o ajustado constantemente, a medida que nuevos datos se van generando. Los modelos nacen para satisfacer cierto dominio específico de determinados problemas, por ejemplo la evaluación del riesgo de créditos financieros, reconocimiento de rostros o patrones visuales, maximizar la calidad del servicio, clasificación de los síntomas patológicos para ciertas enfermedades, optimización de redes informáticas, detección de intrusiones en escenarios de seguridad informática y también se pueden analizar el comportamiento en línea de los usuarios e historial de compras ó navegación  de un cierta \web. Específicamente, un algoritmo de \machinelearning ``infiere patrones y asociaciones entre distintas variables, y conjunto de datos''~\cite[capítulo 8]{guller2015big}, en otras palabras, un algoritmo de \machinelearning aprende para predecir. 


Un modelo es una construcción matemática para capturar patrones de un conjunto de datos. Como señala \emph{Guller}~\MLGuller, básicamente es una función que toma ciertas características de un conjunto de datos, como sus valores iniciales y sus resultados.  Los modelos son las piezas fundamentales de cualquier implementación de \machinelearning, describen los datos observados de un sistema, en muchos casos, estos modelos se aplican a nuevos conjuntos de datos que ayudan a un nuevo modelo lograr aprender nuevos comportamientos y también predecirlos~\MLPDASunila. Se clasifican de la siguiente manera:

\begin{itemize}
\menorEspacioItemize		
 \item Modelos Lógicos
 \item Modelos Geométricos
 \item Modelos Probabilísticos
\end{itemize} 
%La aplicación de algún modelos debe ser definida por un caminio para lograr un objetivo, como el nuestro que es el de lograr predicciones con datos secuenciales. 

Cada uno de los modelos señalados son entrenados por algoritmos de \machinelearning, nos concentraremos en los modelos probabilísticos, los cuales pueden ser entrenados por algoritmos de \emph{regresión lineal}, \emph{forecasting} o \emph{predicción}. Todos los algoritmos mencionados anteriormente intentan de alguna manera determinar que pasará en el futuro, dado a cierta información provista de experiencia o conocimiento previo.

Esta sección se entrega un punto introductorio a ciertos fundamentos y referencias de algoritmos que son mencionados o para contextualizar al lector, estos se agrupan en la siguientes categorías:

\begin{itemize}
	\menorEspacioItemize
	\item \textbf{Regresión}
			% Referencias
% ~\cite{gollapudi2016practical} SUNILA


	El análisis de regresión nos permite modelar matemáticamente la relación entre dos variables utilizando el álgebra simple~\cite{gollapudi2016practical}.
	Es una técnica de clasificación que es particularmente adecuada para un modelo continuo, lineal (mínimos cuadrados), polinomio, y regresiones logísticas. Se encuentran entre las técnicas más utilizadas para mejorar o ajustar un modelo paramétrico. Se considera que la regresión es un caso especializado de la  clasificación, para los cuales las variables de salida son continuas en lugar de categórica.

% Regression techniques are used to predict response variables with numerical outcomes, such as predicting the miles per gallon of a car or predicting the temperature of a city. The input variables may be numeric or categorical. However, what is common with these algorithms is that the output (or response variable) is numeric. We’ll review some of the most commonly used regression techniques including linear regression, neural networks, decision trees, and boosted decision tree regression.
			\mediocm
	\item \textbf{Clasificadores}	
				El fin de un clasificador es extraer conocimiento específico de un dominio a partir de datos históricos. Por ejemplo, un clasificador puede ser construido para identificar una enfermedad a partir de un conjunto de síntomas. El científico recoge la información relativa a la temperatura del cuerpo (variable continua), la congestión (variables discretas alta, media y baja), y el actual de diagnóstico (gripe). Este conjunto de datos se utiliza para crear un modelo como si la temperatura> 102 y la congestión = ALTO ENTONCES paciente tiene la gripe (probabilidad 0.72), que los médicos pueden utilizar en su diagnóstico.


% Classification is a type of supervised machine learning. In supervised learning, the goal is to infer a function using labeled training data. The function can then be used to determine the label for a new dataset (where the labels are unknown). A non-exhaustive list of classification algorithms that can be used for building the model includes decision trees, logistic regression, neural networks, support vector machines, naïve Bayes, and Bayes Point Machines.

% Classification algorithms are used to predict the label for input data (where the label is unknown). Labels are also referred to as classes, groups, or target variables. For example, a telecommunication company wants to predict the following:

% Churn: Customers who have an inclination to switch to a different telecommunication provider
% Propensity to Buy: Customers willing to buy new products or services
% Upselling: Customers willing to buy upgraded services or add-ons
% To achieve this, the telecommunication company builds a classification model using training data (where the labels are known or have already been predefined). In this section, you’ll look at several common classification algorithms that can be used for building the model. Once the model has been built and validated using test data, data scientists at the telecommunication company can use the model to predict churn, propensity to buy, and upselling labels for customers (where the labels are unknown). Consequently, the telecommunication company can use these predictions to design marketing strategies that can reduce the customer churn and offer services to the customers that are more willing to buy new services or upsell.

% Other scenarios where classification algorithms are commonly used include financial institutions, where models are used to determine whether a credit card transaction is a fraudulent case or if a loan application should be approved based on the financial profile of the customer. Hotels and airlines use models to determine whether a customer should be upgraded to a higher level of service (such as from economy to business class, from a normal room to a suite, etc.).

% The classification problem is defined as follows: given an input sample of image, where x1 refers to an item in the sample of size d, the goal of classification is to learn the mapping image, where y ∈ Y is a class.

% An instance of data belongs to one of J groups (or classes), such as image. For example, in a two-class classification problem for the telecommunication scenario, Class C1 refers to customers that will churn and switch to a new telecommunication provider, and Class C2 refers to customers that will not churn.

% To achieve this, labeled training data is first used to train a model using one of the classification algorithms. This is then validated by using test data to determine the number of mistakes made by the trained model (the classifier). Various metrics are used to measure the performance of the classifier. These include measuring the accuracy, precision, recall, and the area under curve (AUC) of the trained model.
			\mediocm
	\item \textbf{Predictores}
			
	Ya extraída y validados ante los últimos datos del modelo, que puede ser utilizado para inferir de los datos futuros. Tomando el ejemplo de un médico recoge los síntomas de la un paciente, como la temperatura corporal o el avance de la enfermedad que esta tratando. Con los datos anteriores puede anticipar en el avanece de la enfermedad que se encuentra tratando a su paciente


	% \item[Optimización]	
	% 	
	No todos los problemas de optimización global son abordable usando métodos de optimización no lineal ó lineal tradicional. Mediante Machine Learning se puede mejorar las posibilidades de que el método de optimización converja hacia una solución (búsqueda inteligente). Imagínese que la lucha contra la propagación de un nuevo virus requiere la optimización de un proceso que puede evolucionar con el tiempo a medida que se descubren más síntomas y casos.



\end{itemize}
 
\uncm


Existen varios algoritmos con distintos estos fines, los anteriores no son la totalidad que ofrece el área de  \machinelearning. Todos los algoritmos pueden ir sufriendo variaciones o alteraciones acorde a como se presenten los datos. Estos deben conocerse para que en el proceso de entrenamiento del  modelo o el uso de estos sea válido. Es lógico pensar que si usa un algoritmo para un escenario que no corresponde este retorne resultado inconsistente, por ende, es importante conocer el dominio y seleccionar correctamente el algoritmo que se utilizará.


Al momento de realizar nuevos algoritmos o modificaciones que ayudan a algunos modelos tradicionales de \machinelearning, se abre un nueva perspectiva para abordar mas problemas con nuevos propósitos. \emph{Sculley y Brodley} en~\cite{Sculley2006} plantean y discuten esta nueva perspectiva con un enfoque más teórico e empírico. La idea fundamental que plantean es que si se tiene un \emph{string} $x$ e $y$, y  se comprimen juntos estos pueden ser más eficientes, es decir,  ambos \emph{strings} comparten la misma información. El caso anterior es la perspectiva de juntar las áreas de \machinelearning y \emph{Data Compression}. 

Si estos \emph{strings} representan grandes volúmenes de datos o largas secuencias de símbolos, usando algunas técnicas o algoritmos de compresión, pueden ayudar considerablemente a enfrentar este escenario, en el cual existe gran complejidad al momento de realizar el entrenamiento. Además, si se consideran estas aproximación, es posible entregar mejores resultado, siendo un beneficio considerable. En el ejemplo anterior como en otras tareas de \machinelearning se puede usar algoritmos de compresión, tanto en tareas tradicionales de clasificación ó agrupamiento.  Esta perspectiva inclusivas no es novedosa y ha aparecido en variados campos de interés~\cite{Sculley2006}, algunas veces con la promesa de reducir los problemas, por ejemplo cuando se realiza la selección implícita de ciertas características de un conjunto de datos. Otro trabajo interesante, es el que hacen los autores  \emph{Li}~\etal~\cite{Li2005}, que lograron implementar una nueva clase de \emph{string kernel}\footnote{Los String kernel, son operaciones que permiten trabajar funciones no lineales, como lineales. Para un explicación detallada ver~\cite{Li2005}.} basada en un algoritmo de compresión y lo utilizaron para manejar un clasificador \texttt{SVM} para la clasificación de géneros musicales. Uno de los algoritmos utilizado por \emph{Sculley}~\cite{Sculley2006}, para evaluar este espacio vectorial común, es de la familia de \lempelziv, más adelante se detallará en la  sección~\ref{ch2:sec-lzfamily}.
% AGREGAR EL EJEMPLO DE DETECCION DE SPAM TROLES

Como se ha señalado, una de las motivaciones que se puede tener al usar técnicas de compresión es ahorrar espacio para ciertos conjuntos de datos que se desean entrenar. Por otro lado también se puede ahorrar tiempo en transmitir o mover ciertos conjuntos de datos, lo que implica que finalmente en la mayoría de los datos se elimina la redundancia de información innecesaria.

Los algoritmos de compresión pueden ser con o sin pérdida de información, abordaremos solamente los algoritmo sin pérdida (\losslessdatacompression). Estos algoritmos en función de los análisis predictivo sobre secuencias han sido adaptados como \lzSieteOcho~\cite{ZivLempel1978} y Active Lezi\cite{Gopalratnam2007}. Por otra parte, ciertos algoritmos de \machinelearning, tales como redes neuronales y \emph{sequential rule mining} se han aplicado para realizar predicciones de secuencias~\cite{Gueniche2015}.

\emph{Begleiter}~\etal en~\cite{Begleiter2004} plantea que existe una  íntima relación entre predicciones  en secuencias discretas y \losslessdatacompression, donde en principio, cualquier algoritmo de compresión sin pérdida puede ser utilizado para realizar predicciones y viceversa (ver \emph{Feder y Merhav} en~\cite{Feder1992}).

También existen algoritmos que son \emph{modelos variables de Markov}, estos modelos están basados en Markov. Se detallará más en la sección~\ref{sec-clas-alg-compreessdata}, además se realizará una aproximación a los \emph{modelos ocultos de Markov} que son más asociados a aproximaciones con \machinelearning.




\mediocm
Hemos introducidos varios conceptos que veremos más adelante, uno de los más importante es el acercamiento de dos áreas para poder complementar una problemática específica y común. Hemos hablado de como ciertos modelos, y como son entrenados por algoritmos. Además hemos visto una primera revisión a la literatura en la cual ha validado a que podemos hacer esta intersección de áreas de conocimiento de manera correcta, para lograr el objetivo de esta tesis, que es las predicciones. En las próximas secciones de este capítulo se estudiará un base teórica para comprender el aporte que puede hacer un algoritmo \losslessdatacompression para un sistema de entrenamiento y predicción de \machinelearning.


 






%%%% todo sobre temas de Machine Learning


\section{Machine Learning para datos secuenciales}\label{ch2:sec-machinelearning-seq-data}
 En \machinelearning se realizan tareas de clasificación, agrupamiento o regresión, como ya hemos introducido. Uno de los modelos que se puede usar son modelos que usan probabilidades, es decir podemos realizar un entrenamiento con una distribución de probabilidades sobre una secuencia de datos. El concepto de modelo de Markov, se basa en la idea anterior, pero no solo se lograr realizar lo anteriormente señado, existe un universo de modelos de markov como \emph{Procesos de deciones de Markov}, \emph{Markov discreto}, \emph{Cadenas de Markov de Monte Carlo para redes Bayesianas}, y  \emph{Modelo oculto de Markov}.

El \emph{Modelo oculto de Markov} (\HMM) ha sido usado tanto en el reconocimiento de voz, traducción de idiomas, clasificación de texto, etiquetado de documentos y compresión. \HMM ha sido una gran herramienta para algoritmo de \machinelearning, se puede ver mas detalle de los trabajos y soluciones que ofrece en \emph{Rabiner}~\cite{}.





\subsubsection{La propiedad Markov}



Las propiedades de Markov es una característica del proceso estocástico, donde la distribución de probabilidades condicionales de un estado a otro depende directamente del estado actual y no de los estado en que alguna vez estuvo, en el pasado. Dado lo anterior se entiende que los estados se producen en un tiempo discret, y la propiedad de Markov es conocida como \emph{cadena de Markov discreta}. Aunque elos procesos de Markov se puede aplicar como ensayo y error en varias aplicaciones, su usabilidad es limita a la resolución de problemas para los cuales las observaciones no dependen de los estados ocultos. Los Modelos ocultos de Markov son una técnica comúnmente aplicado a afrontar este reto.






\subsubsection{Textos por revisar}

% The first-order discrete Markov chain

% Este trabajo se ha centrado en la generación de un modelo predictivo de secuencias discretas sobre un alfabeto finito, con las definiciones anteriores podemos profundizando como usar técnicas de Machine Learning, que nos permitan avanzar y estar en la búsqueda de nuevas comparaciones.

\begin{lstlisting}

El aprendizaje sobre datos secuenciales, como también el reconocimiento de patrones sigue siendo una de los desafíos del área de \machinelearning.
La literatura en estos temas es extensa y se ofrecen muchos acercamientos para análisis y predicción sobre secuencias en un alfabeto finito.

Una de las técnicas mas usadas son basadas sobre los \HMM, siglas en ingles de \hiddenmarkovmodels (Cadenas ocultas de Markov) CITA A RABINEER 1989 . Los \HMM nos ofrecen un estructura flexible que puede modelar distinto orígenes de datos secuenciales. Sin embargo, trabajar con los \HMM requieren una basta compresión en el dominio de problema, para poder modelar todas sus posibles restricciones.


Existen muchos problemas en que el factor de la secuencialidad de los datos se convierten en un principal actor. Hemos estado atacando un escenario en que la ocurrencia de los datos, sin ser afectos al tiempo, el orden que van ocurriendo generan puntos a desarrollar. También dado a la flexibilidad proporcionada, un entrenamiento exitoso requiere un gran conjunto de datos para ser entrenado.

 



Si diéramos una introducción al modelamiento secuencial, es necesario introducir modelos o los efectos probabilistas


De alguna manera hay que mencionar los VMM o Modelos de markov variables, son la base de los algoritmos de predicción usando LZ78



Aquí también se debe dar una intro pequeña a que se utilizará matlab, una de las validaciones que se espera es hacer correr el modelo de LZ para compararlo con los resultados típicos que tienen el LZ.

\end{lstlisting}




\section{Modelo oculto de Markov}\label{ch2:sec-hmm}
 %%%%%%%  
%%%%%%%  
%%%%%%%  

Los Modelos ocultos de Markov o en sus siglas en inglés \HMM (\emph{Hidden Markov Model})  básicamente son un modelo probabilistico basado en los procesos de Markov, también conocido como cadenas de Markov. Una cadena o proceso de Markov es una secuencia de eventos, estos se pueden representar por estados, la probabilidad de cada uno estados sólo depende del estado anterior.


Estos \HMM son modelos dinámicos con un proceso estocástico doble, en el cual el sistema es modelado mediante un proceso de markov con estados no observados directamente, es decir, estados escondidos. Aunque el proceso estocástico es subyacente y no directamente observable, este se puede observar sobre otro conjunto de procesos estocásticos, que producen una secuencia de simbolos observados. Normalmente han sido usados en reconocimiento de voz o habla, reconocimiento de patrones de imágenes ó video, traducción de textos, clasificación de texto, etiquetado de documentos y finalmente también en compresión de datos.

En los modelos tradicionales de Markov, los estados son visibles para el observador, y los estados de transición son visibles a un observador, y los estados de transición son parametizables, usando probabilidades transición. Cada estado tiene una distribución de probabilidad sobre la salida \emph{emitida} (variables observadas).
% Y QUE PASA CON LOS HMM se habla de los tradicionales falta mejorar esta idea




% \subsubsection{Terminología para HMM}


Para poder predecir la sequencia de estado más probable, los \HMM hacen un sistema de observación y transmisión. Una gran diferencia de los modelos ocultos de markov y un proceso de markov clásico es que en los primeros, los estados no son observables.  Una nueva observación se emite con una probabilidad conocida como la probabilidad de emisión cada vez que el estado del sistema o modelo cambios. 

\subsubsection{Fuentes de aleatoriedad para HMM}

Los estados en \HMM son modelos generativos, los cuales modelan la distribución conjunta de  observaciones y estados ocultos.

\begin{itemize}
	\menorEspacioItemize
	\item Transición entre estados.
	\item La emisión de una observación cuando se da un estado.
\end{itemize}


\subsubsection{Escenario formal HMM}

\begin{itemize}
	\menorEspacioItemize
	\item Un conjunto de observaciones
	\item Una secuencia de estados ocultos
	\item Un modelo que maximiza la probabilidad conjunta de las observaciones y estados ocultos, conocido como el modelo Lambda.
\end{itemize}


Un modelo Lambda,$\lambda$, está compuesto de probabilidades $\pi$ iniciales, las probabilidades de las transiciones de estado según la definición de la matriz A, y las probabilidades de los estados que emiten una o más observaciones.







%@TODO: Definir una notación matematica para empezar hablar de HMM


 


Si resumieramos los problemas fundamentales que pueden ser trabajados por \HMM, tendríamos los tres problemas siguientes:

\begin{enumerate}
	\menorEspacioItemize
	\item Evaluación de un modelo: Evaluar la probabilidad de una secuencia de observaciones para un \HMM dada  $(M = (A, B, M)).$
	\item Decodificar la ruta:  EVALUAR la secuencia óptima de un modelo de estados $Q$ (Estados ocultos) para una determinada secuencia de observaciones y un modelo \HMM $M = (A, B, M).$
	\item  Entrenamiento del modelo:   determinar el conjunto de parámetros del modelo que explica mejor la señal observada.

\end{enumerate}









 












%\vspace{1cm}

%\input{sec-algo-viterbi}





%%%% todo sobre temas de Algoritmos de compresión
\input{compression/capitulo-II-compress-data}



\chapter[Predicciones sobre Web Access]{Predicciones sobre Web Access}\label{ch:predicciones-webaccess}

%Predicting the future is mostly a matter of managing not to blink as you witness the present.
%William Gibson

% Predecir es una cosa. Predecir correctamente es otra.

%%%%%%%Gueniche_Fournier-Viger_Raman_Tseng
%Moreover, machine learning algorithms such as neural networks and sequential rule mining have been applied to perform sequence prediction [6, 11].

 

% introduccion a toda esta locura
\chapter[Introducción]{Introducción}
\label{ch:intro}

%chacharara
{
  El área de las ciencias de la computación que se encarga de estudiar todos los datos de maneras en que entregue información relevante, para poder ser estudiada es la Minería de Datos. Hoy en día esta área toma mucha relevancia al encontrarnos en una auge de la información generada por usuarios, redes sociales y variadas plataformas, podemos mencionar que esta es una de las razones para poder requerir disponer de herramientas de analisis que nos permitan saber como se comporta usuarios sobre una web, conocer la frecuencia en que se accede a un recurso en internet, etc.
  
  Sea el caso que entre una conexión cliente-servidor, una gran cantidad de servicios proporcionan datos de acccesos de los usuarios que acceden, sobre estos mismos existe se puede realizar un estudio sobre como predecir cual es la siguiente página que van a visitar.
  }

% Normalmente se ocupa muchos algorimot de maquinas de aprendizaje para poder hacer predicciones en variadas areas,

% Nuestro intere es hacer que las rpecicciones sean modeladas por un compresaor de datos basado en Lempel ziv



%%%%%%%%%%
% Uno de los interes en poder crear un un aplicación de ambas areas es la convergenia de las areas en la cual un proceso de aprendizaje ó predicción secuencial se puede usar con ML y Compress data
% Cuando se hace el pio train se hace un entrenamiento de modelo lo que resulta ser es que genera el arbol que es un trie de LZ que será el predicto
% Una función de predición implementada con MLIB y PIO sería predict () en scala




%%%%%%% CONTEXTO PRELIMINAR 
\section{Contexto Preliminar} \label{sec:preliminar}

  La Web crece constantemente y por ende su infraestructura, también la información que podemos obtener de los  usuarios y  concurrencia de los sistemas, la cual para usuarios finales se traduce en latencia y una mejor o peor experiencia de usuario. Paralelamente se suma un costo exponencial de recursos tanto en tecnologías de desarrollo como servicio que no son optimizados para poder dar una experiencia de usuario con calidad de servicio. Podemos reflexionar, entonces, que el no tener mayores recursos mejorará el rendimiento ni tampoco será lo óptimo para dar una calidad de servicio web, ya que el ancho de banda de Internet no crecerá a la misma proporción.
   
  Adicionalmente, las tecnologías para la creación de web dinámicas e asíncronas han evolucionado a favor de traspasar la carga cliente.
  Hoy en día ya se poseen leguanjes y framework que disminuyen considerablemente la carga de un servidor, por lo cual, un buen servicio web es proveer una balanceada carga dentro del cliente y el servidor, pero cuando se poseen un volumen de datos grandes es requerido tomar decisiones que los recursos y lenguajes no cubren, es ahí el interés de dar inteligencia a los servicios web.

  Predecir los futuros accesos que un usuario tendrá en una determinada web. Entendiendo que la manera en que un usuario navega es su comportamiento registrado en una web, y que se puede analizar, estudiar y registrar mediante \emph{Web Access Log} y a los cuales se puede hacer una minería de datos, Web Usage Minning. El por qué de hacer minería de datos es que cada día la web genera un innumerable cantidad de datos, por lo cual usar un algoritmo que se puedan operan comprimidos presenta un interés ya que además de disminuir el espacio físico o recursos utilizados, este se puede usar como un algoritmo de predicción y trabajar con una mayor cantidad de datos.
  
  Los registros de accesos de manera procesada o pre-procesada, ayudaría a ingenieros de desarrollo web y diseñadores, como a  usuarios finales a tener una experiencia de usuario mejor, disminuyendo por ejemplo la latencia en respuestas por parte de cada petición que realizan.
  
  Hoy en día, las web no pueden ser simplemente dinámicas en contenidos, debe poseer una adaptabilidad a la demanda del usuario o proveer información que permita adaptarse a los eventos, por lo tanto, es de interés el profundizar en este tópico.

  El interés en hacer un estudio sobre esto es poder hacer integraciones en áreas como compresión y la áreas que se dedican a completamente a hacer estudios sobre maquinas de aprendizaje. De por sí solas cada una se abordado independientemente lo cual es un interés converge en un problema en común que se puede resolver de manera eficiente.

  Durante este trabajo se usarán técnicas de compresión de datos, se utilizará una infraestructura y patrón de implementación para modelos de Machine Learning. Adicionalmente toda la experimentación se llevará acabo disponibilizando los algoritmos y modelos como servicio REST, el cual se explicará mas adelante, ya mencionado lo anterior este trabajo es implementable en áreas productivas las cuales pueden presentar interés.

  Definiremos que un usuarios es un la que se conecta a un servicio web, estos pueden ser paginas web informativas, redes sociales, etc. Este usuario establece una conexión directa con una pagina al momento de realizar esta operación, es posible almacenar datos los cuales vamos a llamar "access log" ó registros de accesos, durante el texto se mantendrán las referencias en ingles.

  Un ejemplo de access log podría ser el siguiente:


\begin{verbatim}
  172.31.33.116 - - [26/Nov/2015:00:12:11 +0000] "GET /wp-content/themes/corridacuprum/images/cuprum-footer.png HTTP/1.1" 200 4447 "http://virtual.corridacuprumteleton.cl/?utm_source=masterbase&utm_medium=mail2&utm_campaign=corrida-cuprum-teleton&utm_campaign=8474:%20%23!nombre!%23%2c+PARTICIPA+EN+LA+CORRIDA+VIRTUAL+Y+APOYA++A+LA+TELETON&utm_source=MasterBase%20CUPRUM&utm_medium=email&utm_content=2&utm_term=none" "Mozilla/5.0 (Linux; Android 5.1.1; SAMSUNG SM-G920I Build/LMY47X) AppleWebKit/537.36 (KHTML, like Gecko) SamsungBrowser/3.2 Chrome/38.0.2125.102 Mobile Safari/537.36"
  172.31.33.116 - - [26/Nov/2015:00:12:12 +0000] "GET /wp-content/themes/corridacuprum/images/principal-footer.png HTTP/1.1" 200 1784 "http://virtual.corridacuprumteleton.cl/?utm_source=masterbase&utm_medium=mail2&utm_campaign=corrida-cuprum-teleton&utm_campaign=8474:%20%23!nombre!%23%2c+PARTICIPA+EN+LA+CORRIDA+VIRTUAL+Y+APOYA++A+LA+TELETON&utm_source=MasterBase%20CUPRUM&utm_medium=email&utm_content=2&utm_term=none" "Mozilla/5.0 (Linux; Android 5.1.1; SAMSUNG SM-G920I Build/LMY47X) AppleWebKit/537.36 (KHTML, like Gecko) SamsungBrowser/3.2 Chrome/38.0.2125.102 Mobile Safari/537.36"
  172.31.33.116 - - [26/Nov/2015:00:12:12 +0000] "GET /wp-content/themes/corridacuprum/images/main.png HTTP/1.1" 200 179333 "http://virtual.corridacuprumteleton.cl/wp-content/themes/corridacuprum/style.css?ver=3.8.1" "Mozilla/5.0 (Linux; Android 5.1.1; SAMSUNG SM-G920I Build/LMY47X) AppleWebKit/537.36 (KHTML, like Gecko) SamsungBrowser/3.2 Chrome/38.0.2125.102 Mobile Safari/537.36"
  172.31.33.116 - - [26/Nov/2015:00:12:12 +0000] "GET /wp-content/themes/corridacuprum/fonts/akzidenzgrotesk-eb.woff2 HTTP/1.1" 200 24660 "http://virtual.corridacuprumteleton.cl/wp-content/themes/corridacuprum/style.css?ver=3.8.1" "Mozilla/5.0 (Linux; Android 5.1.1; SAMSUNG SM-G920I Build/LMY47X) AppleWebKit/537.36 (KHTML, like Gecko) SamsungBrowser/3.2 Chrome/38.0.2125.102 Mobile Safari/537.36"
  172.31.33.116 - - [26/Nov/2015:00:12:12 +0000] "GET /wp-content/themes/corridacuprum/fonts/akzidenzgrotesk-mc.woff2 HTTP/1.1" 200 24604 "http://virtual.corridacuprumteleton.cl/wp-content/themes/corridacuprum/style.css?ver=3.8.1" "Mozilla/5.0 (Linux; Android 5.1.1; SAMSUNG SM-G920I Build/LMY47X) AppleWebKit/537.36 (KHTML, like Gecko) SamsungBrowser/3.2 Chrome/38.0.2125.102 Mobile Safari/537.36"
  172.31.33.116 - - [26/Nov/2015:00:12:12 +0000] "GET /wp-content/themes/corridacuprum/images/lines-left.png HTTP/1.1" 200 4860 "http://virtual.corridacuprumteleton.cl/wp-content/themes/corridacuprum/style.css?ver=3.8.1" "Mozilla/5.0 (Linux; Android 5.1.1; SAMSUNG SM-G920I Build/LMY47X) AppleWebKit/537.36 (KHTML, like Gecko) SamsungBrowser/3.2 Chrome/38.0.2125.102 Mobile Safari/537.36"
  172.31.33.116 - - [26/Nov/2015:00:12:12 +0000] "GET /wp-content/themes/corridacuprum/images/lines-right.png HTTP/1.1" 200 4841 "http://virtual.corridacuprumteleton.cl/wp-content/themes/corridacuprum/style.css?ver=3.8.1" "Mozilla/5.0 (Linux; Android 5.1.1; SAMSUNG SM-G920I Build/LMY47X) AppleWebKit/537.36 (KHTML, like Gecko) SamsungBrowser/3.2 Chrome/38.0.2125.102 Mobile Safari/537.36"
  	
  \end{verbatim}

  El ejemplo anterior nos da mucha información interesante como la IP desde donde se conecta, el tipo de navegador, el dispositivo si es un telefono inteligente o un navegador de escritorio, la fecha en que se realizo el acceso y también lo mas relevante el destino del usuario.








% @TODO: SEGUIR TRABAJANDO EN ESTA BREVE INTRODUCCION
%En este tema convergen tres áreas, por un lado existe trabajo para crear estructuras eficientes para predicciones basadas en algoritmos de compresión, como es en el caso de~\cite{Claude2014}, y, por otro lado, el uso de algoritmos de aprendizaje para realizar clustering y predecir el comportamiento basado en el mismo contenido o en la distancia del contenido que visita el usuario actual al contenido clusterizado, como es el caso de ~\cite{Poornalatha2012}, inclusive se han utilizado modelos de Markov en ~\cite{Dongshan2002}  para poder modelar el comportamiento de la web.
%La tercera área son los Sistemas de Recomendación, la cual en este proyecto no se tocará pero si se mencionará el enfoque práctico que presenta área como un foco de múltiples implementaciones. 







%Algoritmos como serivcio
\section{Algoritmos como servicio web }

	Los avances en el desarrollo de nuevas tecnología que brinden mejores experiencias en su uso del día a día, deriva en cómo podemos llevar varios escenarios idealizados a implementaciones empresariales reales. Es bastante común encontrar librerías que son bastante útiles para hacer Minería de Datos, Clustering y muchas operaciones que pueden recurrir en cálculos muy complejos pero no se pueden ofrecer como servicio. Ya en auge de las infraestructuras Cloud, la capacidad de computo que se puede alcanzar no es un problema a lo que antes se enfrenta un Cientista de Datos.


	Una API es un interfaz de programación de Aplicaciones que nos permiten intermediar el Servicio A con el Servicio B. Respectivamente A puede ser el proveedor y B el Demandante de servicios. Si quisiéramos analizar datos que se encuentran dentro de un servidor especifico, estos se podrían consumir por esta interfaz. Existen variados clientes que nos permiten ayudar a esta comunicación, incluso se pueden utilizar por una terminal de Unix la cual es posible que mediante el programa $curl$, el cual permita dialogar con un determinado servicio web.
	
	Ya se dispone de infraestructura como servicio, software como servicio, plataformas como servicios. Dado lo anterior ofrecer estos algoritmos para hacer que las soluciones de desarrollo den valor agregado a la experiencia requerida por el usuario. Así es que hemos decidido utilizar una librería y framework que nos de esta posibilidad. Ofrecer algoritmos a la industria como un servicio que ayuda de manera inteligente y multiplataforma que es el caso de una API. 
	
	Todas las ventajas de este patrón son heredados de las características que ofrece una API, interoperabilidad, evitar problemas de Infraestructura, Resiliencia de Datos, Persistencia de Datos, Análisis y Procesamiento sin afectar un curso operacional de una aplicación. Un ejemplo claro de esto es el análisis de datos en sistemas legados los cuales en plan de mejoras, no poseen la compatibilidad para poder realizarlo. Por otro lado, los algoritmos de compresión o algoritmos de Machine Learning tienden ocupar recursos y este hecho pueden ser la razón para no implementarlos. 
	
	
	
	




%introducción a las Prediccion


\section{Predicción }

  	
En esta sección se presenta formalmente el ambiente de desarrollo que se utilizará durante este trabajo. \emph{PredictionIO} es un servidor de \emph{Machine Learning} de código abierto para Científico de Datos y Desarrolladores que permite crear motores de predicción para aplicaciones en producción, con un bajo tiempo de entrenamiento y despliegue. Principalmente está construido en \emph{Apache Spark, HBase} y \emph{Spray}. 

Este ambiente de trabajo se encuentra en un maduración estable y constante que permite tanto disponer servidores con motores predictivos, como también toda una infraestructura distribuida para hacer que complejos algoritmos sean utilizados para solucionar problemas reales.



% PIO, tiene practicamente todo armadao, ello no hicieron nada nuevo ... solo juntaron  todo...



% He estado investigando y revisando documentación, Yelp, Skype, Hubot de github y otras implementaciones tienen usando prediction.io




% La otra opción es meterle a este "DASE" un algoritmo  que mezcle una representación de cadenas de markov mezclado con LZ78. no se en que punto mezclarlo en el diccionario, o la verdad es como hacer el compresor sea "mas inteligente", encontré un papaer que te adjunto en el cual usan lz78 y lzw, esta interesante ya que le hacen un acercamiento mas al tema de de ser un predictor online.


% Ahora entiendo que la cadenas de markov son y se han ocupado para las predicciones, pero no veo la necesidad de ocuparlas mayormente. Adin me inisiste en que le de una vuelta.... pero mi sensación  es que tengo separada las ideas en dos extremos. 

% Ya revise Suffix Tree para predicciones LZ77 y LZW, también PPM y HMM.

 
 



\subsection{Arquitectura DASE}


Un motor de predicción es un tipo de proceso en \emph{Machine Learning}. Siguiendo una arquitectura de tipo \emph{DASE}, contendríamos los siguientes componentes.



\begin{itemize}

  \item\label{dase-datasource} \textbf{ $[D]$ Data Source y Data Preparator}. Los Data Source leen la data desde la entrada original y la transforman en un formato deseado para hacer análisis de estos. En cambio \emph{Data Preparator} pre-procesa la información y la reenvía a los algoritmos para   hacer el modelo de entrenamiento.


  \item\label{dase-algoritmo} \textbf{ $[A]$ Algoritmo}. Los componentes de \emph{PredictionIO}, dada sus librerías incluyen algoritmos de \emph{Machine Learning}, estos, pueden ser provistos por \emph{Apache Spark} o se pueden incluir algoritmos propios como también de terceros.
    Adicionalmente a los algoritmos podemos asignarle parámetros, para determinar como debiese ser construido el motor ó si es requerido para un cierto algoritmo.



  \item\label{dase-servicio} \textbf{ $[S]$ Servicio}. El componente servicio toma las consultas ó \emph{queries} de predicción y retorna los resultados, en nuestro modelo propuesto en la etapa experimental veremos el siguiente símbolo de una secuencia. 
  Si el motor de predicción tiene múltiples algoritmos, combinará los resultados en uno. Adicionalmente, la lógica específica de negocios puede ser añadida para especificar aún más el resultado final. 
 
  \item\label{dase-eval} \textbf{ $[E]$ Evaluación de Métricas}.
Las métricas de evaluación cuantifican la precisión de la predicción con una puntuación numérica. Puede ser utilizado para la comparación de algoritmos o ajustes de los parámetros del algoritmo.
\end{itemize}




  \tikzstyle{decision} = [diamond, draw,text width=4.5em, text badly centered, node distance=2.5cm, inner sep=0pt]
  \tikzstyle{block} = [rectangle, draw,text width=5em, text centered, rounded corners, minimum height=4em]
  \tikzstyle{line} = [draw, very thick, color=black!50, -latex']
  \tikzstyle{cloud} = [draw, ellipse, node distance=2.5cm,
  minimum height=2em]


\begin{figure}[t]
	\centering	
	\resizebox{0.8\textwidth}{!}{% <------ Don't forget this %
	
		\begin{tikzpicture}[scale=1, node distance = 2.5cm, auto]
		% Place nodes
		
		\node [block] (init) {Data de la Aplicación};
		\draw [color=gray,thick](1.3,1) rectangle (11.2,-3.5);    
		
		\node [block,right of=init] 		  (datasource) {Data Source};
		\node [block,right of=datasource]     (datapreparator) {Data Preparator};
		\node [block,right of=datapreparator] (alg1) {Algoritmo };
		\node [block,right of=alg1] 		  (serving) {Serving};
		\node [block,below of=serving] 		  (evalmetric) {Evaluación Métrica};
		\node [block,right of=serving] 		  (resultpredict) {Resultado Predicción};
		\node [block,below of=resultpredict]  (resulteval) {Resultado Evaluación};
		
		\path [line] (init) -- (datasource);
		\path [line] (datasource) -- (datapreparator);
		\path [line] (datapreparator) -- (alg1);
		\path [line] (alg1) -- (serving);
		\path [line] (serving) -- (evalmetric);
		\path [line] (serving) -- (resultpredict);
		\path [line] (evalmetric) -- (resulteval);    
		
		\end{tikzpicture}
	}
	\caption{Diagrama de componentes Arquitectura DASE.}
	\label{fig:arquitectura-dase}
\end{figure}



\vspace{1cm}

\emph{PredictionIO} ayuda a tener componentes muy modulares, las que ya hemos descrito como  arquitectura \emph{DASE}(\ref{fig:arquitectura-dase})  que puede construir modelos de predicción de manera sencilla. También poder integrarlos con gran facilidad a cualquier sistema o plataforma, por ejemplo, es posible elegir cual de todos los componentes se podrá desplegar al momento de crear un \emph{Engine} (Motor de Predicción.)



\vspace{1cm}
\subsection{Despliegue de motor de predicción}

  Un \emph{Motor de predicción} pone todos los componentes del diseño de arquitectura \emph{DASE} en un estado especifico de despliegue 
  \begin{enumerate}
  		\setlength{\itemsep}{1pt}
  		\setlength{\parskip}{0pt}
  		\setlength{\parsep}{0pt}
    \item Data Source
    \item Data Preparator
    \item Uno o más Algoritmos generadores de Modelos
    \item Un Servicio 
  \end{enumerate}

  Si se especifica más de un algoritmo, cada uno de los resultados de los modelos de predicción se entregará para ser consumido por cualquier cliente.
  Cada \emph{motor de predicción} procesa los datos y construye un modelos de forma independiente. Por lo tanto, todos los motores de predicción que usaremos sirven a su propio conjunto de resultados. Por ejemplo, se puede desplegar dos \emph{motores predictivos} para una aplicación móvil: uno para recomendar noticias a los usuarios y otro para sugerir nuevos amigos a los usuarios.


\vspace{1cm}
\subsection{Evaluación del motor de predicción }

  Para evaluar el \emph{Accuracy} de un motor de predicción, se debe especificar la métrica seleccionad cuando se corre el motor de evaluación, en los capítulos experimentales se verá como se generan métricas y como se desempeña esta métrica para ser evaluada.











\subsection{Modelamiento de eventos}

%https://docs.prediction.io/datacollection/eventmodel/




  El modelamiento de eventos es fundamental para un predictor \emph{online}, el hecho de poder llevar un vector con ciertas propiedades característica\footnote{Característica de un cierto dataset para entrenar.} de una representación vectorial del mundo del \emph{Machine Learning} a un modelamiento secuencial, es en realidad el modelamiento que se debe realizar de como  tener las muestras de datos para generar \emph{Resilient Distributed Dataset} (RDD) que son un parte principal de nuestro ambiente de trabajo con \emph{PredictionIO} para poder acceder posterior o inmediatamente. 

  Un evento lo definiremos como entidad que nos permite dar una representación temporalizada de información que será procesada por un motor de predicción. Analizaremos los eventos que un usuarios realiza para poder acceder a una web. Adicionalmente cuando cada usuario ingresa a una web automáticamente este genera una sesión, desde que que llega hasta que abandona la web.

  Usaremos un \emph{dataset} con información que esta totalmente depurada y recuperada de los access log (provista por Claude \etal~\cite{Claude2014}), los cuales a efectos de temporalidad solo nos interesa conocer la secuencialidad de estos accesos.
  


%Poner algo mas matematico.
% EVENT API 
% https://docs.prediction.io/datacollection/eventapi/

  El modelamiento que realizaremos contempla los siguientes campos :
  \begin{itemize}
    		\setlength{\itemsep}{1pt}
    		\setlength{\parskip}{0pt}
    		\setlength{\parsep}{0pt}
      \item Tipo de Evento: Visitar
      \item Entidad que ejecuta el evento: Usuario
      \item Propiedades:
          \begin{enumerate}
          		\setlength{\itemsep}{1pt}
          		\setlength{\parskip}{0pt}
          		\setlength{\parsep}{0pt}
            \item Página actual
            \item Página siguiente
            \item Cierre de Sesión
          \end{enumerate}
    \end{itemize}



    El interés de tener un modelo totalmente atómico es poder contemplar la información que nos entrega, destacando sus variables y propiedades como restricciones.



\subsection{Ventajas }


  Es posible mezclar y aplicar distintas característica, si el modelo no puede ser persistido por \emph{PredictionIO} automáticamente. Se requiere un objeto  heredado de una clase que permita lograr la persistencia en memoria, esto permite cargar el modelo persistentemente e instanciar automáticamente durante la ejecución del motor de predicciones.

  % Tenga en cuenta que los modelos generados por PAlgorithm no pueden ser persistieron automáticamente por naturaleza y deben implementar estas características, si se desea modelo de persistencia.


  Comprendiendo el concepto de \emph{Resilient Distributed Dataset} (RDD), esta es la abstracción básica de \emph{Apache Spark}, aún más, esto es una de las grandes cualidades de PredictionIO, ya que no solamente podemos disponer de una máquina para hacer estudios o implementar algoritmos, este servidor de \emph{Machine Learning} permite, gracias a sus componentes poder hacer un cluster para entregar mayor eficiencia acorde a los datos o algoritmo a implementar.

  Ya hemos mencionado que un \emph{Resilient Distributed Dataset} (RDD) es una representación inmutable, una colección particionada de elementos que pueden ser operadas en paralelo. Internamente cada RDD tiene cinco principales propiedades:


  \begin{enumerate}
    \item Una lista de particiones.
    \item Una función para procesar cada porción de datos.
    \item Una lista de dependencias en otros RDD. 
    \item Opcionalmente una partición de un RDD puede ser representada como una combinación de llave y valor. 
    \item Opcionalmente, una lista de los lugares preferidos para calcular cada una dividida en (por ejemplo, lugares de bloque para un archivo \emph{HDFS}), para procesamiento en \emph{Clustering}.

  \end{enumerate}




% https://docs.prediction.io/templates/recommendation/customize-serving/




















%@TODO:
% Este tema debería detallarse en las siguientes secciones
\section{Literatura}
En la literatura, el tema de la predicción en la web se ha presentado como un tema concurrente, y ha sido abarcado por varios autores. Tenemos los siguientes trabajos de interés:

\begin{enumerate}
  \item Dongshan y Junyi~\cite{Dongshan2002} destacan que un modelo de Markov puede ayudar a predecir el comportamiento de un usuario, pero con ciertas limitaciones .  Para solucionarlo presentan un nuevo modelo de Markov basado en una representación de \emph{Tree Order Model}, el cual es un híbrido entre un modelo de markov tradicional y una representación de árbol, bautizada como HTMM (por sus siglas en inglés, \emph{Hybrid-Order Tree Markov Model}).
  Su modelo fue presentado en 2002, y da una importancia a conocer la predicción de los \emph{web access}, dada la importancia de creación de redes, la minería de datos, e-commerce, y otras áreas.

  \item Domenech \etal~\cite{Domenech2006}, muestran un estudio de los rendimientos de técnicas de recuperación de datos.
  Las mismas se pueden utilizar para dar una entrada ideal a algoritmos de aprendizaje o algoritmos de predicción. 
  Los conceptos más importantes son las nuevas variables de caracterización, temporalidad, espacio y geografía, que se le suman a la predicción. 
  Además de comenzar un trabajo más elaborado de como tomar una predicción, se introducen conceptos como predicciones genéricas o específicas, variables de uso de recursos a nivel de red ó nivel procesamiento.
  Finalmente, se presenta un modelo predictivo que puede ayudar a disminuir la latencia entre la petición del cliente y la respuesta de la web, dando así un mejor rendimiento y \emph{QoS}.


  % @TODO detallar más explicarlo mas simple, darle mas enfoque al usuario segúnn del punto de vista que de los docuentos 
  % como los autores antteriores.

  \item Chen \etal~\cite{Chen2011} dan una nueva perspectiva enfocada a entregar una clara recomendación a los usuarios basada en la misma propuesta de este proyecto, los access log.
  El primer análisis realizado por los autores cubre las reglas asociativas que requiere un sistema de recomendación, pero en las pruebas propiamente tales encuentran que el análisis de los patrones detectadados dan una representación clara de como optimizar la web, y finalmente mediante sus pruebas logran una recomendación de calidad.

  \item Rajimol y Raju~\cite{Rajimol2012} minaron los patrones de los accesos web, donde el enfoque es usar los registros de acceso para crear subsecuencias y realizar comparaciones.
  La literatura presenta un interés para poder anticipar el patrón de comportamiento de la web.
  % @TODO reflexionar mas sobre este paper

  \item Kewen~\cite{kewen2012} realizó un análisis más profundo del \emph{web usage minning}.
  Parte de la importancia de este trabajo, es que después de minar los registros de accesos, logran reducir la ``\emph{bad data}''.
  %@TODO: Preguntar si este paper se escapa mucho del tema prinicipal, pero parece interesante  

  \item Poornalatha y Raghavendra~\cite{Poornalatha2012} establecen que se pueden utilizar máquinas de aprendizaje para predecir basándose en distintas entre clusters. Estos autores, al igual que Domenech \etal~\cite{Domenech2006} y Dongshan y Junyi~\cite{Dongshan2002}, comparan el objetivo de optimizar los recursos tanto en redes (disminución de latencia) y experiencia de usuario.

  \item Claude \etal~\cite{Claude2014} presentan una estructura de representación eficiente que permite dar una representación de \emph{web access log} y ofrecen las operaciones básicas de WUM.
  
\end{enumerate}



\section{Descripción del Contenido y Contribuciones}


\section{Predictores de estado finito}
	% Referencias
% 
% ~\cite{Begleiter2004}
% ~\cite{}
%
%
% Seccion preliminar para VMM de ~\cite{Begleiter2004}
% https://dl.dropboxusercontent.com/spa/srrxi5k8b49mdt9/m-1owunl.png
% https://dl.dropboxusercontent.com/spa/srrxi5k8b49mdt9/sbq0f8f9.png


Sea $\Sigma$ un alfabeto finito y $q_{i}^{n}$ el $i\mbox{-ésimo}$ símbolo $q$ de un conjunto total de $n$ símbolos. Para entrenar un secuencia $q_{1}^{n}=q_{1},q_{2} \dots q_{n}$, donde $q_{i} \in \Sigma$ y el símbolo compuesto $q_{i}q_{i+1}$, que es la concatenación de $q_{i}$ y $q_{i+1}$, debemos tener un algoritmo que permita leer y usar esta secuencia, además entrenar con los datos procesados y entregar un resultado predictivo con el modelo que se desea generar. El objetivo es entrenar un modelo $M$ que entregue como resultado la probabilidad condicionada para cualquier futuro símbolo dado a una sub-secuencia que ha sido parte del entrenamiento previamente.
Definimos $\Sigma^{*}$ como un conjunto de sub-secuencias de símbolos $\sigma$ que componen $s$, talque $s \ \in \Sigma^{*} $. Específicamente, en cualquier contexto de secuencia $s$ de $\Sigma^{*}$  y  $\sigma \in \Sigma$ símbolos, el aprendizaje de la secuencia dado por el entrenamiento debe dar una distribución de probabilidad $M(\sigma | s )$ que su valor resultado es representada por el modelo $M$ al momento que es consultado o evaluado para predecir.

El rendimiento del modelo predictivo se puede medir mediante una función del promedio de registro de errores $L(M,x_{1}^{T})$ de $M (\cdot | \cdot )$, con respecto a una secuencia $s = x_{1}^{T}$ con $x_{1}^{T}= x_{1},x_{2},....,x_{n} $ %, al acierto del siguiente símbolo $q_{i}$ de la secuencia $s$
  por lo tanto podemos definir $L$ como \begin{equation} L( M , x_{1}^{T} ) = 
- \dfrac{1}{T} 
\sum _{i=1}^{T} \log{ M(x_{i} | x_{i} \cdots x_{i-1}} ),\end{equation}
donde el logaritmo es en base $2$.  El promedio de $L$ es directamente relacionado a  \begin{equation}M(x_{1}^{R}) = \prod_{i=1}^{T} M(x_{i} | x_{i} \cdots x_{i-1} ) \end{equation} y minimizar el promedio de $L$ es completamente equivalente a maximizar la asignación de probabilidades a una secuencia de pruebas $x_{1}^{T}= x_{1},x_{2},\cdots,x_{n} $, teniendo en cuenta que esta equivalencia es totalmente válida. 
Sea $M(x_{1}^{T})$ una asignación de probabilidad consistente para una secuencia completa, la cual satisface \begin{equation}
M(x_{1}^{t-1}) = \sum_{\mbox{$x_t$} \in \Sigma}^{T} M(x_{1} \cdots x_{t-1}x_{t} ) \ , 
\end{equation}para todo $t=1,\cdots\ ,\ T$, induce la asignación de probabilidad,

\begin{equation}
M(x_{t} | x_{1}^{t-1} ) =  \dfrac{M(x_{1}^{t})}{M(x_{1}^{t-1} )},\ t=1,...,T.
\end{equation}


El registro de pérdida $L$ tiene variadas interpretaciones. Tal vez la más importante se encuentra en su equivalencia a \LDC. La cantidad $-\log M (x_{i} | x_{1} \cdots x_{i-1}) $, que también se llama la ``auto--información'', puede ser la de compresión ideal o ``largo de secuencia'' de $x_{i}$, en \emph{bits} por símbolo, con respecto a la distribución de probabilidad condicional  \begin{equation}M (X | x_{1} \cdots x_{i-1}) \ ,\end{equation} esta puede ser  \online(con  una pequeña redundancia arbitraria) usando codificación aritmética (Rissanen y Langdon, 1979)\cite{RissanenLangdon1979}.


Por lo tanto, el promedio de $L$ también mide la tasa de compresión media de una secuencia de prueba, cuando se utilizan las predicciones generadas por $M$, es decir, un bajo promedio de $L$ %log-loss function
sobre la secuencia $x_{1}^{T}$ puede implicar una buena compresión de esta secuencia \cite{Begleiter2004}.

%\footnote{Mas adelante usaremos el término en ingles \emph{Data Source}
Si se supone que el entrenamiento y las secuencias de pruebas fueron generados de una fuente desconocida, para referirnos a fuentes de datos , tanto conocidas como desconocidas. $P$. Definimos una secuencia dada por valores aleatorios $X_{1}^{T} = X_{1} \cdots X_{T} $, podemos decir que claramente la distribución $P$ minimiza unicamente \emph{log-loss} o como la hemos llamado anteriormente $L$, lo cual es \begin{equation}
P = arg\ min_{M} \{ - E_{P} \{\log M( X_{1}^{T} )\}   \}
\end{equation}


Dada la equivalencia de \emph{log-loss} y la compresión, como se ha visto anteriormente, el significado de \emph{log-loss} de $P$ logra la mejor compresión posible, o logra una entropía 
\begin{equation}
	H_{T}(P) = - E \log P( X_{1}^{T} ) 
\end{equation}
Aún no conociendo realmente cual es la distribución de probabilidad de $P$, un entrenamiento genera una aproximación a $M$ usando una secuencia de entrenamiento. La pérdida extra que podemos obtener la llamaremos \emph{Redundancia} y esta dada por el valor de

\begin{equation}
D_{T} ( P || M ) = E_{P} \{ - \log M(X_{1}^{T} - (- \log P(X_{1^{T}})   )  )       \} \ .
\end{equation}
 

Para normalizar la \emph{redundancia} $D_{T} ( P || M ) / T $, de una secuencia de largo $T$, da los  \emph{bits} extra por símbolo (sobre la tasa de la entropía) al comprimir una secuencia utilizando $P$.  

Este ajuste probabilístico motiva un objetivo deseable, al entregar un algoritmo de propósito general para la predicción: minimizar la redundancia de manera uniforme, con respecto a todas las posibles distribuciones. 

Un algoritmo de predicción el cual pueda acotar la redundancia de manera uniforme, con respecto a todas las distribuciones dada una clase debe poseer un cota inferior de redundancia para cualquier \emph{Predictor Universal} y \emph{Compresor Universal} \begin{equation}
\Omega \left(  K \dfrac{\log T}{2 T } \right),
\end{equation} donde $K$ es (más o menos) el número de parámetros del modelo que codifica la distribución $P$ (Rissanen \cite{Rissanen1984}, 1984).






 

Si llamamos al siguiente símbolo $b_{t}$, diremos que el resultado de nuestro predictor es entregar este valor. Dado esto, existe una función de pérdida asociada $L( b_{t},x_{t} )$ para cada predicción realizada. 

El objetivo de cada predictor es tener una función de minimización tal que minimize la fracción de predicciones erróneas, a lo anterior lo llamaremos $T$.% que será:












% TODO:
% Concluir y conectar con el que sigue

\textbf{CONECTOR pendiente a la siguiente subseccion}

\section{Modelos tradicionales}
	%Gueniche_Fournier-Viger_Raman_Tseng
%However, these models suffer from some important limitations [5]. First, most of them assume the Markovian hypothesis that each event solely depends on the previous events. If this hypothesis does not hold, prediction accuracy using these models can severely decrease [5, 3]. Second, all these models are built using only part of the information contained in training sequences. Thus, these models do not use all the information contained in training sequences to perform predic- tions, and this can severely reduce their accuracy. For instance, Markov models typically considers only the last k items of training sequences to perform a pre- diction, where k is the order of the model. One may think that a solution to this problem is to increase the order of Markov models. However, increasing the order of Markov models often induces a very high state complexity, thus making them impractical for many real-life applications [3].

%
%Primero asumimos que cada evento por si 
 


Se ha estudiado como los eventos secuenciales que se obtienen de usuarios \emph{web} cuando navegan se puede modelar, pero reconocer la frecuencia de estos patrones de \emph{webaccess} puede requerir mucha  información de entrada y un alto costo procesamiento.  Indudablemente puede entregar mejor precisión en el resultado de como es el comportamiento de los usuarios, pero no es óptimo para las demandas actuales. 

Estos enfoques tradicionales usan Modelos de Markov, a medida que va aumentando el historial de navegación secuencial el orden de complejidad aumenta directamente proporcional. Lo anterior genera cálculos  complejos y lentos de procesar. Se ha encontrado que la implementación de estos modelos son inadecuados para este propósito~\cite{Dongshan2002} , debido a que no logran abarcar un volumen de datos, ni secuencias de mayores tamaños. Además la escalabilidad de estas propuestas  como señala Dongshan \etal~\cite{Dongshan2002} presentan en la práctica muchas limitaciones técnicas que no permiten ser implementadas. 


 

%CITED Moghaddam_Kabir
%The techniques that rely on sequential patterns such as Markov models and sequential association rules mining contain more precise information about users’ navigation behavior. Association rules were proposed to capture the co-occurrence of buying different items in a supermarket shopping [11]. Association rules indicate groups that are presented together. In [12] study on different kinds of sequential association rules for web document prediction is proposed. It shows how to construct the association rule based prediction models for web log data.





%@TODO


\subsection{Limitaciones de los modelos tradicionales de Markov}



Los procesos de Markov sirven para modelar una gran cantidad de eventos y escenarios, ya hemos explicado anteriormente en las sección \refMLintro e inclusive hemos visto en la sección de compresión (ver sección \refLDCmodelamiento) que Markov ayuda a realizar aproximaciones a ciertos algoritmos de compresión. El uso de Markov también puede tener un enfoque predictivo directamente sin usar los enfoques propuesta en el capítulo 2\ref{}, pero dado a su dependencia y al aumento de su complejidad existen cotas para poder implementarlo, es decir, existen limitaciones a los modelos tradicionales de Markov.

Supongamos que deseamos predecir la siguiente página que visita un usuario, teniendo en cuenta un base histórica de navegación, que anteriormente se ha capturado. Digamos que $p(x_{n})$ es la probabilidad de que el usuario visite la página $n-ésima$, usando un modelo tradicional de Markov directamente dependemos de la página que ya ha visitado. También puede existir el caso equiprobable que el usuario pueda acceder a cualquier página cuando recién comienza la navegación. Lo que hemos descrito anteriormente es los primeros ordenes, es decir, el modelo de Markov de orden cero. es la tasa base de probabilidad incondicional, dada por \begin{equation}
p(x_{n}) = Pr(X_n) , \end{equation} que es la probabilidad de visitar un página \emph{web}. El  modelo de Markov de orden uno, observa la probabilidad en la transición de un página a otra, es decir, la podemos definir como  $x_{1}$ y $x_{2}$, que corresponden a la primera y segunda página visitada por un usuario. La probabilidad de la transición desde una página a la otra página correspondería en este caso a 
\begin{equation} 
p(x_{2} | x_1) = Pr(X_2 = x_{2} | X_1 = x_{1}) 
\end{equation}

Ya en este punto podemos ver una cierta analogía a los modelos discretos que proponemos atacar, entendiendo que nuestros símbolos o elementos de nuestro alfabetos son representaciones de cada pagian web, analogamente representadas en este modelo de Markov, como $x_{1}$ y $x_{2}$. 

En ese momento es facil comprender que para cada vez que realicemos una nueva transición el modelo dependerá de mas estados anteriores o también se entiende facilmente que los tendre que tener un estado anterior con gran memoria. Por lo tanto si usamos el $k$-ésimo orden de un modelo de Markov debemos considerar la probabilidad condicional, que un usuario cambie a una $n$-ésima página nueva dada su anterior visita(infinitamente hacia atrás), teniendo que $k = n -1$ páginas vistas, tendríamos:

\begin{equation}\label{eq:tantito}
% \scriptstyle
p( x_{n} | x_{n-1},..., x_{n-k} ) = Pr(X{n} = x_{n}| X_{n-1} = x_{n-1},..., X_{n-k} = x_{n-k}) .
\end{equation}

Concluímos que los modelos de Markov de orden inferior no pueden predecir con certeza total el futuro de los \webasccesslog, ya que no van lo suficientemente atrás del historial de navegación registrado por \webasccesslog, para específicamente determinar que página accederá el usuario. Los modelos con mayor orden de estados, son distintas combinaciones de las acciones observadas y registradas en el conjunto de datos de los \webasccesslog, entonces, el número de estados tiende a crecer exponencialmente al igual que el orden del modelo.

Este aumento puede limitar significativamente la aplicabilidad de los modelos de Markov para aplicaciones en las que las predicciones rápidas son críticas para el rendimiento, sea en tiempo real o para aplicaciones con restricciones de uso de memoria. Además, muchos ejemplos en los conjuntos de prueba podrían no tener estados correspondientes en los modelos de Markov que da mayor orden, por lo que reduciría su alcance y eficacia. Se puede ver más detalle de nuevas aproximaciones y mas detalles de las evidentes aproximaciones que Dongshan ha discutido y analizado en \cite{Dongshan2002}.

Finalmente en el persucución de nuestro objetivo, para lograr identificar un cierto comportamiento futuro de un usuario al momento de navegar, requiere de buenas predicciones, las que simultaneamente requieren modelos de Markov de mayor orden, pero los modelos de orden superior resultan de mayor complejidad en espacio de estado y cobertura de estados de transición. 







%%%%% Cita pagina 2 Dongshan2002 %%%%%%%%
% Los modelos tradicionales de Markov predicen la siguiente página \emph{web} que un usuario puede acceder considerando el acceso más probable, se itera para  coincidir su secuencia de acceso actual con secuencias de accesos \emph{web} históricas.

% Usando estos modelos los investigadores como  Dongshan \etal~\cite{Dongshan2002} han comparado  el máximo número de elementos  prefijos de cada secuencia histórica,  con los elementos sufijos de la  misma longitud de secuencia de \emph{webaccess} actual del usuario y obteniendo secuencias dada su secuencia histórica con probabilidad más alta en la que los elementos coinciden.





\section{Modelo predictivo simple de navegación }
	% Referencias
% ~\cite{Poornalatha2012}
% ~\cite{Moghaddam2009}
% ~\cite{Begleiter2004}
% ~\cite{Dongshan2002}


Si se tiene un problema de predicción en el cual debemos buscar la siguiente página web que el usuario desea visitar, dada a su página actual o más la página más frecuente~\cite{Poornalatha2012}, es el problema que esta tesis aborda, por lo cual buscamos un modelo que lo solucione con una representación simple.


Varios modelos han sido propuestos para modelar la navegación de un usuario y predecir la siguiente página que el usuario accederá. Se han desarrollado enfoques como  reglas de asociación, búsqueda de patrones secuenciales, \emph{clustering} y \emph{clasificación} son métodos bastante frecuentes~\cite{Moghaddam2009}, filtros colaborativos, etc.. 


En la literatura se han propuesto aproximaciones interesantes. Queremos discutir el modelo de navegación propuesto por \emph{Moghaddam}~\etal en ~\cite{Moghaddam2009}, la selección de este trabajo es una de las aproximaciones más directa a la que presenta esta tesis, pero con un enfoque usando algoritmos de tipo \losslessdatacompression y no \machinelearning. A  diferencia del anterior queremos seguir en la linea de investigación usando un modelo de navegación representado por un \emph{trie} que es la base para \lzSieteOcho, pero incluirlo en las etapas que un sistema de \machinelearning posee, ya que creemos tener un aprendizaje  y una actualización constante de los datos, generan mejores  probabilidad. Además entregan un sistema que responde a la demanda y presenta mejores características con respecto a tener disponibilidad en linea.

Otros de las aproximaciones a la solución de este problema es la desarrollada por {Dongshan}~\etal~\cite{Dongshan2002}, en la cual la aproximación es el desarrollo de un modelo de Markov representado mediante un estructura de datos de árbol, llamado \emph{Hybrid Tree Markov Model} o~\texttt{HTMM}~(ver referencia~\cite{Dongshan2002}).
\uncm
 
 % TODO:
 % Concluir y conectar con el que sigue
 
 \textbf{CONECTOR pendiente a la siguiente subseccion}

 




	
	\subsection{Modelamiento de navegación LZ78}
		% Referencias
% ~\cite{Claude2014}
% ~\cite{Moghaddam2009}
% ~\cite{Begleiter2004}
% 

Sabemos que \lzSieteOcho es un algoritmo que no tendrá pérdida, si buscamos representar la navegación de un usuario no deseamos perder datos, ya que estos mismo nos servirán para realizar una predicción, y esta representación debe ser la base para ese objetivo, un modelo.

Para comenzar con el modelo, debemos tomar las ideas principales que tiene \lzSieteOcho. Primero en base a una sequencia de símbolos crear un diccionario, y después crear un \trie que represente la estructura y modelo. Uno de los factores de los sistemas que están en linea o \webs, es siempre reducir su complejidad y con la simple idea anterior es posible. Segundo debemos hacer que cada sesión que realice el usuario es representado por una secuencia, gracias el trabajo realizado por \emph{Claude}~\etal en~\cite{Claude2014}, podemos solamente preocuparnos de continuar la constucción, ya que los datos ya han sido minados de los \webasccesslog.



Supogamos que $S_{1}$ es la secuencia de un cierto usuario, con el  algoritmo \lzSieteOcho creará un árbol de navegación que representará la navegación de este usuario. Si ingresa otro usuario con una secuencia representada por $S_{2}$, con un comportamiento similar, se podría crear predicciones en base a como ya se ha comportado el usuario anterio. En caso contrario actualizará los nodos de cada símbolos~(representan una página web o una categoría de una web). Si el usuario se mantiene navegando el algoritmo irá actualizando su frecuencia en cada nodo del \trie. Podemos señalar además por el trabajo de~\cite{Begleiter2004}, en general, el número total de secuencias que inserta en el árbol es menos que el algoritmo \PPM.  

% TODO explicar esto con el modelo de navegacion que usa ~\cite{Moghaddam2009}


% Explicamos este algoritmo con un ejemplo. Supongamos que el usuario solicita las páginas \texttt{ABABCBC} secuencialmente. Si utilizamos el algoritmo \lzSieteOcho, entonces generaríamos un \emph{trie} con nodos  \texttt{A, B, AB, BC, C} que deberán insertar. 


% Cuando se inserta una secuencia en el árbol los contadores de las aristas que representan el paso desde la raíz hasta la última petición de secuencia se incrementa en cada inserción. 
% Supongamos ahora que el usuario B pide a la secuencia de páginas \texttt{ABCABCD}, estos generaría cambios en el \emph{trie} y de cumplir las condiciones se incrementarían los contadores. 


% TODO:
% Concluir y conectar con el que sigue

\textbf{CONECTOR pendiente a la siguiente subseccion}




\subsubsection{Herramientas para crear Modelos predictivos}
% AHORA VIENDOLO ES COMO UN PARTE MAS EXPERIMENTAL ESTA SECTION

\section{Predecir con \emph{PredictionIO}}



\section{Predicción }

  	
En esta sección se presenta formalmente el ambiente de desarrollo que se utilizará durante este trabajo. \emph{PredictionIO} es un servidor de \emph{Machine Learning} de código abierto para Científico de Datos y Desarrolladores que permite crear motores de predicción para aplicaciones en producción, con un bajo tiempo de entrenamiento y despliegue. Principalmente está construido en \emph{Apache Spark, HBase} y \emph{Spray}. 

Este ambiente de trabajo se encuentra en un maduración estable y constante que permite tanto disponer servidores con motores predictivos, como también toda una infraestructura distribuida para hacer que complejos algoritmos sean utilizados para solucionar problemas reales.



% PIO, tiene practicamente todo armadao, ello no hicieron nada nuevo ... solo juntaron  todo...



% He estado investigando y revisando documentación, Yelp, Skype, Hubot de github y otras implementaciones tienen usando prediction.io




% La otra opción es meterle a este "DASE" un algoritmo  que mezcle una representación de cadenas de markov mezclado con LZ78. no se en que punto mezclarlo en el diccionario, o la verdad es como hacer el compresor sea "mas inteligente", encontré un papaer que te adjunto en el cual usan lz78 y lzw, esta interesante ya que le hacen un acercamiento mas al tema de de ser un predictor online.


% Ahora entiendo que la cadenas de markov son y se han ocupado para las predicciones, pero no veo la necesidad de ocuparlas mayormente. Adin me inisiste en que le de una vuelta.... pero mi sensación  es que tengo separada las ideas en dos extremos. 

% Ya revise Suffix Tree para predicciones LZ77 y LZW, también PPM y HMM.

 
 



\subsection{Arquitectura DASE}


Un motor de predicción es un tipo de proceso en \emph{Machine Learning}. Siguiendo una arquitectura de tipo \emph{DASE}, contendríamos los siguientes componentes.



\begin{itemize}

  \item\label{dase-datasource} \textbf{ $[D]$ Data Source y Data Preparator}. Los Data Source leen la data desde la entrada original y la transforman en un formato deseado para hacer análisis de estos. En cambio \emph{Data Preparator} pre-procesa la información y la reenvía a los algoritmos para   hacer el modelo de entrenamiento.


  \item\label{dase-algoritmo} \textbf{ $[A]$ Algoritmo}. Los componentes de \emph{PredictionIO}, dada sus librerías incluyen algoritmos de \emph{Machine Learning}, estos, pueden ser provistos por \emph{Apache Spark} o se pueden incluir algoritmos propios como también de terceros.
    Adicionalmente a los algoritmos podemos asignarle parámetros, para determinar como debiese ser construido el motor ó si es requerido para un cierto algoritmo.



  \item\label{dase-servicio} \textbf{ $[S]$ Servicio}. El componente servicio toma las consultas ó \emph{queries} de predicción y retorna los resultados, en nuestro modelo propuesto en la etapa experimental veremos el siguiente símbolo de una secuencia. 
  Si el motor de predicción tiene múltiples algoritmos, combinará los resultados en uno. Adicionalmente, la lógica específica de negocios puede ser añadida para especificar aún más el resultado final. 
 
  \item\label{dase-eval} \textbf{ $[E]$ Evaluación de Métricas}.
Las métricas de evaluación cuantifican la precisión de la predicción con una puntuación numérica. Puede ser utilizado para la comparación de algoritmos o ajustes de los parámetros del algoritmo.
\end{itemize}




  \tikzstyle{decision} = [diamond, draw,text width=4.5em, text badly centered, node distance=2.5cm, inner sep=0pt]
  \tikzstyle{block} = [rectangle, draw,text width=5em, text centered, rounded corners, minimum height=4em]
  \tikzstyle{line} = [draw, very thick, color=black!50, -latex']
  \tikzstyle{cloud} = [draw, ellipse, node distance=2.5cm,
  minimum height=2em]


\begin{figure}[t]
	\centering	
	\resizebox{0.8\textwidth}{!}{% <------ Don't forget this %
	
		\begin{tikzpicture}[scale=1, node distance = 2.5cm, auto]
		% Place nodes
		
		\node [block] (init) {Data de la Aplicación};
		\draw [color=gray,thick](1.3,1) rectangle (11.2,-3.5);    
		
		\node [block,right of=init] 		  (datasource) {Data Source};
		\node [block,right of=datasource]     (datapreparator) {Data Preparator};
		\node [block,right of=datapreparator] (alg1) {Algoritmo };
		\node [block,right of=alg1] 		  (serving) {Serving};
		\node [block,below of=serving] 		  (evalmetric) {Evaluación Métrica};
		\node [block,right of=serving] 		  (resultpredict) {Resultado Predicción};
		\node [block,below of=resultpredict]  (resulteval) {Resultado Evaluación};
		
		\path [line] (init) -- (datasource);
		\path [line] (datasource) -- (datapreparator);
		\path [line] (datapreparator) -- (alg1);
		\path [line] (alg1) -- (serving);
		\path [line] (serving) -- (evalmetric);
		\path [line] (serving) -- (resultpredict);
		\path [line] (evalmetric) -- (resulteval);    
		
		\end{tikzpicture}
	}
	\caption{Diagrama de componentes Arquitectura DASE.}
	\label{fig:arquitectura-dase}
\end{figure}



\vspace{1cm}

\emph{PredictionIO} ayuda a tener componentes muy modulares, las que ya hemos descrito como  arquitectura \emph{DASE}(\ref{fig:arquitectura-dase})  que puede construir modelos de predicción de manera sencilla. También poder integrarlos con gran facilidad a cualquier sistema o plataforma, por ejemplo, es posible elegir cual de todos los componentes se podrá desplegar al momento de crear un \emph{Engine} (Motor de Predicción.)



\vspace{1cm}
\subsection{Despliegue de motor de predicción}

  Un \emph{Motor de predicción} pone todos los componentes del diseño de arquitectura \emph{DASE} en un estado especifico de despliegue 
  \begin{enumerate}
  		\setlength{\itemsep}{1pt}
  		\setlength{\parskip}{0pt}
  		\setlength{\parsep}{0pt}
    \item Data Source
    \item Data Preparator
    \item Uno o más Algoritmos generadores de Modelos
    \item Un Servicio 
  \end{enumerate}

  Si se especifica más de un algoritmo, cada uno de los resultados de los modelos de predicción se entregará para ser consumido por cualquier cliente.
  Cada \emph{motor de predicción} procesa los datos y construye un modelos de forma independiente. Por lo tanto, todos los motores de predicción que usaremos sirven a su propio conjunto de resultados. Por ejemplo, se puede desplegar dos \emph{motores predictivos} para una aplicación móvil: uno para recomendar noticias a los usuarios y otro para sugerir nuevos amigos a los usuarios.


\vspace{1cm}
\subsection{Evaluación del motor de predicción }

  Para evaluar el \emph{Accuracy} de un motor de predicción, se debe especificar la métrica seleccionad cuando se corre el motor de evaluación, en los capítulos experimentales se verá como se generan métricas y como se desempeña esta métrica para ser evaluada.











\subsection{Modelamiento de eventos}

%https://docs.prediction.io/datacollection/eventmodel/




  El modelamiento de eventos es fundamental para un predictor \emph{online}, el hecho de poder llevar un vector con ciertas propiedades característica\footnote{Característica de un cierto dataset para entrenar.} de una representación vectorial del mundo del \emph{Machine Learning} a un modelamiento secuencial, es en realidad el modelamiento que se debe realizar de como  tener las muestras de datos para generar \emph{Resilient Distributed Dataset} (RDD) que son un parte principal de nuestro ambiente de trabajo con \emph{PredictionIO} para poder acceder posterior o inmediatamente. 

  Un evento lo definiremos como entidad que nos permite dar una representación temporalizada de información que será procesada por un motor de predicción. Analizaremos los eventos que un usuarios realiza para poder acceder a una web. Adicionalmente cuando cada usuario ingresa a una web automáticamente este genera una sesión, desde que que llega hasta que abandona la web.

  Usaremos un \emph{dataset} con información que esta totalmente depurada y recuperada de los access log (provista por Claude \etal~\cite{Claude2014}), los cuales a efectos de temporalidad solo nos interesa conocer la secuencialidad de estos accesos.
  


%Poner algo mas matematico.
% EVENT API 
% https://docs.prediction.io/datacollection/eventapi/

  El modelamiento que realizaremos contempla los siguientes campos :
  \begin{itemize}
    		\setlength{\itemsep}{1pt}
    		\setlength{\parskip}{0pt}
    		\setlength{\parsep}{0pt}
      \item Tipo de Evento: Visitar
      \item Entidad que ejecuta el evento: Usuario
      \item Propiedades:
          \begin{enumerate}
          		\setlength{\itemsep}{1pt}
          		\setlength{\parskip}{0pt}
          		\setlength{\parsep}{0pt}
            \item Página actual
            \item Página siguiente
            \item Cierre de Sesión
          \end{enumerate}
    \end{itemize}



    El interés de tener un modelo totalmente atómico es poder contemplar la información que nos entrega, destacando sus variables y propiedades como restricciones.



\subsection{Ventajas }


  Es posible mezclar y aplicar distintas característica, si el modelo no puede ser persistido por \emph{PredictionIO} automáticamente. Se requiere un objeto  heredado de una clase que permita lograr la persistencia en memoria, esto permite cargar el modelo persistentemente e instanciar automáticamente durante la ejecución del motor de predicciones.

  % Tenga en cuenta que los modelos generados por PAlgorithm no pueden ser persistieron automáticamente por naturaleza y deben implementar estas características, si se desea modelo de persistencia.


  Comprendiendo el concepto de \emph{Resilient Distributed Dataset} (RDD), esta es la abstracción básica de \emph{Apache Spark}, aún más, esto es una de las grandes cualidades de PredictionIO, ya que no solamente podemos disponer de una máquina para hacer estudios o implementar algoritmos, este servidor de \emph{Machine Learning} permite, gracias a sus componentes poder hacer un cluster para entregar mayor eficiencia acorde a los datos o algoritmo a implementar.

  Ya hemos mencionado que un \emph{Resilient Distributed Dataset} (RDD) es una representación inmutable, una colección particionada de elementos que pueden ser operadas en paralelo. Internamente cada RDD tiene cinco principales propiedades:


  \begin{enumerate}
    \item Una lista de particiones.
    \item Una función para procesar cada porción de datos.
    \item Una lista de dependencias en otros RDD. 
    \item Opcionalmente una partición de un RDD puede ser representada como una combinación de llave y valor. 
    \item Opcionalmente, una lista de los lugares preferidos para calcular cada una dividida en (por ejemplo, lugares de bloque para un archivo \emph{HDFS}), para procesamiento en \emph{Clustering}.

  \end{enumerate}




% https://docs.prediction.io/templates/recommendation/customize-serving/
















 	 %introducción a las Prediccion



\input{capitulo-IV}
%%%%%%%%%%%%%%%%%%%%%%%%%%%
% Iniciamos el resto de secciones adicionales al contenido: referencias y apendices
\backmatter
%%%%%%%%%%%%%%%%%%%%%%%%%%%
%%Se presenta una revision bibliografica acertada, actual, y exhaustiva (se consulta toda la literatura relevante).
%Presenta y redacta los objetivos, el trabajo realizado, y la validacion de manera detallada. Presenta discusiones detalladas, articuladas, claras, y un buen uso del lenguaje tecnico.
% Bibliografía - El estilo por defecto es IEEE Transactions
\bibliographystyle{ieeetr}
\bibliography{IEEEabrv,referencias}
%%%%%%%%%%%%%%%%%%%%%%%%%%%
% Simbología y glosario
% Utilice un paquete para generar símbolos y glosarios.
% Por ejemplo: nomencl (http://texdoc.net/pkg/nomencl)
% Anexos
\appendix
%%%%%%%%%%%%%%%%%%%%%%%%%%%
\chapter{Anexos técnicos}
\label{ch:anexo-a}


\section{Uso de linea de Comando Prediction.IO}


La interación con \emph{PIO} es a través de una interface de linea de comando, esta sigue el siguiente formato de uso:

 

\begin{lstlisting}[frame=single,basicstyle=\ttfamily\tiny,]
  pio <command> [options] <args>...
\end{lstlisting}



En caso de tener duda usted al igual que los comandos de bash puede ejecutar

\begin{lstlisting}[frame=single,basicstyle=\ttfamily\tiny,]
pio help <command> 
\end{lstlisting}, para ver mas detalles de cada detalle de los comandos disponibles.


Los comandos de PredictionIO se pueden separar en tres categorías:

\begin{itemize}
	\item \textbf{help} Muestra un resumen del uso  \emph{pio help <command>} para leer sobre un sub comando
	
	\item \textbf{Version} Muestra la version instalada de PredictionIO
	
	
	\item \textbf{Status} Muesta la ruta de instalación y el estatus de ejecución de sistema, como también sus dependencias
	
\end{itemize}

Comandos del servidor de eventos



\begin{itemize}
	\item \textbf{app} Muestra un resumen del uso  \emph{pio help <command>} para leer sobre un sub comando
	
	\item \textbf{Version} Muestra la version instalada de PredictionIO
	
	
	\item \textbf{app }  Administra todas las aplicaciones que usa el servidor de eventos
	\item \textbf{pio app data-delete <name> } Borra toda la data contenida por una aplicación específica
	\item \textbf{pio app delete <name> }  Borra una aplicación completa
	\item \textbf{eventserver }  Lanza el servidor de eventos
	\item \textbf{ --ip <value> } Une la IP seleccionada, el valor por defecto es \emph{localhost}
	\item \textbf{access\_key} Administra todas las llaves de acceso al servidor de eventos ó app
	
\end{itemize}


\section {Comandos del Motor de Predicción}

Es requerido que estos comandos se ejecuten desde la misma carpeta que contiene el proyecto ó aplicación desplegada.

Las opciones  --debug y --verbose  muestran información detallada sobre los conectoes de las aplicaciones complementarias a las que esta compuesta PredictionIO

\begin{itemize}
	\item \textbf{build} Construye y compila el proyecto completo desde la carpeta fuente, tiene un flag adicional \emph{-- clean}, para una compilación limpia.

	\item \textbf{train} Ejecuta el entrenamiento declarado en el motor 

	\item \textbf{deploy} Despliega el motor para ser usado meiante REST como Algoritmo como Servicio. Si no existe ninguna nueva instancia  desplegada, por defecto usará la última creada.
	
\end{itemize}


\newpage
\section{Configuraciones para hacer correr IntelliJ con Apache SPARK y Prediction.IO 0.94}

\begin{lstlisting}[frame=single,basicstyle=\ttfamily\tiny,]
Main class: io.prediction.workflow.CreateWorkflow

VM options: -Dspark.master=local -Dlog4j.configuration=file:/Users/jguzman/PredictionIO/conf/log4j.properties


Program arguments: --engine-id dummy --engine-version dummy --engine-variant engine.json


io.prediction.workflow.CreateWorkflow
-Dspark.master=local -Dlog4j.configuration=file:/Users/jguzman/PredictionIO/conf/log4j.properties -Dorg.xerial.snappy.lib.name=libsnappyjava.jnilib 
--engine-id dummy --engine-version dummy --engine-variant engine.json



SPARK_HOME=/Users/jguzman/PredictionIO/vendors/spark-1.4.1/bin
PIO_FS_BASEDIR=/Users/jguzman/.pio_store
PIO_FS_ENGINESDIR=/Users/jguzman/.pio_store/engines
PIO_FS_TMPDIR=/Users/jguzman/.pio_store/tmp
PIO_STORAGE_REPOSITORIES_METADATA_NAME=pio_meta
PIO_STORAGE_REPOSITORIES_METADATA_SOURCE=ELASTICSEARCH
PIO_STORAGE_REPOSITORIES_MODELDATA_NAME=pio_model
PIO_STORAGE_REPOSITORIES_MODELDATA_SOURCE=LOCALFS
PIO_STORAGE_REPOSITORIES_APPDATA_NAME=pio_appdata
PIO_STORAGE_REPOSITORIES_APPDATA_SOURCE=ELASTICSEARCH
PIO_STORAGE_REPOSITORIES_EVENTDATA_NAME=pio_event
PIO_STORAGE_REPOSITORIES_EVENTDATA_SOURCE=HBASE
PIO_STORAGE_SOURCES_ELASTICSEARCH_TYPE=elasticsearch
PIO_STORAGE_SOURCES_ELASTICSEARCH_HOSTS=localhost
PIO_STORAGE_SOURCES_ELASTICSEARCH_PORTS=9300
PIO_STORAGE_SOURCES_LOCALFS_TYPE=localfs
PIO_STORAGE_SOURCES_LOCALFS_HOSTS=/Users/jguzman/.pio_store/models
PIO_STORAGE_SOURCES_LOCALFS_PORTS=0
PIO_STORAGE_SOURCES_HBASE_TYPE=hbase
PIO_STORAGE_SOURCES_HBASE_HOSTS=0
PIO_STORAGE_SOURCES_HBASE_PORTS=0


Main class: io.prediction.workflow.CreateServer
Program Arguments: --engineInstanceId **replace_with_the_id_from_pio_train**




Try -- for more information.
Usage: pio train [--batch <value>] [--skip-sanity-check]
                 [--stop-after-read] [--stop-after-prepare]
                 [--engine-factory <value>] [--engine-params-key <value>]
                 [--scratch-uri <value>]
                 [common options...]

Kick off a training using an engine (variant) to produce an engine instance.
This command will pass all pass-through arguments to its underlying spark-submit
command.

  --batch <value>
      Batch label of the run.
  --skip-sanity-check
      Disable all data sanity check. Useful for speeding up training in
      production.
  --stop-after-read
      Stop the training process after DataSource.read(). Useful for debugging.
  --stop-after-prepare
      Stop the training process after Preparator.prepare(). Useful for
      debugging.
  --engine-factory
      Override engine factory class.
  --engine-params-key
      Retrieve engine parameters programmatically from the engine factory class.
  --scratch-uri
      URI of the working scratch space. Specify this when you want to have all
      necessary files transferred to a remote location. You will usually want to
      specify this when you use --deploy-mode cluster.

\end{lstlisting}

\vspace{1cm}


\section {Llamadas al Servidor de Machine Learning mediante curl }


\begin{lstlisting}

curl -H "Content-Type: application/json"  -d '{"webaccess" : "AC","num" : 10}' http://52.33.180.212:8000/queries.json



\end{lstlisting}



\section{Python SDK para PredictionIO}

Para hacer uso de estos scripts en python es necesario tener instalado el package de prediction para python sdk.

Si se tiene pip instalado correctamente se puede utilizar

\begin{verbatim}
pip install predictionio
\end{verbatim}
ó
\begin{verbatim}
$ easy_install predictionio
\end{verbatim}


Es recomendable tener acceso sudo para evitar problemas con permisos al momento de usar \emph{pip} o \emph{easy\_install} {(ie. sudo pip install predictionio)}.




El siguiente script, permite hacer una sola consulta la cual puede ser ejecutada desde el \emph{cli} de python ó llamando directamente al archivo ejecutable. ( {python test.py})

\begin{lstlisting}[frame=single,basicstyle=\ttfamily\tiny,]
import predictionio

engine_client = predictionio.EngineClient(url="http://localhost:8000")

print engine_client.send_query({"webaccess": "A", "num": 10})
\end{lstlisting}



\newpage
A diferencia del script explicado anteriomente este permite enviar secuencias desde la terminal \emph{cli}, la cual puede seguir en ejecución hasta que el usuario finalice el proceso.

\begin{lstlisting}[frame=single,basicstyle=\ttfamily\tiny,]
"""
Send sample query to prediction engine
"""

import predictionio
import readline

engine_client = predictionio.EngineClient(url="http://localhost:8000")
while True:
    word = raw_input('Enter a Sequences or a single page to predict the next user webaccess: \n')
    print engine_client.send_query({"webaccess": word, "num": 10})

\end{lstlisting}





\section{Programa C++ para hacer splits dentro del Dataset}
Programa para poder pasar la data de msnbc a una representación de símbolos.

\begin{lstlisting}[frame=single,basicstyle=\ttfamily\tiny,]
#include <iostream>     // cout
#include <fstream>      // ifstream
#include <sstream>
#include <algorithm>
#include <string>
#include <cmath>
#include <cstdio>
#include <vector>
#include <map>
#include <iterator>

using namespace std;

/** 
  alias lseq = g++ -std=c++11 letterSequences.cpp -o letterSequences 
  ./letterSequences

% Different categories found in input file:

frontpage news tech local opinion on-air misc weather msn-news health living business msn-sports sports summary bbs travel
**/

 
int main()
{
   map<string, int> mapCategories;

  
   // Inserting data in map
  mapCategories.insert(make_pair("frontpage", 1));
    mapCategories.insert(make_pair("news",    2));
    mapCategories.insert(make_pair("tech",    3));
    mapCategories.insert(make_pair("local",   4));
    mapCategories.insert(make_pair("opinion",   5));
    mapCategories.insert(make_pair("on-air",  6));
    mapCategories.insert(make_pair("misc",    7));
    mapCategories.insert(make_pair("weather",   8));
    mapCategories.insert(make_pair("msn-news",  9));
    mapCategories.insert(make_pair("health",  10));
    mapCategories.insert(make_pair("living",  11));
    mapCategories.insert(make_pair("business",  12));
    mapCategories.insert(make_pair("msn-sports",13));
    mapCategories.insert(make_pair("sports",  14));
    mapCategories.insert(make_pair("summary",   15));
    mapCategories.insert(make_pair("bbs",     16));
    mapCategories.insert(make_pair("travel",  17));
    

   vector<char> alphabet = { 'A','B','C','D','E','F','G',
                'H','I','J','K','L','M','N','O',
                'P','Q','R','S','T','U','V','W',
                'X','Y','Z'};

   // Iterate through all elements in map
   map<string, int>::iterator it = mapCategories.begin();

   ifstream  fin("msnbc990928.seq");
   string    file_line;
   int fold = 0 ;

   while(getline(fin, file_line)){

    string    buf; // Have a buffer string
    stringstream  ss(file_line); // Insert the string into a stream
    vector<string> tokens; // Create vector to hold our words
    
    while (ss >> buf) tokens.push_back(buf);

    if( tokens.size() < 6 ){
      ++fold;
      for (int i = 0; i < tokens.size(); ++i){
        string tmp = tokens.at(i); 
        cout << alphabet.at( stoi(tmp) - 1) << " ";
      }cout<< endl;

    }

    //this value is for make the size of the folds of data
    if( fold == 1000000 ) break;

   }
    return 0;
}
\end{lstlisting}










  
%%%%%%%%%%%%%%%%%%%%%%%%%%%

\clearpage
\printglossary[type=\acronymtype]
\printglossary
\end{document}