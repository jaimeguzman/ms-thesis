% Ejemplo del uso de la template para escribir tesis/memorias de la Universidad Diego Portales.
%
% Eviar bugs a: Adín Ramírez, adin.ramirez (at) mail.udp.cl

% Puede generar borradores si omite la opción "final" de la clase.
% \documentclass{udpthesis}
\documentclass[final]{udpthesis}

% Establecemos el sistema para uso del español
% Babel ya esta cargado dentro de updthesis
\usepackage[T1]{fontenc}%    output
\usepackage[utf8]{inputenc}% input
\usepackage{lmodern}

% Leyendas
\usepackage[font=footnotesize,labelfont=bf,labelsep=period]{caption}

% Agregue acá otros paquetes que le sean de utilidad
% Matemáticas
\usepackage{amsmath}

% Gráficos
\usepackage{graphicx}
\usepackage[font=footnotesize,labelformat=simple]{subfig}
% Cambiamos el formato de las leyendas: finalizan en punto, y en negrita.
\captionsetup{labelsep=period,labelfont=bf}
% Habilitamos el uso de paréntesis al citar las figuras con subfiguras dentro, e.g., Fig. 1(a)
\renewcommand\thesubfigure{(\alph{subfigure})}
\renewcommand\thesubtable{(\alph{subtable})}
\newcommand{\subfigureautorefname}{\figureautorefname}

% Código
% Para generar código fuente usar listings.sty
%\usepackage{listings}
%\usepackage{tikz}
%\lstset{
%  language=[LaTeX]TeX,
%  breaklines=true,
%  basicstyle=\tt\scriptsize,
%  keywordstyle=\color{blue},
%  identifierstyle=\color{magenta},
%  commentstyle=\color{green!40!black},
%  % frame 
%  frame=tb,
%  captionpos=t,
%  xleftmargin=1em,
%  numbersep=0.3em,
%  numbers=left,
%  framexleftmargin=1.1em,
%  framexrightmargin=0pt,
%  % additional letters for accents in spanish
%  literate=%
%    {á}{{\'{a}}}1
%    {é}{{\'{e}}}1
%    {í}{{\'{i}}}1
%    {ó}{{\'{o}}}1
%    {ú}{{\'{u}}}1
%    {ñ}{{\~{n}}}1
%    {Ñ}{{\~{N}}}1
%}
%
%\renewcommand{\lstlistingname}{Código}% Listing -> Código
%\DeclareCaptionFormat{listing}{\rule{\dimexpr\linewidth\relax}{0.4pt}\par\vskip1pt#1#2#3}
%\captionsetup[lstlisting]{format=listing,singlelinecheck=false, margin=0pt,position=bottom}

% O para generar algoritmos en pseudocódigo usar algpseudocode.sty
%\usepackage{algorithm}
%\usepackage{algpseudocode}
%\makeatletter
%\renewcommand{\ALG@name}{Algoritmo}% Algorithm -> Algoritmo
%\makeatother
%\captionsetup[algorithm]{font=footnotesize,labelsep=period}

% Referencias (este paquete ordena y comprime las referencias)
\usepackage{cite}

% Un paquete para generar texto. REMUEVA ESTE PAQUETE AL UTILIZAR ESTA PLANTILLA.
\usepackage{blindtext}


% Establecemos el tema a utilizar. 
% Debe existir el archivo udpthesisEIT.sty en su sistema TeX para poder utilizarlo.
% Por ejemplo, para utilizar el tema de magíster de la EIT deben de utilizar
% \udptheme{EIT-MS}
\udptheme{EIT}


\begin{document}
%% Inicio de la portada
\frontmatter

% Título del tema (no más de 12 palabras)
\title{Modelo Híbrido de LZ-78 y Machine Learning para predicción de comportamientos de usuarios basado en web access log}
% para precisar aún más su tema, use un subtítulo
%\subtitle{Subtítulo explicativo del tema}

% El autor(es) de la tesis
\author{Jaime Guzmán}
\email{mail@jguzman.cl}% utilice un correo que revise después de graduado

% o una lista de autores separados con comas
%\author{Juan Bar, José Foo}
%\email{juan.bar@mail.udp.cl, jose.foo@mail.udp.cl}

% Fecha a aparecer en la tesis
\date{2015}

% Profesor guía
\professor{Adin Ramirez}
% Comité
\committee{Francisco Claude}{Darth Vader}

% Dedicatoria
\dedicatory{Utilice un par de oraciones para dedicar su tesis, o una frase de alguién importante.}

% Agradecimientos
\acknowledgment{Nota redactada sobriamente en la cual se agradece a quienes han colaborado en la elaboración del trabajo. No puede exceder más de una página.}

% Abstrat en inglés
\abstract{%<- evita nueva linea en el abstract
Abstract is the summary in english of the subject your are presenting in this thesis. Should not exceed one page.
}

% Resumen
\resumen{%<- evita nueva linea en el resumen 
%El resumen no debe contener menos de 100 palabras ni mas de 300 palabras.

El siguiente documento comprende el estudio y creación de un algoritmo híbrido entre Machine Learning y Loseless Compression Algorithm. Dado que  Internet crece cada día y los datos crecen en volúmenes del orden de los Terabytes, por lo cual es de interés usar técnicas de compresión para realizar un procesamiento de mayor información con la menor cantidad de recursos.

Hoy existe variadas tipos de web, redes sociales, microbloging, web informativas, etc. El contenido proporcionado a los usuarios finales  ya no es estático y esto permite que los mismo puedan generar, aportar, modificar contenido, dado  esto la ingeniría ofrecidad para construir web esta en constante evolución lo que ha ayudado a generar mas recursos para poder desarrollarla. Muchas de estas nuevas tecnologias han permitido entregar una mejor experiencia al momento de navergarla, aún cuando se ha creado un gran avance sobre el mismo esto no ha permitido crear Webs que sean por sí mismas inteligente y puedan ir anticipando su comportamiento, para por ejemplo; disminuir la latencia desde que se abre una web ya visitada o desde que se navega dentro de un sitio con alta demanda; también desde el punto de vista de la arquitectura como servicio que las hospedan no se ha visto abarcada, dando un aspecto económico a los recursos utilizados. Si bien el crecimiento de los recursos de almacenamiento en la nube se encuentran un apogeo, las redes no crecen a la misma velocidad. 

Este trabajo busca predecir el comportamiento de un usuario dentro de una web, usando técnicas de aprendizaje y algoritmo de compresión. Con este proprosito se trabajará para crear un modelo híbrido.
}

% Generamos la portada
\makecover

% Indices y listas
\tableofcontents% tabla de contenido
\listoftables%    índice de tablas
\listoffigures%   índice de figuras
% puede agregar otras listas o índices acá de ser necesario



% Inicio del contenido
\mainmatter

% Capítulos y secciones del documento
% Aca se incluyen los archivos con el texto de los capitulos
% Incluyo el archivo cap-intro.tex
\chapter[Introducción]{Introducción}
\label{ch:intro}

\section{Motivación}
\label{sec:motivacion}

La motivación de este proyecto de título es poder converger dos áreas de estudio para el mismo problema, predicción del comportamiento de la web. Usar recursos tanto que la industria ofrece (predicction.IO) y la academia para poder crear un algoritmo o modelo que permita dar mejor precisión.

\section{Contexto}
\label{sec:contexto}

  La Web crece constantemente y por ende su infraestructura, también los tipos de usuarios y  también la concurrencia de los mismos sistemas, la cual para usuarios finales se traduce en latencia y una mejor o peor experiencia de usuario. 
  Paralelamente se suma un costo exponencial de recursos tanto en tecnologías de desarrollo como servicio que no son optimizados para poder dar una experiencia de usuario con calidad de servicio. Podemos reflexionar, entonces, que el no tener mayores recursos mejorará el rendimiento ni tampoco será lo óptimo para dar una calidad de servicio web, ya que el ancho de banda de Internet no crecerá a la misma proporción.
   
  Adicionalmente, las tecnologías para la creación de web dinámicas e asíncronas han evolucionado a favor del cliente.
  Hoy en día ya se poseen \emph{MEAN stacks} que disminuyen considerablemente la carga de un servidor, por lo cual, un buen servicio web es proveer una balanceada carga dentro del cliente y el servidor, pero cuando se poseen un volumen de datos considerables es requerido tomar decisiones que los recursos y lenguajes no cubren, es ahí el interés de hacer  web inteligente.

  Es de gran interés predecir los movimientos siguientes que un usuario tendrá en una determinada web.
  Entendiendo que la manera en que un usuario final navega es su comportamiento registrado en una web, y que se puede analizar, estudiar y registrar mediante \emph{Web Access Log} y a los cuales se puede hacer una minería de datos, Web Usage Minning. El por qué de hacer minería de datos es que cada día la web genera un innumerable cantidad de datos, por lo cual usar un algoritmo como LZ 78 presenta un interés ya que además de ser un algoritmo de compresión, este se puede usar como un algoritmo de predicción y trabajar con una mayor cantidad de datos.
  
  Los registros de accesos de manera procesada o pre-procesada, ayudaría a ingenieros de desarrollo web y diseñadores, como a  usuarios finales a tener una experiencia de usuario mejor, disminuyendo por ejemplo la latencia en respuestas por parte de cada petición que realizan.
  
  Hoy en día, las web no pueden ser simplemente dinámicas en contenidos, debe poseer una adaptabilidad a la demanda del usuario o proveer información que permita adaptarse a los eventos, por lo tanto, es de interés el profundizar en este tópico.




  \subsection{Trabajos relacionados}

% @TODO: SEGUIR TRABAJANDO EN ESTA BREVE INTRODUCCION
En este tema convergen tres áreas, por un lado existe trabajo para crear estructuras eficientes para predicciones basadas en algoritmos de compresión, como es en el caso de~\cite{Claude2014}, y, por otro lado, el uso de algoritmos de aprendizaje para realizar clustering y predecir el comportamiento basado en el mismo contenido o en la distancia del contenido que visita el usuario actual al contenido clusterizado, como es el caso de ~\cite{Poornalatha2012}, inclusive se han utilizado modelos de Markov en ~\cite{Dongshan2002}  para poder modelar el comportamiento de la web.
La tercera área son los Sistemas de Recomendación, la cual en este proyecto no se tocará pero si se mencionará el enfoque práctico que presenta área como un foco de múltiples implementaciones. 


En la literatura, el tema de la predicción en la web se ha presentado como un tema concurrente, y ha sido abarcado por varios autores. Tenemos los siguientes trabajos de interés:

\begin{enumerate}
  \item Dongshan y Junyi~\cite{Dongshan2002} destacan que un modelo de Markov puede ayudar a predecir el comportamiento de un usuario, pero con ciertas limitaciones .  Para solucionarlo presentan un nuevo modelo de Markov basado en una representación de \emph{Tree Order Model}, el cual es un híbrido entre un modelo de markov tradicional y una representación de árbol, bautizada como HTMM (por sus siglas en inglés, \emph{Hybrid-Order Tree Markov Model}).
  Su modelo fue presentado en 2002, y da una importancia a conocer la predicción de los \emph{web access}, dada la importancia de creación de redes, la minería de datos, e-commerce, y otras áreas.

  \item Domenech \etal~\cite{Domenech2006}, muestran un estudio de los rendimientos de técnicas de recuperación de datos.
  Las mismas se pueden utilizar para dar una entrada ideal a algoritmos de aprendizaje o algoritmos de predicción. 
  Los conceptos más importantes son las nuevas variables de caracterización, temporalidad, espacio y geografía, que se le suman a la predicción. 
  Además de comenzar un trabajo más elaborado de como tomar una predicción, se introducen conceptos como predicciones genéricas o específicas, variables de uso de recursos a nivel de red ó nivel procesamiento.
  Finalmente, se presenta un modelo predictivo que puede ayudar a disminuir la latencia entre la petición del cliente y la respuesta de la web, dando así un mejor rendimiento y \emph{QoS}.


  % @TODO detallar más explicarlo mas simple, darle mas enfoque al usuario segúnn del punto de vista que de los docuentos 
  % como los autores antteriores.

  \item Chen \etal~\cite{Chen2011} dan una nueva perspectiva enfocada a entregar una clara recomendación a los usuarios basada en la misma propuesta de este proyecto, los access log.
  El primer análisis realizado por los autores cubre las reglas asociativas que requiere un sistema de recomendación, pero en las pruebas propiamente tales encuentran que el análisis de los patrones detectadados dan una representación clara de como optimizar la web, y finalmente mediante sus pruebas logran una recomendación de calidad.

  \item Rajimol y Raju~\cite{Rajimol2012} minaron los patrones de los accesos web, donde el enfoque es usar los registros de acceso para crear subsecuencias y realizar comparaciones.
  La literatura presenta un interés para poder anticipar el patrón de comportamiento de la web.
  % @TODO reflexionar mas sobre este paper

  \item Kewen~\cite{kewen2012} realizó un análisis más profundo del \emph{web usage minning}.
  Parte de la importancia de este trabajo, es que después de minar los registros de accesos, logran reducir la ``\emph{bad data}''.
  %@TODO: Preguntar si este paper se escapa mucho del tema prinicipal, pero parece interesante  

  \item Poornalatha y Raghavendra~\cite{Poornalatha2012} establecen que se pueden utilizar máquinas de aprendizaje para predecir basándose en distintas entre clusters. Estos autores, al igual que Domenech \etal~\cite{Domenech2006} y Dongshan y Junyi~\cite{Dongshan2002}, comparan el objetivo de optimizar los recursos tanto en redes (disminución de latencia) y experiencia de usuario.

  \item Claude \etal~\cite{Claude2014} presentan una estructura de representación eficiente que permite dar una representación de \emph{web access log} y ofrecen las operaciones básicas de WUM.
\end{enumerate}
% Incluyo el archivo cap-tema.tex
\chapter[Tema]{Compresión}
\label{ch:tema}

%empezar habalr sibre lempel ziv
Los algoritmo de compresión si perdida, son un área en constante desarrollo




\section{Tipos de Algoritmo de compresión}
 
	\begin{enumerate}
		\item PPM
		\item LZ 77
		\item LZ 78
		\item LZ W
		
	 	
	\end{enumerate}
	
 
 
 
 

% Genero toda las referencias para demostrar el uso de la bibliografía
% No es necesario que utilice este comando en su documento.
\nocite{*}

\chapter[Tema]{Compresión}
\label{ch:tema}

%empezar habalr sibre lempel ziv
Los algoritmo de compresión si perdida, son un área en constante desarrollo




\section{Tipos de Algoritmo de compresión}
 
	\begin{enumerate}
		\item PPM
		\item LZ 77
		\item LZ 78
		\item LZ W
		
	 	
	\end{enumerate}
	
 
 
 
 

% Genero toda las referencias para demostrar el uso de la bibliografía
% No es necesario que utilice este comando en su documento.
\nocite{*}


% incluya otros archivos según su necesidad
% 1.-resumen de archivos para tesis

Predicción de Markov de Orden Variable

Este documento se refiere a los algoritmos de predicción de secuencias discretas en un alfabeto finito, utilizando variables modelos Markov. La clase de este tipo de algoritmos es amplio y comprende en principio cualquier algoritmo de compresión sin pérdida. Nos enfocamos en seis prominentes algoritmos de predicción, incluyendo  (CTW), (PPM) y (PST).  Discutiremos las propiedades de estos algoritmos y podemos comparar su rendimiento con secuencias reales de tres dominios.
La comparación se hace con respecto a la predicción calidad medido por el promedio de pérdida de registro. También comparamos algoritmos de clasificación basados en estos predictores con respecto a un gran número de clasificación de  tareas. Nuestros resultados indican que una "descomposición" CTW (una variante del algoritmo CTW) y PPM superan todos los demás algoritmos de predicción de las tareas. Sorprendentemente, un algoritmo diferente, que es una modificación del algoritmo de compresión Lempel-Ziv supera todos los algoritmos de  problemas de clasificación.

UN NUEVO MODELO DE MARKOV PARA EL ACCESO A LA WEB PREDICCIÓN

Predecir con exactitud el comportamiento del usuario al acceder a la Web puede reducir al mínimo la latencia que percibe el usuario, que es crucial en el rápido y creciente World Wide Web. A pesar de que los modelos Markov han ayudado a predecir comportamientos de acceso de usuario, tienen graves limitaciones. Los modelos Híbridos de árbol, Markov predicen acceso a la Web, precisamente al mismo tiempo que ofrecen un alto nivel de cobertura y escalabilidad.

La World Wide Web es una gran base de datos donde se almacena  y se accesa a la información, permite a los usuarios navegar a través de enlaces y ver con los exploradores. El tráfico de Internet ha aumentado considerablemente debido a la popularidad de la Web y como consecuencia los usuarios perciben la latencia. La solución obvia de incrementar el ancho de banda, no es viable, ya que no podemos cambiar fácilmente la infraestructura de la Web (Internet) sin gran costo económico. Sin embargo, si se puede predecir las búsquedas del futuro usuario, podríamos poner esas páginas en el lado del cliente de caché cuando el navegador es gratuito. Cuando un usuario solicita una de las páginas, el navegador puede recuperarlo directamente desde la memoria caché.

Gran parte de las actuales investigaciones han examinado modelos y buscan predecir comportamientos acceso de usuario en la Web para mejorar los motores de búsqueda, y a entender los modelos compra influencia para predecir Web access, necesitamos un método para modelar y analizar secuencias de acceso Web. Con esta información, podemos deducir las solicitudes de los usuarios.

Algunos investigadores han usado modelos Markov tradicionales, que a menudo son empleados para estudiar los procesos estocásticos y predecir comportamientos acceso de usuario. En general, se utiliza la secuencia de páginas Web el usuario ha accedido a que la entrada, con el objetivo de construir modelos de Markov que pueden predecir la página a la que el usuario lo más probable es acceder a la siguiente.  usado el N-hop Markov modelos para mejorar las estrategias de prelectura cachés Web,  Markov modelos para predecir el siguiente página accede el usuario;  Lo Mejor y  utilizaron modelos Markov para clasificar las sesiones de usuario.  sin embargo, pusieron a prueba la eficacia de los diferentes modelos de Markov predicción para el acceso a la Web y tradicionales modelos de Markov son inadecuados para este propósito. Por lo tanto, necesitamos un nuevo modelo de Markov predicción para el acceso a la Web.

El híbrido de fin de árbol modelo de Markov puede predecir Web access precisamente, lo que proporciona una alta cobertura y una buena escalabilidad. HTMM inteligente combina dos métodos: una estructura de árbol modelo de Markov que agrega el método acceso secuencias de coincidencia de patrones y un híbrido de método que combina diferentes modelos de Markov. Las evaluaciones del rendimiento comparando nuestros HTMM Markov modelos tradicionales a confirmar su utilidad.




Memoria dinámica y eficiente página web modelo de predicción de LZ78 y LZW algoritmos

Acceso a la Web predicción ha despertado un gran interés en los últimos años. Prelectura Web y algunos sistemas de personalización utilizar algoritmos predicción. La mayoría de sus aplicaciones que predecir el siguiente usuario página web tienen un componente que no fuera la preparación de datos y una sección en línea que proporciona contenido personalizado para los usuarios basándose en sus actuales actividades de navegación. En este trabajo presentamos un modelo de predicción que no tiene un componente sin conexión y colocar en la memoria con una buena precisión. El algoritmo se basa en la LZ78 y LZW los algoritmos que están adaptadas para modelar la navegación del usuario en la web. Nuestro modelo reduce complejidad computacional que es un problema grave en los países en desarrollo sistemas de predicción en línea. La evaluación del desempeño se presenta mediante registros web real. Esta evaluación muestra que nuestro modelo necesita mucho menos memoria que PPM familia de algoritmos con una buena precisión.


Utilizando modelos de compresión para filtrar Comentarios Troll

Internet está evolucionando. ¿Cómo se genera el contenido ha cambiado y en la actualidad, los usuarios y lectores de un sitio puede crear contenido. Pueden expresarse mostrando sus sentimientos u opiniones comentando diversas historias o comentarios de otros usuarios en sitios web de noticias sociales. Este hecho ha llevado a efectos secundarios negativos: la aparición de troll los usuarios y sus contenidos que buscan deliberadamente polémica. En este trabajo proponemos un nuevo método para filtrar trolling comentarios utilizando modelos de compresión. Normalmente, espacio vectorial representación del modelo utilización es bastante común, pero estos filtros pueden ser atacados. Con este fin, se validan nuestro enfoque con datos de "Meneame", un popular sitio de noticias social española, la formación varios modelos de compresión, que demuestra que nuestro método puede mantener altos índices de precisión, mientras que este tipo de filtros difícil de derrotar.




% Iniciamos el resto de secciones adicionales al contenido: referencias y apendices
\backmatter


% Bibliografía
% referencias.bib es el archivo con la base de datos bibliografica
% se recomienda utilizar un manejador de referencias: Jabref (jabref.sourceforge.net)
% El estilo por defecto es IEEE Transactions
\bibliographystyle{ieeetr}
% Acá puede incluir uno más archivos de referencia
\bibliography{IEEEabrv,referencias}


% Simbología y glosario
% Utilice un paquete para generar símbolos y glosarios.
% Por ejemplo: nomencl (http://texdoc.net/pkg/nomencl)


% Anexos
\appendix

% Aca se incluyen los archivos con el texto de los anexos
% Por ejemplo, anexo.tex
\chapter{Primer anexo}
\label{ch:anexo-a}


Configuraciones para hacer correr IntelliJ con Apache SPARK y Prediction.IO 0.94


\begin{lstlisting}[frame=single,basicstyle=\ttfamily\tiny,]
Main class: io.prediction.workflow.CreateWorkflow

VM options: -Dspark.master=local -Dlog4j.configuration=file:/Users/jguzman/PredictionIO/conf/log4j.properties


Program arguments: --engine-id dummy --engine-version dummy --engine-variant engine.json


io.prediction.workflow.CreateWorkflow
-Dspark.master=local -Dlog4j.configuration=file:/Users/jguzman/PredictionIO/conf/log4j.properties -Dorg.xerial.snappy.lib.name=libsnappyjava.jnilib 
--engine-id dummy --engine-version dummy --engine-variant engine.json



SPARK_HOME=/Users/jguzman/PredictionIO/vendors/spark-1.4.1/bin
PIO_FS_BASEDIR=/Users/jguzman/.pio_store
PIO_FS_ENGINESDIR=/Users/jguzman/.pio_store/engines
PIO_FS_TMPDIR=/Users/jguzman/.pio_store/tmp
PIO_STORAGE_REPOSITORIES_METADATA_NAME=pio_meta
PIO_STORAGE_REPOSITORIES_METADATA_SOURCE=ELASTICSEARCH
PIO_STORAGE_REPOSITORIES_MODELDATA_NAME=pio_model
PIO_STORAGE_REPOSITORIES_MODELDATA_SOURCE=LOCALFS
PIO_STORAGE_REPOSITORIES_APPDATA_NAME=pio_appdata
PIO_STORAGE_REPOSITORIES_APPDATA_SOURCE=ELASTICSEARCH
PIO_STORAGE_REPOSITORIES_EVENTDATA_NAME=pio_event
PIO_STORAGE_REPOSITORIES_EVENTDATA_SOURCE=HBASE
PIO_STORAGE_SOURCES_ELASTICSEARCH_TYPE=elasticsearch
PIO_STORAGE_SOURCES_ELASTICSEARCH_HOSTS=localhost
PIO_STORAGE_SOURCES_ELASTICSEARCH_PORTS=9300
PIO_STORAGE_SOURCES_LOCALFS_TYPE=localfs
PIO_STORAGE_SOURCES_LOCALFS_HOSTS=/Users/jguzman/.pio_store/models
PIO_STORAGE_SOURCES_LOCALFS_PORTS=0
PIO_STORAGE_SOURCES_HBASE_TYPE=hbase
PIO_STORAGE_SOURCES_HBASE_HOSTS=0
PIO_STORAGE_SOURCES_HBASE_PORTS=0


Main class: io.prediction.workflow.CreateServer
Program Arguments: --engineInstanceId **replace_with_the_id_from_pio_train**




Try -- for more information.
Usage: pio train [--batch <value>] [--skip-sanity-check]
                 [--stop-after-read] [--stop-after-prepare]
                 [--engine-factory <value>] [--engine-params-key <value>]
                 [--scratch-uri <value>]
                 [common options...]

Kick off a training using an engine (variant) to produce an engine instance.
This command will pass all pass-through arguments to its underlying spark-submit
command.

  --batch <value>
      Batch label of the run.
  --skip-sanity-check
      Disable all data sanity check. Useful for speeding up training in
      production.
  --stop-after-read
      Stop the training process after DataSource.read(). Useful for debugging.
  --stop-after-prepare
      Stop the training process after Preparator.prepare(). Useful for
      debugging.
  --engine-factory
      Override engine factory class.
  --engine-params-key
      Retrieve engine parameters programmatically from the engine factory class.
  --scratch-uri
      URI of the working scratch space. Specify this when you want to have all
      necessary files transferred to a remote location. You will usually want to
      specify this when you use --deploy-mode cluster.

\end{lstlisting}

\vspace{1cm}

Como hacer llamadas curl desde la consola o terminal Linux


\begin{lstlisting}

curl -H "Content-Type: application/json"  -d '{"webaccess" : "AC","num" : 10}' http://52.33.180.212:8000/queries.json



\end{lstlisting}





%\blindtext[5]


\chapter{Segundo anexo}
\label{ch:anexo-b}


%\blindtext[10]



% puede incluir más archivos de anexos
% \input{anexo-dos}

\end{document}
% that's all folks