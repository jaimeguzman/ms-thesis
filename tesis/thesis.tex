% Ejemplo del uso de la template para escribir tesis/memorias de la Universidad Diego Portales.
%
% Eviar bugs a: Adín Ramírez, adin.ramirez (at) mail.udp.cl

% Puede generar borradores si omite la opción "final" de la clase.
% \documentclass{udpthesis}
\documentclass[final]{udpthesis}

% Establecemos el sistema para uso del español
% Babel ya esta cargado dentro de updthesis
\usepackage[T1]{fontenc}%    output
\usepackage[utf8]{inputenc}% input
\usepackage{lmodern}

% Leyendas
\usepackage[font=footnotesize,labelfont=bf,labelsep=period]{caption}

% Agregue acá otros paquetes que le sean de utilidad
% Matemáticas
\usepackage{amsmath}

% Gráficos
\usepackage{graphicx}
\usepackage[font=footnotesize,labelformat=simple]{subfig}
% Cambiamos el formato de las leyendas: finalizan en punto, y en negrita.
\captionsetup{labelsep=period,labelfont=bf}
% Habilitamos el uso de paréntesis al citar las figuras con subfiguras dentro, e.g., Fig. 1(a)
\renewcommand\thesubfigure{(\alph{subfigure})}
\renewcommand\thesubtable{(\alph{subtable})}
\newcommand{\subfigureautorefname}{\figureautorefname}

% Código
% Para generar código fuente usar listings.sty
%\usepackage{listings}
%\usepackage{tikz}
%\lstset{
%  language=[LaTeX]TeX,
%  breaklines=true,
%  basicstyle=\tt\scriptsize,
%  keywordstyle=\color{blue},
%  identifierstyle=\color{magenta},
%  commentstyle=\color{green!40!black},
%  % frame 
%  frame=tb,
%  captionpos=t,
%  xleftmargin=1em,
%  numbersep=0.3em,
%  numbers=left,
%  framexleftmargin=1.1em,
%  framexrightmargin=0pt,
%  % additional letters for accents in spanish
%  literate=%
%    {á}{{\'{a}}}1
%    {é}{{\'{e}}}1
%    {í}{{\'{i}}}1
%    {ó}{{\'{o}}}1
%    {ú}{{\'{u}}}1
%    {ñ}{{\~{n}}}1
%    {Ñ}{{\~{N}}}1
%}
%
%\renewcommand{\lstlistingname}{Código}% Listing -> Código
%\DeclareCaptionFormat{listing}{\rule{\dimexpr\linewidth\relax}{0.4pt}\par\vskip1pt#1#2#3}
%\captionsetup[lstlisting]{format=listing,singlelinecheck=false, margin=0pt,position=bottom}

% O para generar algoritmos en pseudocódigo usar algpseudocode.sty
%\usepackage{algorithm}
%\usepackage{algpseudocode}
%\makeatletter
%\renewcommand{\ALG@name}{Algoritmo}% Algorithm -> Algoritmo
%\makeatother
%\captionsetup[algorithm]{font=footnotesize,labelsep=period}

% Referencias (este paquete ordena y comprime las referencias)
\usepackage{cite}

% Un paquete para generar texto. REMUEVA ESTE PAQUETE AL UTILIZAR ESTA PLANTILLA.
\usepackage{blindtext}


% Establecemos el tema a utilizar. 
% Debe existir el archivo udpthesisEIT.sty en su sistema TeX para poder utilizarlo.
% Por ejemplo, para utilizar el tema de magíster de la EIT deben de utilizar
% \udptheme{EIT-MS}
\udptheme{EIT}


\begin{document}
%% Inicio de la portada
\frontmatter

% Título del tema (no más de 12 palabras)
\title{Modelo Híbrido de LZ-78 y Machine Learning para predicción de comportamientos de usuarios basado en web access log}
% para precisar aún más su tema, use un subtítulo
%\subtitle{Subtítulo explicativo del tema}

% El autor(es) de la tesis
\author{Jaime Guzmán}
\email{mail@jguzman.cl}% utilice un correo que revise después de graduado

% o una lista de autores separados con comas
%\author{Juan Bar, José Foo}
%\email{juan.bar@mail.udp.cl, jose.foo@mail.udp.cl}

% Fecha a aparecer en la tesis
\date{2015}

% Profesor guía
\professor{Adin Ramirez}
% Comité
\committee{Francisco Claude}{Darth Vader}

% Dedicatoria
\dedicatory{Utilice un par de oraciones para dedicar su tesis, o una frase de alguién importante.}

% Agradecimientos
\acknowledgment{Nota redactada sobriamente en la cual se agradece a quienes han colaborado en la elaboración del trabajo. No puede exceder más de una página.}

% Abstrac en inglés
\abstract{%<- evita nueva linea en el abstract
Abstract is the summary in english of the subject your are presenting in this thesis. Should not exceed one page.}

% Resumen
\resumen{%<- evita nueva linea en el resumen 
El resumen no debe contener menos de 100 palabras ni mas de 300 palabras.}

% Generamos la portada
\makecover

% Indices y listas
\tableofcontents% tabla de contenido
\listoftables%    índice de tablas
\listoffigures%   índice de figuras
% puede agregar otras listas o índices acá de ser necesario



% Inicio del contenido
\mainmatter

% Capítulos y secciones del documento
% Aca se incluyen los archivos con el texto de los capitulos
% Incluyo el archivo cap-intro.tex
\input{cap-intro}
% Incluyo el archivo cap-tema.tex
\chapter[Tema]{Compresión}
\label{ch:tema}




% genero más documento de forma aleatoria


% Genero toda las referencias para demostrar el uso de la bibliografía
% No es necesario que utilice este comando en su documento.
\nocite{*}

% incluya otros archivos según su necesidad
% 1.-resumen de archivos para tesis

Predicción de Markov de Orden Variable

Este documento se refiere a los algoritmos de predicción de secuencias discretas en un alfabeto finito, utilizando variables modelos Markov. La clase de este tipo de algoritmos es amplio y comprende en principio cualquier algoritmo de compresión sin pérdida. Nos enfocamos en seis prominentes algoritmos de predicción, incluyendo  (CTW), (PPM) y (PST).  Discutiremos las propiedades de estos algoritmos y podemos comparar su rendimiento con secuencias reales de tres dominios.
La comparación se hace con respecto a la predicción calidad medido por el promedio de pérdida de registro. También comparamos algoritmos de clasificación basados en estos predictores con respecto a un gran número de clasificación de  tareas. Nuestros resultados indican que una "descomposición" CTW (una variante del algoritmo CTW) y PPM superan todos los demás algoritmos de predicción de las tareas. Sorprendentemente, un algoritmo diferente, que es una modificación del algoritmo de compresión Lempel-Ziv supera todos los algoritmos de  problemas de clasificación.

UN NUEVO MODELO DE MARKOV PARA EL ACCESO A LA WEB PREDICCIÓN

Predecir con exactitud el comportamiento del usuario al acceder a la Web puede reducir al mínimo la latencia que percibe el usuario, que es crucial en el rápido y creciente World Wide Web. A pesar de que los modelos Markov han ayudado a predecir comportamientos de acceso de usuario, tienen graves limitaciones. Los modelos Híbridos de árbol, Markov predicen acceso a la Web, precisamente al mismo tiempo que ofrecen un alto nivel de cobertura y escalabilidad.

La World Wide Web es una gran base de datos donde se almacena  y se accesa a la información, permite a los usuarios navegar a través de enlaces y ver con los exploradores. El tráfico de Internet ha aumentado considerablemente debido a la popularidad de la Web y como consecuencia los usuarios perciben la latencia. La solución obvia de incrementar el ancho de banda, no es viable, ya que no podemos cambiar fácilmente la infraestructura de la Web (Internet) sin gran costo económico. Sin embargo, si se puede predecir las búsquedas del futuro usuario, podríamos poner esas páginas en el lado del cliente de caché cuando el navegador es gratuito. Cuando un usuario solicita una de las páginas, el navegador puede recuperarlo directamente desde la memoria caché.

Gran parte de las actuales investigaciones han examinado modelos y buscan predecir comportamientos acceso de usuario en la Web para mejorar los motores de búsqueda, y a entender los modelos compra influencia para predecir Web access, necesitamos un método para modelar y analizar secuencias de acceso Web. Con esta información, podemos deducir las solicitudes de los usuarios.

Algunos investigadores han usado modelos Markov tradicionales, que a menudo son empleados para estudiar los procesos estocásticos y predecir comportamientos acceso de usuario. En general, se utiliza la secuencia de páginas Web el usuario ha accedido a que la entrada, con el objetivo de construir modelos de Markov que pueden predecir la página a la que el usuario lo más probable es acceder a la siguiente.  usado el N-hop Markov modelos para mejorar las estrategias de prelectura cachés Web,  Markov modelos para predecir el siguiente página accede el usuario;  Lo Mejor y  utilizaron modelos Markov para clasificar las sesiones de usuario.  sin embargo, pusieron a prueba la eficacia de los diferentes modelos de Markov predicción para el acceso a la Web y tradicionales modelos de Markov son inadecuados para este propósito. Por lo tanto, necesitamos un nuevo modelo de Markov predicción para el acceso a la Web.

El híbrido de fin de árbol modelo de Markov puede predecir Web access precisamente, lo que proporciona una alta cobertura y una buena escalabilidad. HTMM inteligente combina dos métodos: una estructura de árbol modelo de Markov que agrega el método acceso secuencias de coincidencia de patrones y un híbrido de método que combina diferentes modelos de Markov. Las evaluaciones del rendimiento comparando nuestros HTMM Markov modelos tradicionales a confirmar su utilidad.




Memoria dinámica y eficiente página web modelo de predicción de LZ78 y LZW algoritmos

Acceso a la Web predicción ha despertado un gran interés en los últimos años. Prelectura Web y algunos sistemas de personalización utilizar algoritmos predicción. La mayoría de sus aplicaciones que predecir el siguiente usuario página web tienen un componente que no fuera la preparación de datos y una sección en línea que proporciona contenido personalizado para los usuarios basándose en sus actuales actividades de navegación. En este trabajo presentamos un modelo de predicción que no tiene un componente sin conexión y colocar en la memoria con una buena precisión. El algoritmo se basa en la LZ78 y LZW los algoritmos que están adaptadas para modelar la navegación del usuario en la web. Nuestro modelo reduce complejidad computacional que es un problema grave en los países en desarrollo sistemas de predicción en línea. La evaluación del desempeño se presenta mediante registros web real. Esta evaluación muestra que nuestro modelo necesita mucho menos memoria que PPM familia de algoritmos con una buena precisión.


Utilizando modelos de compresión para filtrar Comentarios Troll

Internet está evolucionando. ¿Cómo se genera el contenido ha cambiado y en la actualidad, los usuarios y lectores de un sitio puede crear contenido. Pueden expresarse mostrando sus sentimientos u opiniones comentando diversas historias o comentarios de otros usuarios en sitios web de noticias sociales. Este hecho ha llevado a efectos secundarios negativos: la aparición de troll los usuarios y sus contenidos que buscan deliberadamente polémica. En este trabajo proponemos un nuevo método para filtrar trolling comentarios utilizando modelos de compresión. Normalmente, espacio vectorial representación del modelo utilización es bastante común, pero estos filtros pueden ser atacados. Con este fin, se validan nuestro enfoque con datos de "Meneame", un popular sitio de noticias social española, la formación varios modelos de compresión, que demuestra que nuestro método puede mantener altos índices de precisión, mientras que este tipo de filtros difícil de derrotar.




% Iniciamos el resto de secciones adicionales al contenido: referencias y apendices
\backmatter


% Bibliografía
% referencias.bib es el archivo con la base de datos bibliografica
% se recomienda utilizar un manejador de referencias: Jabref (jabref.sourceforge.net)
% El estilo por defecto es IEEE Transactions
\bibliographystyle{ieeetr}
% Acá puede incluir uno más archivos de referencia
\bibliography{IEEEabrv,referencias}


% Simbología y glosario
% Utilice un paquete para generar símbolos y glosarios.
% Por ejemplo: nomencl (http://texdoc.net/pkg/nomencl)


% Anexos
\appendix

% Aca se incluyen los archivos con el texto de los anexos
% Por ejemplo, anexo.tex
\chapter{Anexos técnicos}
\label{ch:anexo-a}


\section{Uso de linea de Comando Prediction.IO}


La interación con \emph{PIO} es a través de una interface de linea de comando, esta sigue el siguiente formato de uso:

 

\begin{lstlisting}[frame=single,basicstyle=\ttfamily\tiny,]
  pio <command> [options] <args>...
\end{lstlisting}



En caso de tener duda usted al igual que los comandos de bash puede ejecutar

\begin{lstlisting}[frame=single,basicstyle=\ttfamily\tiny,]
pio help <command> 
\end{lstlisting}, para ver mas detalles de cada detalle de los comandos disponibles.


Los comandos de PredictionIO se pueden separar en tres categorías:

\begin{itemize}
	\item \textbf{help} Muestra un resumen del uso  \emph{pio help <command>} para leer sobre un sub comando
	
	\item \textbf{Version} Muestra la version instalada de PredictionIO
	
	
	\item \textbf{Status} Muesta la ruta de instalación y el estatus de ejecución de sistema, como también sus dependencias
	
\end{itemize}

Comandos del servidor de eventos



\begin{itemize}
	\item \textbf{app} Muestra un resumen del uso  \emph{pio help <command>} para leer sobre un sub comando
	
	\item \textbf{Version} Muestra la version instalada de PredictionIO
	
	
	\item \textbf{app }  Administra todas las aplicaciones que usa el servidor de eventos
	\item \textbf{pio app data-delete <name> } Borra toda la data contenida por una aplicación específica
	\item \textbf{pio app delete <name> }  Borra una aplicación completa
	\item \textbf{eventserver }  Lanza el servidor de eventos
	\item \textbf{ --ip <value> } Une la IP seleccionada, el valor por defecto es \emph{localhost}
	\item \textbf{access\_key} Administra todas las llaves de acceso al servidor de eventos ó app
	
\end{itemize}


\section {Comandos del Motor de Predicción}

Es requerido que estos comandos se ejecuten desde la misma carpeta que contiene el proyecto ó aplicación desplegada.

Las opciones  --debug y --verbose  muestran información detallada sobre los conectoes de las aplicaciones complementarias a las que esta compuesta PredictionIO

\begin{itemize}
	\item \textbf{build} Construye y compila el proyecto completo desde la carpeta fuente, tiene un flag adicional \emph{-- clean}, para una compilación limpia.

	\item \textbf{train} Ejecuta el entrenamiento declarado en el motor 

	\item \textbf{deploy} Despliega el motor para ser usado meiante REST como Algoritmo como Servicio. Si no existe ninguna nueva instancia  desplegada, por defecto usará la última creada.
	
\end{itemize}


\newpage
\section{Configuraciones para hacer correr IntelliJ con Apache SPARK y Prediction.IO 0.94}

\begin{lstlisting}[frame=single,basicstyle=\ttfamily\tiny,]
Main class: io.prediction.workflow.CreateWorkflow

VM options: -Dspark.master=local -Dlog4j.configuration=file:/Users/jguzman/PredictionIO/conf/log4j.properties


Program arguments: --engine-id dummy --engine-version dummy --engine-variant engine.json


io.prediction.workflow.CreateWorkflow
-Dspark.master=local -Dlog4j.configuration=file:/Users/jguzman/PredictionIO/conf/log4j.properties -Dorg.xerial.snappy.lib.name=libsnappyjava.jnilib 
--engine-id dummy --engine-version dummy --engine-variant engine.json



SPARK_HOME=/Users/jguzman/PredictionIO/vendors/spark-1.4.1/bin
PIO_FS_BASEDIR=/Users/jguzman/.pio_store
PIO_FS_ENGINESDIR=/Users/jguzman/.pio_store/engines
PIO_FS_TMPDIR=/Users/jguzman/.pio_store/tmp
PIO_STORAGE_REPOSITORIES_METADATA_NAME=pio_meta
PIO_STORAGE_REPOSITORIES_METADATA_SOURCE=ELASTICSEARCH
PIO_STORAGE_REPOSITORIES_MODELDATA_NAME=pio_model
PIO_STORAGE_REPOSITORIES_MODELDATA_SOURCE=LOCALFS
PIO_STORAGE_REPOSITORIES_APPDATA_NAME=pio_appdata
PIO_STORAGE_REPOSITORIES_APPDATA_SOURCE=ELASTICSEARCH
PIO_STORAGE_REPOSITORIES_EVENTDATA_NAME=pio_event
PIO_STORAGE_REPOSITORIES_EVENTDATA_SOURCE=HBASE
PIO_STORAGE_SOURCES_ELASTICSEARCH_TYPE=elasticsearch
PIO_STORAGE_SOURCES_ELASTICSEARCH_HOSTS=localhost
PIO_STORAGE_SOURCES_ELASTICSEARCH_PORTS=9300
PIO_STORAGE_SOURCES_LOCALFS_TYPE=localfs
PIO_STORAGE_SOURCES_LOCALFS_HOSTS=/Users/jguzman/.pio_store/models
PIO_STORAGE_SOURCES_LOCALFS_PORTS=0
PIO_STORAGE_SOURCES_HBASE_TYPE=hbase
PIO_STORAGE_SOURCES_HBASE_HOSTS=0
PIO_STORAGE_SOURCES_HBASE_PORTS=0


Main class: io.prediction.workflow.CreateServer
Program Arguments: --engineInstanceId **replace_with_the_id_from_pio_train**




Try -- for more information.
Usage: pio train [--batch <value>] [--skip-sanity-check]
                 [--stop-after-read] [--stop-after-prepare]
                 [--engine-factory <value>] [--engine-params-key <value>]
                 [--scratch-uri <value>]
                 [common options...]

Kick off a training using an engine (variant) to produce an engine instance.
This command will pass all pass-through arguments to its underlying spark-submit
command.

  --batch <value>
      Batch label of the run.
  --skip-sanity-check
      Disable all data sanity check. Useful for speeding up training in
      production.
  --stop-after-read
      Stop the training process after DataSource.read(). Useful for debugging.
  --stop-after-prepare
      Stop the training process after Preparator.prepare(). Useful for
      debugging.
  --engine-factory
      Override engine factory class.
  --engine-params-key
      Retrieve engine parameters programmatically from the engine factory class.
  --scratch-uri
      URI of the working scratch space. Specify this when you want to have all
      necessary files transferred to a remote location. You will usually want to
      specify this when you use --deploy-mode cluster.

\end{lstlisting}

\vspace{1cm}


\section {Llamadas al Servidor de Machine Learning mediante curl }


\begin{lstlisting}

curl -H "Content-Type: application/json"  -d '{"webaccess" : "AC","num" : 10}' http://52.33.180.212:8000/queries.json



\end{lstlisting}



\section{Python SDK para PredictionIO}

Para hacer uso de estos scripts en python es necesario tener instalado el package de prediction para python sdk.

Si se tiene pip instalado correctamente se puede utilizar

\begin{verbatim}
pip install predictionio
\end{verbatim}
ó
\begin{verbatim}
$ easy_install predictionio
\end{verbatim}


Es recomendable tener acceso sudo para evitar problemas con permisos al momento de usar \emph{pip} o \emph{easy\_install} {(ie. sudo pip install predictionio)}.




El siguiente script, permite hacer una sola consulta la cual puede ser ejecutada desde el \emph{cli} de python ó llamando directamente al archivo ejecutable. ( {python test.py})

\begin{lstlisting}[frame=single,basicstyle=\ttfamily\tiny,]
import predictionio

engine_client = predictionio.EngineClient(url="http://localhost:8000")

print engine_client.send_query({"webaccess": "A", "num": 10})
\end{lstlisting}



\newpage
A diferencia del script explicado anteriomente este permite enviar secuencias desde la terminal \emph{cli}, la cual puede seguir en ejecución hasta que el usuario finalice el proceso.

\begin{lstlisting}[frame=single,basicstyle=\ttfamily\tiny,]
"""
Send sample query to prediction engine
"""

import predictionio
import readline

engine_client = predictionio.EngineClient(url="http://localhost:8000")
while True:
    word = raw_input('Enter a Sequences or a single page to predict the next user webaccess: \n')
    print engine_client.send_query({"webaccess": word, "num": 10})

\end{lstlisting}





\section{Programa C++ para hacer splits dentro del Dataset}
Programa para poder pasar la data de msnbc a una representación de símbolos.

\begin{lstlisting}[frame=single,basicstyle=\ttfamily\tiny,]
#include <iostream>     // cout
#include <fstream>      // ifstream
#include <sstream>
#include <algorithm>
#include <string>
#include <cmath>
#include <cstdio>
#include <vector>
#include <map>
#include <iterator>

using namespace std;

/** 
  alias lseq = g++ -std=c++11 letterSequences.cpp -o letterSequences 
  ./letterSequences

% Different categories found in input file:

frontpage news tech local opinion on-air misc weather msn-news health living business msn-sports sports summary bbs travel
**/

 
int main()
{
   map<string, int> mapCategories;

  
   // Inserting data in map
  mapCategories.insert(make_pair("frontpage", 1));
    mapCategories.insert(make_pair("news",    2));
    mapCategories.insert(make_pair("tech",    3));
    mapCategories.insert(make_pair("local",   4));
    mapCategories.insert(make_pair("opinion",   5));
    mapCategories.insert(make_pair("on-air",  6));
    mapCategories.insert(make_pair("misc",    7));
    mapCategories.insert(make_pair("weather",   8));
    mapCategories.insert(make_pair("msn-news",  9));
    mapCategories.insert(make_pair("health",  10));
    mapCategories.insert(make_pair("living",  11));
    mapCategories.insert(make_pair("business",  12));
    mapCategories.insert(make_pair("msn-sports",13));
    mapCategories.insert(make_pair("sports",  14));
    mapCategories.insert(make_pair("summary",   15));
    mapCategories.insert(make_pair("bbs",     16));
    mapCategories.insert(make_pair("travel",  17));
    

   vector<char> alphabet = { 'A','B','C','D','E','F','G',
                'H','I','J','K','L','M','N','O',
                'P','Q','R','S','T','U','V','W',
                'X','Y','Z'};

   // Iterate through all elements in map
   map<string, int>::iterator it = mapCategories.begin();

   ifstream  fin("msnbc990928.seq");
   string    file_line;
   int fold = 0 ;

   while(getline(fin, file_line)){

    string    buf; // Have a buffer string
    stringstream  ss(file_line); // Insert the string into a stream
    vector<string> tokens; // Create vector to hold our words
    
    while (ss >> buf) tokens.push_back(buf);

    if( tokens.size() < 6 ){
      ++fold;
      for (int i = 0; i < tokens.size(); ++i){
        string tmp = tokens.at(i); 
        cout << alphabet.at( stoi(tmp) - 1) << " ";
      }cout<< endl;

    }

    //this value is for make the size of the folds of data
    if( fold == 1000000 ) break;

   }
    return 0;
}
\end{lstlisting}











% puede incluir más archivos de anexos
% \input{anexo-dos}

\end{document}
% that's all folks