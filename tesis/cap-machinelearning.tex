\chapter[Experimental]{Experimental}
\label{ch:tema}
%Los ob jetivos no esta ́n diferenciados entre generales y espec ́ıficos.

%e presentan proce- sos que validan el tra- ba jo comparando con otros m ́etodos del es- tado del arte. Se pre- senta una validacio ́n depara ́metros,yto- das las dimensiones que fueron plantea- das son exploradas en profundidad.


%Se detalla y explica la recoleccio ́n (de apli- car) y ana ́lisis de da- tos y resultados.
%Se distinguen los re- sultados segu ́n las va- riables investigadas, y se examinan en virtud de lo explicado en el marco te ́orico.
%Se interpretan con claridad (es decir, se consideran distintos factores y anteceden- tes descritos hasta el momento) los resultados obtenidos.


Lo que se busca es usar como un modelo de predicción un algoritmo de la familia de compresores de Lempel Ziv. Se usará para predecir secuencias finitas discretas de accesos web. Al momento de generar el arbol este creará una represetnación trie de un diccionario de simbolos, el cual no utilizaremos para poder hacer predicciones.

En base a lo anterior y teniendo funciando el algoritmo para crear un modelo de predicción, la integraremos con el servidor de Machine Learning, Prediction.IO que ya se ha explicado anteriormente.



%\subsection{Ambiente experimental}

Para los ambientes experimentales se han dispuestos dos máquinas para realizar las pruebas.

\subsubsection{Máquinas}
\begin{itemize}
	\item Procesador 2,8 GHz Intel Core i7, 16 GB de Memoria RAM y Sistema Operativo OSX
	\item Procesadores Intel Xeon E5-2670 v2 (Ivy Bridge) de alta frecuencia 32 vCPU, 244 GB de Memoria RAM y Sistema Operativo Ubuntu 14.14 
\end{itemize}

\subsubsection{Software}

\begin{itemize}
	\item C++11
%	\item Java version "1.8.0_51"
%	\item Java(TM) SE Runtime Environment (build 1.8.0_51-b16)
%	\item Java HotSpot(TM) 64-Bit Server VM (build 25.51-b03, mixed mode)
	\item Scala code runner version 2.11.7 -- Copyright 2002-2013, LAMP/EPFL
	\item SBT 0.13.9 
	\item Python 2.7.10
	\item Prediction.IO 0.9.4
	\item elasticsearch 1.4.4	
	\item Apache Spark-1.4.1
	\item Hbase 1.0.0
	\item Zookeeper 

\end{itemize}





\subsection{Resultados Experimentales}


%Las conclusiones son deducidas lo ́gicamen- te de los resultados obtenidos y de la interpretacio ́npresen- tada, adema ́s esta ́n conectadas al marco teo ́rico.
%Las conclusiones muestran el logro de los ob jetivos.
%Se presentan proyec- ciones va ́lidas y valio- sas a partir del traba- jo realizado.
%Se detallan claramen- te las limitaciones del traba jo realizado.
\section{Conclusiones y Resultados}





 
 
 




 

% Genero toda las referencias para demostrar el uso de la bibliografía
% No es necesario que utilice este comando en su documento.
% \nocite{*}
