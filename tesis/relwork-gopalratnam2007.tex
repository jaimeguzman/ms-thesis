Existen áreas que para lograr su realización la predicción en secuencias de eventos que por lo general se puede modelar como procesos estocásticos.
Acorde con su trabajo se concuerdan que la predicción es una componente importante en una serie de ámbitos en Inteligencia Artificial y \emph{Machine Learning}, con el fin de que sistemas inteligentes puedan tomar decisiones cada vez más informadas y confiables.  Gopalratnam \etal\cite{Gopalratnam2007} estudian y dan un gran estudio al estado del arte de la predicciones secuenciales de eventos.

Gopalratnam \& Cook, 2007 \etal\cite{Gopalratnam2007} proponen un algoritmo \emph{On-Demand} que considera varios modelos de Markov para su creación.  Este algoritmo de predicción secuencial se basa en un enfoque de teoría de la información, y se basa en \texttt{LZ78} de la fammilia de algoritmos de compresión de datos de \emph{Lempel Ziv}. La eficacia de este algoritmo en un típico {Ambiente Inteligente}, \emph{Smart Home} (la Casa Inteligente), es demostrada mediante el empleo de este algoritmo para predecir el uso de dispositivos en el hogar, este trabajo ha sido una incursión en el campo de \emph{Internet of Things},(\texttt{IoT}). El rendimiento de este algoritmo se probó en conjuntos de datos sintéticos que son representativos de las interacciones típicas entre una casa inteligente y sus habiantes. Además, para el ambiente de la Casa Inteligente, se introduce un método de aprendizaje de una medida del tiempo relativo entre las acciones usando \emph{ActiveLezi} (\texttt{ALZ})\label{acro-ActiveLezi}, y demostrar la eficacia de este enfoque en la síntesis de datos inteligente de la Casa Inteligente.

El funcionamiento es basado en almacenar la frecuencia del patrón de \emph{input} en un \emph{trie} acorde al algoritmo de compresión de \texttt{LZ78} para superar algunos de los problemas que surgen con \texttt{LZ78}, se usa una ventana de largo variable de los símbolos previamente usados en la construcción del \emph{trie}. El tamaño de la ventana crece con el número de las diferentes sub-secuencias que se van viendo en la entrada de cada secuencia nueva que ingresa.  Sea ${\mbox{suff}}_{l}$,  el sufijo de largo $l+1$ ,el sufijo de longitud $l=1$ de las inmediatamente historial de interacción.% a, que es un hacha .... la probabilidad se define de la siguiente forma recursiva.:

Gopalratnam \& Cook, 2007 \etal\cite{Gopalratnam2007} efectivamente modelan procesos secuenciales, y es extremadamente útil para la predicción de procesos donde los eventos dependen de la historia de un evento anterior. Esto es debido a la capacidad del algoritmo para construir un modelo preciso de la fuente de los eventos que se genera, una característica heredada de su información de antecedentes teóricos(``base de conocimiento histórica'') y el algoritmo de compresión de texto \texttt{LZ78}. La eficacia para el aprendizaje de una medida de tiempo también se puede atribuir al hecho de que \texttt{ALZ} (\ref{acro-ActiveLezi}) es un potente predictor secuencial. Los sólidos principios teóricos en los que se fundamenta \texttt{ALZ} también significan que \texttt{ALZ} es un óptimo predictor universal, y se puede utilizar en una variedad de escenarios de predicción.
	
  	
  	  