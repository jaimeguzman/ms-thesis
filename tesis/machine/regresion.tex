% Referencias
% ~\cite{gollapudi2016practical} SUNILA


	El análisis de regresión nos permite modelar matemáticamente la relación entre dos variables utilizando el álgebra simple~\cite{gollapudi2016practical}.
	Es una técnica de clasificación que es particularmente adecuada para un modelo continuo, lineal (mínimos cuadrados), polinomio, y regresiones logísticas. Se encuentran entre las técnicas más utilizadas para mejorar o ajustar un modelo paramétrico. Se considera que la regresión es un caso especializado de la  clasificación, para los cuales las variables de salida son continuas en lugar de categórica.

% Regression techniques are used to predict response variables with numerical outcomes, such as predicting the miles per gallon of a car or predicting the temperature of a city. The input variables may be numeric or categorical. However, what is common with these algorithms is that the output (or response variable) is numeric. We’ll review some of the most commonly used regression techniques including linear regression, neural networks, decision trees, and boosted decision tree regression.