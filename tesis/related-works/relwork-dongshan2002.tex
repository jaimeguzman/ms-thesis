Destacan que un modelo de Markov puede ayudar a predecir el comportamiento de un usuario, pero con ciertas limitaciones. Para solucionarlo presentan un nuevo modelo de Markov basado en una representación de \emph{Tree Order Model}, el cual es un híbrido entre un modelo de Markov tradicional y una representación de árbol, bautizada como \texttt{HTMM} (por sus siglas en inglés, \emph{Hybrid-Order Tree Markov Model}).
Su modelo fue presentado en 2002, y es relevante conocer la predicción de los \emph{webaccess}, dada la importancia de creación de redes, la minería de datos, \emph{e-commerce}, y otras áreas.
