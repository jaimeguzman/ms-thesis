Internet is growing every day and data volumes grow in the order of 
terabytes, for this reason it is of interest to use compression 
techniques for larger-scale processing of information with the fewest 
possible resources. Today there are various types of web: social networks, microbloging, web information, etc. The content provided to end 
users is not static and this allows them to create, contribute, and 
modify content, so Web Development Industry is constantly evolving to generate 
better solutions and friendly interaction with the users. 
Many of these new technologies have to deliver a better experience when browsing, the breakthrough has made it possible to create Web that are themselves intelligent and anticipating their behavior can go to, for example, decrease latency from a website and accessed or since navigate within a site with high concurrency; also from the point of view of the current web service patterns that are consumed immediately give a wide application area and provides a strategic aspect to the use of the information. While the growth of storage resources in the cloud is in full swing, networks don't grow at the same speed, which gives an area of interest to deepen.


So in this thesis we  propose the creation of a hybrid model between Machine Learning and Lossless Data Compression  for predict the next page that some user can access within a web; using training techniques and compression algorithms like well know \texttt{LZ78}, on a server Machine Learning. To this end we will work to create and study a predictive model that uses both areas and provide the results of the predictive model with integrated online component to any type of platform.