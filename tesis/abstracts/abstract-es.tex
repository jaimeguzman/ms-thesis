%El resumen no debe contener menos de 100 palabras ni mas de 300 palabras.

Internet tiene un crecimiento exponencial  y los datos se generan en grandes volúmenes, en el orden de los \emph{Terabytes}. Hoy en día existen varios tipos de \emph{webs}: redes sociales, microbloging, \emph{web} informativas, etc. estas genera un interés en usar técnicas de compresión para realizar procesamiento a mayor escala de información con la menor cantidad de recursos posibles. El contenido proporcionado a los usuarios finales ya no es estático y esto permite que los mismos puedan  aportar, modificar ó eliminar contenido. Así  la industria del desarrollo \emph{web} está en constante evolución para generar más variados recursos para ser usados. Muchas de estas nuevas tecnologías han permitido entregar mejor experiencia al momento de navegar en el lado del cliente, pero el gran avance realizado aún no permite crear \emph{web} que sean en sí inteligente por si misma y pueda ir anticipando la  navegación de un usuario o la recomendación de un contenido, tenemos los siguientes ejemplos; disminuir la latencia desde que se comienza a navegar una \emph{web,} prediciendo los futuros accesos del usuario o aún mejor reconociendo patrones discretos en la navegación de un usuario,  de un sitio concurrente; también desde el enfoque de los patrones de desarrollo \emph{web} actuales, los servicio  se consumen inmediatamente y dan una área de integración y  aplicación extensa como el caso estudio que abordaremos o los ejemplos que se señalarán, aportando un aspecto estratégico al uso de la información con procesamiento inmediato para la ayuda de elección decisiones.
Si bien el crecimiento del almacenamiento de datos en la nube se encuentra en auge en la industria, las redes de telecomunicaciones  no crecen a la misma velocidad, lo cual genera un campo de interés para investigar nuevas implementación de modelos predictivos en grandes volúmenes de datos. 

En este trabajo proponemos  la creación de un modelo híbrido entre \emph{Machine Learning} y algoritmos de tipo \emph{Lossless Data Compression} para predecir la siguiente secuencia de acceso que un usuario puede realizará en una \emph{web}; usando un modelo de navegación predictiva basado en un Algoritmo de Compresión sin pérdida como \texttt{LZ78}, sobre los componentes de un servidor de \emph{Machine Learning}. Con este propósito se intentará crear y estudiar un modelo predictivo que use ambas áreas y  ofrezca resultados del modelo predictivo con una componente \emph{online} integrable a cualquier tipo de plataforma cliente.