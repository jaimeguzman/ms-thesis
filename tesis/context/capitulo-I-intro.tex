%
% Introduccion 
%
%
Internet es una fuente de información que crece constantemente y es visitada con una gran frecuencia, posee billones de \webs públicas. Dado este rápido desarrollo y accesos a grandes colecciones de datos se vuelve fundamental y desafiante lograr anticipar las acciones que un usuario realice, tanto para investigadores e ingenieros. Los beneficios de lograr predicciones con la suficiente precisión pueden  mejorar la experiencia de usuarios en base a su comportamiento ó mejorar la experiencia de compras en comercios electrónicos, por ejemplo como un escenario de estudio.

Cuando se navega en un sitio \web se puede recolecta una gran cantidad de registros guardados desde el lado del servidor, los cuales por ejemplo son datos de accesos de la sesión realizada,  también estos registros representan usuarios de una \web que se encuentran activos (\online), llamaremos en adelante a estos registros de acceso: \webasccesslog. En estos registros se pueden realizar variados análisis, un escenario es cuando existe un usuario el cual intenta comprar un cierto producto en una \web de comercio electrónico. Este sigue un cierto comportamiento hasta el momento de generar una comprar, supongamos que el usuario se encuentra en un estado de búsqueda o visita todas las posibles elecciones, puede volver a las ya vista, abandonar la actual página de cierto productos, hasta encontrar lo que desea y comprar.

Existen varias área que buscan un estudio de este problema, como también al ejemplo que hemos mencionado, el de las predicciones de páginas \web, también conocido como \texttt{WPP}. Algunos de estos enfoques para  abordarlo, por ejemplo son:  Análisis Predictivos de Datos, \emph{Web Usage Minning}, \emph{Web Access Pattern}, \machinelearning y predicciones secuenciales con algoritmos de compresión. Esta tesis usará estas dos ultimas áreas para la implementación de un modelo predictivo simple, en un servidor de \machinelearning. La  propuesta se basa en la implementación del algoritmo \lzSieteOcho que está adaptado para modelar y representar la navegación secuencial del usuario. Nuestra propuesta disminuye el tiempo de actualización del modelo predictivo, con actualizaciones a medida que se va usando. Además la puesta en marcha del servicio, que es un desventaja en el desarrollo de sistemas predictivos \online.


Predecir estos  accesos no es trivial, aún los modelos predictivos que se han implementado no logran dar con un patrón para la navegación de usuario, de manera genérica. Dado la diversidad de perfiles de usuarios que se pueden encontrar en ciertas \webs  y distintos flujos de navegación de contenido de las mismas.  

En este trabajo se presenta una implementación de un modelo de predicción \online, que poseen un entrenamiento inicial \offline si es deseado y que se puede consumir como una \API de servicios \REST, la cual permite una integración ha variadas plataformas clientes y sistemas que no tenga un componente \online y con una buena exactitud de predicción. 

% conector:
En las próximas secciones pretendemos que el lector se contextualice más en el escenario que deseamos realizar predicciones, trabajos relacionados que presentan otros enfoques y los fundamentos que soportan esta tesis al trabajar con \webasccesslog.



