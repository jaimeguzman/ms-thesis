%
% Introduccion 
%
%
Internet es una fuente de información que crece constantemente y es visitada con una gran frecuencia, posee billones de web públicas. Dado este rápido desarrollo y accesos a grandes colecciones de datos, se vuelve fundamental y desafiante lograr anticipar las acciones de un usuario realice para investigadores e ingenierios. Los beneficios de lograr predicciones con la suficiente precisión pueden por ejemplo, mejorar la experiencia de usuarios en base a su comportamiento, registrados por \webasccesslog, ó mejorar la experiencia de compras en comercios electrónicos.

Existen varias área que buscan un estudio de este problema, el de las predicciones web de usuarios, también conocido como WPP. Algunos de estos enfoques para  abordarlo, por ejemplo son:  Análisis Predictivos de Patrones, Web Usage Minning, Web Access Pattern, Machine Learning y predicciones secuenciales con algoritmos de compresión. Esta tesis usará estas dos ultimas áreas para la implementación de un modelo predictivo simple y en un servidor de Machine Learning.

La implementación  propuesto se basa en la implementación del algoritmo \lzSieteOcho que está adaptado para modelar y representar la navegación secuencial del usuario. Nuestra propuesta disminuye el tiempo de actualización del modelo predictivo, con actualizaciones a medida que se va usando. Además la puesta en marcha del servicio, que es un desventaja en el desarrollo de sistemas predictivos \online.


Cuando se navega en un sitio web se puede recolecta una gran cantidad de registros almacenados en el lado del servidor, los cuales pueden ser por ejemplo, datos de accesos de la sesión realizada, también estos  registros representan usuarios de la \web  que se encuentran activos, llamaremos en adelante a estos registros de acceso: \webasccesslog,   los cuales se pueden realizar variadas análisis, por ejemplo:  Existe un usuario el cual intenta comprar un cierto producto en una web de comercio electrónico. Este sigue un cierto comportamiento ya que hasta el momento de comprar, se genera un un estado de búsqueda o visita en todas las elecciones posibles, puede volver como abandonar variados productos, hasta encontrar y comprar, o no. 

Predecir estos  accesos no es trivial, aún los modelos predictivos que se han implementado no logran dar con un patrón para la navegación de usuario, de manera genérica. Dado la diversidad de perfiles de usuarios que se pueden encontrar en ciertas \webs  y distintos flujos de navegación de contenido de las mismas.  

En este trabajo se presenta un modelo de predicción \online, que poseen un entrenamiento inicial \offline si es deseado y que se puede consumir como una \API de servicios \REST, la cual permite una integración ha variadas plataformas clientes y sistemas que no tenga un componente \online y con una buena exactitud de predicción. 

% conector:
En las próximas secciones pretendemos que el lector se contextualice más en el escenario que deseamos realizar predicciones, trabajos relacionados que presentan otros enfoques y los fundamentos que soportan esta tesis al trabajar con \webasccesslog.



