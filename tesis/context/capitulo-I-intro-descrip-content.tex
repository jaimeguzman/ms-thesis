%
% ch2 ~\ref{ch:Compresion-Machine-Learning}
% ch3 ~\ref{ch:predicciones-webaccess}
% ch4 ~\ref{ch:experimetal-all}
% anexo ~\ref{ch:anexos}
%
\section{Estructura de la tesis} 




Para una mejor lectura y compresión del proyecto el documento ha sido organizado de la siguiente forma: 

 % En el Capítulo \ref{cap:ch1-intro} hemos realizado una introducción y  explicado un contexto preliminar que rodea este trabajo y hemos definido el problema que buscamos abordar, como las posibles aproximaciones de soluciones, también hemos dado un interés en el uso de algoritmo que puedan ser consumidos como un servicio \emph{web}.

En el Capítulo~\ref{ch:Compresion-Machine-Learning}, presentaremos los temas de \machinelearning y \losslessdatacompression, para llevar acabo un estudio de predicciones sobre \webasccesslog, para poder predecir la siguiente página web que un usuario accede.


% explicaremos todos los conceptos básicos para el entendimiento sobre esta investigación, como también conceptos para el uso de \emph{PredictionIO} y cerraremos con todos los trabajos relacionados más recientes que involucran nuestra investigación. Adicionalmente veremos como poder dar una inducción a un entorno de trabajo llamado \emph{PredictionIO},el cual nos ayudará a entregar los algoritmos como servicios consumibles, por cualquier aplicación cliente que pueda comunicarse con un servidor \emph{web}. 
% veremos las predicciones sobre \emph{webaccess log}, los modelos propuestos por varios investigadores y sus limitaciones, daremos una revisión del trabajo realizado por \emph{Rissanen}\cite{Rissanen1984} que da el inicio a esta área de Investigación.
% En el Capítulo \ref{ch:Compresion-Machine-Learning} veremos los temas de \emph{Machine Learning} y \emph{Lossless Compression Data} y como pretendemos crear un modelo predictivo con recursos de ambas áreas. Para cerrar este capitulo explicaremos el uso de del algoritmo Lempel \& Ziv, el uso para secuencias discretas su convergencia a un modelo de predicción eficiente y exacto. 

En el Capítulo~\ref{ch:predicciones-webaccess} se describe como crear un modelo predictivo usando un algoritmo de compresión y se presentan los fundamentos teóricos.





Finalmente el Capítulo \ref{ch:experimetal-all}, se presentará nuestros experimentos realizados con nuestra implementación. Posteriormente, veremos nuestras conclusiones y discusiones, como también nuestro trabajos futuros.


% sobre la implementación de un algoritmo de compresión en un servidor de \emph{Machine Learning}: \emph{PredictionIO}, analizaremos el comportamiento del algoritmo y como se desempeña este en distintos ambientes. propondremos discusiones de como mejorar nuestra implementación y los trabajos a futuro que puede presentar esta investigación.

% Explicaremos nuestras conclusiones obtenidas en base al desarrollo e implementación de la propuesta.
% Ademas dejaremos planteado nuestros trabajos futuros.


Además se adjunta un anexo, que corresponde a una guía básica para él uso de \emph{PredictionIO}, todos los datos y nuestra implementación se puede encontrar en nuestro repositorio \textbf{git público}\footnote{Repositorio público, \url{https://github.com/jaimeguzman/PredictionIO-LZmodel}}, en este se encontrarán los datos para replicar los experimentos que hemos realizado. 



