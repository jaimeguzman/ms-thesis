
Un alfabeto es un conjunto ordenado de elementos. Una de sus características es que la cantidad de elementos puede ser un valor discreto. Representamos por $\Sigma$ al alfabeto y los elementos de este  alfabeto por $\sigma^{n}$ símbolos, tal que $\sigma^{n}= \sigma_{1}, \dots, \sigma_{n} \mbox{ tal que } \sigma_{i} \in \Sigma$, donde {$\sigma_{1}$}\label{concept-alphabet-homepage} es definido como la página inicial de una cierta secuencia  o sesión de usuario. Sea $|\Sigma|$ la longitud de secuencia o la cardinalidad del alfabeto daremos como restricción que $|\Sigma| \geq 2$ símbolos~\cite{Dmitry2002} necesarios para realizar experimentos acotados. Posteriormente en el capítulo experimental~(\ref{ch:experimetal-all}) usaremos un conjunto de datos de tamaño $|\Sigma| = 17\  \mbox{símbolos}$, para datos reales recolectados de \emph{MSNBC}\cite{Claude2014} y un $|\Sigma|$ variable para experimentos con datos sintéticos.

El alfabeto sera utilizado simbólicamente como la representación de varios nodos contenido en un \emph{trie} que {modela la navegación de usuarios}~(\ref{sec:nuestromodelopredict-mlldc}) para un sitio \emph{web}.