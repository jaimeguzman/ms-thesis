% Referencias
% ~\cite{Poornalatha2012}
% ~\cite{Moghaddam2009}
% ~\cite{Begleiter2004}
% ~\cite{Dongshan2002}


Si se tiene un problema de predicción en el cual debemos buscar la siguiente página web que el usuario desea visitar, dada a su página actual o más la págin más frecuente~\cite{Poornalatha2012}, es el problema que esta tesis aborda, por lo cual buscamos un modelo que lo solucione con una representación simple.


Varios modelos han sido propuestos para modelar la navegación de un usuario y predecir la siguiente página que el usuario accederá. Se han desarrollado enfoques como  reglas de asociación, busqueda de patrones secuenciales, \textbf{clustering} y \emph{clasificación} son métodos bastante frecuentes~\cite{Moghaddam2009}, filtros colaborativos, etc.. 


En la literatura se han propuesto aproximaciones interesantes. Queremos discutir el modelo de navegación propuesto por \emph{Moghaddam}~\etal en ~\cite{Moghaddam2009}, la selección de este trabajo es una de las aproximaciones más directa a la que presenta esta tesis, pero con un enfoque usando algoritmos de tipo \losslessdatacompression y no \machinelearning. A  diferencia del anterior queremos seguir en la linea de investigación usando un modelo de navegación representado por un \emph{trie} que es la base para \lzSieteOcho, pero incluirlo en las etapas que un sistema de \machinelearning posee, ya que creemos que tener un aprendizaje  y una actualización constante de los datos, generan mejores  probabilidad. Además entregan un sistema que responde a la demanda y presenta mejores características con respecto a tener disponibilidad en linea.

Otros de las aproximaciones a la solución de este problema es la desarrollada por {Dongshan}~\etal~\cite{Dongshan2002}, en la cual la aproximación es el desarrollo de un modelo de Markov representado mediante un estructura de datos de árbol, llamado \emph{Hybrid Tree Markov Model} o~\texttt{HTMM}~(ver referencia~\cite{Dongshan2002}).
\uncm
 



