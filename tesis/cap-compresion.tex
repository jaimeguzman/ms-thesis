\chapter[Compresion y Machine Learning]{Compresion y Machine Learning}
\label{ch:Compresion-Machine-Learning}

% @TODO: 
% - hablar mas de que LZ78 es basado en un diccicionario.


%empezar habalr sibre lempel ziv

Los algoritmos de compresión han sido estudiados e investigados por durante varios años, la motivación
fundamental es poder optimizar el espacio, para mayor uso o almacenamiento de datos. Estos algoritmos se encuentran sin saberlo en nuestro día a día, desde el nucleo de un sistema operativo como linux hasta por ejemplo los formatos \emph{zip}, \emph{rar},\emph{7z}, también en formatos de imágenes y audios, etc. los cuales son útiles para poder optimizar una transferencia de archivos de un equipo a otro mediante Internet o simplemente comprimir datos para respaldar en dispositivos físicos, etc.

La motivación de profundizar en el área de compresión de datos se debe a una de la razones mencionadas con anterioridad, Internet. Esta red de redes, constante crea nuevos contenidos, registros, imágenes etc. los cuales no es conveniente mover de un lugar a otro mediante un transferencia directa, estos archivos crecen innumerablemente y aquí es uno de los mayores aporte que poseen los algoritmos de compresión con relación a nuestra red de redes. 

A diferencia del volúmenes de datos las infraestructura de redes y su velocidad no crece directamente proporcional, que esto genera un sin fin de problemas para los usuarios e industria web. La latencia  es el tiempo de respuesta que demora un usuario en solicitar, hacer un \emph{REQUEST}, a un servidor, simplemente es un el tiempo de respuesta desde iniciada una acción demandada. 
Uno de los grandes ejemplos que tenemos en la web es la proliferación de archivos comprimidos para su descarga, los cuales en su interior poseen variados recursos multimedia, texto, etc.

Las propiedades de estos algoritmo no solo permiten juntar un set de archivos y lograr un tasa de compresión optima para ser transmitido por Internet, también pueden ayudar a realizar análisis en grandes volúmenes de información, por ejemplo; el análisis de texto, clasificación de proteínas, moderación de contenidos en web y predicciones del comportamiento de usuarios que navegan en un sitio de Internet. Sobre este último punto es nuestro mayor interés debido a ya las antes mencionadas similitudes que poseen un algoritmo de compresión y un modelo variable de Markov.

Para introducir el camino se debe presentar formalmente los algoritmos de compresión y su clasificación más general. Entre ellos tenemos los algoritmos con pérdida y sin pérdida, nos enfocaremos en los algoritmos \emph{Lossless Compression Algorithm}, algoritmos de compresión sin pérdida.









%Sintetis de paper: Using Compresion models for filtering troll comments
\section{Modelos de Compresión}
 
 \subsection{Prediction by Partial Match (PPM)}
 
	El algoritmo de predicción por certeza parcial es considerado uno de los mejores algoritmos del tipo \emph{Lossless Compression Algoriths}. El algoritmo requiere un tope superior $D$ en el maximo del orden de Markov de un modelo variable de Markov (\emph{VMM}) para construirse. 
	PPM maneja el problema de frecuencia cero usando dos mecanismo  llamados
	
	\begin{itemize}
		\item Escape
		\item Exclusion
	\end{itemize}
	
% @TODO: Trabajar mas en este tema. y mencionar mas adelante porque usar LZ y no este, rendimiento


 \subsection{Probabilistic Suffix Tree (PST)}
 
 Los arboles de sufijos implementados como un algoritmo de predicción intentan construir el único y mejor VMM con limite superior $D$, acorde a la secuencia de entrenamiento de entrada. Esto asume que un limite superior a la orden de Markov de un "fuente certera" es conocida como \emph{learner}.
 
 %@TODO: explayar mas aca
 
 
 \subsection{Cadenas de Markov Dinámicas}
 
 
 Los Algoritmos DMC ó \emph{Dinamyc Markov Compression} son modelos de información con máquinas de estados finitos. Las asociaciones están hechas entre todos los simbolos posibles en el alfabeto origen y la distribución de probabilidad sobre todos los simbolos en el alfabeto. 
 Esta distribución de probabilidad es usada para predecir el siguiente digito binario. 
 Los \emph{DMC} comienzan en un estado ya previamente definido, cambiando de estado cuando nuevos bits son leídos desde la entrada. La frecuencia de transmisión ya sea un 0 or un 1 son sumados cuando un nuevo simbolo entra. La estructura puede también 
 ser actualizada usando \emph{state cloning method}.
 
 

% Genero toda las referencias para demostrar el uso de la bibliografía
% No es necesario que utilice este comando en su documento.
\nocite{*}
