% Cita a Rkonow Claude & Navarro

Se usarán las secuencias disponibles de \emph{webaccess logs} pública del sitio MSNBC. El set de datos provienen de los registros de un servidor {IIS} (Internet Information Services) msnbc.com de un día completo de la fecha  28 de Septiembre de 1999. 
Contiene secuencias de acceso web de 989,818 usuarios con un promedio de 5,7 categorías web visitas por secuencia, el tamaño del alfabeto de este conjunto de datos es, $\sigma \ = 17$.




Utilizaremos un simple programa\footnote{Este programa se encuentra junto en el anexo de la memoria} en \texttt{C++ }para filtrar las sesiones de los usuarios, creando nuevos set de datos y transformarlos a símbolos, caracteres en mayúscula, por ejemplo en la figura \ref{fig:sesiones-ds-msbc-num}  y figura \ref{fig:sesiones-ds-basicos} se encuentra una muestra de las entradas que disponemos para trabajar. El set de datos de la figura \ref{fig:sesiones-ds-basicos}, es el formato que utilizaremos para generar el modelo.


\begin{figure}[t] 
	\centering
	\begin{lstlisting}[frame=single,basicstyle=\ttfamily\tiny,]
	1 1 
	2 
	3 2 2 4 2 2 2 3 3 
	5 
	1 
	6 
	1 1 
	6 
	6 7 7 7 6 6 8 8 8 8 
	6 9 4 4 4 10 3 10 5 10 4 4 4 
	\end{lstlisting}	
	\caption{Secuencia de datos con indices numéricos del set de datos MSNBC}
	\label{fig:sesiones-ds-msbc-num}
\end{figure}




\begin{figure}[t] 
	\centering
	\begin{lstlisting}[frame=single,basicstyle=\ttfamily\tiny,]
	% Different categories found in input file:
	
	frontpage news tech local opinion on-air misc weather msn-news health living business msn-sports sports summary bbs travel
	
	% Sequences:
	A A 
	B 
	C B B D B B B C C 
	A A 
	F 
	F G G G F F H H H H 
	F I D D D J C J E J D D D 
	A A A K A A A 
	L L 
	A A 
	H H H H H H 
	\end{lstlisting}
	\caption{Ejemplo de sesiones de usuarios en símbolos dataset MSNBC.}
	\label{fig:sesiones-ds-basicos}
\end{figure}






Ejemplo de sub-dataset creados
\begin{itemize}
	\setlength{\itemsep}{0.9pt}
	\setlength{\parskip}{0pt}
	\setlength{\parsep}{0pt}
	\item Secuencias de \emph{webaccess} con sesiones de con un umbral mínimo.
	\item Secuencias de \emph{webaccess} con sesiones de largo equivalente.
	\item Secuencias de \emph{webaccess} con sesiones de más de 100 secciones web visitadas.
	\item Secuencias de \emph{webaccess} con sesiones acotadas superiormente.
\end{itemize}


Trabajaremos con aproximadamente $1.000.000$ de registros los cuales hemos realizados distintos subconjuntos para hacer una validación cruzada:

\begin{itemize}
	\setlength{\itemsep}{1pt}
	\setlength{\parskip}{0pt}
	\setlength{\parsep}{0pt}
	\item 10 Sesiones de usuarios
	\item 100 Sesiones de usuarios
	\item 1.000 Sesiones de usuarios
	\item 10.000 Sesiones de usuarios
	\item 50.000 Sesiones de usuarios
	\item 100.000 Sesiones de usuarios
	\item 500.000 Sesiones de usuarios
	\item 1.000.000 Sesiones de usuarios
\end{itemize}


Es importante señalar que la división de nuestro set de datos es debido a que no conocemos la naturaleza de los mismos, pero con nuestro modelo podemos estudiarlos para ir detectando patrones que puedan ser claves para medir el rendimiento en función a los criterios de volumen de datos versus \emph{Accurracy}.






\begin{table}[]
	\centering
	\label{table-list-symbol}
	\begin{tabular}{cl}
		Page       & Symbol \\ \cline{1-2}
		frontpage  & A      \\
		news       & B      \\
		tech       & C      \\
		local      & D      \\
		opinion    & E      \\
		on-air     & F      \\
		misc       & G      \\
		weather    & H      \\
		msn-news   & I      \\
		health     & J      \\
		living     & K      \\
		business   & L      \\
		msn-sports & M      \\
		sports     & N      \\
		summary    & O      \\
		bbs        & P      \\
		travel     & Q      \\ 
	\end{tabular}
	\caption{Equivalencia secciones msnbc y símbolo representativo.}
\end{table}