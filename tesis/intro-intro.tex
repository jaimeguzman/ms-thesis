{
Los nuevos avances tecnológicos y la inclusión de la ciencia de la computación en distintos campos han permitido crear grandes colecciones de datos que se pueden analizar, clasificar y procesar. Encontrar información útil y relevante dentro de estos grandes volúmenes de datos ayuda a mejorar las decisiones basándose en un conocimiento histórico de estos mismo y a la vez determinando patrones cuando se van generando a medida que transcurre el tiempo. 


Hoy en día las aplicaciones en que los usuarios se enfrentan no pueden ser estáticas y no tener características que ayuden a predecir la forma en que interactúan los usuarios. Estas nuevas características deseadas toman mucha relevancia al encontrarnos en un auge de información generada por varios usuarios, redes sociales y variadas plataformas. Podemos mencionar que ésta es una de las razones para poder  crear o usar  herramientas para el análisis y predicción de datos que nos permitan saber cómo se comportan un patrón encontrado dentro de una \emph{web}, conocer su frecuencia de accesos a un recurso en Internet, etc.

Al navegar en una \emph{web} establecemos una conexión {cliente--servidor}, en ese preciso instante una gran cantidad de registros son almacenados en el lado del servidor, los cuales pueden ser por ejemplo, datos de accesos de la sesión realizada. Estos registros  son normalmente usuarios de la \emph{web}  que se encuentran activos, llamaremos en adelante a estos registros de acceso: \emph{webaccess logs},  de los cuales se pueden realizar variadas investigaciones, por ejemplo:  cómo predecir cuál es la siguiente acceso de un usuario activo en una colección secuencial de estos registros. 

Ya hemos mencionado  que  en general las webs, redes sociales y en sí \emph{Internet} genera un gran volumen de datos constantemente, y se deben usar técnicas específicas que nos den resultados cuantitativos para darle una interpretación a nuestro análisis de datos predictivo. Una de las técnicas que toma sentido y nos da un gran aporte en este ámbito es el uso de Minería de datos  en los \emph{webaccess logs}.  Estás técnicas están a la disposición  de estudios y análisis de datos tanto para las industria, como para la academia y  se han convertido en un tema importante en esta área  de investigación, en los últimos años, en la cual podemos extender el campo usando estás técnicas en la búsqueda de nuevas aproximaciones para modelos predictivos en demanda. También se pueden utilizar para mejorar el rendimiento del caché en navegadores web como lo ha mencionado \emph{Moghaddam} \etal~\cite{Moghaddam2009},  detección de intereses de usuarios y también recomendaciones de páginas o bienes relacionados para los sitios {web} de comercio electrónico, mejorar  resultados de los motores de búsqueda y personalización de contenido web con preferencias personalizadas por el usuario a medida que va navegando. Todos los ejemplos anteriormente mencionados usando un modelo predictivo con disponibilidad inmediata.



Predecir estos  accesos no es trivial, aún los modelos predictivos que se han implementado no logran dar con un patrón para la navegación de usuario, de manera genérica. Dado la diversidad de perfiles de usuarios que se pueden encontrar en ciertas \emph{webs}  y distintos flujos de navegación de contenido de las mismas.  

Muchas técnicas de recuperación de datos y algunos sistemas de personalización usan algoritmos de predicción, podemos señalar que existen técnicas con componente online que entregan resultados inmediatos o \emph{online}, es bastante usual crear un entrenamiento estático y con un set de datos de volumen fijo, a este lo llamamos contrariamente una componente \emph{offline} de un modelo de predicción . La mayoría de las aplicaciones actuales que predicen la siguiente página web de un usuario tienen una componente  \emph{offline} que hacen la tarea de preparar los datos y su componente \emph{online}  proporciona contenido personalizado a los usuarios en función de sus actividades de navegación actuales.

Un modelo genérico no podría ser válido dada a la naturaleza básica de los conjuntos de datos que podemos recuperar de una \emph{web}, pero si un modelo predictivo liviano que vaya aprendiendo acorde a los patrones que se van recopilando en tiempo real, esto da inicio a tener un modelo predictivo con componentes tanto \emph{online} como \emph{offline}, el problema anterior es uno de nuestros intereses, consumir el resultado de un modelo de predicción confiable en que podamos ver los siguiente accesos de los usuarios y tomar acciones inmediatas. Muchos algoritmos de \emph{Machine Learning} usan análisis de datos predictivos para generar modelos predictivos en variadas áreas y este último es nuestro segundo interés, crear un modelo predictivo que se pueda integrar con los componentes de un servicio de predicción discreta que normalmente es abarcado por el campo de \emph{Machine Learning}.  

En este trabajo se presenta un modelo de predicción \emph{online}, que poseen un entrenamiento inicial \emph{offline} si es deseado y que se puede consumir como una \texttt{API} de servicios \texttt{REST}, la cual permite una integración ha variadas plataformas clientes y sistemas que no tenga un componente \emph{online} y con una buena exactitud de predicción. 

Nuestro modelo propuesto se basa en la implementación del algoritmo \texttt{LZ78} que está adaptado para modelar y representar la navegación secuencial del usuario. Nuestra propuesta disminuye la complejidad computacional  y la puesta en marcha del servicio, que es un desventaja en el desarrollo de sistemas predictivos \emph{online}.
}
%%%%%%Moghaddam_Kabir
%Web access prediction has attracted significant attention in recent years. Web prefetching and some personalization systems use prediction algorithms. Most current applications that predict the next user web page have an offline component that does the data preparation task and an online section that provides personalized content to the users based on their current navigational activities. In this paper we present an online prediction model that does not have an offline component and fit in the memory with good prediction accuracy. Our algorithm is based on LZ78 and LZW algorithms that are adapted for modeling the user navigation in web. 
%  A performance evaluation is presented using real web logs. 
%This evaluation shows that our model needs much less memory than PPM family of algorithms with good prediction accuracy.

  
%When user requests inserted and deleted incrementally the online models are desirable. 
%In this paper we present efficient techniques for modeling user navigation behavior. 
%Our model is online so changes in user request patterns will update our prediction model incrementally.
%%%%%%%%%%%%%%%%