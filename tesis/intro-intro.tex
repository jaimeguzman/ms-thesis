%chacharara
{
Los nuevos avances tecnológicos y la inclusión de la ciencia de la computación en distintas campos han permitido hacer colecciones de datos muy grandes, las cuales se deben procesar, clasificar y analizar. Encontrar información útil dentro de estos grandes volúmenes de datos significa poder mejorar las decisiones teniendo como premisa la base de conocimientos históricos que se van constantemente almacenando.

La minería de datos basada en \emph{webaccess logs} se ha convertido en una área importante de investigación en los últimos años. Se puede utilizar para mejorar el rendimiento de la caché web, la detección de intereses de los usuarios y así recomendar páginas o bienes relacionados para los sitios web de comercio electrónico, mejorar los resultados de los motores de búsqueda y por ejemplo personalizar el contenido de la web con las preferencias deseadas para el usuario. En la predicción de navegación web logrando entender el patrón de navegación del usuario y luego la predicción de las páginas siguientes es el principal problema que buscamos solucionar. Con un sistema de predicciónofiable podemos ver la siguiente acción de los usuarios de Internet y tomar acciones. 


Normalmente se ocupan muchos algoritmos de \emph{Machine Learning} para poder hacer predicciones en variadas áreas. Hoy en día las aplicaciones en que los usuarios se enfrentan no pueden ser estáticas y sin tener un comportamiento que ayuden a predecir la forma en que actúan los usuarios. Esta área toma mucha relevancia al encontrarnos en un auge de la información generada por usuarios, redes sociales y variadas plataformas. Podemos mencionar que ésta es una de las razones para poder disponer de herramientas para análisis y predicción que nos permitan saber cómo se comportan los usuarios dentro de una web, conocer la frecuencia en que se accede a un recurso en Internet, etc. 

En el momento en que un usuario entra a una web se establece una conexión cliente-servidor, una gran cantidad de servicios proporcionan datos de accesos de los usuarios que se encuentran activos en una \emph{web}, sobre estos mismos \emph{webaccess logs} y se pueden realizar variadas investigaciones sobre cómo predecir cuál es la siguiente página que podrá visitar. 


Las predicciones en los registros de \emph{webaccess logs} ha atraído una significativa atención de varios investigadores en los últimos años. Muchas técnicas de recuperación de datos y algunos sistemas de personalización usan algoritmos de predicción. La mayoría de las aplicaciones actuales que predicen la siguiente página web de un usuario tienen una componente en línea que hace la tarea de preparación de datos y una sección en línea que proporciona contenido personalizado a los usuarios en función de sus actividades de navegación actuales. En este trabajo se presenta un modelo de predicción en línea que se puede consumir como un servicio de API el cual da una integración ha variadas plataformas y sistemas que no tenga un componente en línea y con una buena precisión de la predicción. Nuestro algoritmo se basa en algoritmos LZ78 que están adaptados para modelar y representar la navegación secuencia del usuario en una web. Nuestro modelo disminuye la complejidad computacional y de implementación, que es un problema grave en el desarrollo de sistemas predictivos en línea.


  }
