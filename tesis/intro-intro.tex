{
Los nuevos avances tecnológicos y la inclusión de la ciencia de la computación en distintas campos han permitido crear grandes colecciones de datos, estas se pueden procesar, clasificar y analizar. Poder encontrar información útil y relevante dentro de estos grandes volúmenes de información pueden ayudar a mejorar las decisiones teniendo como base de conocimientos histórico y los datos constantemente se van almacenando.

En el momento en que un usuario entra a una web se establece una conexión cliente--servidor, una gran cantidad de registros del lado del servidor almacena datos de accesos de los usuarios que se encuentran activos en una \emph{web}. Sobre estos mismos \emph{webaccess logs}, se pueden realizar variadas investigaciones sobre cómo predecir cuál es la siguiente acceso. 

La minería de datos basada en \emph{webaccess logs} se ha convertido en una área importante de investigación en los últimos años de la cual podemos continuar usando estás técnicas en buscar aproximaciones a nuevos modelos predictivos. Se puede utilizar para mejorar el rendimiento de caché en navegadores web \etal~\cite{Moghaddam2009},  detección de intereses de usuarios y también recomendaciones de páginas o bienes relacionados para los sitios web de comercio electrónico, mejorar los resultados de motores de búsqueda y por ejemplo personalizar el contenido de la web con preferencias deseadas por el usuario a medida que va navegando.

Predecir estos futuros accesos no es trivial, aún los modelos predictivos que se han implementado no logran dar con un patrón para la navegación de usuario. Dado que cada web tiene distintos perfiles de usuarios un modelo genérico no podría ser válido, pero si lo es un modelo predictivo que vaya aprendiendo acorde a los patrones que se van recopilando, esto da inicio a que se debe tener un modelo predictivo con componentes tanto \emph{online} como \emph{offline}. Este problema es el interés de nuestro trabajo, encontrar un modelo de predicción confiable en que podamos ver los siguiente accesos de los usuarios y tomar acciones. 

Se ocupan muchos algoritmos de \emph{Machine Learning} para poder hacer predicciones en variadas áreas. Hoy en día las aplicaciones en que los usuarios se enfrentan no pueden ser estáticas y sin tener características que ayuden a predecir la forma en que interactúan los usuarios. Esta área toma mucha relevancia al encontrarnos en un auge de la información generada por varios usuarios, redes sociales y variadas plataformas. Podemos mencionar que ésta es una de las razones para poder disponer de herramientas para análisis y predicción que nos permitan saber cómo se comportan los usuarios dentro de una web, conocer la frecuencia en que se accede a un recurso en Internet, etc. 

Las predicciones sobre \emph{webaccess logs} ha atraído una significativa atención de varios investigadores en los últimos años. Muchas técnicas de recuperación de datos y algunos sistemas de personalización usan algoritmos de predicción. La mayoría de las aplicaciones actuales que predicen la siguiente página web de un usuario tienen una componente  \emph{offline} que hace la tarea de preparación de datos y una sección en línea que proporciona contenido personalizado a los usuarios en función de sus actividades de navegación actuales. En este trabajo se presenta un modelo de predicción en línea que se puede consumir como una \emph{API} de servicios, la cual da una integración ha variadas plataformas y sistemas que no tenga un componente \emph{online} y con una buena exactitud de predicción. 

Nuestro modelo propuesto se basa en la implementación del algoritmo \texttt{LZ78} que está adaptados para modelar y representar la navegación secuencial del usuario. Nuestro modelo disminuye la complejidad computacional  y de implementación, que es un desventaja en el desarrollo de sistemas predictivos \emph{online}.
}
%%%%%%Moghaddam_Kabir
%Web access prediction has attracted significant attention in recent years. Web prefetching and some personalization systems use prediction algorithms. Most current applications that predict the next user web page have an offline component that does the data preparation task and an online section that provides personalized content to the users based on their current navigational activities. In this paper we present an online prediction model that does not have an offline component and fit in the memory with good prediction accuracy. Our algorithm is based on LZ78 and LZW algorithms that are adapted for modeling the user navigation in web. 
%  A performance evaluation is presented using real web logs. 
%This evaluation shows that our model needs much less memory than PPM family of algorithms with good prediction accuracy.

  
%When user requests inserted and deleted incrementally the online models are desirable. 
%In this paper we present efficient techniques for modeling user navigation behavior. 
%Our model is online so changes in user request patterns will update our prediction model incrementally.
%%%%%%%%%%%%%%%%