Aún cuando se presenta varios antecedentes, podemos decir que nuestro modelo ocupa bastante menos memoria pero esto va directamente relacionado el tamaño del \emph{trie} de predicción generado.

%
Aun cuando el \emph{trie} que representa totalmente el modelo de navegación del los usuarios de un sitio web, este no puede generalizar completamente el comportamiento estocástico de los usuarios y/o agentes que acceden a los recursos de \emph{MSNBC} o un cualquier web en general.

Una de las mayores ventajas de nuestro modelo es que al estar embebido en un servidor de Machine Learning, cada nuevo evento que ingresa para la siguiente nueva ejecución esta estará mejor preparado, podríamos decir que la recolección de data de cada evento en particular nos ayudaría a que nuestro modelo en futuros trabajos vaya aprendiendo mas y más precisamente.


A medida que la altura del árbol va creciendo este genera mayor demora en cuanto a la creación del \emph{trie}.
% Pero a mayor altura hay mayor precisión.

Hemos validado que hacer un modelo de navegación de usuarios es un perfecta implementación de un predictor usando un árbol de la familia de \emph{Lempel} \& \emph{Ziv}.

A nuestro modelo le es afectado secuencias de menor tamaño, por lo cual se debe trabajar para hacer un aprendizaje de que momento omitirlo o no. Estas sesiones de bajo número de secuencias genera un bajo porcentaje de \emph{accurracy}. 


% Cual es el minimo entrenamiento para lograr a Predecir ?

Hemos presentado un modelo liviano el cual puede ser utilizado para predecir secuencias de \emph{webaccess} en demanda.






%In this paper we studied the empirical performance of a number of prominent prediction algorithms. We focused on prediction settings that are more closely related to those required

%%%%%%%%% On Prediction Using Variable Order Markov Models

%$%%%%%%%%
%%n this paper we studied the empirical performance of a number of prominent prediction algorithms. We focused on prediction settings that are more closely related to those required


%On Prediction Using Variable Order Markov Models
%by machine learning practitioners dealing with discrete sequences.
%However, somewhat surprisingly, the best predictor under the log-loss is not the best classifier. On the contrary, the consistently best protein classifier is based on the mediocre lz-ms predictor! This algo- rithm is a simple modification of the well-known Lempel-Ziv-78 (lz78) prediction algorithm, which can capture VMMs with large contexts. The surprisingly good classification accuracy achieved by this algorithm may be of independent interest to protein analysis research and clearly deserves further investigatio 
% Genero toda las referencias para demostrar el uso de la bibliografía
% No es necesario que utilice este comando en su document
%
%Conclusion of this paper Gopalratnam Cook
%a

lz modelos eficazmente procesos secuenciales, y es extremadamente útil para la predicción de los procesos donde los eventos son dependientes de la historia evento anterior. Esto es debido a la capacidad del algoritmo para construir un modelo preciso de la fuente de los eventos que se generan, una característica heredada de su información de fondo teórico y el algoritmo de compresión de texto \texttt{LZ78}.
%
La eficacia del método para el aprendizaje de una medida de tiempo también se puede atribuir a hecho de que ALZ es un fuerte predictor secuencial. Los principios teóricos de sonido en el que ALZ se fundamenta también significan que ALZ es un universal Quiniela óptima, y se puede utilizar en una variedad de escenarios de predicción.
%conclusion lcoa
%

Dado que la mayoría de los predictores funcionan de 
manera offline, uno de los aporte de tener estar estructura de algoritmos como servicios es poder tener un motor de predicción en linea.

\section{Contribución de esta Tesis}


En este trabajo,  nuestra principal contribución es el desarrollo del primer servidor de predicción de secuencias discretas con funcionalidad online y \emph{offline} para secuencias de  sitios web. Adicionalmente nuestro modelo en conjunto \emph{PredictionIO}, es el primer servidor de Machine Learning usando un modelo de predicción con un algoritmo de \emph{Lossless Data Compression}. Hay varios temas propuestos en el que este trabajo pueda tener iteraciones  futuras de estas sistema. 