%El resumen no debe contener menos de 100 palabras ni mas de 300 palabras.

Internet crece cada día y los datos crecen en volúmenes del orden de los Terabytes, por lo cual es de interés usar técnicas de compresión para realizar procesamiento a mayor escala de información con la menor cantidad de recursos posibles. Hoy en día existen variados tipos de web, redes sociales, microbloging, web informativas, etc. El contenido proporcionado a los usuarios finales ya no es estático y esto permite que los mismos puedan generar, aportar, modificar contenido, así es que la industria del desarrollo web está en constante evolución para generar más recursos para poder desarrollarla. Muchas de estas nuevas tecnologías han permitido entregar una mejor experiencia al momento de navegar, el gran avance realizado no ha permitido crear Web que sean por sí mismas inteligentes y puedan ir anticipando su comportamiento ó recomendando, por ejemplo; disminuir la latencia desde que se abre una web o el reconocimiento discreto de patrones cuando se navega dentro de un sitio con alta demanda de usuarios; también desde el punto de vista de los patrones web actuales los servicio que se consumen inmediatamente dan una área de aplicación extensa y aporta un aspecto estratégico al uso de de la información. Si bien el crecimiento de los recursos de almacenamiento en la nube se encuentra en apogeo, las redes de telecomunicaciones  no crecen a la misma velocidad, lo cual da un área de interés para profundizar. 

En este trabajo proponemos  la creación de un modelo híbrido entre \emph{Machine Learning} y Algoritmos de tipo \emph{Lossless Data Compression} para predecir la siguiente secuencias de acceso que usuario puede realizará dentro en una web; usando modelamiento de navegación basado en un Algoritmo de Compresión como \texttt{LZ78}, sobre un servidor de Machine Learning. Con este propósito se trabajará para crear y estudiar un modelo predictivo que use ambas áreas y  ofrecer los resultados del modelo predictivo con una componente \emph{online} integrable a cualquier tipo de plataforma.