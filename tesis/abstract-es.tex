%El resumen no debe contener menos de 100 palabras ni mas de 300 palabras.

Internet crece exponencialmente  y los datos son generados en grandes volúmenes, en el orden de los \emph{Terabytes}. Hoy en día existen varios tipos de web; redes sociales, microbloging, web informativas, etc. por lo cual es de interés usar técnicas de compresión para realizar procesamiento a mayor escala de información con la menor cantidad de recursos posibles. El contenido proporcionado a los usuarios finales ya no es estático y esto permite que los mismos puedan  aportar, modificar ó eliminar contenido. Así  la industria del desarrollo web está en constante evolución para generar más recursos para poder desarrollarse. Muchas de estas nuevas tecnologías han permitido entregar mejor experiencia al momento de navegar del lado del cliente, pero el gran avance realizado no ha permitido crear \emph{web} que sean por sí mismas inteligentes y puedan ir anticipando la  navegación de un usuario ó recomendación de contenido, por ejemplo; disminuir la latencia desde que comienza a navegar una web o el reconocimiento discreto de patrones de navegación de usuario,  dentro de un sitio con alta demanda de usuarios; también desde el punto de vista de los patrones de desarrollo web actuales los servicio que se consumen inmediatamente dan una área de aplicación e integración extensa y aporta un aspecto estratégico al uso de la información. Si bien el crecimiento del almacenamiento de datos en la nube se encuentra en apogeo, las redes de telecomunicaciones  no crecen a la misma velocidad, lo cual es un campo de interés para profundizar. 

En este trabajo proponemos  la creación de un modelo híbrido entre \emph{Machine Learning} y Algoritmos de tipo \emph{Lossless Data Compression} para predecir la siguiente secuencias de acceso que usuario puede realizará dentro en una web; usando modelamiento de navegación basado en un Algoritmo de Compresión como \texttt{LZ78}, sobre un servidor de Machine Learning. Con este propósito se trabajará para crear y estudiar un modelo predictivo que use ambas áreas y  ofrecer los resultados del modelo predictivo con una componente \emph{online} integrable a cualquier tipo de plataforma.