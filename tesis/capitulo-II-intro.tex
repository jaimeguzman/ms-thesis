

\chapter[Machine Learning y Lossless Data Compression]{Machine Learning y Lossless Data Compression}\label{ch:Compresion-Machine-Learning}
%%%%%%%%%%%%%%%%%%%%%%%%%%%%%%%%%%%%%%%%
% Que busco inducir de machine learnig
% Que buscon inducir de LDC
% Como los cruzo
% Donde el lector puede ir buscando lo que necesita para comprender
% Una idea donde se junta la prediccion con la prediccion son los VMM

%%%%% SECCION DE MACHINE LEARNIGN PARA ANALISIS DE DATOS SECUENCIALES

Una de las áreas que comprende la Inteligencia Artificial es \machinelearning, en esta área de conocimiento existen algortimos y metodología que permiten realizar un entrenamiento a un sistema, con el fin de aprender desde un conjunto de datos. En los últimos años esta área ha tenido mayor atención, debido a la gran cantidad de datos que están disponibles y el mayor interes por encontrar información relevante dentro de estos datos. Anteriormente existían limitación de recursos, que hoy el \cloudcomputing lograr solucionar, abriendo instancias para nuevas investigaciones y desarrollo. Adicionalmente al tener mejor infraestructura para funcionar existe una gran demanda de soluciones a problemas y necesidades que la industria presenta, esto vuelve a que la exista una competencia en la extración de conocimiento de ciertos conjuntos de datos.

El conocimiento en este contexto, es a menudo definido como un modelo que puede ser actualizado o ajustado constantemente, a medida que nuevos datos se van generando. Los modelos nacen para satisfacer cierto dominio específico, por ejemplo la evaluación del riesgo de créditos financieros, reconocimiento de rostros o patrones visuales, maximizar la calidad del servicio, clasificación de los síntomas patológicos para ciertas enfermedades, optimización de redes informáticas, detección de intrusiones en escenarios de seguridad informática y también se pueden analizar el comportamiento en línea de los usuarios e historial de compras ó navegación  de un cierta \web. Específicamente, un algoritmo de \machinelearning ``infiere patrones y asociaciones entre distintas variables, y conjunto de datos''~\cite[capítulo 8]{guller2015big}, en otras palabras, un algoritmo de \machinelearning aprende para predecir. 


Un modelo es una construcción matemática para capturar patrones de un conjunto de datos. Como señala \emph{Guller}~\MLGuller, básicamente es una función que toma ciertas características de un conjunto de datos, como sus valores iniciales y sus resultados.  Los modelos son las piezas fundamentales de cualquier implementación de \machinelearning, describen los datos observados de un sistema, en muchos casos, estos modelos se aplican a nuevos conjuntos de datos que ayudan a un nuevo modelo lograr aprender nuevos comportamientos y también predecirlos~\MLPDASunila. Se clasifican de la siguiente manera:

\begin{itemize}
\menorEspacioItemize		
 \item Modelos Lógicos
 \item Modelos Geométricos
 \item Modelos Probabilísticos
\end{itemize} 
%La aplicación de algún modelos debe ser definida por un caminio para lograr un objetivo, como el nuestro que es el de lograr predicciones con datos secuenciales. 

Cada uno de los modelos señalados son entrenados por algoritmos de \machinelearning, nos concentraremos en los modelos probabilísticos, los cuales pueden ser entrenados por algoritmos de \emph{regresión lineal}, \emph{forecasting} o \emph{predicción}. Todos los algoritmos mencionados anteriormente intentan de alguna manera determinar que pasará en el futuro, dado a cierta información provista de experiencia o conocimiento previo.

Esta sección entrega un punto introductorio a ciertos fundamentos y referencias de algoritmos que son mencionados. Estos se agrupan en la siguientes categorías:

\begin{itemize}
	\menorEspacioItemize
	\item \textbf{Regresión}
			% Referencias
% ~\cite{gollapudi2016practical} SUNILA


	El análisis de regresión nos permite modelar matemáticamente la relación entre dos variables utilizando el álgebra simple~\cite{gollapudi2016practical}.
	Es una técnica de clasificación que es particularmente adecuada para un modelo continuo, lineal (mínimos cuadrados), polinomio, y regresiones logísticas. Se encuentran entre las técnicas más utilizadas para mejorar o ajustar un modelo paramétrico. Se considera que la regresión es un caso especializado de la  clasificación, para los cuales las variables de salida son continuas en lugar de categórica.

% Regression techniques are used to predict response variables with numerical outcomes, such as predicting the miles per gallon of a car or predicting the temperature of a city. The input variables may be numeric or categorical. However, what is common with these algorithms is that the output (or response variable) is numeric. We’ll review some of the most commonly used regression techniques including linear regression, neural networks, decision trees, and boosted decision tree regression.
	\item \textbf{Clasificadores}	
				El fin de un clasificador es extraer conocimiento específico de un dominio a partir de datos históricos. Por ejemplo, un clasificador puede ser construido para identificar una enfermedad a partir de un conjunto de síntomas. El científico recoge la información relativa a la temperatura del cuerpo (variable continua), la congestión (variables discretas alta, media y baja), y el actual de diagnóstico (gripe). Este conjunto de datos se utiliza para crear un modelo como si la temperatura> 102 y la congestión = ALTO ENTONCES paciente tiene la gripe (probabilidad 0.72), que los médicos pueden utilizar en su diagnóstico.


% Classification is a type of supervised machine learning. In supervised learning, the goal is to infer a function using labeled training data. The function can then be used to determine the label for a new dataset (where the labels are unknown). A non-exhaustive list of classification algorithms that can be used for building the model includes decision trees, logistic regression, neural networks, support vector machines, naïve Bayes, and Bayes Point Machines.

% Classification algorithms are used to predict the label for input data (where the label is unknown). Labels are also referred to as classes, groups, or target variables. For example, a telecommunication company wants to predict the following:

% Churn: Customers who have an inclination to switch to a different telecommunication provider
% Propensity to Buy: Customers willing to buy new products or services
% Upselling: Customers willing to buy upgraded services or add-ons
% To achieve this, the telecommunication company builds a classification model using training data (where the labels are known or have already been predefined). In this section, you’ll look at several common classification algorithms that can be used for building the model. Once the model has been built and validated using test data, data scientists at the telecommunication company can use the model to predict churn, propensity to buy, and upselling labels for customers (where the labels are unknown). Consequently, the telecommunication company can use these predictions to design marketing strategies that can reduce the customer churn and offer services to the customers that are more willing to buy new services or upsell.

% Other scenarios where classification algorithms are commonly used include financial institutions, where models are used to determine whether a credit card transaction is a fraudulent case or if a loan application should be approved based on the financial profile of the customer. Hotels and airlines use models to determine whether a customer should be upgraded to a higher level of service (such as from economy to business class, from a normal room to a suite, etc.).

% The classification problem is defined as follows: given an input sample of image, where x1 refers to an item in the sample of size d, the goal of classification is to learn the mapping image, where y ∈ Y is a class.

% An instance of data belongs to one of J groups (or classes), such as image. For example, in a two-class classification problem for the telecommunication scenario, Class C1 refers to customers that will churn and switch to a new telecommunication provider, and Class C2 refers to customers that will not churn.

% To achieve this, labeled training data is first used to train a model using one of the classification algorithms. This is then validated by using test data to determine the number of mistakes made by the trained model (the classifier). Various metrics are used to measure the performance of the classifier. These include measuring the accuracy, precision, recall, and the area under curve (AUC) of the trained model.
	\item \textbf{Predictores}
			
	Ya extraída y validados ante los últimos datos del modelo, que puede ser utilizado para inferir de los datos futuros. Tomando el ejemplo de un médico recoge los síntomas de la un paciente, como la temperatura corporal o el avance de la enfermedad que esta tratando. Con los datos anteriores puede anticipar en el avanece de la enfermedad que se encuentra tratando a su paciente


	% \item[Optimización]	
	% 	
	No todos los problemas de optimización global son abordable usando métodos de optimización no lineal ó lineal tradicional. Mediante Machine Learning se puede mejorar las posibilidades de que el método de optimización converja hacia una solución (búsqueda inteligente). Imagínese que la lucha contra la propagación de un nuevo virus requiere la optimización de un proceso que puede evolucionar con el tiempo a medida que se descubren más síntomas y casos.



\end{itemize}
 
\uncm


Existen varios algoritmos con distintos estos fines, los anteriores no son la totalidad que ofrece el área de  \machinelearning. Todos los algoritmos pueden ir sufriendo variaciones o alteraciones acorde a como se presenten los datos. Estos deben conocerse para que en el proceso de entrenamiento del  modelo o el uso de estos sea válido. Es lógico pensar que si usa un algoritmo para un escenario que no corresponde este retorne resultado inconsistente, por ende, es importante conocer el dominio y seleccionar correctamente el algoritmo que se utilizará.


Al momento de realizar nuevos algoritmos o modificaciones que ayudan a algunos modelos tradicionales de \machinelearning, se abre un nueva perspectiva para abordar mas problemas con nuevos propósitos. \emph{Sculley y Brodley} en~\cite{Sculley2006} plantean y discuten esta nueva perspectiva con un enfoque más teórico e empírico. La idea fundamental que plantean es que si se tiene un \emph{string} $x$ e $y$, y  se comprimen juntos estos pueden ser más eficientes, es decir,  ambos \emph{strings} comparten la misma información. El caso anterior es la perspectiva de juntar las áreas de \machinelearning y \emph{Data Compression}. 

Si estos \emph{string} representarán grandes volúmenes de datos o largas secuencias de símbolos, si se usan técnicas o algoritmos de compresión, pueden ayudar a enfrentar este escenario, en el cual existe gran complejidad al momento de realizar el entrenamiento. Además, si se consideran estas aproximación, es posible entregar mejores resultado siendo  un beneficio considerable. En el ejemplo anterior como en otras tareas se puede usar algoritmos de compresión sobre \machinelearning, tanto en tareas tradicionales de clasificación ó agrupamiento.  Esta perspectiva inclusivas no son nuevas y ha aparecido en variados campos~\cite{Sculley2006}, algunas veces con la promesa de reducir los problemas, cuando se realiza la selección implícita de ciertas características de un conjunto de datos. 

Li~\etal en~\cite{Li2005} lograron implementar una nueva clase de \emph{string kernel}\footnote{Los String kernel, son operaciones que permiten trabajar funciones no lineales, como lineales. Para un explicación detallada ver~\cite{Li2005}.} basada en un algoritmo de compresión y lo utilizaron para manejar un clasificador \texttt{SVM} para la clasificación de géneros musicales. El algoritmo utilizado por \cite{Sculley2006} es de la familia de {Lempel and Ziv}, el cual lo veremos mas adelante en la sección ~\ref{}.
% AGREGAR EL EJEMPLO DE DETECCION DE SPAM TROLES

Como se ha señalado, una de las motivaciones que se puede tener al usar técnicas de compresión, es ahorrar espacio para ciertos conjuntos de datos que se desean entrenar. Por otro lado también se puede ahorrar tiempo en transmitir o mover ciertos conjuntos de datos, lo que implica que finalmente en la mayoria de los datos se elimina la redundancia de información innecesaria.

Los algoritmos de compresión pueden ser con o sin pérdida de informacion, abordaremos solamente los algoritmo sin pérdida (\losslessdatacompression). Estos en función de los análisis predictivo no presente explícitamente una acercamiento, pero poseen una estrecha relación,temáticas que usan ciertos algoritmos de tipo probabilistas de \machinelearning. Estos algoritmos son los \emph{modelos variables de markov}, son tipos de modelos los cuales están basados en las cadenas de Markov, esto se detallará con mayor precisión en la sección~\ref{}, además se realizará una aproximación a los \emph{modelos ocultos de markov} que son mas asociados a estudios de \machinelearning.





\subsubsection{Resumen}

En esta introducción hemos introducidos varios conceptos que veremos mas adelante, uno de los más importante es el acercamiento de dos areas para poder complementar una solución específica. Hemos hablado de como ciertos modelos son entrenados por algoritmos. Además hemos visto una primera revisión a la literatura en la cual ha validado a que podemos hacer esta intersección de áreas de manera correcta, para lograr un objetivo como es las predicciones. En las proximas secciones de este capitulo se entregará una base mas detallada para comprender el aporte de que puede hacer un algoritmo \losslessdatacompression para un modelo de entrenamiento y predicción de \machinelearning.


% Summary

% In this chapter, you learned about different regression, classification, and clustering algorithms. You learned how each of the algorithms work, and the type of problems for which they are suited. The goal of this chapter is to provide you with the foundation for using these algorithms to solve the various problems covered in upcoming chapters. In addition, the resources provided in this chapter will help you learn more deeply about state-of-art machine learning and the theory behind some of these algorithms.





