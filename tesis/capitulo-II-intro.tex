%%%%%%%%%%%%%%%%%%%%%%%%%%%%%%%%%%%%%%%%
% Que busco inducir de machine learnig
% Que buscon inducir de LDC
% Como los cruzo
% Donde el lector puede ir buscando lo que necesita para comprender
% Una idea donde se junta la prediccion con la prediccion son los VMM

%%%%% SECCION DE MACHINE LEARNIGN PARA ANALISIS DE DATOS SECUENCIALES

Una de las áreas que comprende la Inteligencia Artificial es \machinelearning, la cual permite realizar entrenamiento a un sistema para aprender desde un conjunto de datos, un proceso de inducción de conocimiento. Este tema ha tenido atención durante los últimos años, dado a las mejoras en recursos \cloudcomputing que están disponibles para nuevas investigaciones y desarrollo, también existe un gran demanda de soluciones a problema o una necesidad competitiva de extracción de conocimiento de ciertos conjunto de datos.

El conocimiento es a menudo definido como un modelo que puede ser actualizado o ajustado constantemente a medida que nuevos datos se van generando y usando. Los modelos son  de dominio específico que van por ejemplo desde la evaluación del riesgo de crédito financieros, reconocimiento de rostros, maximizar la calidad del servicio, la clasificación de los síntomas patológicos de ciertas enfermedades, optimización de redes informáticas, detección de intrusiones de seguridad y también se pueden analizar el comportamiento en línea de los usuarios e historial de compras ó navegación  de un cierta \webs. Específicamente, un algoritmo de \machinelearning infiere patrones y asociaciones entre distintas variables y conjunto de datos~\cite{guller2015big}, en otras palabras un algoritmo de \machinelearning aprende para predecir desde los datos. Se pueden clasificar las aproximaciones a soluciones de la siguiente manera:

\begin{itemize}
	
	\item[Clasificadores]	
			El fin de un clasificador es extraer conocimiento específico de un dominio a partir de datos históricos. Por ejemplo, un clasificador puede ser construido para identificar una enfermedad a partir de un conjunto de síntomas. El científico recoge la información relativa a la temperatura del cuerpo (variable continua), la congestión (variables discretas alta, media y baja), y el actual de diagnóstico (gripe). Este conjunto de datos se utiliza para crear un modelo como si la temperatura> 102 y la congestión = ALTO ENTONCES paciente tiene la gripe (probabilidad 0.72), que los médicos pueden utilizar en su diagnóstico.


% Classification is a type of supervised machine learning. In supervised learning, the goal is to infer a function using labeled training data. The function can then be used to determine the label for a new dataset (where the labels are unknown). A non-exhaustive list of classification algorithms that can be used for building the model includes decision trees, logistic regression, neural networks, support vector machines, naïve Bayes, and Bayes Point Machines.

% Classification algorithms are used to predict the label for input data (where the label is unknown). Labels are also referred to as classes, groups, or target variables. For example, a telecommunication company wants to predict the following:

% Churn: Customers who have an inclination to switch to a different telecommunication provider
% Propensity to Buy: Customers willing to buy new products or services
% Upselling: Customers willing to buy upgraded services or add-ons
% To achieve this, the telecommunication company builds a classification model using training data (where the labels are known or have already been predefined). In this section, you’ll look at several common classification algorithms that can be used for building the model. Once the model has been built and validated using test data, data scientists at the telecommunication company can use the model to predict churn, propensity to buy, and upselling labels for customers (where the labels are unknown). Consequently, the telecommunication company can use these predictions to design marketing strategies that can reduce the customer churn and offer services to the customers that are more willing to buy new services or upsell.

% Other scenarios where classification algorithms are commonly used include financial institutions, where models are used to determine whether a credit card transaction is a fraudulent case or if a loan application should be approved based on the financial profile of the customer. Hotels and airlines use models to determine whether a customer should be upgraded to a higher level of service (such as from economy to business class, from a normal room to a suite, etc.).

% The classification problem is defined as follows: given an input sample of image, where x1 refers to an item in the sample of size d, the goal of classification is to learn the mapping image, where y ∈ Y is a class.

% An instance of data belongs to one of J groups (or classes), such as image. For example, in a two-class classification problem for the telecommunication scenario, Class C1 refers to customers that will churn and switch to a new telecommunication provider, and Class C2 refers to customers that will not churn.

% To achieve this, labeled training data is first used to train a model using one of the classification algorithms. This is then validated by using test data to determine the number of mistakes made by the trained model (the classifier). Various metrics are used to measure the performance of the classifier. These include measuring the accuracy, precision, recall, and the area under curve (AUC) of the trained model.

	\item[Predictores]		
		
	Ya extraída y validados ante los últimos datos del modelo, que puede ser utilizado para inferir de los datos futuros. Tomando el ejemplo de un médico recoge los síntomas de la un paciente, como la temperatura corporal o el avance de la enfermedad que esta tratando. Con los datos anteriores puede anticipar en el avanece de la enfermedad que se encuentra tratando a su paciente



	\item[Optimización]	
		
	No todos los problemas de optimización global son abordable usando métodos de optimización no lineal ó lineal tradicional. Mediante Machine Learning se puede mejorar las posibilidades de que el método de optimización converja hacia una solución (búsqueda inteligente). Imagínese que la lucha contra la propagación de un nuevo virus requiere la optimización de un proceso que puede evolucionar con el tiempo a medida que se descubren más síntomas y casos.




	\item[Regresión]
		% Referencias
% ~\cite{gollapudi2016practical} SUNILA


	El análisis de regresión nos permite modelar matemáticamente la relación entre dos variables utilizando el álgebra simple~\cite{gollapudi2016practical}.
	Es una técnica de clasificación que es particularmente adecuada para un modelo continuo, lineal (mínimos cuadrados), polinomio, y regresiones logísticas. Se encuentran entre las técnicas más utilizadas para mejorar o ajustar un modelo paramétrico. Se considera que la regresión es un caso especializado de la  clasificación, para los cuales las variables de salida son continuas en lugar de categórica.

% Regression techniques are used to predict response variables with numerical outcomes, such as predicting the miles per gallon of a car or predicting the temperature of a city. The input variables may be numeric or categorical. However, what is common with these algorithms is that the output (or response variable) is numeric. We’ll review some of the most commonly used regression techniques including linear regression, neural networks, decision trees, and boosted decision tree regression.

\end{itemize}

Las clasificaciones anteriores no son la totalidad que ofrece \machinelearning, pero para acercarse a los análisis predictivos de datos, es suficiente.

% AQUI PODRIA IR LA IDEA DE EXPLICACION DE MODELO 




% BUSCAR IDEAS DE COMO INTRODUCIR EL ANALISIS PREDICTIVO DE DATOS CON ML

% dejar en claro que no son las unicas clasificaciones

%IDEA: 
Existen variadas propuestas que buscan modelar la navegación de usuarios, aun que acorde a MOHA y la literatura en general estas no son \online

% TODA:  INTRODUCIR UN POCO IDEAS DE HMM EN ESTA INTRODUCCION DE CAPITULO










% TODO: hacer una introduccion a LDC y como usare el super mega LZ en el modelo.


La Construcción de un modelo de análisis de datos predictivos para \machinelearning, que pueda capturar  la mejor relación entre las características descriptivas que se buscan y el objetivo (target) de un dataset.

Un criterio que debe ser simple de comprender es que si se desea construir modelo que realice prediccion númerica, este entregue un resultado consistente. osea un número que esta acorde al dataset que ha sido utilizado para entrenar.





Por otro lado el área de \losslessdatacompression en función de los análisis predictivo no presente explícitamente una acercamiento, pero se torna fundamental su compresión para hacer uso de tecnologías que puedan trabajar con grandes volumenes de infomación.




Además como se señala y detallará en la sección de compresión, existen algoritmos de compresión que han tenido base común en las bases matemáticas que usan ciertos algoritmos de tipo probabilistas de \machinelearning. Estos algoritmos son las \emph{modelos variables de markov}










