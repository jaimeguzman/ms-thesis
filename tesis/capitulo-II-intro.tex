

\chapter[Machine Learning y Lossless Data Compression]{Machine Learning y Lossless Data Compression}\label{ch:Compresion-Machine-Learning}
%%%%%%%%%%%%%%%%%%%%%%%%%%%%%%%%%%%%%%%%
% Que busco inducir de machine learnig
% Que buscon inducir de LDC
% Como los cruzo
% Donde el lector puede ir buscando lo que necesita para comprender
% Una idea donde se junta la prediccion con la prediccion son los VMM

%%%%% SECCION DE MACHINE LEARNIGN PARA ANALISIS DE DATOS SECUENCIALES

Una de las áreas que comprende la Inteligencia Artificial es \machinelearning, esta permite realizar entrenamientos a un sistema para aprender desde un conjunto de datos. Este tema ha tenido atención durante los últimos años, dado a las mejoras en disponibilidad de recursos que el \cloudcomputing ofrece para nuevas investigaciones y desarrollo, también existe un gran demanda de soluciones a problemas y una necesidad competitiva de extracción de conocimiento de ciertos conjunto de datos.

El conocimiento es a menudo definido como un modelo que puede ser actualizado o ajustado constantemente a medida que nuevos datos se van generando y usando. Los modelos son  de dominio específico que van por ejemplo desde la evaluación del riesgo de crédito financieros, reconocimiento de rostros, maximizar la calidad del servicio, la clasificación de los síntomas patológicos de ciertas enfermedades, optimización de redes informáticas, detección de intrusiones de seguridad y también se pueden analizar el comportamiento en línea de los usuarios e historial de compras ó navegación  de un cierta \webs. Específicamente, un algoritmo de \machinelearning ``infiere patrones y asociaciones entre distintas variables y conjunto de datos''~\cite[capítulo 8]{guller2015big}, en otras palabras un algoritmo de \machinelearning aprende para predecir. 


Un modelo es una construcción matemática para capturar patrones de un conjunto de datos. Como señala \emph{Guller}~\MLGuller, básicamente un modelo es una función que toma ciertas características de un conjunto de datos como sus valores iniciales y resultados.  Estos son la pieza fundamental de cualquier implementación de \machinelearning, describen lo datos observados de un sistema. En muchos casos estos modelos se aplican a nuevos conjuntos de datos  que ayudan a un modelo aprender nuevos comportamientos y también predecirlos~\MLPDASunila. Se clasifican de la siguiente manera:

\begin{itemize}
			\setlength{\itemsep}{1pt}
			\setlength{\parskip}{0pt}
			\setlength{\parsep}{0pt}
 \item Modelos Lógicos
 \item Modelos Geométricos
 \item Modelos Probabilisticos
\end{itemize} 


Cada uno de estos modelos son entrenados por algoritmos de \machinelearning, nos concentraremos en los modelos probabilísticos estos pueden ser entrenados por algoritmos de \emph{regresión lineal}, \emph{forecasting} o \emph{predicción}. Todos los algoritmos mencionados anteriormente intentan de alguna manera determinar que pasará en el futuro, dado a la información provista de experiencia o conocimiento previo.


%La aplicación de algún modelos debe ser definida por un caminio para lograr un objetivo, como el nuestro que es el de lograr predicciones con datos secuenciales.  



Esta sección entrega un base a ciertos fundamentos y referencias de algoritmos que son mencionados y refereidos. Se agrupa los algoritmos en  las siguientes categorías:

\begin{itemize}
			\setlength{\itemsep}{1pt}
			\setlength{\parskip}{0pt}
			\setlength{\parsep}{0pt}
	\item Regresión
	\item Clasificadores	
	\item Predictores
	% \item[Optimización]	
	% 	
	No todos los problemas de optimización global son abordable usando métodos de optimización no lineal ó lineal tradicional. Mediante Machine Learning se puede mejorar las posibilidades de que el método de optimización converja hacia una solución (búsqueda inteligente). Imagínese que la lucha contra la propagación de un nuevo virus requiere la optimización de un proceso que puede evolucionar con el tiempo a medida que se descubren más síntomas y casos.



\end{itemize}



\subsubsection{Regresión}
		
	Es una técnica de clasificación que es particularmente adecuada para un modelo continuo, lineal (mínimos cuadrados), polinomio, y regresiones logísticas se encuentran entre las técnicas más utilizadas para mejorar o ajustar un modelo paramétrico, o función, $y = f ( x j )$, a un conjunto de datos. Se considera la regresión un caso especializado de clasificación para los cuales las variables de salida son continuas en lugar de categórica.


\subsubsection{Clasificadores}
		% Referencias
% ~\cite{} 


	Los clasificadores son un tipo de algoritmo de aprendizaje supervisado. Una de las ideas principales es inferir una función etiquetando dato de entrenamiento, con lo anterior puede extraer conocimiento específico de un dominio a partir de datos históricos. Por ejemplo, un clasificador puede ser construido para identificar una enfermedad a partir de un conjunto de síntomas.
	
 % TODO PONER UN EJEMPLO DEL PAPER QUE PREDICE CON UN CLASIFICADOR

 

% Classification is a type of supervised machine learning. In supervised learning, the goal is to infer a function using labeled training data. The function can then be used to determine the label for a new dataset (where the labels are unknown). A non-exhaustive list of classification algorithms that can be used for building the model includes decision trees, logistic regression, neural networks, support vector machines, naïve Bayes, and Bayes Point Machines.

% Classification algorithms are used to predict the label for input data (where the label is unknown). Labels are also referred to as classes, groups, or target variables. For example, a telecommunication company wants to predict the following:

% Churn: Customers who have an inclination to switch to a different telecommunication provider
% Propensity to Buy: Customers willing to buy new products or services
% Upselling: Customers willing to buy upgraded services or add-ons
% To achieve this, the telecommunication company builds a classification model using training data (where the labels are known or have already been predefined). In this section, you’ll look at several common classification algorithms that can be used for building the model. Once the model has been built and validated using test data, data scientists at the telecommunication company can use the model to predict churn, propensity to buy, and upselling labels for customers (where the labels are unknown). Consequently, the telecommunication company can use these predictions to design marketing strategies that can reduce the customer churn and offer services to the customers that are more willing to buy new services or upsell.

% Other scenarios where classification algorithms are commonly used include financial institutions, where models are used to determine whether a credit card transaction is a fraudulent case or if a loan application should be approved based on the financial profile of the customer. Hotels and airlines use models to determine whether a customer should be upgraded to a higher level of service (such as from economy to business class, from a normal room to a suite, etc.).

% The classification problem is defined as follows: given an input sample of image, where x1 refers to an item in the sample of size d, the goal of classification is to learn the mapping image, where y ∈ Y is a class.

% An instance of data belongs to one of J groups (or classes), such as image. For example, in a two-class classification problem for the telecommunication scenario, Class C1 refers to customers that will churn and switch to a new telecommunication provider, and Class C2 refers to customers that will not churn.

% To achieve this, labeled training data is first used to train a model using one of the classification algorithms. This is then validated by using test data to determine the number of mistakes made by the trained model (the classifier). Various metrics are used to measure the performance of the classifier. These include measuring the accuracy, precision, recall, and the area under curve (AUC) of the trained model.
\subsubsection{Predictores}
		% Referencias
% ~\cite{} 

  Los Algoritmos para realizar Predicciones, son el resultado de un proceso de análisis, extración y modelamiento de la información estudiada. Lo anterior puede ser utilizado para inferir sobre datos futuros. Este tema se profundizará y detallará en el capítulo~\ref{ch:predicciones-webaccess}.



\vspace{1cm}

Existen una gran cantidad de algoritmos, los anteriores no son la totalidad que ofrece \machinelearning, todos los modelos pueden ir sufriendo variaciones o alteraciones acorde a como se presenten los datos. Estos deben conocerse para que el modelo o el uso de estos sea valido, es lógico pensar que si usa un algoritmo para un escenario que no corresponde este sea totalmente invalido o inconsistente.



Al momento de realizar nuevos algoritmos o modificaciones que ayuden a modelos tradicionales de \machinelearning, se crear un nueva perspectiva para abordar variados problemas. \emph{Sculley y Brodley} en~\cite{Sculley2006} plantean y discuten estas perspectiva con un enfoque más teórico e empírico, la idea fundamental que planten es que si se tiene un \emph{string} $x$ e $y$ y al ser comprimidos juntos son eficaces, ambos \emph{string} comparten la misma información. Si estos \emph{string} representarán grandes volúmenes de datos puede ser un escenario el cual puede sumar complejidad al momento de realizar un entrenamiento, sin usar técnicas de compresión, pero si se considera esta aproximación es posible entregar mejores resultado, siendo un beneficio a favor. En el ejemplo anterior como en otras tareas se puede usar algoritmos de compresión sobre \machinelearning, tanto en tareas tradicionales de clasificación y agrupamiento.  Esta perspectiva no es nuevo y ha aparecido en variados campos~\cite{Sculley2006}, algunas veces con la promesa de reducir los problemas en la selección implícita de ciertas características de un conjunto de datos. 

Li~\etal en~\cite{Li2005} lograron implementar una nueva clase \emph{string kernel}\footnote{Los String kernel, son operaciones que permiten trabajar funciones no lineales, como lineales. Para un explicación detallada ver~\cite{Li2005}.} basada en un algoritmo de compresión y lo utilizaron para manejar un clasificador \texttt{SVM} para la clasificación de géneros musicales. El algoritmo utilizado por \cite{Sculley2006} es de la familia de {Lempel and Ziv}, el cual lo veremos mas adelante en la sección ~\ref{}.


Una de las motivaciones que se puede tener al usar compresión es ahorrar espacio para ciertos conjuntos de datos que se desean entrenar, también ahorrar tiempo en transmitir o mover cierto conjunto de datos y finalmente la mayoría de los datos en general presentantan redundancia de información.

Los algoritmos de compresión pueden ser con o sin pérdida de información, abordaremos solamente los algoritmo sin pérdida (\losslessdatacompression), estos en función de los análisis predictivo no presente explícitamente una acercamiento, pero poseen una estrecha relación,temáticas que usan ciertos algoritmos de tipo probabilistas de \machinelearning. Estos algoritmos son los \emph{modelos variables de markov}, son tipos de modelos los cuales están basados en las cadenas de Markov, esto se detallará con mayor precisión en la sección~\ref{}, además se realizará una aproximación a los \emph{modelos ocultos de markov} que son mas asociados a estudios de \machinelearning.





%\subsubsection{Resumen}

En esta introducción hemos introducidos varios conceptos que veremos mas adelante, uno de los más importante es el acercamiento de dos áreas para poder complementar una solución específica. Hemos hablado de como ciertos modelos son entrenados por algoritmos. Además hemos visto una primera revisión a la literatura en la cual ha validado a que podemos hacer esta intersección de áreas de manera correcta, para lograr un objetivo como es las predicciones. En las próximas secciones de este capitulo se entregará una base mas detallada para comprender el aporte de que puede hacer un algoritmo \losslessdatacompression para un modelo de entrenamiento y predicción de \machinelearning.





